Μουσάων Ἑλικωνιάδων ἀρχώμεθ' ἀείδειν,

αἵ θ' Ἑλικῶνος ἔχουσιν ὄρος μέγα τε ζάθεόν τε, {[}vírgula no fim

καί τε περὶ κρήνην ἰοειδέα πόσσ' ἁπαλοῖσιν {[}mudar tipo de sigma

ὀρχεῦνται καὶ βωμὸν ἐρισθενέος Κρονίωνος· {[}trocar . por ·

καί τε λοεσσάμεναι τέρενα χρόα Περμησσοῖο 5

ἠ' Ἵππου κρήνης ἠ' Ὀλμειοῦ ζαθέοιο

ἀκροτάτῳ Ἑλικῶνι χοροὺς ἐνεποιήσαντο, {[}vírgula no fim

καλοὺς ἱμερόεντας, ἐπερρώσαντο δὲ ποσσίν. {[}deletar vírgula / trocar :
por ,

ἔνθεν ἀπορνύμεναι κεκαλυμμέναι ἠέρι πολλῷ {[}deletar duas vírgulas

ἐννύχιαι στεῖχον περικαλλέα ὄσσαν ἱεῖσαι, 10

ὑμνεῦσαι Δία τ' αἰγίοχον καὶ πότνιαν Ἥρην

Ἀργείην, χρυσέοισι πεδίλοις ἐμβεβαυῖαν, {[}retirar o ``trema'' do iota

κούρην τ' αἰγιόχοιο Διὸς γλαυκῶπιν Ἀθήνην

Φοῖβόν τ' Ἀπόλλωνα καὶ Ἄρτεμιν ἰοχέαιραν

ἠδὲ Ποσειδάωνα γαιήοχον ἐννοσίγαιον 15 {[}trocar \emph{e} por \emph{ai}
/ deletar as 2 vírgulas

καὶ Θέμιν αἰδοίην ἑλικοβλέφαρόν τ' Ἀφροδίτην

Ἥβην τε χρυσοστέφανον καλήν τε Διώνην

Λητώ τ' Ἰαπετόν τε ἰδὲ Κρόνον ἀγκυλομήτην

Ἠῶ τ' Ἠέλιόν τε μέγαν λαμπράν τε Σελήνην

Γαῖάν τ' Ὠκεανόν τε μέγαν καὶ Νύκτα μέλαιναν 20

ἄλλων τ' ἀθανάτων ἱερὸν γένος αἰὲν ἐόντων.

αἵ νύ ποθ' Ἡσίοδον καλὴν ἐδίδαξαν ἀοιδήν,

ἄρνας ποιμαίνονθ' Ἑλικῶνος ὕπο ζαθέοιο.

τόνδε δέ με πρώτιστα θεαὶ πρὸς μῦθον ἔειπον,

Μοῦσαι Ὀλυμπιάδες, κοῦραι Διὸς αἰγιόχοιο· 25 {[}trocar : por ·

``ποιμένες ἄγραυλοι, κάκ' ἐλέγχεα, γαστέρες οἶον, {[}aspas

ἴδμεν ψεύδεα πολλὰ λέγειν ἐτύμοισιν ὁμοῖα,

ἴδμεν δ' εὖτ' ἐθέλωμεν ἀληθέα γηρύσασθαι.'' {[}aspas / deletar as 2
vírgulas

ὣς ἔφασαν κοῦραι μεγάλου Διὸς ἀρτιέπειαι,

καί μοι σκῆπτρον ἔδον δάφνης ἐριθηλέος ὄζον 30

δρέψασαι, θηητόν· ἐνέπνευσαν δέ μοι αὐδὴν {[}trocar : por ·

θέσπιν, ἵνα κλείοιμι τά τ' ἐσσόμενα πρό τ' ἐόντα, {[}trocar . por ,

καί μ' ἐκέλονθ' ὑμνεῖν μακάρων γένος αἰὲν ἐόντων,

σφᾶς δ' αὐτὰς πρῶτόν τε καὶ ὕστατον αἰὲν ἀείδειν.

ἀλλὰ τίη μοι ταῦτα περὶ δρῦν ἢ περὶ πέτρην; 35 {[}trocar · por ;

τύνη, Μουσάων ἀρχώμεθα, ταὶ Διὶ πατρὶ

ὑμνεῦσαι τέρπουσι μέγαν νόον ἐντὸς Ὀλύμπου,

εἴρευσαι τά τ' ἐόντα τά τ' ἐσσόμενα πρό τ' ἐόντα,

φωνῇ ὁμηρεῦσαι, τῶν δ' ἀκάματος ῥέει αὐδὴ {[}trocar : por ,

ἐκ στομάτων ἡδεῖα· γελᾷ δέ τε δώματα πατρὸς 40 {[}trocar : por ·

Ζηνὸς ἐριγδούποιο θεᾶν ὀπὶ λειριοέσσῃ

σκιδναμένῃ, ἠχεῖ δὲ κάρη νιφόεντος Ὀλύμπου {[}trocar : por ,

δώματά τ' ἀθανάτων· αἱ δ' ἄμβροτον ὄσσαν ἱεῖσαι {[}{[}trocar , por · /
deletar acento

θεῶν γένος αἰδοῖον πρῶτον κλείουσιν ἀοιδῇ

ἐξ ἀρχῆς, οὓς Γαῖα καὶ Οὐρανὸς εὐρὺς ἔτικτεν, 45

οἵ τ' ἐκ τῶν ἐγένοντο, θεοὶ δωτῆρες ἐάων· {[}acrescentar vírgula /
trocar . por ·

δεύτερον αὖτε Ζῆνα θεῶν πατέρ' ἠδὲ καὶ ἀνδρῶν, {[}deletar 1ª vírgula

ἀρχόμεναί θ' ὑμνεῦσι καὶ ἐκλήγουσαι ἀοιδῆς,

ὅσσον φέρτατός ἐστι θεῶν κάρτει τε μέγιστος·

{[}trocar duas letras de lugar entre si / deletar acento no iota /
{[}trocar . por ·

αὖτις δ' ἀνθρώπων τε γένος κρατερῶν τε Γιγάντων 50

ὑμνεῦσαι τέρπουσι Διὸς νόον ἐντὸς Ὀλύμπου

Μοῦσαι Ὀλυμπιάδες, κοῦραι Διὸς αἰγιόχοιο.

τὰς ἐν Πιερίῃ Κρονίδῃ τέκε πατρὶ μιγεῖσα

Μνημοσύνη, γουνοῖσιν Ἐλευθῆρος μεδέουσα,

λησμοσύνην τε κακῶν ἄμπαυμά τε μερμηράων. 55

ἐννέα γάρ οἱ νύκτας ἐμίσγετο μητίετα Ζεὺς

νόσφιν ἀπ' ἀθανάτων ἱερὸν λέχος εἰσαναβαίνων· {[}trocar : por ·

ἀλλ' ὅτε δή ῥ' ἐνιαυτὸς ἔην, περὶ δ' ἔτραπον ὧραι

μηνῶν φθινόντων, περὶ δ' ἤματα πόλλ' ἐτελέσθη,

ἡ δ' ἔτεκ' ἐννέα κούρας, ὁμόφρονας, ᾗσιν ἀοιδὴ 60 {[}deletar acento /
acrescentar ,

μέμβλεται ἐν στήθεσσιν, ἀκηδέα θυμὸν ἐχούσαις,

τυτθὸν ἀπ' ἀκροτάτης κορυφῆς νιφόεντος Ὀλύμπου· {[}trocar . por ·

ἔνθά σφιν λιπαροί τε χοροὶ καὶ δώματα καλά, {[}trocar . por ,

πὰρ δ' αὐτῇς Χάριτές τε καὶ Ἵμερος οἰκί' ἔχουσιν

ἐν θαλίῃς· ἐρατὴν δὲ διὰ στόμα ὄσσαν ἱεῖσαι 65 {[}trocar : por ·

μέλπονται, πάντων τε νόμους καὶ ἤθεα κεδνὰ {[}acrescentar vírgula

ἀθανάτων κλείουσιν, ἐπήρατον ὄσσαν ἱεῖσαι.

αἳ τότ' ἴσαν πρὸς Ὄλυμπον, ἀγαλλόμεναι ὀπὶ καλῇ, {[}acrescentar vírgula

ἀμβροσίῃ μολπῇ· περὶ δ' ἴαχε γαῖα μέλαινα {[}trocar : por ·

ὑμνεύσαις, ἐρατὸς δὲ ποδῶν ὕπο δοῦπος ὀρώρει 70

νισομένων πατέρ' εἰς ὅν· ὁ δ' οὐρανῷ ἐμβασιλεύει, {[}trocar : por · /
deletar acento

αὐτὸς ἔχων βροντὴν ἠδ' αἰθαλόεντα κεραυνόν,

κάρτει νικήσας πατέρα Κρόνον· εὖ δὲ ἕκαστα {[}trocar : por ·

ἀθανάτοις διέταξε ὁμῶς καὶ ἐπέφραδε τιμάς.

ταῦτ' ἄρα Μοῦσαι ἄειδον Ὀλύμπια δώματ' ἔχουσαι,75 {[}deletar 1ª vírgula
no meio

ἐννέα θυγατέρες μεγάλου Διὸς ἐκγεγαυῖαι,

Κλειώ τ' Εὐτέρπη τε Θάλειά τε Μελπομένη τε

Τερψιχόρη τ' Ἐρατώ τε Πολύμνιά τ' Οὐρανίη τε

Καλλιόπη θ'· ἡ δὲ προφερεστάτη ἐστὶν ἁπασέων. {[}trocar : por · /
deletar acento

ἡ γὰρ καὶ βασιλεῦσιν ἅμ' αἰδοίοισιν ὀπηδεῖ. 80 {[}deletar acento, manter
espírito

ὅντινα τιμήσουσι Διὸς κοῦραι μεγάλοιο

γεινόμενόν τε ἴδωσι διοτρεφέων βασιλήων,

τῷ μὲν ἐπὶ γλώσσῃ γλυκερὴν χείουσιν ἐέρσην,

τοῦ δ' ἔπε' ἐκ στόματος ῥεῖ μείλιχα· οἱ δέ νυ λαοὶ {[}trocar : por · /
trocar \emph{de} por \emph{nu}

πάντες ἐς αὐτὸν ὁρῶσι διακρίνοντα θέμιστας 85

ἰθείῃσι δίκῃσιν· ὁ δ' ἀσφαλέως ἀγορεύων {[}trocar : por ·

αἶψά τι καὶ μέγα νεῖκος ἐπισταμένως κατέπαυσε· {[}trocar : por ·

τούνεκα γὰρ βασιλῆες ἐχέφρονες, οὕνεκα λαοῖς {[}deletar espírito

βλαπτομένοις ἀγορῆφι μετάτροπα ἔργα τελεῦσι

ῥηιδίως, μαλακοῖσι παραιφάμενοι ἐπέεσσιν· 90 {[}trocar . por ·

ἐρχόμενον δ' ἀν' ἀγῶνα θεὸν ὣς ἱλάσκονται

αἰδοῖ μειλιχίῃ, μετὰ δὲ πρέπει ἀγρομένοισι. {[}trocar : por .

τοίη Μουσάων ἱερὴ δόσις ἀνθρώποισιν.

ἐκ γάρ τοι Μουσέων καὶ ἑκηβόλου Ἀπόλλωνος

ἄνδρες ἀοιδοὶ ἔασιν ἐπὶ χθόνα καὶ κιθαρισταί, 95

ἐκ δὲ Διὸς βασιλῆες· ὁ δ' ὄλβιος, ὅντινα Μοῦσαι {[}trocar : por ·

φίλωνται· γλυκερή οἱ ἀπὸ στόματος ῥέει αὐδή. {[}trocar : por ·

εἰ γάρ τις καὶ πένθος ἔχων νεοκηδέι θυμῷ

ἄζηται κραδίην ἀκαχήμενος, αὐτὰρ ἀοιδὸς

Μουσάων θεράπων κλεῖα προτέρων ἀνθρώπων 100 {[}alterar vogal e acento

ὑμνήσει μάκαράς τε θεοὺς οἳ Ὄλυμπον ἔχουσιν, {[}deletar vírgula

αἶψ' ὅ γε δυσφροσυνέων ἐπιλήθεται οὐδέ τι κηδέων

μέμνηται· ταχέως δὲ παρέτραπε δῶρα θεάων. {[}trocar : por ·

χαίρετε τέκνα Διός, δότε δ' ἱμερόεσσαν ἀοιδήν· {[}trocar . por ·

κλείετε δ' ἀθανάτων ἱερὸν γένος αἰὲν ἐόντων, 105

οἳ Γῆς ἐξεγένοντο καὶ Οὐρανοῦ ἀστερόεντος,

Νυκτός τε δνοφερῆς, οὕς θ' ἁλμυρὸς ἔτρεφε Πόντος.

εἴπατε δ' ὡς τὰ πρῶτα θεοὶ καὶ γαῖα γένοντο {[}deletar vírgula

καὶ ποταμοὶ καὶ πόντος ἀπείριτος οἴδματι θυίων {[}deletar as duas
vírgulas no fim

ἄστρά τε λαμπετόωντα καὶ οὐρανὸς εὐρὺς ὕπερθεν· 110 {[}colocar · no fim

οἵ τ' ἐκ τῶν ἐγένοντο, θεοὶ δωτῆρες ἐάων· {[}por vírgula / colocar · no
fim

ὥς τ' ἄφενος δάσσαντο καὶ ὡς τιμὰς διέλοντο, {[}por virgula no fim

ἠδὲ καὶ ὡς τὰ πρῶτα πολύπτυχον ἔσχον Ὄλυμπον.

ταῦτά μοι ἔσπετε Μοῦσαι Ὀλύμπια δώματ' ἔχουσαι {[}deletar vírgula

ἐξ ἀρχῆς, καὶ εἴπαθ', ὅτι πρῶτον γένετ' αὐτῶν. 115

ἤτοι μὲν πρώτιστα Χάος γένετ'· αὐτὰρ ἔπειτα {[}trocar , por ·

Γαῖ' εὐρύστερνος, πάντων ἕδος ἀσφαλὲς αἰεὶ

ἀθανάτων οἳ ἔχουσι κάρη νιφόεντος Ὀλύμπου {[}deletar as 2 vírgulas

Τάρταρά τ' ἠερόεντα μυχῷ χθονὸς εὐρυοδείης,

ἠδ' Ἔρος, ὃς κάλλιστος ἐν ἀθανάτοισι θεοῖσι, 120

λυσιμελής, πάντων τε θεῶν πάντων τ' ἀνθρώπων

δάμναται ἐν στήθεσσι νόον καὶ ἐπίφρονα βουλήν.

TIRAR O SINAL DE ``PARÁGRAFO'', AQUI E LOGO ABAIXO

ἐκ Χάεος δ' Ἔρεβός τε μέλαινά τε Νὺξ ἐγένοντο· {[}trocar : por ·

Νυκτὸς δ' αὖτ' Αἰθήρ τε καὶ Ἡμέρη ἐξεγένοντο,

οὓς τέκε κυσαμένη Ἐρέβει φιλότητι μιγεῖσα. 125

Γαῖα δέ τοι πρῶτον μὲν ἐγείνατο ἶσον ἑωυτῇ

Οὐρανὸν ἀστερόενθ', ἵνα μιν περὶ πάντα καλύπτοι,

ὄφρ' εἴη μακάρεσσι θεοῖς ἕδος ἀσφαλὲς αἰεί, {[}trocar . por ,

γείνατο δ' οὔρεα μακρά, θεᾶν χαρίεντας ἐναύλους {[}deletar vírgula no
fim

Νυμφέων, αἳ ναίουσιν ἀν' οὔρεα βησσήεντα, 130 {[}trocar . por , no fim

ἠδὲ καὶ ἀτρύγετον πέλαγος τέκεν οἴδματι θυῖον,

{[}juntar e mudar acento/espírito / deletar vírgula

Πόντον, ἄτερ φιλότητος ἐφιμέρου· αὐτὰρ ἔπειτα {[}trocar : por ·

Οὐρανῷ εὐνηθεῖσα τέκ' Ὠκεανὸν βαθυδίνην {[}deletar vírgula no fim

Κοῖόν τε Κρεῖόν θ' Ὑπερίονά τ' Ἰαπετόν τε

Θείαν τε Ῥείαν τε Θέμιν τε Μνημοσύνην τε 135

Φοίβην τε χρυσοστέφανον Τηθύν τ' ἐρατεινήν.

τοὺς δὲ μέθ' ὁπλότατος γένετο Κρόνος ἀγκυλομήτης,

δεινότατος παίδων, θαλερὸν δ' ἤχθηρε τοκῆα. {[}trocar : por ,

γείνατο δ' αὖ Κύκλωπας ὑπέρβιον ἦτορ ἔχοντας,

Βρόντην τε Στερόπην τε καὶ Ἄργην ὀβριμόθυμον, 140

οἳ Ζηνὶ βροντήν τ' ἔδοσαν τεῦξάν τε κεραυνόν.

οἱ δ' ἤτοι τὰ μὲν ἄλλα θεοῖς ἐναλίγκιοι ἦσαν,

{[}deletar acento e juntar/separar as palavras de outra forma

μοῦνος δ' ὀφθαλμὸς μέσσῳ ἐνέκειτο μετώπῳ· {[}trocar . por ·

Κύκλωπες δ' ὄνομ' ἦσαν ἐπώνυμον, οὕνεκ' ἄρά σφεων

κυκλοτερὴς ὀφθαλμὸς ἕεις ἐνέκειτο μετώπῳ· 145 {[}trocar : por ·

ἰσχὺς δ' ἠδὲ βίη καὶ μηχαναὶ ἦσαν ἐπ' ἔργοις.

ἄλλοι δ' αὖ Γαίης τε καὶ Οὐρανοῦ ἐξεγένοντο

τρεῖς παῖδες μεγάλοι \textless{}τε\textgreater{} καὶ ὄβριμοι, οὐκ
ὀνομαστοί, {[}acrescentar colchetes

Κόττος τε Βριάρεώς τε Γύγης θ', ὑπερήφανα τέκνα. {[}acrescentar gama

τῶν ἑκατὸν μὲν χεῖρες ἀπ' ὤμων ἀίσσοντο, 150

ἄπλαστοι, κεφαλαὶ δὲ ἑκάστῳ πεντήκοντα

ἐξ ὤμων ἐπέφυκον ἐπὶ στιβαροῖσι μέλεσσιν· {[}trocar : por ·

ἰσχὺς δ' ἄπλητος κρατερὴ μεγάλῳ ἐπὶ εἴδει.

ὅσσοι γὰρ Γαίης τε καὶ Οὐρανοῦ ἐξεγένοντο,

δεινότατοι παίδων, σφετέρῳ δ' ἤχθοντο τοκῆι 155

ἐξ ἀρχῆς· καὶ τῶν μὲν ὅπως τις πρῶτα γένοιτο, {[}trocar : por ·

πάντας ἀποκρύπτασκε καὶ ἐς φάος οὐκ ἀνίεσκε

Γαίης ἐν κευθμῶνι, κακῷ δ' ἐπετέρπετο ἔργῳ, {[}vírgula no fim

Οὐρανός· ἡ δ' ἐντὸς στοναχίζετο Γαῖα πελώρη

{[}trocar . por · / deletar acento

στεινομένη, δολίην δὲ κακὴν ἐπεφράσσατο τέχνην. 160

{[}trocar : por , / deletar e trocar letras

αἶψα δὲ ποιήσασα γένος πολιοῦ ἀδάμαντος

τεῦξε μέγα δρέπανον καὶ ἐπέφραδε παισὶ φίλοισιν· {[}trocar . por ·

εἶπε δὲ θαρσύνουσα, φίλον τετιημένη ἦτορ· {[}trocar : por ·

``παῖδες ἐμοὶ καὶ πατρὸς ἀτασθάλου, αἴ κ' ἐθέλητε {[}aspas

πείθεσθαι· πατρός κε κακὴν τεισαίμεθα λώβην 165 {[}trocar , por · /
acrescentar ε

ὑμετέρου· πρότερος γὰρ ἀεικέα μήσατο ἔργα.'' {[}trocar : por · / aspas

ὣς φάτο· τοὺς δ' ἄρα πάντας ἕλεν δέος, οὐδέ τις αὐτῶν {[}trocar : por ·

φθέγξατο. θαρσήσας δὲ μέγας Κρόνος ἀγκυλομήτης

αἶψ' αὖτις μύθοισι προσηύδα μητέρα κεδνήν·

{[}deletar iota e por apóstrofo / trocar : por ·

``μῆτερ, ἐγώ κεν τοῦτό γ' ὑποσχόμενος τελέσαιμι 170 {[}aspas

ἔργον, ἐπεὶ πατρός γε δυσωνύμου οὐκ ἀλεγίζω

ἡμετέρου· πρότερος γὰρ ἀεικέα μήσατο ἔργα.'' {[}trocar : por · / aspas

ὣς φάτο· γήθησεν δὲ μέγα φρεσὶ Γαῖα πελώρη· {[}trocar : por ·

εἷσε δέ μιν κρύψασα λόχῳ, ἐνέθηκε δὲ χερσὶν {[}trocar : por ,

ἅρπην καρχαρόδοντα, δόλον δ' ὑπεθήκατο πάντα. 175 {[}trocar : por ,

ἦλθε δὲ νύκτ' ἐπάγων μέγας Οὐρανός, ἀμφὶ δὲ Γαίῃ

ἱμείρων φιλότητος ἐπέσχετο, καί ῥ' ἐτανύσθη

πάντῃ· ὁ δ' ἐκ λοχέοιο πάις ὠρέξατο χειρὶ

{[}colocar iota subscrito no eta / trocar : por ·

σκαιῇ, δεξιτερῇ δὲ πελώριον ἔλλαβεν ἅρπην,

μακρὴν καρχαρόδοντα, φίλου δ' ἀπὸ μήδεα πατρὸς 180

ἐσσυμένως ἤμησε, πάλιν δ' ἔρριψε φέρεσθαι

ἐξοπίσω. τὰ μὲν οὔ τι ἐτώσια ἔκφυγε χειρός· {[}trocar : por . / trocar :
por ·

ὅσσαι γὰρ ῥαθάμιγγες ἀπέσσυθεν αἱματόεσσαι,

πάσας δέξατο Γαῖα· περιπλομένων δ' ἐνιαυτῶν {[}trocar : por ·

γείνατ' Ἐρινῦς τε κρατερὰς μεγάλους τε Γίγαντας, 185

τεύχεσι λαμπομένους, δολίχ' ἔγχεα χερσὶν ἔχοντας,

Νύμφας θ' ἃς Μελίας καλέουσ' ἐπ' ἀπείρονα γαῖαν. {[}trocar tipo de sigma

μήδεα δ' ὡς τὸ πρῶτον ἀποτμήξας ἀδάμαντι

κάββαλ' ἀπ' ἠπείροιο πολυκλύστῳ ἐνὶ πόντῳ,

ὣς φέρετ' ἂμ πέλαγος πουλὺν χρόνον, ἀμφὶ δὲ λευκὸς 190

ἀφρὸς ἀπ' ἀθανάτου χροὸς ὤρνυτο· τῷ δ' ἔνι κούρη {[}trocar : por ·

ἐθρέφθη· πρῶτον δὲ Κυθήροισι ζαθέοισιν {[}trocar : por ·

ἔπλητ', ἔνθεν ἔπειτα περίρρυτον ἵκετο Κύπρον.

ἐκ δ' ἔβη αἰδοίη καλὴ θεός, ἀμφὶ δὲ ποίη

ποσσὶν ὕπο ῥαδινοῖσιν ἀέξετο· τὴν δ' Ἀφροδίτην 195 {[}trocar : por ·

ἀφρογενέα τε θεὰν καὶ ἐυστέφανον Κυθέρειαν

κικλήσκουσι θεοί τε καὶ ἀνέρες, οὕνεκ' ἐν ἀφρῷ

θρέφθη· ἀτὰρ Κυθέρειαν, ὅτι προσέκυρσε Κυθήροις· {[}trocar : por · /
{[}trocar : por ·

Κυπρογενέα δ', ὅτι γέντο περικλύστῳ ἐνὶ Κύπρῳ· {[}trocar : por · /
alterar letras

ἠδὲ φιλομμειδέα, ὅτι μηδέων ἐξεφαάνθη. 200

τῇ δ' Ἔρος ὡμάρτησε καὶ Ἵμερος ἔσπετο καλὸς

γεινομένῃ τὰ πρῶτα θεῶν τ' ἐς φῦλον ἰούσῃ· {[}trocar . por ·

ταύτην δ' ἐξ ἀρχῆς τιμὴν ἔχει ἠδὲ λέλογχε

μοῖραν ἐν ἀνθρώποισι καὶ ἀθανάτοισι θεοῖσι,

παρθενίους τ' ὀάρους μειδήματά τ' ἐξαπάτας τε 205

τέρψίν τε γλυκερὴν φιλότητά τε μειλιχίην τε.

τοὺς δὲ πατὴρ Τιτῆνας ἐπίκλησιν καλέεσκε

παῖδας νεικείων μέγας Οὐρανός, οὓς τέκεν αὐτός· {[}trocar : por ·

φάσκε δὲ τιταίνοντας ἀτασθαλίῃ μέγα ῥέξαι

ἔργον, τοῖο δ' ἔπειτα τίσιν μετόπισθεν ἔσεσθαι. 210

Νὺξ δ' ἔτεκε στυγερόν τε Μόρον καὶ Κῆρα μέλαιναν {[}maiúscula

καὶ Θάνατον, τέκε δ' Ὕπνον, ἔτικτε δὲ φῦλον Ὀνείρων. {[}trocar : por ·

οὔ τινι κοιμηθεῖσα θεῶν τέκε Νὺξ ἐρεβεννή. {[}trocar , por .

δεύτερον αὖ Μῶμον καὶ Ὀιζὺν ἀλγινόεσσαν

Ἑσπερίδας θ', αἷς μῆλα πέρην κλυτοῦ Ὠκεανοῖο 215 {[}troca de letra; iota
subs.

χρύσεα καλὰ μέλουσι φέροντά τε δένδρεα καρπόν· {[}trocar : por ·

καὶ Μοίρας καὶ Κῆρας ἐγείνατο νηλεοποίνους,

{[}Κλωθώ τε Λάχεσίν τε καὶ Ἄτροπον, αἵ τε βροτοῖσι {[}acrescentar
colchete / espaço

γεινομένοισι διδοῦσιν ἔχειν ἀγαθόν τε κακόν τε,{]} {[}idem

αἵ τ' ἀνδρῶν τε θεῶν τε παραιβασίας ἐφέπουσιν, 220 {[}por um espaço /
trocar : por ·

οὐδέ ποτε λήγουσι θεαὶ δεινοῖο χόλοιο,

πρίν γ' ἀπὸ τῷ δώωσι κακὴν ὄπιν, ὅστις ἁμάρτῃ.

τίκτε δὲ καὶ Νέμεσιν πῆμα θνητοῖσι βροτοῖσι {[}deletar virgula no fim

Νὺξ ὀλοή· μετὰ τὴν δ' Ἀπάτην τέκε καὶ Φιλότητα {[}trocar : por ·

Γῆράς τ' οὐλόμενον, καὶ Ἔριν τέκε καρτερόθυμον. 225

αὐτὰρ Ἔρις στυγερὴ τέκε μὲν Πόνον ἀλγινόεντα

Λήθην τε Λιμόν τε καὶ Ἄλγεα δακρυόεντα

Ὑσμίνας τε Μάχας τε Φόνους τ' Ἀνδροκτασίας τε

Νείκεά τε Ψεύδεά τε Λόγους τ' Ἀμφιλλογίας τε {[}maiúscula / deletar
sigma

Δυσνομίην τ' Ἄτην τε, συνήθεας ἀλλήλῃσιν, 230

Ὅρκόν θ', ὃς δὴ πλεῖστον ἐπιχθονίους ἀνθρώπους

πημαίνει, ὅτε κέν τις ἑκὼν ἐπίορκον ὀμόσσῃ· {[}trocar . por ·

Νηρέα δ' ἀψευδέα καὶ ἀληθέα γείνατο Πόντος {[}deletar virgula no fim

πρεσβύτατον παίδων· αὐτὰρ καλέουσι γέροντα, {[}trocar : por ·

οὕνεκα νημερτής τε καὶ ἤπιος, οὐδὲ θεμίστων 235

λήθεται, ἀλλὰ δίκαια καὶ ἤπια δήνεα οἶδεν· {[}trocar : por ·

αὖτις δ' αὖ Θαύμαντα μέγαν καὶ ἀγήνορα Φόρκυν

Γαίῃ μισγόμενος καὶ Κητὼ καλλιπάρηον {[}trocar por eta sem iotasubscrito

Εὐρυβίην τ' ἀδάμαντος ἐνὶ φρεσὶ θυμὸν ἔχουσαν.

Νηρῆος δ' ἐγένοντο μεγήριτα τέκνα θεάων 240

πόντῳ ἐν ἀτρυγέτῳ καὶ Δωρίδος ἠυκόμοιο,

κούρης Ὠκεανοῖο τελήεντος ποταμοῖο, {[}deletar a vírgula

Πρωθώ τ' Εὐκράντη τε Σαώ τ' Ἀμφιτρίτη τε {[}trocar rô por lambda

Εὐδώρη τε Θέτις τε Γαλήνη τε Γλαύκη τε,

Κυμοθόη Σπειώ τε θοὴ Θαλίη τ' ἐρόεσσα 245

{[}teta inicial minúsculo / deletar e refazer a sequencia seguinte

Πασιθέη τ' Ἐρατώ τε καὶ Εὐνίκη ῥοδόπηχυς

καὶ Μελίτη χαρίεσσα καὶ Εὐλιμένη καὶ Ἀγαυὴ

Δωτώ τε Πρωτώ τε Φέρουσά τε Δυναμένη τε

Νησαίη τε καὶ Ἀκταίη καὶ Πρωτομέδεια,

Δωρὶς καὶ Πανόπη καὶ εὐειδὴς Γαλάτεια 250 {[}alterar vogais

Ἱπποθόη τ' ἐρόεσσα καὶ Ἱππονόη ῥοδόπηχυς

Κυμοδόκη θ', ἣ κύματ' ἐν ἠεροειδέι πόντῳ

πνοιάς τε ζαέων ἀνέμων σὺν Κυματολήγῃ

ῥεῖα πρηΰνει καὶ ἐυσφύρῳ Ἀμφιτρίτῃ,

Κυμώ τ' Ἠιόνη τε ἐυστέφανός θ' Ἁλιμήδη 255

Γλαυκονόμη τε φιλομμειδὴς καὶ Ποντοπόρεια

Λειαγόρη τε καὶ Εὐαγόρη καὶ Λαομέδεια {[}alterar vogais

Πουλυνόη τε καὶ Αὐτονόη καὶ Λυσιάνασσα

Εὐάρνη τε φυὴν ἐρατὴ καὶ εἶδος ἄμωμος

καὶ Ψαμάθη χαρίεσσα δέμας δίη τε Μενίππη 260

Νησώ τ' Εὐπόμπη τε Θεμιστώ τε Προνόη τε

Νημερτής θ', ἣ πατρὸς ἔχει νόον ἀθανάτοιο.

αὗται μὲν Νηρῆος ἀμύμονος ἐξεγένοντο

κοῦραι πεντήκοντα, ἀμύμονα ἔργ' εἰδυῖαι· {[}trocar . por ·

Θαύμας δ' Ὠκεανοῖο βαθυρρείταο θύγατρα 265

ἠγάγετ' Ἠλέκτρην· ἡ δ' ὠκεῖαν τέκεν Ἶριν {[}trocar : por · / deletar
acento

ἠυκόμους θ' Ἁρπυίας, Ἀελλώ τ' Ὠκυπέτην τε, {[}acrescentar vírgula

αἵ ῥ' ἀνέμων πνοιῇσι καὶ οἰωνοῖς ἅμ' ἕπονται

ὠκείῃς πτερύγεσσι· μεταχρόνιαι γὰρ ἴαλλον. {[}trocar : por ·

Φόρκυι δ' αὖ Κητὼ γραίας τέκε καλλιπαρήους 270

{[}deletar `trema' / trocar gama minúsculo por maisc. / eliminar iota
subescrito

ἐκ γενετῆς πολιάς, τὰς δὴ Γραίας καλέουσιν

ἀθάνατοί τε θεοὶ χαμαὶ ἐρχόμενοί τ' ἄνθρωποι,

Πεμφρηδώ τ' εὔπεπλον Ἐνυώ τε κροκόπεπλον, {[}mover espírito

Γοργούς θ', αἳ ναίουσι πέρην κλυτοῦ Ὠκεανοῖο

ἐσχατιῇ πρὸς νυκτός, ἵν' Ἑσπερίδες λιγύφωνοι, 275

Σθεννώ τ' Εὐρυάλη τε Μέδουσά τε λυγρὰ παθοῦσα· {[}trocar . por ·

ἡ μὲν ἔην θνητή, αἱ δ' ἀθάνατοι καὶ ἀγήρῳ, {[}deletar acento nos dois
artigos

αἱ δύο· τῇ δὲ μιῇ παρελέξατο Κυανοχαίτης {[}trocar : por ·

ἐν μαλακῷ λειμῶνι καὶ ἄνθεσιν εἰαρινοῖσι. {[}deletar nu final

τῆς ὅτε δὴ Περσεὺς κεφαλὴν ἀπεδειροτόμησεν, 280 {[}deletar
delta+apóstrofe

ἐξέθορε Χρυσάωρ τε μέγας καὶ Πήγασος ἵππος. {[}reescrever início

τῷ μὲν ἐπώνυμον ἦν, ὅτ' ἄρ' Ὠκεανοῦ παρὰ πηγὰς

{[}inserir epsilon / acrescentar \emph{ar'} / trocar \emph{peri} por
\emph{para}

γένθ', ὁ δ' ἄορ χρύσειον ἔχων μετὰ χερσὶ φίλῃσι. {[}deletar nu final

χὠ μὲν ἀποπτάμενος, προλιπὼν χθόνα μητέρα μήλων,

ἵκετ' ἐς ἀθανάτους· Ζηνὸς δ' ἐν δώμασι ναίει 285 {[}trocar : por ·

βροντήν τε στεροπήν τε φέρων Διὶ μητιόεντι· {[}trocar . por ·

Χρυσάωρ δ' ἔτεκε τρικέφαλον Γηρυονῆα

μιχθεὶς Καλλιρόῃ κούρῃ κλυτοῦ Ὠκεανοῖο· {[}trocar . por ·

τὸν μὲν ἄρ' ἐξενάριξε βίη Ἡρακληείη

βουσὶ πάρ' εἰλιπόδεσσι περιρρύτῳ εἰν Ἐρυθείῃ 290

ἤματι τῷ, ὅτε περ βοῦς ἤλασεν εὐρυμετώπους {[}acrescentar vírgula

Τίρυνθ' εἰς ἱερήν, διαβὰς πόρον Ὠκεανοῖο, {[}vírgula no final

Ὄρθόν τε κτείνας καὶ βουκόλον Εὐρυτίωνα

{[}acrescentar um acento: manter o outro na mesma palavra

σταθμῷ ἐν ἠερόεντι πέρην κλυτοῦ Ὠκεανοῖο.

ἡ δ' ἔτεκ' ἄλλο πέλωρον ἀμήχανον, οὐδὲν ἐοικὸς 295 {[}deletar acento

θνητοῖς ἀνθρώποις οὐδ' ἀθανάτοισι θεοῖσι, {[}deletar nu final

σπῆι ἔνι γλαφυρῷ, θείην κρατερόφρον' Ἔχιδναν, {[}por vírgula

ἥμισυ μὲν νύμφην ἑλικώπιδα καλλιπάρηον, {[}trocar por eta sem
iotasubscrito

ἥμισυ δ' αὖτε πέλωρον ὄφιν δεινόν τε μέγαν τε

αἰόλον ὠμηστήν, ζαθέης ὑπὸ κεύθεσι γαίης. 300 {[}por vírgula

ἔνθα δέ οἱ σπέος ἐστὶ κάτω κοίλῃ ὑπὸ πέτρῃ

τηλοῦ ἀπ' ἀθανάτων τε θεῶν θνητῶν τ' ἀνθρώπων, {[}trocar , por ·

ἔνθ' ἄρα οἱ δάσσαντο θεοὶ κλυτὰ δώματα ναίειν.

ἡ δ' ἔρυτ' εἰν Ἀρίμοισιν ὑπὸ χθόνα λυγρὴ Ἔχιδνα, {[}deletar acento

ἀθάνατος νύμφη καὶ ἀγήραος ἤματα πάντα. 305

τῇ δὲ Τυφάονά φασι μιγήμεναι ἐν φιλότητι

δεινόν θ' ὑβριστήν τ' ἄνομόν θ' ἑλικώπιδι κούρῃ· {[}trocar : por ·

ἡ δ' ὑποκυσαμένη τέκετο κρατερόφρονα τέκνα. {[}deletar acento

Ὄρθον μὲν πρῶτον κύνα γείνατο Γηρυονῆι· {[}trocar : por ·

δεύτερον αὖτις ἔτικτεν ἀμήχανον, οὔ τι φατειόν, 310 {[}por vírgula

Κέρβερον ὠμηστήν, Ἀίδεω κύνα χαλκεόφωνον,

πεντηκοντακέφαλον, ἀναιδέα τε κρατερόν τε· {[}trocar : por ·

τὸ τρίτον Ὕδρην αὖτις ἐγείνατο λύγρ' εἰδυῖαν {[}alterações nas últimas
palavras

Λερναίην, ἣν θρέψε θεὰ λευκώλενος Ἥρη

ἄπλητον κοτέουσα βίῃ Ἡρακληείῃ. 315

καὶ τὴν μὲν Διὸς υἱὸς ἐνήρατο νηλέι χαλκῷ

Ἀμφιτρυωνιάδης σὺν ἀρηιφίλῳ Ἰολάῳ

Ἡρακλέης βουλῇσιν Ἀθηναίης ἀγελείης· {[}trocar . por ·

ἡ δὲ Χίμαιραν ἔτικτε πνέουσαν ἀμαιμάκετον πῦρ,

{[}deletar acento / eliminar a entrada de partágrafo no início do verso

δεινήν τε μεγάλην τε ποδώκεά τε κρατερήν τε. 320 {[}trocar : por .

τῆς ἦν τρεῖς κεφαλαί· μία μὲν χαροποῖο λέοντος, {[}deletar \emph{d'} /
trocar : por ·

ἡ δὲ χιμαίρης, ἡ δ' ὄφιος κρατεροῖο δράκοντος.

{[}deletar acento nos dois artigos / deletar virgula

πρόσθε λέων, ὄπιθεν δὲ δράκων, μέσση δὲ χίμαιρα,

δεινὸν ἀποπνείουσα πυρὸς μένος αἰθομένοιο.

τὴν μὲν Πήγασος εἷλε καὶ ἐσθλὸς Βελλεροφόντης. 325

ἡ δ' ἄρα Φῖκ' ὀλοὴν τέκε Καδμείοισιν ὄλεθρον, {[}deletar acento

Ὄρθῳ ὑποδμηθεῖσα, Νεμειαῖόν τε λέοντα, {[}acrescentar virgula

τόν ῥ' Ἥρη θρέψασα Διὸς κυδρὴ παράκοιτις

γουνοῖσιν κατένασσε Νεμείης, πῆμ' ἀνθρώποις.

ἔνθ' ἄρ' ὅ γ' οἰκείων ἐλεφαίρετο φῦλ' ἀνθρώπων, 330

κοιρανέων Τρητοῖο Νεμείης ἠδ' Ἀπέσαντος· {[}trocar : por ·

ἀλλά ἑ ἲς ἐδάμασσε βίης Ἡρακληείης.

Κητὼ δ' ὁπλότατον Φόρκυι φιλότητι μιγεῖσα

γείνατο δεινὸν ὄφιν, ὃς ἐρεμνῆς κεύθεσι γαίης

πείρασιν ἐν μεγάλοις παγχρύσεα μῆλα φυλάσσει. 335

τοῦτο μὲν ἐκ Κητοῦς καὶ Φόρκυνος γένος ἐστί. {[}deletar nu final

Τηθὺς δ' Ὠκεανῷ ποταμοὺς τέκε δινήεντας,

Νεῖλόν τ' Ἀλφειόν τε καὶ Ἠριδανὸν βαθυδίνην,

Στρυμόνα Μαίανδρόν τε καὶ Ἴστρον καλλιρέεθρον

Φᾶσίν τε Ῥῆσόν τ' Ἀχελῷόν τ' ἀργυροδίνην 340

Νέσσόν τε Ῥοδίον θ' Ἁλιάκμονά θ' Ἑπτάπορόν τε

Γρήνικόν τε καὶ Αἴσηπον θεῖόν τε Σιμοῦντα

Πηνειόν τε καὶ Ἕρμον ἐυρρείτην τε Κάικον

Σαγγάριόν τε μέγαν Λάδωνά τε Παρθένιόν τε

Εὔηνόν τε καὶ Ἀλδῆσκον θεῖόν τε Σκάμανδρον· 345

{[}trocar rô por lambda e acentos / {[}trocar . por ·

τίκτε δὲ θυγατέρων ἱερὸν γένος, αἳ κατὰ γαῖαν

ἄνδρας κουρίζουσι σὺν Ἀπόλλωνι ἄνακτι

καὶ ποταμοῖς, ταύτην δὲ Διὸς πάρα μοῖραν ἔχουσι, {[}pi minúsculo

Πειθώ τ' Ἀδμήτη τε Ἰάνθη τ' Ἠλέκτρη τε

Δωρίς τε Πρυμνώ τε καὶ Οὐρανίη θεοειδὴς 350

Ἱππώ τε Κλυμένη τε Ῥόδειά τε Καλλιρόη τε

Ζευξώ τε Κλυτίη τε Ἰδυῖά τε Πασιθόη τε

Πληξαύρη τε Γαλαξαύρη τ' ἐρατή τε Διώνη

Μηλόβοσίς τε Θόη τε καὶ εὐειδὴς Πολυδώρη {[}trocar por \emph{phi}
maiúsculo

Κερκηίς τε φυὴν ἐρατὴ Πλουτώ τε βοῶπις 355

Περσηίς τ' Ἰάνειρά τ' Ἀκάστη τε Ξάνθη τε

Πετραίη τ' ἐρόεσσα Μενεσθώ τ' Εὐρώπη τε

Μῆτίς τ' Εὐρυνόμη τε Τελεστώ τε κροκόπεπλος

Χρυσηίς τ' Ἀσίη τε καὶ ἱμερόεσσα Καλυψὼ

Εὐδώρη τε Τύχη τε καὶ Ἀμφιρὼ Ὠκυρόη τε 360

καὶ Στύξ, ἣ δή σφεων προφερεστάτη ἐστὶν ἁπασέων.

αὗται ἄρ' Ὠκεανοῦ καὶ Τηθύος ἐξεγένοντο

πρεσβύταται κοῦραι· πολλαί γε μέν εἰσι καὶ ἄλλαι· {[}trocar : por ·

τρὶς γὰρ χίλιαί εἰσι τανίσφυροι Ὠκεανῖναι,

αἵ ῥα πολυσπερέες γαῖαν καὶ βένθεα λίμνης 365

πάντῃ ὁμῶς ἐφέπουσι, θεάων ἀγλαὰ τέκνα. {[}colocar iota subscrito no eta

τόσσοι δ' αὖθ' ἕτεροι ποταμοὶ καναχηδὰ ῥέοντες,

υἱέες Ὠκεανοῦ, τοὺς γείνατο πότνια Τηθύς· {[}trocar : por ·

τῶν ὄνομ' ἀργαλέον πάντων βροτὸν ἄνδρα ἐνισπεῖν, {[}trocar pelo
acusativo singular

οἱ δὲ ἕκαστοι ἴσασιν, ὅσοι περιναιετάουσι. 370 {[}deletar acento

Θεία δ' Ἠέλιόν τε μέγαν λαμπράν τε Σελήνην {[}trocar por maiúscula

Ἠῶ θ', ἣ πάντεσσιν ἐπιχθονίοισι φαείνει

ἀθανάτοις τε θεοῖσι τοὶ οὐρανὸν εὐρὺν ἔχουσι,

γείναθ' ὑποδμηθεῖσ' Ὑπερίονος ἐν φιλότητι. {[}mudar tipo de sigma

Κρείῳ δ' Εὐρυβίη τέκεν ἐν φιλότητι μιγεῖσα 375 {[}acrescentar epsilon

Ἀστραῖόν τε μέγαν Πάλλαντά τε δῖα θεάων

Πέρσην θ', ὃς καὶ πᾶσι μετέπρεπεν ἰδμοσύνῃσιν.

Ἀστραίῳ δ' Ἠὼς ἀνέμους τέκε καρτεροθύμους,

ἀργεστὴν Ζέφυρον Βορέην τ' αἰψηροκέλευθον {[}mudar acento de posição

καὶ Νότον, ἐν φιλότητι θεὰ θεῷ εὐνηθεῖσα. 380

τοὺς δὲ μέτ' ἀστέρα τίκτεν Ἑωσφόρον Ἠριγένεια

ἄστρά τε λαμπετόωντα, τά τ' οὐρανὸς ἐστεφάνωται. {[}acrescentar acento:
manter o outro

Στὺξ δ' ἔτεκ' Ὠκεανοῦ θυγάτηρ Πάλλαντι μιγεῖσα

Ζῆλον καὶ Νίκην καλλίσφυρον ἐν μεγάροισι {[}deletar :

καὶ Κράτος ἠδὲ Βίην ἀριδείκετα γείνατο τέκνα. 385 {[}trocar , por .

τῶν οὐκ ἔστ' ἀπάνευθε Διὸς δόμος, οὐδέ τις ἕδρη,

οὐδ' ὁδός, ὅππῃ μὴ κείνοις θεὸς ἡγεμονεύει,

{[}mudar η por ῃ / mudar ῃ por ει

ἀλλ' αἰεὶ πὰρ Ζηνὶ βαρυκτύπῳ ἑδριόωνται.

ὣς γὰρ ἐβούλευσε Στὺξ ἄφθιτος Ὠκεανίνη

ἤματι τῷ, ὅτε πάντας Ὀλύμπιος ἀστεροπητὴς 390

ἀθανάτους ἐκάλεσσε θεοὺς ἐς μακρὸν Ὄλυμπον,

εἶπε δ', ὃς ἂν μετὰ εἷο θεῶν Τιτῆσι μάχοιτο,

μή τιν' ἀπορραίσειν γεράων, τιμὴν δὲ ἕκαστον

ἑξέμεν, ἣν τὸ πάρος γε μετ' ἀθανάτοισι θεοῖσι.

{[}acrescentar vírgula / deletar nu final / acrescentar ponto final

τὸν δ' ἔφαθ', ὅστις ἄτιμος ὑπὸ Κρόνου ἠδ' ἀγέραστος, 395

τιμῆς καὶ γεράων ἐπιβησέμεν, ἣ θέμις ἐστίν.

{[}trocar ἧ por ἣ

ἦλθε δ' ἄρα πρώτη Στὺξ ἄφθιτος Οὔλυμπόνδε

σὺν σφοῖσιν παίδεσσι φίλου διὰ μήδεα πατρός· {[}trocar . por ·

τὴν δὲ Ζεὺς τίμησε, περισσὰ δὲ δῶρα ἔδωκεν.

αὐτὴν μὲν γὰρ ἔθηκε θεῶν μέγαν ἔμμεναι ὅρκον, 400

παῖδας δ' ἤματα πάντα ἑοῦ μεταναιέτας εἶναι.

ὣς δ' αὔτως πάντεσσι διαμπερές, ὥς περ ὑπέστη,

ἐξετέλεσσ'· αὐτὸς δὲ μέγα κρατεῖ ἠδὲ ἀνάσσει. {[}mudar tipo de sigma

Φοίβη δ' αὖ Κοίου πολυήρατον ἦλθεν ἐς εὐνήν· {[}maíusc. / trocar : por ·

κυσαμένη δἤπειτα θεὰ θεοῦ ἐν φιλότητι 405 {[}juntar as palavras: acento!

Λητὼ κυανόπεπλον ἐγείνατο, μείλιχον αἰεί,

ἤπιον ἀνθρώποισι καὶ ἀθανάτοισι θεοῖσι, {[}deletar nu

μείλιχον ἐξ ἀρχῆς, ἀγανώτατον ἐντὸς Ὀλύμπου.

γείνατο δ' Ἀστερίην εὐώνυμον, ἥν ποτε Πέρσης

ἠγάγετ' ἐς μέγα δῶμα φίλην κεκλῆσθαι ἄκοιτιν. 410

ἡ δ' ὑποκυσαμένη Ἑκάτην τέκε, τὴν περὶ πάντων {[}deletar acento

Ζεὺς Κρονίδης τίμησε· πόρεν δέ οἱ ἀγλαὰ δῶρα, {[}trocar : por ·

μοῖραν ἔχειν γαίης τε καὶ ἀτρυγέτοιο θαλάσσης.

ἡ δὲ καὶ ἀστερόεντος ἀπ' οὐρανοῦ ἔμμορε τιμῆς, {[}deletar acento / por
vírgula

ἀθανάτοις τε θεοῖσι τετιμένη ἐστὶ μάλιστα. 415

καὶ γὰρ νῦν, ὅτε πού τις ἐπιχθονίων ἀνθρώπων

ἔρδων ἱερὰ καλὰ κατὰ νόμον ἱλάσκηται,

κικλήσκει Ἑκάτην· πολλή τέ οἱ ἔσπετο τιμὴ {[}tirar iota subescrito

ῥεῖα μάλ', ᾧ πρόφρων γε θεὰ ὑποδέξεται εὐχάς,

καί τέ οἱ ὄλβον ὀπάζει, ἐπεὶ δύναμίς γε πάρεστιν. 420

ὅσσοι γὰρ Γαίης τε καὶ Οὐρανοῦ ἐξεγένοντο

καὶ τιμὴν ἔλαχον, τούτων ἔχει αἶσαν ἁπάντων· {[}trocar . por ·

οὐδέ τί μιν Κρονίδης ἐβιήσατο οὐδέ τ' ἀπηύρα,

ὅσσ' ἔλαχεν Τιτῆσι μέτα προτέροισι θεοῖσιν, {[}mudar tipo de sigma

{[}transpor o verso 425 da 1ª edição para o lugar do verso 434; quanto
ao resto, mesma ordem{]}

ἀλλ' ἔχει, ὡς τὸ πρῶτον ἀπ' ἀρχῆς ἔπλετο δασμός. 425 {[}trocar , por .

οὐδ', ὅτι μουνογενής, ἧσσον θεὰ ἔμμορε τιμῆς {[}deletar vírgula no fim

καὶ γεράων γαίῃ τε καὶ οὐρανῷ ἠδὲ θαλάσσῃ, {[}trocar ς por ων / trocar :
por ,

ἀλλ' ἔτι καὶ πολὺ μᾶλλον, ἐπεὶ Ζεὺς τίεται αὐτήν.

ᾧ δ' ἐθέλῃ, μεγάλως παραγίνεται ἠδ' ὀνίνησιν· {[}trocar vogais / trocar
: por ·

ἔν τ' ἀγορῇ λαοῖσι μεταπρέπει, ὅν κ' ἐθέλῃσιν· 430 {[}trocar : por ·

ἠδ' ὁπότ' ἐς πόλεμον φθισήνορα θωρήσσωνται

ἀνέρες, ἔνθα θεὰ παραγίνεται, οἷς κ' ἐθέλῃσι

νίκην προφρονέως ὀπάσαι καὶ κῦδος ὀρέξαι.

ἔν τε δίκῃ βασιλεῦσι παρ' αἰδοίοισι καθίζει,

ἐσθλὴ δ' αὖθ' ὁπότ' ἄνδρες ἀεθλεύωσ' ἐν ἀγῶνι· 435

{[}deletar \emph{in} e por apóstrofo / trocar , por ·

ἔνθα θεὰ καὶ τοῖς παραγίνεται ἠδ' ὀνίνησι· {[}trocar ; por ·

νικήσας δὲ βίῃ καὶ κάρτει, καλὸν ἄεθλον {[}deletar trema

ῥεῖα φέρει χαίρων τε, τοκεῦσι δὲ κῦδος ὀπάζει.

ἐσθλὴ δ' ἱππήεσσι παρεστάμεν, οἷς κ' ἐθέλῃσιν· {[}trocar . por ,

καὶ τοῖς, οἳ γλαυκὴν δυσπέμφελον ἐργάζονται, 440

εὔχονται δ' Ἑκάτῃ καὶ ἐρικτύπῳ Ἐννοσιγαίῳ,

ῥηιδίως ἄγρην κυδρὴ θεὸς ὤπασε πολλήν,

ῥεῖα δ' ἀφείλετο φαινομένην, ἐθέλουσά γε θυμῷ.

ἐσθλὴ δ' ἐν σταθμοῖσι σὺν Ἑρμῇ ληίδ' ἀέξειν· {[}trocar : por ·

βουκολίας δὲ βοῶν τε καὶ αἰπόλια πλατέ' αἰγῶν 445

{[}trocar \emph{de} por \emph{d'} e a palavra seguinte

ποίμνας τ' εἰροπόκων ὀίων, θυμῷ γ' ἐθέλουσα,

ἐξ ὀλίγων βριάει κἀκ πολλῶν μείονα θῆκεν.

οὕτω τοι καὶ μουνογενὴς ἐκ μητρὸς ἐοῦσα

πᾶσι μετ' ἀθανάτοισι τετίμηται γεράεσσι. {[}deletar nu final

θῆκε δέ μιν Κρονίδης κουροτρόφον, οἳ μετ' ἐκείνην450

ὀφθαλμοῖσιν ἴδοντο φάος πολυδερκέος Ἠοῦς.

οὕτως ἐξ ἀρχῆς κουροτρόφος, αἳ δέ τε τιμαί.

Ῥείη δὲ δμηθεῖσα Κρόνῳ τέκε φαίδιμα τέκνα, {[}deletar signo de parágrafo

Ἱστίην Δήμητρα καὶ Ἥρην χρυσοπέδιλον, {[}por vírgula no fim

ἴφθιμόν τ' Ἀίδην, ὃς ὑπὸ χθονὶ δώματα ναίει 455

νηλεὲς ἦτορ ἔχων, καὶ ἐρίκτυπον Ἐννοσίγαιον,

Ζῆνά τε μητιόεντα, θεῶν πατέρ' ἠδὲ καὶ ἀνδρῶν,

τοῦ καὶ ὑπὸ βροντῆς πελεμίζεται εὐρεῖα χθών.

καὶ τοὺς μὲν κατέπινε μέγας Κρόνος, ὥς τις ἕκαστος

νηδύος ἐξ ἱερῆς μητρὸς πρὸς γούναθ' ἵκοιτο, 460

τὰ φρονέων, ἵνα μή τις ἀγαυῶν Οὐρανιώνων

ἄλλος ἐν ἀθανάτοισιν ἔχοι βασιληίδα τιμήν.

πεύθετο γὰρ Γαίης τε καὶ Οὐρανοῦ ἀστερόεντος {[}tirar vírgula no fim

οὕνεκά οἱ πέπρωτο ἑῷ ὑπὸ παιδὶ δαμῆναι, {[}por vírgula no fim

καὶ κρατερῷ περ ἐόντι, Διὸς μεγάλου διὰ βουλάς. 465 {[}trocar : por .

τῷ ὅ γ' ἄρ' οὐκ ἀλαοσκοπιὴν ἔχεν, ἀλλὰ δοκεύων {[}alterar configuração
palavras

παῖδας ἑοὺς κατέπινε· Ῥέην δ' ἔχε πένθος ἄλαστον. {[}trocar : por ·

ἀλλ' ὅτε δὴ Δί' ἔμελλε θεῶν πατέρ' ἠδὲ καὶ ἀνδρῶν

τέξεσθαι, τότ' ἔπειτα φίλους λιτάνευε τοκῆας

τοὺς αὐτῆς, Γαῖάν τε καὶ Οὐρανὸν ἀστερόεντα, 470

μῆτιν συμφράσσασθαι, ὅπως λελάθοιτο τεκοῦσα

παῖδα φίλον, τείσαιτο δ' ἐρινῦς πατρὸς ἑοῖο {[}acrescentar epsilon

παίδων \textless{}θ'\textgreater{} οὓς κατέπινε μέγας Κρόνος
ἀγκυλομήτης. {[}por \textless{}\textgreater{} e tirar vírgula

οἱ δὲ θυγατρὶ φίλῃ μάλα μὲν κλύον ἠδ' ἐπίθοντο, {[}deletar acento

καί οἱ πεφραδέτην, ὅσα περ πέπρωτο γενέσθαι 475

ἀμφὶ Κρόνῳ βασιλῆι καὶ υἱέι καρτεροθύμῳ· {[}trocar . por ·

πέμψαν δ' ἐς Λύκτον, Κρήτης ἐς πίονα δῆμον,

ὁππότ' ἄρ' ὁπλότατον παίδων ἤμελλε τεκέσθαι, {[}trocar palavras de lugar
/ trocar ξ por κ

Ζῆνα μέγαν· τὸν μέν οἱ ἐδέξατο Γαῖα πελώρη {[}trocar : por ·

Κρήτῃ ἐν εὐρείῃ τρεφέμεν ἀτιταλλέμεναί τε. 480 {[}trocar alfa por
epsilon

ἔνθά μιν ἷκτο φέρουσα θοὴν διὰ νύκτα μέλαιναν, {[}vírgula no fim

πρώτην ἐς Λύκτον· κρύψεν δέ ἑ χερσὶ λαβοῦσα {[}trocar : por ·

ἄντρῳ ἐν ἠλιβάτῳ, ζαθέης ὑπὸ κεύθεσι γαίης,

Αἰγαίῳ ἐν ὄρει πεπυκασμένῳ ὑλήεντι.

τῷ δὲ σπαργανίσασα μέγαν λίθον ἐγγυάλιξεν 485

Οὐρανίδῃ μέγ' ἄνακτι, θεῶν προτέρων βασιλῆι.

τὸν τόθ' ἑλὼν χείρεσσιν ἑὴν ἐσκάτθετο νηδύν, {[}vírgula no fim

σχέτλιος, οὐδ' ἐνόησε μετὰ φρεσίν, ὥς οἱ ὀπίσσω {[}trocar : por ,

ἀντὶ λίθου ἑὸς υἱὸς ἀνίκητος καὶ ἀκηδὴς

λείπεθ', ὅ μιν τάχ' ἔμελλε βίῃ καὶ χερσὶ δαμάσσας 490

τιμῆς ἐξελάαν, ὁ δ' ἐν ἀθανάτοισιν ἀνάξειν. {[}trocar vogais / deletar
acento

καρπαλίμως δ' ἄρ' ἔπειτα μένος καὶ φαίδιμα γυῖα

ηὔξετο τοῖο ἄνακτος· ἐπιπλομένου δ' ἐνιαυτοῦ, {[}trocar : por · /
singular 2x

Γαίης ἐννεσίῃσι πολυφραδέεσσι δολωθείς, {[}vírgula no fim

ὃν γόνον ἂψ ἀνέηκε μέγας Κρόνος ἀγκυλομήτης, 495 {[}vírgula no fim

νικηθεὶς τέχνῃσι βίηφί τε παιδὸς ἑοῖο.

πρῶτον δ' ἐξήμησε λίθον, πύματον καταπίνων·

{[}trocar vogais / deletar pronome relativo / mudanças no verbo final /
trocar : por ·

τὸν μὲν Ζεὺς στήριξε κατὰ χθονὸς εὐρυοδείης

Πυθοῖ ἐν ἠγαθέῃ, γυάλοις ὕπο Παρνησσοῖο, {[}acrescentar um sigma

σῆμ' ἔμεν ἐξοπίσω, θαῦμα θνητοῖσι βροτοῖσι. 500 {[}deletar nu

λῦσε δὲ πατροκασιγνήτους ὀλοῶν ὑπὸ δεσμῶν, {[}, no fim

Οὐρανίδας, οὓς δῆσε πατὴρ ἀεσιφροσύνῃσιν· {[}trocar : por ·

οἵ οἱ ἀπεμνήσαντο χάριν εὐεργεσιάων,

δῶκαν δὲ βροντὴν ἠδ' αἰθαλόεντα κεραυνὸν

καὶ στεροπήν· τὸ πρὶν δὲ πελώρη Γαῖα κεκεύθει· 505 {[}trocar : por · /
trocar : por ·

τοῖς πίσυνος θνητοῖσι καὶ ἀθανάτοισιν ἀνάσσει.

κούρην δ' Ἰαπετὸς καλλίσφυρον Ὠκεανίνην

ἠγάγετο Κλυμένην καὶ ὁμὸν λέχος εἰσανέβαινεν.

ἡ δέ οἱ Ἄτλαντα κρατερόφρονα γείνατο παῖδα, {[}deletar acento / trocar :
por ,

τίκτε δ' ὑπερκύδαντα Μενοίτιον ἠδὲ Προμηθέα, 510 {[}vírgula no fim

ποικίλον αἰολόμητιν, ἁμαρτίνοόν τ' Ἐπιμηθέα· {[}trocar : por ·

ὃς κακὸν ἐξ ἀρχῆς γένετ' ἀνδράσιν ἀλφηστῇσι· {[}trocar : por ·

πρῶτος γάρ ῥα Διὸς πλαστὴν ὑπέδεκτο γυναῖκα

παρθένον. ὑβριστὴν δὲ Μενοίτιον εὐρύοπα Ζεὺς

εἰς ἔρεβος κατέπεμψε βαλὼν ψολόεντι κεραυνῷ 515 {[}minúscula:
acento/espírito!

εἵνεκ' ἀτασθαλίης τε καὶ ἠνορέης ὑπερόπλου.

Ἄτλας δ' οὐρανὸν εὐρὺν ἔχει κρατερῆς ὑπ' ἀνάγκης, {[}vírgula no fim

πείρασιν ἐν γαίης πρόπαρ' Ἑσπερίδων λιγυφώνων {[}deletar 2 vírgulas

ἑστηώς, κεφαλῇ τε καὶ ἀκαμάτῃσι χέρεσσι· {[}deletar , / deletar
\emph{nu} / trocar : por ·

ταύτην γάρ οἱ μοῖραν ἐδάσσατο μητίετα Ζεύς. 520

δῆσε δ' ἀλυκτοπέδῃσι Προμηθέα ποικιλόβουλον, {[}, no fim

δεσμοῖς ἀργαλέοισι, μέσον διὰ κίον' ἐλάσσας· {[}, / trocar : por ·

καί οἱ ἐπ' αἰετὸν ὦρσε τανύπτερον· αὐτὰρ ὅ γ' ἧπαρ {[}trocar : por ·

ἤσθιεν ἀθάνατον, τὸ δ' ἀέξετο ἶσον ἁπάντῃ {[}colocar iota subscrito no
eta

νυκτός, ὅσον πρόπαν ἦμαρ ἔδοι τανυσίπτερος ὄρνις. 525 {[},

τὸν μὲν ἄρ' Ἀλκμήνης καλλισφύρου ἄλκιμος υἱὸς

Ἡρακλέης ἔκτεινε, κακὴν δ' ἀπὸ νοῦσον ἄλαλκεν

Ἰαπετιονίδῃ καὶ ἐλύσατο δυσφροσυνάων, {[} ,

οὐκ ἀέκητι Ζηνὸς Ὀλυμπίου ὕψι μέδοντος, {[}separar as palavras / acento

ὄφρ' Ἡρακλῆος Θηβαγενέος κλέος εἴη 530

πλεῖον ἔτ' ἢ τὸ πάροιθεν ἐπὶ χθόνα πουλυβότειραν.

ταῦτ' ἄρα ἁζόμενος τίμα ἀριδείκετον υἱόν· {[}trocar : por ·

καί περ χωόμενος παύθη χόλου, ὃν πρὶν ἔχεσκεν,

οὕνεκ' ἐρίζετο βουλὰς ὑπερμενέι Κρονίωνι.

καὶ γὰρ ὅτ' ἐκρίνοντο θεοὶ θνητοί τ' ἄνθρωποι 535

Μηκώνῃ, τότ' ἔπειτα μέγαν βοῦν πρόφρονι θυμῷ

δασσάμενος προύθηκε, Διὸς νόον ἐξαπαφίσκων. {[}mudar vogal

τῷ μὲν γὰρ σάρκάς τε καὶ ἔγκατα πίονα δημῷ {[}singular

ἐν ῥινῷ κατέθηκε, καλύψας γαστρὶ βοείῃ, {[} ,

τοῖς δ' αὖτ' ὀστέα λευκὰ βοὸς δολίῃ ἐπὶ τέχνῃ 540 {[}plural

εὐθετίσας κατέθηκε, καλύψας ἀργέτι δημῷ. {[} ,

δὴ τότε μιν προσέειπε πατὴρ ἀνδρῶν τε θεῶν τε· {[}trocar : por ·

``Ἰαπετιονίδη, πάντων ἀριδείκετ' ἀνάκτων, {[}aspas

ὦ πέπον, ὡς ἑτεροζήλως διεδάσσαο μοίρας.'' {[}aspas

ὣς φάτο κερτομέων Ζεὺς ἄφθιτα μήδεα εἰδώς· 545 {[}trocar . por ·

τὸν δ' αὖτε προσέειπε Προμηθεὺς ἀγκυλομήτης,

ἦκ' ἐπιμειδήσας, δολίης δ' οὐ λήθετο τέχνης· {[}trocar : por ·

``Ζεῦ κύδιστε μέγιστε θεῶν αἰειγενετάων, {[}aspas

τῶν δ' ἕλευ ὁπποτέρην σε ἐνὶ φρεσὶ θυμὸς ἀνώγει.'' {[}aspas

φῆ ῥα δολοφρονέων· Ζεὺς δ' ἄφθιτα μήδεα εἰδὼς 550 {[}trocar : por ·

γνῶ ῥ' οὐδ' ἠγνοίησε δόλον· κακὰ δ' ὄσσετο θυμῷ {[}trocar : por ·

θνητοῖς ἀνθρώποισι, τὰ καὶ τελέεσθαι ἔμελλε.

χερσὶ δ' ὅ γ' ἀμφοτέρῃσιν ἀνείλετο λευκὸν ἄλειφαρ, {[}trocar . por ,

χώσατο δὲ φρένας ἀμφί, χόλος δέ μιν ἵκετο θυμόν,

ὡς ἴδεν ὀστέα λευκὰ βοὸς δολίῃ ἐπὶ τέχῃ. 555

ἐκ τοῦ δ' ἀθανάτοισιν ἐπὶ χθονὶ φῦλ' ἀνθρώπων

καίουσ' ὀστέα λευκὰ θυηέντων ἐπὶ βωμῶν. {[}mudar tipo de sigma

τὸν δὲ μέγ' ὀχθήσας προσέφη νεφεληγερέτα Ζεύς· {[}trocar : por ·

``Ἰαπετιονίδη, πάντων πέρι μήδεα εἰδώς, {[}aspas

ὦ πέπον, οὐκ ἄρα πω δολίης ἐπελήθεο τέχνης.'' 560 {[}aspas / trocar
\emph{iota} por \emph{epsilon}

ὣς φάτο χωόμενος Ζεὺς ἄφθιτα μήδεα εἰδώς. {[}deletar : e colocar . no
lugar

ἐκ τούτου δἤπειτα χόλου μεμνημένος αἰεὶ {[}deletar o δ e acrescentar um
χ

οὐκ ἐδίδου μελίῃσι πυρὸς μένος ἀκαμάτοιο

θνητοῖς ἀνθρώποις οἳ ἐπὶ χθονὶ ναιετάουσιν· {[}deletar . e colocar · no
lugar

ἀλλά μιν ἐξαπάτησεν ἐὺς πάις Ἰαπετοῖο 565

κλέψας ἀκαμάτοιο πυρὸς τηλέσκοπον αὐγὴν

ἐν κοίλῳ νάρθηκι· δάκεν δ' ἄρα νειόθι θυμὸν

\begin{quote}
{[}deletar : e colocar · no lugar / substituir δέ ἑ por δ' ἄρα / deletar
a vírgula no fim do verso
\end{quote}

Ζῆν' ὑψιβρεμέτην, ἐχόλωσε δέ μιν φίλον ἦτορ,

ὡς ἴδ' ἐν ἀνθρώποισι πυρὸς τηλέσκοπον αὐγήν.

αὐτίκα δ' ἀντὶ πυρὸς τεῦξεν κακὸν ἀνθρώποισι· 570 {[}deletar : e colocar
· no lugar

γαίης γὰρ σύμπλασσε περικλυτὸς Ἀμφιγυήεις

παρθένῳ αἰδοίῃ ἴκελον Κρονίδεω διὰ βουλάς· {[}deletar . e colocar · no
lugar

ζῶσε δὲ καὶ κόσμησε θεὰ γλαυκῶπις Ἀθήνη

ἀργυφέῃ ἐσθῆτι· κατὰ κρῆθεν δὲ καλύπτρην {[}deletar : e colocar · no
lugar

δαιδαλέην χείρεσσι κατέσχεθε, θαῦμα ἰδέσθαι· 575 {[}deletar : e colocar
· no lugar

ἀμφὶ δέ οἱ στεφάνους νεοθηλέας, ἄνθεα ποίης,

\begin{quote}
{[}deletar a 1ª vírgula / substituir νεοθηλέος por νεοθηλέας / vírgula
depois de νεοθηλέας
\end{quote}

ἱμερτοὺς περίθηκε καρήατι Παλλὰς Ἀθήνη· {[}no fim, deletar . e colocar ·

ἀμφὶ δέ οἱ στεφάνην χρυσέην κεφαλῆφιν ἔθηκε,

τὴν αὐτὸς ποίησε περικλυτὸς Ἀμφιγυήεις

ἀσκήσας παλάμῃσι, χαριζόμενος Διὶ πατρί. 580

τῇ δ' ἔνι δαίδαλα πολλὰ τετεύχατο, θαῦμα ἰδέσθαι,

κνώδαλ' ὅσ' ἤπειρος δεινὰ τρέφει ἠδὲ θάλασσα·

\begin{quote}
{[}deletar vírgula após κνώδαλ' / mudar tipo de sigma / trocar πολλὰ por
δεινὰ / no fim do verso, deletar , e colocar ·
\end{quote}

τῶν ὅ γε πόλλ' ἐνέθηκε, χάρις δ' ἐπὶ πᾶσιν ἄητο,

\begin{quote}
{[}deletar os dois travessões (manter as vírgulas) / deletar ἀπελάμπετο
πολλή e substituir por ἐπὶ πᾶσιν ἄητο, {[}
\end{quote}

θαυμάσια, ζωοῖσιν ἐοικότα φωνήεσσιν.

αὐτὰρ ἐπεὶ δὴ τεῦξε καλὸν κακὸν ἀντ' ἀγαθοῖο, 585 {[}deletar . e colocar
, no lugar

ἐξάγαγ' ἔνθά περ ἄλλοι ἔσαν θεοὶ ἠδ' ἄνθρωποι,

\begin{quote}
{[}deletar vírgula após ἐξάγαγ' e acrescentar acento na última sílaba de
ἔνθά {[}a palavra deve ter dois acentos
\end{quote}

κόσμῳ ἀγαλλομένην γλαυκώπιδος Ὀβριμοπάτρης· {[}deletar . e colocar · no
lugar

θαῦμα δ' ἔχ' ἀθανάτους τε θεοὺς θνητούς τ' ἀνθρώπους,

ὡς εἶδον δόλον αἰπύν, ἀμήχανον ἀνθρώποισιν.

ἐκ τῆς γὰρ γένος ἐστὶ γυναικῶν θηλυτεράων, 590

τῆς γὰρ ὀλοίιόν ἐστι γένος καὶ φῦλα γυναικῶν,

\begin{quote}
{[}em ὀλώιόν, deletar ώ e substituir por οί (dois iotas seguidos, o 1º é
acentuado, o segundo não
\end{quote}

πῆμα μέγα θνητοῖσι, σὺν ἀνδράσι ναιετάουσαι,

\begin{quote}
{[}trocar μέγ' αἵ por μέγα / acrescentar uma vírgula após θνητοῖσι /

trocar μετ' por σὺν / trocar as duas últimas letras da última palavra:
ιν por αι

acrescentar uma vírgula no fim
\end{quote}

οὐλομένης Πενίης οὐ σύμφοροι, ἀλλὰ Κόροιο.

{[}trocar um π minúsculo por um Π maiúsculo, e um κ minúsculo por um Κ
maiúsculo

ὡς δ' ὁπότ' ἐν σμήνεσσι κατηρεφέεσσι μέλισσαι

κηφῆνας βόσκωσι, κακῶν ξυνήονας ἔργων· 595 {[}deletar -- e colocar · no
lugar

αἱ μέν τε πρόπαν ἦμαρ ἐς ἠέλιον καταδύντα

{[}tirar acento agudo e manter espírito áspero no iota da primeira
palavra

ἠμάτιαι σπεύδουσι τιθεῖσί τε κηρία λευκά,

οἱ δ' ἔντοσθε μένοντες ἐπηρεφέας κατὰ σίμβλους {[}deletar acento

{[}tirar acento agudo e manter espírito áspero no iota da primeira
palavra

ἀλλότριον κάματον σφετέρην ἐς γαστέρ' ἀμῶνται· {[}deletar -- e colocar ·
no lugar

ὣς δ' αὔτως ἄνδρεσσι κακὸν θνητοῖσι γυναῖκας 600

Ζεὺς ὑψιβρεμέτης θῆκε, ξυνήονας ἔργων

ἀργαλέων. ἕτερον δὲ πόρεν κακὸν ἀντ' ἀγαθοῖο,

{[} deletar o 1º ; e substituir por . / o 2º, por ,

ὅς κε γάμον φεύγων καὶ μέρμερα ἔργα γυναικῶν

μὴ γῆμαι ἐθέλῃ, ὀλοὸν δ' ἐπὶ γῆρας ἵκηται

χήτει γηροκόμοιο· ὁ δ' οὐ βιότου γ' ἐπιδευὴς 605 {[}deletar : e colocar
· no lugar

ζώει, ἀποφθιμένου δὲ διὰ ζωὴν δατέονται

χηρωσταί. ᾧ δ' αὖτε γάμου μετὰ μοῖρα γένηται, {[}deletar : e colocar .
no lugar

κεδνὴν δ' ἔσχεν ἄκοιτιν, ἀρηρυῖαν πραπίδεσσι, {[}vírgula após ἄκοιτιν

τῷ δέ τ' ἀπ' αἰῶνος κακὸν ἐσθλῷ ἀντιφερίζει

ἐμμενές· ὃς δέ κε τέτμῃ ἀταρτηροῖο γενέθλης, 610 {[}deletar : e colocar
· no lugar

ζώει ἐνὶ στήθεσσιν ἔχων ἀλίαστον ἀνίην

θυμῷ καὶ κραδίῃ, καὶ ἀνήκεστον κακόν ἐστιν.

ὣς οὐκ ἔστι Διὸς κλέψαι νόον οὐδὲ παρελθεῖν.

οὐδὲ γὰρ Ἰαπετιονίδης ἀκάκητα Προμηθεὺς

τοῖό γ' ὑπεξήλυξε βαρὺν χόλον, ἀλλ' ὑπ' ἀνάγκης 615

καὶ πολύιδριν ἐόντα μέγας κατὰ δεσμὸς ἐρύκει.

Ὀβριάρεῳ δ' ὡς πρῶτα πατὴρ ὠδύσσατο θυμῷ {[}tirar sinal de parágrafo

Κόττῳ τ' ἠδὲ Γύγῃ, δῆσε κρατερῷ ἐνὶ δεσμῷ, {[}acrescentar gama / deletar
\emph{nu}

ἠνορέην ὑπέροπλον ἀγώμενος ἠδὲ καὶ εἶδος

καὶ μέγεθος· κατένασσε δ' ὑπὸ χθονὸς εὐρυοδείης. 620 {[}trocar : por ·

ἔνθ' οἵ γ' ἄλγε' ἔχοντες ὑπὸ χθονὶ ναιετάοντες

εἵατ' ἐπ' ἐσχατιῇ μεγάλης ἐν πείρασι γαίης {[}deletar vírgula no fim

δηθὰ μάλ' ἀχνύμενοι, κραδίῃ μέγα πένθος ἔχοντες.

ἀλλά σφεας Κρονίδης τε καὶ ἀθάνατοι θεοὶ ἄλλοι {[}deletar vírgula no fim

οὓς τέκεν ἠύκομος Ῥείη Κρόνου ἐν φιλότητι 625 {[}deletar vírgula no fim

Γαίης φραδμοσύνῃσιν ἀνήγαγον ἐς φάος αὖτις· {[}trocar : por ·

αὐτὴ γάρ σφιν ἅπαντα διηνεκέως κατέλεξε, {[}, no fim

σὺν κείνοις νίκην τε καὶ ἀγλαὸν εὖχος ἀρέσθαι.

δηρὸν γὰρ μάρναντο πόνον θυμαλγέ' ἔχοντες

ἀντίον ἀλλήλοισι διὰ κρατερὰς ὑσμίνας 631

{[}acrescentar o número dos versos aqui e abaixo / deletar , no fim

Τιτῆνές τε θεοὶ καὶ ὅσοι Κρόνου ἐξεγένοντο, 630

οἱ μὲν ἀφ' ὑψηλῆς Ὄθρυος Τιτῆνες ἀγαυοί, 632 {[}deletar acento

οἱ δ' ἄρ' ἀπ' Οὐλύμποιο θεοὶ δωτῆρες ἐάων {[}deletar acento {[} deletar
2 vírgulas

οὓς τέκεν ἠύκομος Ῥείη Κρόνῳ εὐνηθεῖσα.

οἵ ῥα τότ' ἀλλήλοισι πόνον θυμαλγέ' ἔχοντες 635 {[}trocar uma palavra

συνεχέως ἐμάχοντο δέκα πλείους ἐνιαυτούς· {[}trocar : por ·

οὐδέ τις ἦν ἔριδος χαλεπῆς λύσις οὐδὲ τελευτὴ

οὐδετέροις, ἶσον δὲ τέλος τέτατο πτολέμοιο.

ἀλλ' ὅτε δὴ κείνοισι παρέσχεθεν ἄρμενα πάντα,

νέκταρ τ' ἀμβροσίην τε, τά περ θεοὶ αὐτοὶ ἔδουσι, 640

πάντων \textless{}τ'\textgreater{} ἐν στήθεσσιν ἀέξετο θυμὸς ἀγήνωρ,
{[}acrescentar \textless{}τ'\textgreater{} / trocar . por ,

ὡς νέκταρ τ' ἐπάσαντο καὶ ἀμβροσίην ἐρατεινήν,

δὴ τότε τοῖς μετέειπε πατὴρ ἀνδρῶν τε θεῶν τε· {[}trocar : por ·

``κέκλυτέ μευ Γαίης τε καὶ Οὐρανοῦ ἀγλαὰ τέκνα, {[}aspas

ὄφρ' εἴπω τά με θυμὸς ἐνὶ στήθεσσι κελεύει. 645 {[}deletar ,

ἤδη γὰρ μάλα δηρὸν ἐναντίοι ἀλλήλοισι

νίκης καὶ κάρτευς πέρι μαρνάμεθ' ἤματα πάντα, {[}trocar algumas letras /
, no fim

Τιτῆνές τε θεοὶ καὶ ὅσοι Κρόνου ἐκγενόμεσθα.

ὑμεῖς δὲ μεγάλην τε βίην καὶ χεῖρας ἀάπτους

φαίνετε Τιτήνεσσιν ἐναντίον ἐν δαῒ λυγρῇ, 650

μνησάμενοι φιλότητος ἐνηέος, ὅσσα παθόντες

ἐς φάος ἂψ ἀφίκεσθε δυσηλεγέος ὑπὸ δεσμοῦ

ἡμετέρας διὰ βουλὰς ὑπὸ ζόφου ἠερόεντος.'' {[}aspas

ὣς φάτο· τὸν δ' αἶψ' αὖτις ἀμείβετο Κόττος ἀμύμων· {[}trocar : por · /
trocar : por ·

``δαιμόνι', οὐκ ἀδάητα πιφαύσκεαι, ἀλλὰ καὶ αὐτοὶ 655 {[}aspas / trocar
: por ,

ἴδμεν ὅ τοι περὶ μὲν πραπίδες, περὶ δ' ἐστὶ νόημα, {[}deletar 1ª ,

ἀλκτὴρ δ' ἀθανάτοισιν ἀρῆς γένεο κρυεροῖο, {[}trocar . por ,

σῇσι δ' ἐπιφροσύνῃσιν ὑπὸ ζόφου ἠερόεντος

ἄψορρον ἐξαῦτις ἀμειλίκτων ὑπὸ δεσμῶν

ἠλύθομεν, Κρόνου υἱὲ ἄναξ, ἀνάελπτα παθόντες. 660

τῷ καὶ νῦν ἀτενεῖ τε νόῳ καὶ πρόφρονι θυμῷ {[}2 palavras alteradas

ῥυσόμεθα κράτος ὑμὸν ἐν αἰνῇ δηιοτῆτι, {[}vírgula no fim

μαρνάμενοι Τιτῆσιν ἀνὰ κρατερὰς ὑσμίνας.'' {[}aspas

ὣς φάτ'· ἐπῄνησαν δὲ θεοὶ δωτῆρες ἐάων {[}trocar : por ·

μῦθον ἀκούσαντες· πολέμου δ' ἐλιλαίετο θυμὸς 665 {[}trocar : por ·

μᾶλλον ἔτ' ἢ τὸ πάροιθε· μάχην δ' ἀμέγαρτον ἔγειραν {[}trocar : por ·

πάντες, θήλειαί τε καὶ ἄρσενες, ἤματι κείνῳ,

Τιτῆνές τε θεοὶ καὶ ὅσοι Κρόνου ἐξεγένοντο,

οὕς τε Ζεὺς ἐρέβεσφιν ὑπὸ χθονὸς ἧκε φόωσδε,

{[}minúscula: espírito! / deletar \emph{u} / vírgula no fim do verso

δεινοί τε κρατεροί τε, βίην ὑπέροπλον ἔχοντες. 670

τῶν ἑκατὸν μὲν χεῖρες ἀπ' ὤμων ἀίσσοντο

πᾶσιν ὁμῶς, κεφαλαὶ δὲ ἑκάστῳ πεντήκοντα

ἐξ ὤμων ἐπέφυκον ἐπὶ στιβαροῖσι μέλεσσιν.

οἳ τότε Τιτήνεσσι κατέσταθεν ἐν δαῒ λυγρῇ

πέτρας ἠλιβάτους στιβαρῇς ἐν χερσὶν ἔχοντες· 675 {[}trocar . por ·

Τιτῆνες δ' ἑτέρωθεν ἐκαρτύναντο φάλαγγας

προφρονέως· χειρῶν τε βίης θ' ἅμα ἔργον ἔφαινον {[}trocar , por ·

ἀμφότεροι, δεινὸν δὲ περίαχε πόντος ἀπείρων, {[}trocar : por ,

γῆ δὲ μέγ' ἐσμαράγησεν, ἐπέστενε δ' οὐρανὸς εὐρὺς

σειόμενος, πεδόθεν δὲ τινάσσετο μακρὸς Ὄλυμπος 680

ῥιπῇ ὕπ' ἀθανάτων, ἔνοσις δ' ἵκανε βαρεῖα

τάρταρον ἠερόεντα ποδῶν αἰπεῖά τ' ἰωὴ

{[}tau minúsculo / colocar acento (dois na mesma palavra)

ἀσπέτου ἰωχμοῖο βολάων τε κρατεράων. {[}trocar : por .

ὣς ἄρ' ἐπ' ἀλλήλοις ἵεσαν βέλεα στονόεντα· {[}trocar . por ·

φωνὴ δ' ἀμφοτέρων ἵκετ' οὐρανὸν ἀστερόεντα 685

κεκλομένων· οἱ δὲ ξύνισαν μεγάλῳ ἀλαλητῷ. {[}deletar acento

οὐδ' ἄρ' ἔτι Ζεὺς ἴσχεν ἑὸν μένος, ἀλλά νυ τοῦ γε

εἶθαρ μὲν μένεος πλῆντο φρένες, ἐκ δέ τε πᾶσαν

φαῖνε βίην· ἄμυδις δ' ἄρ' ἀπ' οὐρανοῦ ἠδ' ἀπ' Ὀλύμπου {[}trocar : por ·

ἀστράπτων ἔστειχε συνωχαδόν, οἱ δὲ κεραυνοὶ 690 {[}trocar : por ,

ἴκταρ ἅμα βροντῇ τε καὶ ἀστεροπῇ ποτέοντο

χειρὸς ἄπο στιβαρῆς, ἱερὴν φλόγα εἰλυφόωντες, {[}vírgula no fim

ταρφέες· ἀμφὶ δὲ γαῖα φερέσβιος ἐσμαράγιζε {[}trocar : por ·

καιομένη, λάκε δ' ἀμφὶ περὶ μεγάλ' ἄσπετος ὕλη· {[}trocar \emph{u} por
\emph{e} / trocar . por ·

ἔζεε δὲ χθὼν πᾶσα καὶ Ὠκεανοῖο ῥέεθρα 695

πόντός τ' ἀτρύγετος· τοὺς δ' ἄμφεπε θερμὸς ἀυτμὴ {[}trocar : por ·

Τιτῆνας χθονίους, φλὸξ δ' αἰθέρα δῖαν ἵκανεν

ἄσπετος, ὄσσε δ' ἄμερδε καὶ ἰφθίμων περ ἐόντων

αὐγὴ μαρμαίρουσα κεραυνοῦ τε στεροπῆς τε.

καῦμα δὲ θεσπέσιον κάτεχεν χάος· εἴσατο δ' ἄντα 700 {[}\emph{kh}
minúsculo / trocar : por ·

ὀφθαλμοῖσιν ἰδεῖν ἠδ' οὔασιν ὄσσαν ἀκοῦσαι

αὔτως, ὡς ὅτε γαῖα καὶ οὐρανὸς εὐρὺς ὕπερθε {[}trocar \emph{ei} por
\emph{hote}

πίλνατο· τοῖος γάρ κε μέγας ὑπὸ δοῦπος ὀρώρει, {[}trocar : por ·

τῆς μὲν ἐρειπομένης, τοῦ δ' ὑψόθεν ἐξεριπόντος· {[}trocar : por ·

τόσσος δοῦπος ἔγεντο θεῶν ἔριδι ξυνιόντων. 705

σὺν δ' ἄνεμοι ἔνοσίν τε κονίην τ' ἐσφαράγιζον

βροντήν τε στεροπήν τε καὶ αἰθαλόεντα κεραυνόν,

κῆλα Διὸς μεγάλοιο, φέρον δ' ἰαχήν τ' ἐνοπήν τε

ἐς μέσον ἀμφοτέρων· ὄτοβος δ' ἄπλητος ὀρώρει {[}trocar : por ·

σμερδαλέης ἔριδος, κάρτευς δ' ἀνεφαίνετο ἔργον. 710

ἐκλίνθη δὲ μάχη· πρὶν δ' ἀλλήλοις ἐπέχοντες {[}trocar : por ·

ἐμμενέως ἐμάχοντο διὰ κρατερὰς ὑσμίνας.

οἱ δ' ἄρ' ἐνὶ πρώτοισι μάχην δριμεῖαν ἔγειραν, {[}deletar acento /
vírgula no fim

Κόττος τε Βριάρεώς τε Γύγης τ' ἄατος πολέμοιο· {[}trocar , por · /
acrescentar gama

οἵ ῥα τριηκοσίας πέτρας στιβαρέων ἀπὸ χειρῶν 715 {[}acrescentar
\emph{epsilon} e mudar acento

πέμπον ἐπασσυτέρας, κατὰ δ' ἐσκίασαν βελέεσσι

Τιτῆνας· καὶ τοὺς μὲν ὑπὸ χθονὸς εὐρυοδείης {[}trocar , por ·

πέμψαν καὶ δεσμοῖσιν ἐν ἀργαλέοισιν ἔδησαν,

νικήσαντες χερσὶν ὑπερθύμους περ ἐόντας, {[}inverter a ordem dos termos

τόσσον ἔνερθ' ὑπὸ γῆς ὅσον οὐρανός ἐστ' ἀπὸ γαίης· 720 {[}deletar , /
trocar : por ·

τόσσον γάρ τ' ἀπὸ γῆς ἐς τάρταρον ἠερόεντα.

{[}deletar sinal de parágrafo / \emph{tau} minúsculo

ἐννέα γὰρ νύκτας τε καὶ ἤματα χάλκεος ἄκμων

οὐρανόθεν κατιών, δεκάτῃ κ' ἐς γαῖαν ἵκοιτο· {[}trocar : por ·

{[}ἶσον δ' αὖτ' ἀπὸ γῆς ἐς τάρταρον ἠερόεντα·{]}

ἐννέα δ' αὖ νύκτας τε καὶ ἤματα χάλκεος ἄκμων

ἐκ γαίης κατιών, δεκάτῃ κ' ἐς τάρταρον ἵκοι. 725 {[}acrescentar vírgula
/ \emph{tau} minúsculo

τὸν πέρι χάλκεον ἕρκος ἐλήλαται· ἀμφὶ δέ μιν νὺξ {[}trocar : por ·

τριστοιχὶ κέχυται περὶ δειρήν· αὐτὰρ ὕπερθε

{[}deletar epsilon / trocar : por · / deletar \emph{nu} no fim do verso

γῆς ῥίζαι πεφύασι καὶ ἀτρυγέτοιο θαλάσσης.

ἔνθα θεοὶ Τιτῆνες ὑπὸ ζόφῳ ἠερόεντι

κεκρύφαται βουλῇσι Διὸς νεφεληγερέταο, 730 {[}, no fim

χώρῳ ἐν εὐρώεντι, πελώρης ἔσχατα γαίης.

τοῖς οὐκ ἐξιτόν ἐστι, θύρας δ' ἐπέθηκε Ποσειδέων {[}trocar . por ,

χαλκείας, τεῖχος δ' ἐπελήλαται ἀμφοτέρωθεν.

{[}deletar o epsilon do \emph{de} e por ' / trocar a palavra seguinte

ἔνθα Γύγης Κόττος τε καὶ Ὀβριάρεως μεγάθυμος {[}acrescentar gama

ναίουσιν, φύλακες πιστοὶ Διὸς αἰγιόχοιο. 735

ἔνθα δὲ γῆς δνοφερῆς καὶ ταρτάρου ἠερόεντος {[}\emph{tau} minúsculo

πόντου τ' ἀτρυγέτοιο καὶ οὐρανοῦ ἀστερόεντος

ἑξείης πάντων πηγαὶ καὶ πείρατ' ἔασιν,

ἀργαλέ' εὐρώεντα, τά τε στυγέουσι θεοί περ· {[}trocar , por ·

χάσμα μέγ', οὐδέ κε πάντα τελεσφόρον εἰς ἐνιαυτὸν 740

οὖδας ἵκοιτ', εἰ πρῶτα πυλέων ἔντοσθε γένοιτο,

ἀλλά κεν ἔνθα καὶ ἔνθα φέροι πρὸ θύελλα θυέλλης {[}mudar de dativo pra
genitivo

ἀργαλέη· δεινὸν δὲ καὶ ἀθανάτοισι θεοῖσι {[}trocar : por ·

τοῦτο τέρας· καὶ Νυκτὸς ἐρεμνῆς οἰκία δεινὰ

{[}trocar : por · / deletar letras e acrescentar um \emph{mu}

ἕστηκεν νεφέλῃς κεκαλυμμένα κυανέῃσι. 745 {[}deletar um \emph{nu} no fim

τῶν πρόσθ' Ἰαπετοῖο πάις ἔχει οὐρανὸν εὐρὺν

ἑστηὼς κεφαλῇ τε καὶ ἀκαμάτῃσι χέρεσσιν

ἀστεμφέως, ὅθι Νύξ τε καὶ Ἡμέρη ἆσσον ἰοῦσαι

ἀλλήλας προσέειπον ἀμειβόμεναι μέγαν οὐδὸν {[}deletar vírgula

χάλκεον· ἡ μὲν ἔσω καταβήσεται, ἡ δὲ θύραζε 750 {[}trocar : por · /
deletar acento 2x

ἔρχεται, οὐδέ ποτ' ἀμφοτέρας δόμος ἐντὸς ἐέργει,

ἀλλ' αἰεὶ ἑτέρη γε δόμων ἔκτοσθεν ἐοῦσα

γαῖαν ἐπιστρέφεται, ἡ δ' αὖ δόμου ἐντὸς ἐοῦσα {[}deletar acento

μίμνει τὴν αὐτῆς ὥρην ὁδοῦ, ἔστ' ἂν ἵκηται· {[}trocar , por ·

ἡ μὲν ἐπιχθονίοισι φάος πολυδερκὲς ἔχουσα, 755 {[}deletar acento

ἡ δ' Ὕπνον μετὰ χερσί, κασίγνητον Θανάτοιο, {[}deletar acento / trocar .
por ,

Νὺξ ὀλοή, νεφέλῃ κεκαλυμμένη ἠεροειδεῖ.

ἔνθα δὲ Νυκτὸς παῖδες ἐρεμνῆς οἰκί' ἔχουσιν,

Ὕπνος καὶ Θάνατος, δεινοὶ θεοί· οὐδέ ποτ' αὐτοὺς {[}trocar : por ·

Ἠέλιος φαέθων ἐπιδέρκεται ἀκτίνεσσιν 760

οὐρανὸν εἰσανιὼν οὐδ' οὐρανόθεν καταβαίνων. {[}juntar e alterar
acento/espírito

τῶν ἕτερος μὲν γῆν τε καὶ εὐρέα νῶτα θαλάσσης {[}por partícula e alterar
palavra

ἥσυχος ἀνστρέφεται καὶ μείλιχος ἀνθρώποισι,

τοῦ δὲ σιδηρέη μὲν κραδίη, χάλκεον δέ οἱ ἦτορ

νηλεὲς ἐν στήθεσσιν· ἔχει δ' ὃν πρῶτα λάβῃσιν 765 {[}trocar : por ·

ἀνθρώπων· ἐχθρὸς δὲ καὶ ἀθανάτοισι θεοῖσιν. {[}trocar : por ·

ἔνθα θεοῦ χθονίου πρόσθεν δόμοι ἠχήεντες

ἰφθίμου τ' Ἀίδεω καὶ ἐπαινῆς Περσεφονείης

ἑστᾶσιν, δεινὸς δὲ κύων προπάροιθε φυλάσσει, {[}vírgula no fim

νηλειής, τέχνην δὲ κακὴν ἔχει· ἐς μὲν ἰόντας 770 {[}trocar : por ·

σαίνει ὁμῶς οὐρῇ τε καὶ οὔασιν ἀμφοτέροισιν,

ἐξελθεῖν δ' οὐκ αὖτις ἐᾷ πάλιν, ἀλλὰ δοκεύων

ἐσθίει, ὅν κε λάβῃσι πυλέων ἔκτοσθεν ἰόντα.

ἰφθίμου τ' Ἀίδεω καὶ ἐπαινῆς Περσεφονείης.

ἔνθα δὲ ναιετάει στυγερὴ θεὸς ἀθανάτοισι, 775

δεινὴ Στύξ, θυγάτηρ ἀψορρόου Ὠκεανοῖο

πρεσβυτάτη· νόσφιν δὲ θεῶν κλυτὰ δώματα ναίει {[}trocar : por ·

μακρῇσιν πέτρῃσι κατηρεφέ'· ἀμφὶ δὲ πάντῃ {[}trocar : por · / iota
subscrito no eta

κίοσιν ἀργυρέοισι πρὸς οὐρανὸν ἐστήρικται.

παῦρα δὲ Θαύμαντος θυγάτηρ πόδας ὠκέα Ἶρις 780

ἀγγελίῃ πωλεῖται ἐπ' εὐρέα νῶτα θαλάσσης. {[}mudar de acusativo pra
dativo

ὁππότ' ἔρις καὶ νεῖκος ἐν ἀθανάτοισιν ὄρηται, {[}vírgula no fim

καί ῥ' ὅστις ψεύδηται Ὀλύμπια δώματ' ἐχόντων,

Ζεὺς δέ τε Ἶριν ἔπεμψε θεῶν μέγαν ὅρκον ἐνεῖκαι

τηλόθεν ἐν χρυσέῃ προχόῳ πολυώνυμον ὕδωρ, 785 {[}vírgula no fim

ψυχρόν, ὅ τ' ἐκ πέτρης καταλείβεται ἠλιβάτοιο {[}separar as duas letras

ὑψηλῆς· πολλὸν δὲ ὑπὸ χθονὸς εὐρυοδείης {[}trocar : por ·

ἐξ ἱεροῦ ποταμοῖο ῥέει διὰ νύκτα μέλαιναν· {[}colocar · no fim

Ὠκεανοῖο κέρας, δεκάτη δ' ἐπὶ μοῖρα δέδασται· {[}trocar : por , / trocar
: por ·

ἐννέα μὲν περὶ γῆν τε καὶ εὐρέα νῶτα θαλάσσης 790

δίνῃς ἀργυρέῃς εἱλιγμένος εἰς ἅλα πίπτει,

ἡ δὲ μί' ἐκ πέτρης προρέει, μέγα πῆμα θεοῖσιν. {[}deletar acento / por
vírgula

ὅς κεν τὴν ἐπίορκον ἀπολλείψας ἐπομόσσῃ

ἀθανάτων οἳ ἔχουσι κάρη νιφόεντος Ὀλύμπου,

κεῖται νήυτμος τετελεσμένον εἰς ἐνιαυτόν· 795 {[}trocar : por ·

οὐδέ ποτ' ἀμβροσίης καὶ νέκταρος ἔρχεται ἆσσον

βρώσιος, ἀλλά τε κεῖται ἀνάπνευστος καὶ ἄναυδος

στρωτοῖς ἐν λεχέεσσι, κακὸν δ' ἐπὶ κῶμα καλύπτει.

αὐτὰρ ἐπὴν νοῦσον τελέσει μέγαν εἰς ἐνιαυτόν,

ἄλλος δ' ἐξ ἄλλου δέχεται χαλεπώτερος ἆθλος· 800 {[}trocar . por ·

εἰνάετες δὲ θεῶν ἀπαμείρεται αἰὲν ἐόντων,

οὐδέ ποτ' ἐς βουλὴν ἐπιμίσγεται οὐδ' ἐπὶ δαῖτας

ἐννέα πάντ' ἔτεα· δεκάτῳ δ' ἐπιμίσγεται αὖτις {[}trocar : por ·

εἴρας ἐς ἀθανάτων οἳ Ὀλύμπια δώματ' ἔχουσι. {[}deletar ,

τοῖον ἄρ' ὅρκον ἔθεντο θεοὶ Στυγὸς ἄφθιτον ὕδωρ, 805 {[} , no fim

ὠγύγιον· τὸ δ' ἵησι καταστυφέλου διὰ χώρου. {[}trocar , por ·

ἔνθα δὲ γῆς δνοφερῆς καὶ ταρτάρου ἠερόεντος {[}\emph{tau} minúsculo

πόντου τ' ἀτρυγέτοιο καὶ οὐρανοῦ ἀστερόεντος

ἑξείης πάντων πηγαὶ καὶ πείρατ' ἔασιν, {[}vírgula no fim

ἀργαλέ' εὐρώεντα, τά τε στυγέουσι θεοί περ. 810

ἔνθα δὲ μαρμάρεαί τε πύλαι καὶ χάλκεος οὐδός, {[}, no fim

ἀστεμφὲς ῥίζῃσι διηνεκέεσσιν ἀρηρώς, {[}deletar a ,

αὐτοφυής· πρόσθεν δὲ θεῶν ἔκτοσθεν ἁπάντων {[}trocar : por ·

Τιτῆνες ναίουσι, πέρην χάεος ζοφεροῖο. {[}\emph{khi} minúsculo

αὐτὰρ ἐρισμαράγοιο Διὸς κλειτοὶ ἐπίκουροι 815

δώματα ναιετάουσιν ἐπ' Ὠκεανοῖο θεμέθλοις,

Κόττος τ' ἠδὲ Γύγης· Βριάρεών γε μὲν ἠὺν ἐόντα {[}trocar : por · /
acrescentar gama

γαμβρὸν ἑὸν ποίησε βαρύκτυπος Ἐννοσίγαιος,

δῶκε δὲ Κυμοπόλειαν ὀπυίειν, θυγατέρα ἥν.

αὐτὰρ ἐπεὶ Τιτῆνας ἀπ' οὐρανοῦ ἐξέλασε Ζεύς, 820 {[}tirar marca de
parágrafo

ὁπλότατον τέκε παῖδα Τυφωέα Γαῖα πελώρη

Ταρτάρου ἐν φιλότητι διὰ χρυσῆν Ἀφροδίτην· {[}trocar : por ·

οὗ χεῖρες †μὲν ἔασιν ἐπ' ἰσχύι ἔργματ' ἔχουσαι,†

καὶ πόδες ἀκάματοι κρατεροῦ θεοῦ· ἐκ δέ οἱ ὤμων {[}trocar : por ·

ἦν ἑκατὸν κεφαλαὶ ὄφιος δεινοῖο δράκοντος, 825 {[}deletar ,

γλώσσῃσι δνοφερῇσι λελιχμότες· ἐκ δέ οἱ ὄσσων {[}trocar , por ·

θεσπεσίῃς κεφαλῇσιν ὑπ' ὀφρύσι πῦρ ἀμάρυσσεν· {[}trocar : por ·

πασέων δ' ἐκ κεφαλέων πῦρ καίετο δερκομένοιο· {[}trocar : por ·

φωναὶ δ' ἐν πάσῃσιν ἔσαν δεινῇς κεφαλῇσι,

παντοίην ὄπ' ἰεῖσαι ἀθέσφατον· ἄλλοτε μὲν γὰρ 830 {[}trocar : por ·

φθέγγονθ' ὥς τε θεοῖσι συνιέμεν, ἄλλοτε δ' αὖτε

ταύρου ἐριβρύχεω μένος ἀσχέτου ὄσσαν ἀγαύρου, {[}deletar 1ª e 2ª ,

ἄλλοτε δ' αὖτε λέοντος ἀναιδέα θυμὸν ἔχοντος,

ἄλλοτε δ' αὖ σκυλάκεσσιν ἐοικότα, θαύματ' ἀκοῦσαι,

ἄλλοτε δ' αὖ ῥοίζεσχ', ὑπὸ δ' ἤχεεν οὔρεα μακρά. 835

καί νύ κεν ἔπλετο ἔργον ἀμήχανον ἤματι κείνῳ,

καί κεν ὅ γε θνητοῖσι καὶ ἀθανάτοισιν ἄναξεν,

εἰ μὴ ἄρ' ὀξὺ νόησε πατὴρ ἀνδρῶν τε θεῶν τε· {[}trocar . por ·

σκληρὸν δ' ἐβρόντησε καὶ ὄβριμον, ἀμφὶ δὲ γαῖα

σμερδαλέον κονάβησε καὶ οὐρανὸς εὐρὺς ὕπερθε 840

πόντός τ' Ὠκεανοῦ τε ῥοαὶ καὶ Τάρταρα γαίης. {[}por um 2º acento

ποσσὶ δ' ὕπ' ἀθανάτοισι μέγας πελεμίζετ' Ὄλυμπος

ὀρνυμένοιο ἄνακτος· ἐπεστονάχιζε δὲ γαῖα. {[}trocar : por · / trocar
\emph{e} por \emph{o}

καῦμα δ' ὑπ' ἀμφοτέρων κάτεχεν ἰοειδέα πόντον

βροντῆς τε στεροπῆς τε πυρός τ' ἀπὸ τοῖο πελώρου 845 {[}deletar , no fim

πρηστήρων ἀνέμων τε κεραυνοῦ τε φλεγέθοντος· {[}trocar . por ·

ἔζεε δὲ χθὼν πᾶσα καὶ οὐρανὸς ἠδὲ θάλασσα· {[}trocar : por ·

θυῖε δ' ἄρ' ἀμφ' ἀκτὰς περί τ' ἀμφί τε κύματα μακρὰ

ῥιπῇ ὕπ' ἀθανάτων, ἔνοσις δ' ἄσβεστος ὀρώρει· {[}trocar : por ·

τρέε δ' Ἀίδης ἐνέροισι καταφθιμένοισιν ἀνάσσων 850

Τιτῆνές θ' ὑποταρτάριοι Κρόνον ἀμφὶς ἐόντες {[}deletar 2 ,

ἀσβέστου κελάδοιο καὶ αἰνῆς δηιοτῆτος. {[}. no fim

Ζεὺς δ' ἐπεὶ οὖν κόρθυνεν ἑὸν μένος, εἵλετο δ' ὅπλα,

βροντήν τε στεροπήν τε καὶ αἰθαλόεντα κεραυνόν,

πλῆξεν ἀπ' Οὐλύμποιο ἐπάλμενος· ἀμφὶ δὲ πάσας 855 {[}trocar : por ·

ἔπρεσε θεσπεσίας κεφαλὰς δεινοῖο πελώρου.

αὐτὰρ ἐπεὶ δή μιν δάμασε πληγῇσιν ἱμάσσας,

ἤριπε γυιωθείς, στονάχιζε δὲ γαῖα πελώρη· {[}trocar . por · / trocar
\emph{e} por \emph{o}

φλὸξ δὲ κεραυνωθέντος ἀπέσσυτο τοῖο ἄνακτος

οὔρεος ἐν βήσσῃσιν ἀιδνῆς παιπαλοέσσης 860

{[}sem iota sobescrito, 2x / deletar , no fim

πληγέντος, πολλὴ δὲ πελώρη καίετο γαῖα {[}trocar . por ,

αὐτμῇ θεσπεσίῃ, καὶ ἐτήκετο κασσίτερος ὣς {[}por ,

τέχνῃ ὑπ' αἰζηῶν ἐν ἐυτρήτοις χοάνοισι

θαλφθείς, ἠὲ σίδηρος, ὅ περ κρατερώτατός ἐστιν, {[}trocar . por ,

οὔρεος ἐν βήσσῃσι δαμαζόμενος πυρὶ κηλέῳ 865

τήκεται ἐν χθονὶ δίῃ ὑφ' Ἡφαίστου παλάμῃσιν· {[}trocar . por ·

ὣς ἄρα τήκετο γαῖα σέλαι πυρὸς αἰθομένοιο.

ῥῖψε δέ μιν θυμῷ ἀκαχὼν ἐς τάρταρον εὐρύν. {[}\emph{tau} minúsculo

ἐκ δὲ Τυφωέος ἔστ' ἀνέμων μένος ὑγρὸν ἀέντων,

νόσφι Νότου Βορέω τε καὶ ἀργεστέω Ζεφύροιο· 870 {[}trocar : por ·

οἵ γε μὲν ἐκ θεόφιν γενεήν, θνητοῖς μέγ' ὄνειαρ. {[}adicionar \emph{nu}
final / trocar : por .

αἱ δ' ἄλλαι μὰψ αὖραι ἐπιπνείουσι θάλασσαν·

{[}trocar \emph{o} por \emph{a}, 2x / juntar duas palavras, com
alteração de espirito/acento / trocar : por ·

αἳ δή τοι πίπτουσαι ἐς ἠεροειδέα πόντον,

πῆμα μέγα θνητοῖσι, κακῇ θυίουσιν ἀέλλῃ· {[}trocar : por ·

ἄλλοτε δ' ἄλλαι ἄεισι διασκιδνᾶσί τε νῆας 875

ναύτας τε φθείρουσι· κακοῦ δ' οὐ γίνεται ἀλκὴ {[}trocar : por · /
deletar \emph{gama}

ἀνδράσιν, οἳ κείνῃσι συνάντωνται κατὰ πόντον. {[}trocar : por .

αἱ δ' αὖ καὶ κατὰ γαῖαν ἀπείριτον ἀνθεμόεσσαν

ἔργ' ἐρατὰ φθείρουσι χαμαιγενέων ἀνθρώπων, {[}deletar ,

πιμπλεῖσαι κόνιός τε καὶ ἀργαλέου κολοσυρτοῦ. 880

αὐτὰρ ἐπεί ῥα πόνον μάκαρες θεοὶ ἐξετέλεσσαν, {[}deletar sinal de
parágrafo

Τιτήνεσσι δὲ τιμάων κρίναντο βίηφι,

δή ῥα τότ' ὤτρυνον βασιλευέμεν ἠδὲ ἀνάσσειν

Γαίης φραδμοσύνῃσιν Ὀλύμπιον εὐρύοπα Ζῆν

ἀθανάτων· ὁ δὲ τοῖσιν ἐὺ διεδάσσατο τιμάς. 885

{[}trocar : por · / deletar acento / alterar fim da palavra / . no fim
do verso

Ζεὺς δὲ θεῶν βασιλεὺς πρώτην ἄλοχον θέτο Μῆτιν, {[}vírgula no fim do
verso

πλεῖστα θεῶν εἰδυῖαν ἰδὲ θνητῶν ἀνθρώπων. {[} \emph{te} \textgreater{}
\emph{theôn} / deletar \emph{epsilon}

ἀλλ' ὅτε δὴ ἄρ' ἔμελλε θεὰν γλαυκῶπιν Ἀθήνην

τέξεσθαι, τότ' ἔπειτα δόλῳ φρένας ἐξαπατήσας

αἱμυλίοισι λόγοισιν ἑὴν ἐσκάτθετο νηδύν, 890 {[}, no fim do verso

Γαίης φραδμοσύνῃσι καὶ Οὐρανοῦ ἀστερόεντος· {[}trocar . por ·

τὼς γάρ οἱ φρασάτην, ἵνα μὴ βασιληίδα τιμὴν

ἄλλος ἔχοι Διὸς ἀντὶ θεῶν αἰειγενετάων.

ἐκ γὰρ τῆς εἵμαρτο περίφρονα τέκνα γενέσθαι· {[}trocar : por ·

πρώτην μὲν κούρην γλαυκώπιδα Τριτογένειαν, 895 {[}, no fim do verso

ἶσον ἔχουσαν πατρὶ μένος καὶ ἐπίφρονα βουλήν, {[}trocar . por ,

αὐτὰρ ἔπειτ' ἄρα παῖδα θεῶν βασιλῆα καὶ ἀνδρῶν

ἤμελλεν τέξεσθαι, ὑπέρβιον ἦτορ ἔχοντα· {[}trocar : por ·

ἀλλ' ἄρα μιν Ζεὺς πρόσθεν ἑὴν ἐσκάτθετο νηδύν,

ὥς οἱ συμφράσσαιτο θεὰ ἀγαθόν τε κακόν τε. 900

{[}deletar \emph{dê} / acrescentar \emph{sum} / . no fim

δεύτερον ἠγάγετο λιπαρὴν Θέμιν, ἣ τέκεν Ὥρας,

Εὐνομίην τε Δίκην τε καὶ Εἰρήνην τεθαλυῖαν,

αἵ τ' ἔργ' ὠρεύουσι καταθνητοῖσι βροτοῖσι,

Μοίρας θ', ᾗς πλείστην τιμὴν πόρε μητίετα Ζεύς, {[}adicionar
\emph{sigma}

Κλωθώ τε Λάχεσίν τε καὶ Ἄτροπον, αἵ τε διδοῦσι 905 {[}separar em dois

θνητοῖς ἀνθρώποισιν ἔχειν ἀγαθόν τε κακόν τε.

τρεῖς δέ οἱ Εὐρυνόμη Χάριτας τέκε καλλιπαρήους, {[}adicionar acento

Ὠκεανοῦ κούρη πολυήρατον εἶδος ἔχουσα,

Ἀγλαΐην τε καὶ Εὐφροσύνην Θαλίην τ' ἐρατεινήν· {[}trocar : por ·

τῶν καὶ ἀπὸ βλεφάρων ἔρος εἴβετο δερκομενάων 910

λυσιμελής· καλὸν δέ θ' ὑπ' ὀφρύσι δερκιόωνται. {[}trocar : por ·

αὐτὰρ ὁ Δήμητρος πολυφόρβης ἐς λέχος ἦλθεν· {[}trocar , por ·

ἣ τέκε Περσεφόνην λευκώλενον, ἣν Ἀιδωνεὺς

ἥρπασεν ἧς παρὰ μητρός, ἔδωκε δὲ μητίετα Ζεύς. {[}trocar : por ,

Μνημοσύνης δ' ἐξαῦτις ἐράσσατο καλλικόμοιο, 915 {[}\emph{mu} maiúsculo

ἐξ ἧς οἱ Μοῦσαι χρυσάμπυκες ἐξεγένοντο

ἐννέα, τῇσιν ἅδον θαλίαι καὶ τέρψις ἀοιδῆς.

Λητὼ δ' Ἀπόλλωνα καὶ Ἄρτεμιν ἰοχέαιραν {[}deletar , no fim

ἱμερόεντα γόνον περὶ πάντων Οὐρανιώνων {[}deletar , no fim

γείνατ' ἄρ' αἰγιόχοιο Διὸς φιλότητι μιγεῖσα. 920

λοισθοτάτην δ' Ἥρην θαλερὴν ποιήσατ' ἄκοιτιν· {[}trocar : por ·

ἡ δ' Ἥβην καὶ Ἄρηα καὶ Εἰλείθυιαν ἔτικτε {[}deletar acento

μιχθεῖσ' ἐν φιλότητι θεῶν βασιλῆι καὶ ἀνδρῶν. {[}mudar tipo de sigma

αὐτὸς δ' ἐκ κεφαλῆς γλαυκώπιδα γείνατ' Ἀθήνην, {[}2 palavras distintas
no fim

δεινὴν ἐγρεκύδοιμον ἀγέστρατον ἀτρυτώνην, 925 {[}\emph{alfa} minúsculo /
, no fim

πότνιαν, ᾗ κέλαδοί τε ἅδον πόλεμοί τε μάχαι τε· {[}trocar , por ·

Ἥρη δ' Ἥφαιστον κλυτὸν οὐ φιλότητι μιγεῖσα

γείνατο, καὶ ζαμένησε καὶ ἤρισεν ᾧ παρακοίτῃ,

ἐκ πάντων τέχνῃσι κεκασμένον Οὐρανιώνων.

ἐκ δ' Ἀμφιτρίτης καὶ ἐρικτύπου Ἐννοσιγαίου 930

Τρίτων εὐρυβίης γένετο μέγας, ὅς τε θαλάσσης {[}separar; atente ao sigma

πυθμέν' ἔχων παρὰ μητρὶ φίλῃ καὶ πατρὶ ἄνακτι

ναίει χρύσεα δῶ, δεινὸς θεός. αὐτὰρ Ἄρηι

ῥινοτόρῳ Κυθέρεια Φόβον καὶ Δεῖμον ἔτικτε, {[}adicionar ,

δεινούς, οἵ τ' ἀνδρῶν πυκινὰς κλονέουσι φάλαγγας 935

ἐν πολέμῳ κρυόεντι σὺν Ἄρηι πτολιπόρθῳ,

Ἁρμονίην θ', ἣν Κάδμος ὑπέρθυμος θέτ' ἄκοιτιν. {[}adicionar .

Ζηνὶ δ' ἄρ' Ἀτλαντὶς Μαίη τέκε κύδιμον Ἑρμῆν,

κήρυκ' ἀθανάτων, ἱερὸν λέχος εἰσαναβᾶσα.

Καδμηὶς δ' ἄρα οἱ Σεμέλη τέκε φαίδιμον υἱὸν 940 {[}alterar letras finais

μιχθεῖσ' ἐν φιλότητι, Διώνυσον πολυγηθέα, {[}mudar tipo de sigma

ἀθάνατον θνητή· νῦν δ' ἀμφότεροι θεοί εἰσιν. {[}trocar : por ·

Ἀλκμήνη δ' ἄρ' ἔτικτε βίην Ἡρακληείην

μιχθεῖσ' ἐν φιλότητι Διὸς νεφεληγερέταο. {[}mudar tipo de sigma

Ἀγλαΐην δ' Ἥφαιστος ἀγακλυτὸς ἀμφιγυήεις 945 {[}\emph{alfa} maiúsculo /
deletar , no fim

ὁπλοτάτην Χαρίτων θαλερὴν ποιήσατ' ἄκοιτιν.

χρυσοκόμης δὲ Διώνυσος ξανθὴν Ἀριάδνην,

κούρην Μίνωος, θαλερὴν ποιήσατ' ἄκοιτιν· {[}trocar . por ·

τὴν δέ οἱ ἀθάνατον καὶ ἀγήρων θῆκε Κρονίων.

Ἥβην δ' Ἀλκμήνης καλλισφύρου ἄλκιμος υἱός, 950 {[}inicial maiúscula

ἲς Ἡρακλῆος, τελέσας στονόεντας ἀέθλους,

παῖδα Διὸς μεγάλοιο καὶ Ἥρης χρυσοπεδίλου,

αἰδοίην θέτ' ἄκοιτιν ἐν Οὐλύμπῳ νιφόεντι· {[}trocar , por ·

ὄλβιος, ὃς μέγα ἔργον ἐν ἀθανάτοισιν ἀνύσσας

ναίει ἀπήμαντος καὶ ἀγήραος ἤματα πάντα. 955

Ἠελίῳ δ' ἀκάμαντι τέκε κλυτὸς Ὠκεανίνη {[}inicial maiúscula

Περσηὶς Κίρκην τε καὶ Αἰήτην βασιλῆα.

Αἰήτης δ' υἱὸς φαεσιμβρότου Ἠελίοιο

κούρην Ὠκεανοῖο τελήεντος ποταμοῖο

γῆμε θεῶν βουλῇσιν, Ἰδυῖαν καλλιπάρηον· 960

{[}adicionar vírgula / eta sem iotasubscrito / trocar . por ·

ἣ δή οἱ Μήδειαν ἐύσφυρον ἐν φιλότητι

γείναθ' ὑποδμηθεῖσα διὰ χρυσῆν Ἀφροδίτην. {[}adicionar ponto final

ὑμεῖς μὲν νῦν χαίρετ', Ὀλύμπια δώματ' ἔχοντες, {[}tirar sinal de
parágrafo

νῆσοί τ' ἤπειροί τε καὶ ἁλμυρὸς ἔνδοθι πόντος· {[}trocar . por ·

νῦν δὲ θεάων φῦλον ἀείσατε, ἡδυέπειαι 965

Μοῦσαι Ὀλυμπιάδες, κοῦραι Διὸς αἰγιόχοιο,

ὅσσαι δὴ θνητοῖσι παρ' ἀνδράσιν εὐνηθεῖσαι

ἀθάναται γείναντο θεοῖς ἐπιείκελα τέκνα.

Δημήτηρ μὲν Πλοῦτον ἐγείνατο δῖα θεάων,

Ἰασίῳ ἥρωι μιγεῖσ' ἐρατῇ φιλότητι 970 {[}mudar tipo de sigma

νειῷ ἔνι τριπόλῳ, Κρήτης ἐν πίονι δήμῳ,

ἐσθλόν, ὃς εἶσ' ἐπὶ γῆν τε καὶ εὐρέα νῶτα θαλάσσης {[}mudar tipo de
sigma

πᾶσαν· τῷ δὲ τυχόντι καὶ οὗ κ' ἐς χεῖρας ἵκηται, {[}trocar : por ·

τὸν δὴ ἀφνειὸν ἔθηκε, πολὺν δέ οἱ ὤπασεν ὄλβον.

Κάδμῳ δ' Ἁρμονίη, θυγάτηρ χρυσῆς Ἀφροδίτης, 975

Ἰνὼ καὶ Σεμέλην καὶ Ἀγαυὴν καλλιπάρηον {[}trocar por eta sem
iotasubscrito

Αὐτονόην θ', ἣν γῆμεν Ἀρισταῖος βαθυχαίτης,

γείνατο καὶ Πολύδωρον ἐυστεφάνῳ ἐνὶ Θήβῃ.

κούρη δ' Ὠκεανοῦ Χρυσάορι καρτεροθύμῳ

μιχθεῖσ' ἐν φιλότητι πολυχρύσου Ἀφροδίτης 980

{[}mudar tipo de sigma / deletar , no fim

Καλλιρόη τέκε παῖδα βροτῶν κάρτιστον ἁπάντων,

Γηρυονέα, τὸν κτεῖνε βίη Ἡρακληείη

βοῶν ἕνεκ' εἰλιπόδων ἀμφιρρύτῳ εἰν Ἐρυθείῃ.

Τιθωνῷ δ' Ἠὼς τέκε Μέμνονα χαλκοκορυστήν,

Αἰθιόπων βασιλῆα, καὶ Ἠμαθίωνα ἄνακτα. 985

αὐτάρ τοι Κεφάλῳ φιτύσατο φαίδιμον υἱόν, {[}mudar a palavra

ἴφθιμον Φαέθοντα, θεοῖς ἐπιείκελον ἄνδρα· {[}trocar . por ·

τόν ῥα νέον τέρεν ἄνθος ἔχοντ' ἐρικυδέος ἥβης

παῖδ' ἀταλὰ φρονέοντα φιλομμειδὴς Ἀφροδίτη

ὦρτ' ἀνερειψαμένη, καί μιν ζαθέοις ἐνὶ νηοῖς 990 {[}mudar \emph{a}
\textgreater{} \emph{e} / adicionar \emph{i}

νηοπόλον μύχιον ποιήσατο, δαίμονα δῖον. {[}\emph{nu} \textgreater{}
\emph{mu}

κούρην δ' Αἰήταο διοτρεφέος βασιλῆος

Αἰσονίδης βουλῇσι θεῶν αἰειγενετάων

ἦγε παρ' Αἰήτεω, τελέσας στονόεντας ἀέθλους,

τοὺς πολλοὺς ἐπέτελλε μέγας βασιλεὺς ὑπερήνωρ, 995

ὑβριστὴς Πελίης καὶ ἀτάσθαλος ὀβριμοεργός· {[}trocar . por ·

τοὺς τελέσας ἐς Ἰωλκὸν ἀφίκετο πολλὰ μογήσας

ὠκείης ἐπὶ νηὸς ἄγων ἑλικώπιδα κούρην

Αἰσονίδης, καί μιν θαλερὴν ποιήσατ' ἄκοιτιν.

καί ῥ' ἥ γε δμηθεῖσ' ὑπ' Ἰήσονι ποιμένι λαῶν 1000 {[}mudar tipo de sigma

Μήδειον τέκε παῖδα, τὸν οὔρεσιν ἔτρεφε Χείρων

Φιλλυρίδης· μεγάλου δὲ Διὸς νόος ἐξετελεῖτο. {[}trocar : por ·

αὐτὰρ Νηρῆος κοῦραι ἁλίοιο γέροντος,

ἤτοι μὲν Φῶκον Ψαμάθη τέκε δῖα θεάων

Αἰακοῦ ἐν φιλότητι διὰ χρυσῆν Ἀφροδίτην· 1005 {[}trocar , por ·

Πηλεῖ δὲ δμηθεῖσα θεὰ Θέτις ἀργυρόπεζα

γείνατ' Ἀχιλλῆα ῥηξήνορα θυμολέοντα.

Αἰνείαν δ' ἄρ' ἔτικτεν ἐυστέφανος Κυθέρεια,

Ἀγχίσῃ ἥρωι μιγεῖσ' ἐρατῇ φιλότητι {[}mudar tipo de sigma

Ἴδης ἐν κορυφῇσι πολυπτύχου ἠνεμοέσσης. 1010

Κίρκη δ' Ἠελίου θυγάτηρ Ὑπεριονίδαο

γείνατ' Ὀδυσσῆος ταλασίφρονος ἐν φιλότητι

Ἄγριον ἠδὲ Λατῖνον ἀμύμονά τε κρατερόν τε· {[}trocar : por ·

{[}Τηλέγονον δὲ ἔτικτε διὰ χρυσῆν Ἀφροδίτην·{]}

οἳ δή τοι μάλα τῆλε μυχῷ νήσων ἱεράων 1015

πᾶσιν Τυρσηνοῖσιν ἀγακλειτοῖσιν ἄνασσον.

Ναυσίθοον δ' Ὀδυσῆι Καλυψὼ δῖα θεάων

γείνατο Ναυσίνοόν τε μιγεῖσ' ἐρατῇ φιλότητι. {[}mudar tipo de sigma

αὗται μὲν θνητοῖσι παρ' ἀνδράσιν εὐνηθεῖσαι

ἀθάναται γείναντο θεοῖς ἐπιείκελα τέκνα. 1020

{[}νῦν δὲ γυναικῶν φῦλον ἀείσατε, ἡδυέπειαι

Μοῦσαι Ὀλυμπιάδες, κοῦραι Διὸς αἰγιόχοιο.{]}

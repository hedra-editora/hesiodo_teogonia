\begingroup\widowpenalties = 3 10000 9000 6000
\chapter*{Introdução\smallskip\subtitulo{A linguagem e a narrativa\break desvelam o cosmo}}
\addcontentsline{toc}{chapter}{Introdução, \textit{por Christian Werner}}

\begin{flushright}
\textsc{christian werner}
\end{flushright}

\setlength{\epigraphwidth}{.65\textwidth}
\begin{epigraphs} 
\qitem{
Trepava ser o mais honesto de todos, ou o mais
danado, no tremeluz, conforme as quantas. Soava
no que falava, artes que falava, diferente
na autoridade, mas com uma autoridade muito veloz.}
{\textsc{joão guimarães rosa},\\
\textit{Grande sertão: Veredas}}
\end{epigraphs}

\noindent{}\textit{O mais honesto ou o mais danado} é como Riobaldo descreve Zé
Bebelo na parte inicial do romance. Trata-se de uma figura que ele
admira, pelas formas de sua astúcia e autoridade moderna, \textit{rápida}, em
contraste com aquela \textit{lenta} e arcaica de Joca Ramiro, o grande chefe
dos jagunços. Tal autoridade, porém, paulatinamente se revela fazer jus
ao mal que ecoa no nome Zé Bebelo, \textit{bellum}, ``guerra'', e belzebu,
cuja negatividade é contrária à justiça moderna desdobrada no discurso
da personagem. Dito de outra forma, Zé Bebelo move-se entre o arcaico e
o moderno, o mítico e o racional.\footnote{Minha interpretação de Zé
  Bebelo se apoia em Rosenfield (2006).}

\textit{Mutatis mutandis} pode-se dizer o mesmo do poema de Hesíodo e de
sua personagem central, Zeus. Também esse poema explora os meandros da
justiça e da soberania como idealizações dependentes da astúcia, \textit{mētis}, essa qualidade ou habilidade essencialmente múltipla e
imanente, focada no aqui e agora da experiência sempre cambiante.\footnote{``No
tremeluz\ldots{} muito veloz''.} E assim como em Rosa, mito e razão não
se revelam formas de pensamento opostas ou incompatíveis, em particular,
pela modo como, em Hesíodo, a linguagem e a narrativa desvelam o cosmo.

\section{Hesíodo: o poeta e sua época\protect\footnote{\MakeUppercase{A}baixo, procurei manter a
  indicação bibliográfica reduzida ao mínimo, sobretudo quando me apoio
  pontualmente no argumento de determinado autor. \MakeUppercase{P}ara uma bibliografia
  mais ampla, cf. a mencionada no final.}}

Diferentemente dos poemas de Homero, os de Hesíodo se associam, eles
próprios, a um poeta e a um lugar como espaço de sua gestação: o poeta
da \textit{Teogonia} se nomeia e se vincula ao entorno do monte Hélicon na
Beócia (22--23). Trata-se de uma região no centro da Grécia, cuja cidade
principal, no passado e hoje, é Tebas. Suas montanhas principais são o
Parnasso, junto a Delfos, o Citéron --- onde Édipo foi exposto --- e o
Hélicon, com sua fonte Hipocrene, ``Fonte do Cavalo'', estes dois
mencionados no início do poema em associação às Musas (1--8).

Indicações temporais, porém, estão virtualmente ausentes do poema, o que
permite reconstituições diversas, todas elas imprecisas e sujeitas a
críticas. Uma delas, feita pelos antigos, é associar Hesíodo a outros
poetas da tradição hexamétrica grega arcaica --- Museu, Orfeu e,
sobretudo, Homero --- e estabelecer uma cronologia relativa, para o que
um critério poderia ser a autoridade: a maior seria a do poeta mais
velho (Koning 2010). Modernamente, a cronologia relativa reaparece
fundamentada no exame linguístico-estatístico do \textit{corpus}
hexamétrico restante (Andersen \& Haug 2012). Assim, Janko (1982), um
trabalho seminal, definiu como sequência cronológica de composição
\textit{Ilíada}, \textit{Odisseia}, \textit{Teogonia} e \textit{Trabalhos e
dias}.

\textls[-10]{Outra forma de contextualizar os poemas no tempo está ligado a
tentativas de reconstituir os séculos \textsc{viii} e \textsc{vii} a.C. 
como a época na qual se sedimentaram uma série de fenômenos culturais e 
políticos que acabaram por definir as sociedades gregas, em especial o 
surgimento da \textit{polis} como principal organização política e social, 
o templo de Apolo e seu oráculo em Delfos como um santuário de todos os gregos,
festivais de cunho religioso, como os Jogos Olímpicos, que passaram a
atrair participantes de uma ampla gama de territórios grego, a
reintrodução da escrita, o culto aos heróis etc. Trata-se de fenômenos
que definem o que Gregory Nagy (1999), na esteira de Snodgrass (1971),
chama de pan-helenismo,\footnote{Moraes (2019, p.\,12) entende ``o
  pan-helenismo como um discurso político capaz de prover uma sensação
  de pertencimento às comunidades de língua grega, baseado em critérios
  simultaneamente culturais e políticos de caráter aglutinador e que
  atuou na produção e reprodução da identidade helênica''.} e do qual
faria parte a produção e recepção da \textit{Teogonia}.}\looseness=-1

A introdução paulatina, com adaptações, nos territórios gregos, nos
quais se falavam dialetos diversos, de um alfabeto de origem fenícia em
torno do século \textsc{viii} a.C. foi um dos responsáveis pela modificação
gradual de diversas práticas sociais, entre elas, a produção e recepção
de poesia. Os poemas podiam ser cantados ou recitados, e, quando
cantados por um coro (o que não é o caso da poesia épica como a
\textit{Teogonia}), esse produzia figuras de dança. É exatamente assim que
as Musas são representadas no início do poema (1--11), em contraste com o
cantor individual Hesíodo. Composições corais eram apresentadas em
ocasiões específicas, muitas vinculadas ao calendário religioso de
determinadas localidades. Quanto à poesia hesiódica, o contexto de
performance é desconhecido por nós. De fato, como se verá mais abaixo
por meio do nome de Hesíodo, é necessário tratar com cuidado os
elementos que parecem atar o poema à realidade.\looseness=-1

\section{Estrutura do poema}

Há diferentes maneiras de conceber a estrutura da \textit{Teogonia}. A de
Thalmann (1984, p.\,38--39), traduzida abaixo, tem a vantagem de
identificar em sua sequência de partes singulares e mais ou menos
independentes, uma moldura em anel (ainda que incompleta), marcada pela
repetição das letras em ordem inversa.

\begin{itemize}
\item \textsc{a.} 1--115\quad \textit{Proêmio}
\item \textsc{b.} 116--210\quad \textit{Os primeiros deuses e os Titãs; primeiro estágio do mito de sucessão}
\item \textsc{c.} 211--32\quad \textit{Prole de Noite}
\item \textsc{d-1.} 233--336\quad \textit{Prole de Mar, incluindo as Nereidas}
\item \textsc{d-2.} 337--70\quad \textit{Prole de Oceano, incluindo as Oceânides}
\item \textsc{e-1.} 371--403\quad \textit{Uniões de outros Titãs e o episódio de Estige}
\item \textsc{e-2.} 404--52\quad \textit{Uniões de outros Titãs e o episódio de Hécate}
\item \textsc{f-1.} 453--506\quad \textit{União de Reia e Crono; segundo estágio do mito de sucessão}
\item \textsc{g.} 507--616\quad \textit{Prole de Jápeto e o episódio de Prometeu}
\item \textsc{f-2.} 617--720\quad \textit{Batalha com os Titãs (Titanomaquia) e fim do segundo estágio}
\item \textsc{c.} 721--819\quad \textit{Descrição do Tártaro}
\item \textsc{b.} 820--80\quad \textit{Batalha com Tifeu, o último inimigo de Zeus}
\item \textsc{a.} 881--929\,(?)\quad \textit{Zeus torna-se rei e divide as honras; união com Astúcia e demais}\footnote{O ponto de interrogação indica que não há consenso que verso nos manuscritos do poema marcaria o fim da composição (Kelly 2007).} 
\end{itemize}

A estrutura em anel, na qual se retomam léxico e temas, é uma forma
retórica assaz trivial na poesia grega. Em Homero, por exemplo, o final
de um discurso pode retomar o tópico do início, indicando ao receptor
que o discurso está chegando ao fim.

Repare-se que proêmio do poema é longo se comparado com o início de
outras composições hexamétricas arcaicas identificado como tal. Nele,
\textit{grosso modo}, o aedo costuma estabelecer algum tipo de vínculo com
a Musa, a divindade da qual depende a performance de seu canto, e a
definir o tema geral do poema. Isto \textit{também} é feito na
\textit{Teogonia}, mas, de um modo bastante sofisticado, o tema principal
do poema --- a autoridade as ações de Zeus --- são interligadas àquelas
das Musas e do aedo.

Com isso, o corte entre o chamado proêmio e o restante do poema é bem
menos abrupto que aquele que se verifica na \textit{Ilíada} e na
\textit{Odisseia}: no proêmio nos podemos ver Zeus sendo celebrado como
deus supremo pelas Musas, e isso, de fato, é o que faz o poema como um
todo, pois, embora Zeus não seja o \textit{primeiro} deus, do ponto de
vista da sequência do poema, é como se ele fosse, já que nenhum deus é
tão poderoso ou merece ser tão celebrado como ele.

\section{Hino às Musas: o proêmio do poema}

Por certo é significativo que o narrador da \textit{Teogonia} --- ao
contrário do narrador dos poemas homéricos --- se nomeie no início do
poema\footnote{Mas apenas uma única vez.} no momento mesmo em que é narrado seu
encontro singular com a entidade religiosa tradicional que confere
autoridade a seu canto e garante a precisão de seu conteúdo, as Musas.
Os primeiros 115 versos do poema compõem um proêmio, no qual se celebram
essas divindades (1--103) e se demarca explicitamente o conteúdo do canto
a seguir (104--15). O trecho se assemelha a uma forma poético-religiosa
tradicional em várias sociedades antigas, o canto que celebra as
honrarias ou áreas de atuação, \textit{timē} no singular, de um deus e
que, mais tarde, passou a ser denominado ``hino'', \textit{humnos}.\footnote{O substantivo (que aparece uma vez na
  \textit{Odisseia}) e o verbo cognato, diversas vezes na
  \textit{Teogonia}, que traduzi por ``louvar'' ou ``cantar''
  (Torrano traduz consistentemente pelo neologismo ``hinear''), não
  parecem ser associados primordialmente a deuses nesses textos.} Com
efeito, tal tipo de canto ganhou na Grécia Antiga, em algum momento, uma
versão narrativa no contexto da tradição hexamétrica: são os hinos
homéricos longos ou médios (Ribeiro Antunes \textit{et al.} 2011; Antunes
2015). O que há de muito particular nesse hino da \textit{Teogonia},
porém, é que somente os gregos conheceram essas divindades coletivas
responsáveis por uma esfera cultural que podemos chamar de poesia, mas
que envolvia também música e dança.

Ao celebrar as Musas antes de apresentar o canto que elas propiciam, ou
seja, a cosmogonia e teogonia que começam no verso 116, o poeta também
fala da relação que há entre ele próprio e essas divindades, pois o
valor de verdade, ou seja, a autoridade do canto que apresenta depende
dessa relação. Como pode um mortal falar de eventos pretensamente reais
que não presenciou --- o surgimento do mundo conhecido e de todas as
divindades, bem como dos mortais que com elas dormiram --- se não
apresentar e fundamentar sua relação com certa autoridade transcendente,
já que não há uma tradição textual canônica e uniforme independente do
poema? Nesse sentido, não é mais possível, para nós, saber com certeza
se algum dia houve um poeta chamado Hesíodo e que foi o autor do poema
que conhecemos, ou se \textit{Hesíodo} teria sido uma autoridade
\textit{mítica} inseparável de certa tradição poética e que seria
reencarnada a cada apresentação do poema, um pouco como o ator que
reencarnaria, com uma máscara ritual, nas apresentações teatrais
atenienses no século \textsc{v} a.C., as figuras tradicionais do mito (Nagy
1990). Nesse diapasão, a iniciação no canto, conduzida pelas Musas, pela
qual teria passado o poeta Hesíodo (9--34) também faria parte desse
contexto mítico.

Isso pode ser exemplificado pelo nome \textit{Hesíodo}. Por certo não é
possível \textit{provar} que não tenha existido uma figura histórica com
esse nome responsável pela composição de um ou mais poemas associados ao
nome (Cingano 2009). Além disso, a etimologia do nome não é segura e tem
sido interpretada de diferentes modos (Most 2006, p.\,\textsc{xiv--xvi}).
Meier-Brügger (1990), por exemplo, rediscutiu todas as hipóteses e
defendeu que \textit{Hesíodo} significa ``aquele que se compraz com
caminhos'', o que pode ser interpretado metapoeticamente. Contudo, o
contexto imediato da única vez em que o nome é mencionado no poema
parece indicar que a expressão \textit{ossan hieisai}, ``voz emitindo'',
repetida diversas vezes no proêmio (10, 43, 65 e 67), seria uma glosa de
\textit{Hesíodo} (Nagy 1990, p.\,47--48; Vergados 2020, p.\,43--46), um exemplo
entre vários do que Vergados (2020) define como o pensamento etimológico do autor.

Outro elemento saliente no proêmio é Zeus. Na verdade, como soberano dos
deuses e dos homens, ou seja, como deus responsável pela estrutura
sociopolítica final do cosmo e, dessa forma, também pela manutenção de
sua dimensão física, não é raro Zeus desempenhar algum papel nos hinos
aos deuses que conhecemos, sobretudo, os hinos homéricos maiores. Sua
presença no proêmio da \textit{Teogonia}, porém, é ubíqua, e não apenas
como pai das Musas e seu público primeiro e principal,\footnote{Não nessa ordem
na sequência do poema.} mas também como o deus que, em vista do que
representa, é particularmente associado ao poder político exercido pelos
reis, \textit{basileus} no singular, no mundo humano. Não surpreende,
assim, que, no final do proêmio, as Musas sejam apresentadas como
sombremaneira ligadas não só aos poetas (94--103), mas também aos reis
(80--93), uma figura que, no contexto hesiódico, não representa um
monarca com amplos poderes, mas uma figura que, na esfera pública, age
sobretudo na função de um juiz (Gagarin 1992). O tipo de poder real
exercido por Zeus no poema --- o poder é absoluto e hereditário --- não é
homólogo àquele dos líderes políticos da época. O rei humano é antes de
tudo um aristocrata com prestígio local que participa da administração
da justiça. Que reis e poetas, porém, são figuras dissociáveis, isso
fica claro no destaque dado a Apolo nessa passagem; de qualquer forma, o
proêmio sugere que, entre os homens, poetas são figuras bastante
próximas dos reis (Laks 1996).\looseness=-1

\section{Abismo, «Khaos», e o início do cosmo}

Para chegar a Zeus e o modo como esse controla o cosmo, o tema central
do poema, Hesíodo inicia do começo, ou seja, de Abismo (116), um espaço
vazio cuja delimitação primeira surge na sequência, Terra, \textit{Gaia}.
Não se trata, porém, da Terra tal qual a conhecemos, mas de um espaço
físico ainda descaracterizado, ou melhor, marcado pela sua função
futura, ser o espaço de atuação dos deuses responsáveis pelo equilíbrio
cósmico, que vai, imageticamente, do Olimpo ao ínfero Tártaro. Antes de
Terra começar a gerar suas formas particulares, Montanhas e Mar, e das
divindades aparecerem, duas coisas fundamentais são necessárias, a
presença de Eros (120), o desejo sem o qual não há geração, e as
potências que permitem a sucessão temporal, Escuridão, Noite, Éter e Dia
(123--25).

Todos os deuses descendem de duas linhagens principais, a de Abismo e a
de Terra, mas entre elas não há nenhuma união. Os descendentes de Abismo
são, em sua maioria, potências cuja essência é negativa, como Noite, Morte,
Agonia etc.; várias delas, além disso, expressam ações e emoções que
permeiam os eventos violentos narrados na sucessão de gerações da
linhagem de Terra, como Briga, Disputas, Batalhas etc. A linhagem de
Abismo, portanto, através da descendência de Noite, \textit{Nux}, e Briga, \textit{Eris}, revela que a separação entre Terra e Abismo nunca é total\footnote{As ações e emoções representadas como descendência de Abismo são
executadas ou sentidas pelos descendentes de Terra.} e assim ilustra uma
constante no poema: o encadeamento das linhagens entre si e também delas
com as histórias que se sucedem mostram um poema no qual os catálogos
dos deuses nascentes e as narrativas nas quais os deuses estão
envolvidos não devem ser separados. Trata-se de uma articulação de
imagens, ações e ideias que pressupõe uma temporalidade própria --- ou
melhor, diversas temporalidades (Loney 2018) --- que revela uma mescla
entre o tempo da narrativa genealógica, o tempo da sucessão de um
deus-rei para o seguinte e o tempo da narração. É a partir disso que o
leitor deve entender, por exemplo, que um deus às vezes já apareça como
personagem no poema antes de o narrador mencionar seu nascimento
propriamente dito.

\section{Genealogias divinas}

No poema, teogonia e cosmogonia são inseparáveis à medida que o espaço
se constitui e as genealogias divinas se sucedem. As divindades que
passam pelo poema --- mais de 300 --- são de diversos tipos no que diz
respeito a cultos e mitos (West 1966): 

\begin{enumerate}
\item Os deuses do panteão --- sobretudo os Olímpicos, como Zeus, Apolo, Atena e Ártemis ---, cultuados pela Hélade mas de uma forma mais específica que aquela com que aparecem
no poema (por exemplo, vinculados a certo lugar ou templo específicos);

\item Deuses presentes nas histórias míticas, mas que provavelmente nunca
foram exatamente objetos de culto, como Atlas e, enquanto coletividade, provavelmente os Titãs; 

\item Partes do cosmo divinizados, como Terra, Noite, Montanhas; alguns eram cultuados;

\item Personificações. Elementos que,
para nós, são abstratos, mas não o eram para os gregos; 

\item Aqueles sobre os quais nada sabemos fora de Hesíodo, ou seja, podem ser parte de
um recurso típico dessa tradição, que permitiria a \textit{criação} de
divindades para compor catálogos ou expressar caraterísticas de uma
linhagem. Algo que não deve ser confundido com ficção nem com inovação.
\end{enumerate}

Essa tipologia, porém, não deve ser tomada como algo estático e
invariável. Eros, por exemplo, pode ser pensado como um deus de culto ou
não. Com efeito, o poema não pode ser om retrato de uma estrutura
religiosa fixa, pois essa não existia. Pelo contrário, ele e a tradição
da qual faz parte deveriam ser antes pensados como uma tentativa de
enquadrar, de dar certa forma a uma vivência religiosa que é
essencialmente plural no tempo e no espaço. O lance astuto incorporado
pela tradição --- ou pelo autor do poema --- é justamente procurar
apresentar como um sistema obviamente fixo algo que é necessariamente
variável. A isso está ligado seu sucesso pan-helênico.

\section{Afrodite}

Um dos modos do poeta expressar o que cada divindade tem de específico é
a derivação do seu nome e de seus epítetos. Uma das construções mais
desenvolvidas que exemplificam é a que trata do nascimento, a partir do
esperma de Céu, \textit{Ouranos}, de Afrodite (192--200):

\begin{verse}
\textit{
{[}\ldots{}{]} primeiro da numinosa Citera achegou-se,\\
e então de lá atingiu o oceânico Chipre.\\
E saiu a respeitada, bela deusa, e grama em volta\\
crescia sob os pés esbeltos: a ela Afrodite\\ %\num{195}
espumogênita e Citereia bela-coroa\\
chamam deuses e varões, porque na espuma\footnote{\textit{Aphros}.}\\
foi criada; Citereia, pois alcançou Citera;\\
cipriogênita, pois nasceu em Chipre cercado-de-\qb{}-mar;\\
e ama-sorriso,\footnote{\textit{Philommeidea}.} pois da genitália\footnote{\textit{Mēdōn}.} surgiu.
}
\end{verse}

Ora, à medida que o narrador, devido ao encontro que teve com as Musas,
garante estar falando a verdade, ao mostrar, por meio do próprio nome ---
aceito em toda a Hélade --- do deus que as histórias que ele conta como
que estão inscritas na identidade verbal mesma do deus, ele confronta
histórias de outras tradições que não revelariam o mesmo
conhecimento profundo e inequívoco da realidade por ele dominado. A
filiação da Afrodite de Homero --- ela é filha de Zeus e de Dione --- como
que sucumbe às \textit{provas} dadas na \textit{Teogonia}, cuja lógica só tem
espaço para uma Afrodite, a filha de Céu.

O surgimento de Afrodite é um dos nascimentos que marcam o fim da
supremacia de Céu sobre o cosmo incipiente, ou seja, um momento de crise
que antecede o equilíbrio cosmológico verificado ainda hoje pelos
ouvintes do poema no seu cotidiano. Depois de Céu, também Crono, seu
herdeiro como deus patriarca detentor do poder soberano, é derrotado;
somente Zeus, como rei dos deuses e homens, sempre tem sucesso nos
conflitos que enfrenta. No século \textsc{xx} percebeu-se que o chamado \textit{mito de
sucessão}, fundamental para o entendimento do poema, composto por três
gerações de deuses e seus \textit{patriarcas}, Céu, Crono e Zeus, e os
conflitos principais que cada uma enfrenta --- a castração de Céu, o
nascimento de Zeus possibilitado pelo truque da pedra aplicado por Reia
e o combate de Zeus contra os Titãs e, posteriormente, Tifeu --- guarda
semelhanças em graus diversos com mitos equivalentes transmitidos por
outras culturas antigas do Oriente, como a babilônia e hurro-hitita
(Rutherford 2009, Kelly 2019). O intercâmbio verificado entre essas
culturas problematiza, assim, a origem necessariamente nebulosa mas
certamente não helenocêntrica do poema, ou pelo menos de parte dele. A
maioria dos intérpretes concorda, hoje, que, de Homero e Hesíodo a
Platão, não deve ter havido nada parecido com um \textit{milagre grego},
ainda que não possamos sempre rastrear com precisão como teriam ocorrido
os diversos casos de intercâmbio entre as culturas orientais e a grega
(Burkert 1992, West 1997, Rutherford 2009, Haubold 2013).

\section{Astúcia «versus» força e criaturas prodigiosas}

Os eventos do mito de sucessão são permeados por um par de opostos
complementares fundamental na mitologia, vale dizer, na cultura grega,
\textit{astúcia} e \textit{força} (Detienne \& Vernant 2008). É ele, por exemplo,
que subjaz à oposição entre os heróis máximos dos dois poemas homéricos,
Odisseu e Aquiles, o primeiro, o astuto por excelência, o segundo, o
herói grego mais temido pelos troianos devido à sua força. Também é essa
oposição que mostra, em diversas fábulas, animais mais fracos
fisicamente derrotando os mais fortes ou velozes. No caso da
\textit{Teogonia}, desde o início a astúcia tem a particularidade de ser
uma característica essencialmente feminina. É de Terra o plano ardiloso
que permite a derrota de Céu; Farsa, \textit{Apatē}, é filha de Noite; e
Astúcia, \textit{Mētis} --- além de Persuasão, \textit{Peithō} ---, é uma das
dezenas de filhas de Oceano. No mito de sucesso, a divindade que usar
apenas uma das qualidades ou a usar de modo desproporcional em relação à
outra sempre sucumbe a adversários que combinam as duas de forma mais
eficaz.

Por outro lado, é a Terra que está ligada à geração dos seres
tradicionalmente chamados de monstros (270--335), Équidna, Hidra de
Lerna, Leão de Nemeia, Medusa, Pégaso, Cérbero, Quimera etc. O que
caracteriza tais criaturas como uma coletividade é que elas não se
assemelham nem aos deuses, nem aos homens, nem aos animais, mas são sempre
seres estranhamente mistos, dotados --- assim como sua ancestral primeira
--- de um inominável, enorme poder, algo que faz deles seres incapazes de
serem conquistados pelos mortais, ou seja, ``impossíveis'',
\textit{amêkhanos}. Nesse sentido, e tendo em vista a história do termo
\textit{monstro}, Zanon (2018) mostrou ser mais apropriado chamar essas
criaturas de \textit{prodígios}. Os únicos que as superaram foram certos
heróis, homens muito superiores em força e astúcia que os homens de hoje
e que, além disso, foram auxiliados por deuses.

\textls[-10]{Pela lógica da narrativa, as criaturas prodigiosas parecem ser uma
espécie de tentativa mal sucedida de continuar o desenvolvimento do
cosmo (Clay 2003), já que, em sua maioria, não têm função alguma salvo
contribuírem para a fama do herói que os derrotou. Além disso, por meio
delas se mostra que, assim como, no plano humano, mortais comuns se
opõem a heróis, no divino, deuses se opõem a monstros. Além disso, como
notou Pucci (2009), alguns deuses da geração de Zeus utilizam, eles
próprios, uma criatura para obter determinado fim pessoal, o que
sinaliza que o equilíbrio cósmico continua instável. Os monstros
presentes no poema indicam, para o leitor do presente, que, por ora, a
fertilidade feminina consubstanciada em Terra e que, na sua forma mais
frenética e disforme, gerou tais criaturas --- veja que nos versos 319 e
326 não fica claro quem é a mãe do respectivo monstro, o que parece
acentuar o desregramento ---, foi dominada e regrada por um elemento
masculino, mas esse não será, necessariamente, o fim da história. No
século \textsc{xx} e \textsc{xxi}, \textit{monstros} continuam a assombrar a fantasia humana,
seja na forma de ameaças espaciais ou da guerra atômica, seja como
consequência da forma com que o homem trata o planeta em que habita ---
ou seja, novamente é Gaia quem parece deter a palavra final e, desta
vez, inalienável.}

\section{Estige e Hécate}

Como que a contrabalançar o peso negativo dessas criaturas, na sequência
nascem duas coletividades benfazejas, os Rios e as Oceaninas (337--70),
e, entre essas últimas, destacam-se duas figuras femininas, Estige e
Hécate (383--452). Ambas aparecem na narrativa, de forma anacrônica, para
serem cooptadas por Zeus, cujo nascimento ainda não ocorreu. Isso se
deve, como já foi mencionado acima, pela lógica própria do poema. As
duas divindades femininas não só se opõem à negatividade essencialmente
feminina dos monstros, mas também preparam a narrativa por vir. Estige
ela mesma e seus filhos antecipam a vitória cósmica de Zeus e o novo
equilíbrio que ele vai instaurar e manter. Esse equilíbrio, porém, não é
resultado de uma tábula rasa, mas dá continuidade ao que já estivera
equilibrado durante a supremacia de Crono.\looseness=-1

Hécate, por sua vez, é a deusa que permite a primeira irrupção mais
substancial dos homens no poema. Como o objetivo do poema é revelar a
ordem do cosmo e as prerrogativas dos deuses e celebrá-los, é esperada a
posição absolutamente marginal que o gênero humano ocupa no poema (Clay
2003). Os homens e seu modo de vida são os protagonistas de outro poema
atribuído a Hesíodo, \textit{Trabalhos e dias}. Isso não significa, porém,
que, do ponto de vista dos próprios deuses, ou seja, em última análise,
da própria \textit{Teogonia}, as características da fronteira que separa
deuses e homens não sejam relevantes. Essas aparecem com clareza em dois
episódios que emolduram o nascimento de Zeus, a celebração de Hécate e a
história de Prometeu.

Se aos heróis --- esses humanos mortais que, vale assinalar, estão no
meio do caminho entre deuses e homens --- é dada uma razão de ser durante
o catálogo de monstros, a relação entre Zeus e Hécate, num momento do
poema em que se enfatiza o equilíbrio cósmico resultante das
responsabilidades diversas atribuídas a cada deus, revela que esse
equilíbrio é indissociável da presença, na terra, dos homens. Dito de
outro modo: para pensar-se, figurar-se o modo como os deuses são no
mundo por meio da sequência de eventos que levou à ordem presente,
utiliza-se também um retrato simplificado e razoavelmente genérico das
práticas cultuais humanas. Deuses, cosmo e homens não existem um sem o
outro. O trecho dedicado a Hécate, porém, revela também que a vida
humana, mais que marcada por certo equilíbrio, é permeada pelo
imponderável: por mais que os homens propiciem os deuses, nada garante
que serão auxiliados por eles.

Não possuímos nenhum testemunho histórico independente da
\textit{Teogonia} que aponte para a importância cultual, mesmo que apenas
local, de Hécate sugerida pelo destaque que lhe dado no poema. Isso é um
forte indício de que comentadores como Clay (2003) estão corretos ao
defender que a figura dessa deusa é usada para se falar de Zeus e da
relação entre os homens e os deuses inaugurada por ele. Menos certa é a
relação entre o nome de Hécate, a maneira como o poeta se refere ao seu
modo de atuação --- ``se ela quiser'' etc. --- e o acaso.

\section{Zeus e Prometeu}

O nascimento de Zeus narrado logo depois (453--91) é o evento que permite
a queda de Crono e a ascensão do terceiro soberano dos deuses. A astúcia
de Terra é a responsável pela castração de Céu, a libertação, ou
nascimento, de seus filhos, os Titãs, e a tomada de
poder por parte do filho mais novo, Crono. De forma homóloga, é a
astúcia da esposa de Crono, Reia, auxiliada pelos conselhos de Céu e
Terra, que permite que seus filhos vejam a luz do Sol e Zeus destrone o
pai. Desta vez, porém, há uma verdadeira competição entre astutos: como
todo bom rei, Crono é previdente, e, ao aprender parcialmente com o
erro de seu pai, decide engolir todos os filhos \textit{após} esses serem
paridos por sua esposa, com o que, porém, ainda exercita de uma forma
arbitrária sua força. Reia, porém, o ludibria no nascimento de Zeus, de
sorte que esse, por meio de uma série de manobras contadas rapidamente
no poema, pode ocupar a regência do cosmo. Ainda que, pelo menos em
parte, nesse momento da narrativa Zeus não seja representado como um
agente deliberando sozinho, seu poder é de pronto ligado às duas esferas
mencionadas acima, astúcia e força. Por enquanto, sua astúcia ainda é
aquela da mãe e da avó; sua força, porém, está ligada ao seu primeiro
ato como soberano --- do ponto de vista da lógica da narrativa: a
libertação dos Ciclopes (501--6), aqueles que lhe fornecerão os raios e o
trovão, atributos que, por certo, funcionam como armas mas também são
simbólicos, já que apontam para sua ligação com o céu.\looseness=-1

O primeiro conflito resolvido por Zeus, porém, envolve a astúcia
(507--616). Trata-se do momento em que deuses e homens se distinguiram,
se separaram em definitivo por ocasião de um banquete festivo para o
qual Prometeu separou a carne de um boi. Marcam esse evento a origem do
sacrifício, a conquista do fogo e a criação da mulher humana. O texto
não procura descrever detalhadamente a linhagem humana que não dominava
o fogo, ainda compartilhava da companhia dos deuses e não conhecia a
reprodução sexual; isso é feito, sob viés distinto, em \textit{Trabalhos e
dias}. Todavia, como o narrador deixa claro que Zeus aceita a repartição
da carne do boi feita por Prometeu para o banquete porque ele tinha em
mente males destinados \textit{aos homens mortais} (551--52), podemos supor
que, nesse momento de sua regência, quando Zeus ainda precisa consolidar
seu poder, os homens,\footnote{Os versos 50 e 185--87 talvez sugiram que esses
fossem guerreiros gigantes nascidos da terra, figuras que conhecemos de
outros relatos.} em conluio com Prometeu, representam uma ameaça que
precisa ser dominada antes que seja tarde demais. A previdência é um
atributo indispensável do soberano que quiser manter seu poder. Ao
contrário de Zeus, que antecipa o movimento do provável inimigo, Crono
falhou em sua tática de engolir os filhos: bastou que um escapasse para
ele ser destronado.

Outro momento fundamental da história de Prometeu é a criação da
primeira mulher. Ao contrário do que ocorre em \textit{Trabalhos e dias},
aqui o narrador não informa seu nome, que lá é Pandora. Como em todos os
eventos que marcam o episódio de Prometeu, bem e mal estão
indissociavelmente ligados (Vernant 1992 e 2002): nessa etiologia do
sacrifício, os ossos, que não podem ser digeridos (mal), são encobertos
pela gordura que solta delicioso aroma (bem), ao passo que a carne (bem)
é disfarçada sob o repelente estômago (mal). Assinale-se que o disfarce
--- e, consequentemente, a habilidade de reconhecer o que está disfarçado
--- também faz parte do domínio da astúcia: se Prometeu é astuto, Zeus o
é em ainda mais alto grau. Os ossos, bem como o aroma da gordura
queimada, são, por outro lado, sinais da imortalidade divina (bem), ao
passo que a carne deliciosa, o alimento perecível, comida pelos homens
aponta para sua mortalidade (mal). A a adoção da carne em sua dieta,
escondida no estômago do boi, deixa claro que os homens são escravos de
seu próprio estômago e precisam satisfazê-lo se não quiserem perecer.

No caso da Mulher, ela é dada aos homens em troca do fogo: ao passo que
o fogo permite que os homens sejam civilizados e não comam carne crua, a
mulher terá que ser por eles alimentada, caso queiram sobreviver por
intermédio de um herdeiro. De fato, fogo e mulher precisam ser
constantemente alimentados para que o homem não pereça. O sacrifício, o
fogo e a bela mulher, portanto, indicam que há elementos que apontam
para uma presença do divino no centro da vida humana, mas eles são tão
tênues como a fumaça que sobe do sacrifício para o céu e tão artificiais
quanto os enfeites da coroa da primeira mulher, contra a qual o homem
não tem defesa alguma.

\section{Titanomaquia}

Após essa separação entre deuses e homens levada a cabo por Zeus graças
à astúcia, a separação seguinte, entre os deuses da geração de seu pai,
os Titãs, e os da sua própria, os Olímpicos, é conseguida devido à
supremacia alcançada sobretudo por meio da força. O episódio conhecido
como Titanomaquia (617--720) mostra que o cosmo ficou mais complexo que
quando sobre ele regia Céu, pois se, para vencer seu pai, num primeiro
momento, Zeus contou com pelo menos dois ardis arquitetados pela mãe e
pela avó --- entregar a Crono uma pedra no lugar do bebê Zeus e,
posteriormente, fazê-lo vomitar todos os irmãos de Zeus que com ele por
fim lutariam contra os deuses mais velhos ---, num segundo momento, a
astúcia deixa de ser suficiente.

É de novo Terra quem aconselha ao neto libertar aqueles que haviam sido
presos por Céu e assim mantidos por Crono abaixo da terra, os
Cem-Braços. Trata-se de uma força descomunal que os dois soberanos
anteriores acharam por bem simplesmente manter paralisada, paralisia
homóloga àquela que tentaram, sem sucesso, implementar contra seus
filhos. Zeus, porém, consegue convencê-los a serem seus aliados e eles
se mostram decisivos no combate contra os Titãs, gratos por serem
trazidos de volta à luz.

Luz e trevas: essa polaridade marca toda a Titanomaquia, pois os Titãs,
uma vez vencidos, passam a ocupar o espaço subterrâneo onde antes
estiveram os Cem-Braços que, porém, agora tem uma honra, uma função no
cosmo, a de serem os eternos guardas dos deuses outrora poderosos, os
Titãs. Essa polaridade, ademais, também prepara o episódio seguinte,
pois o esforço de Zeus para vencer os Titãs como que traz o cosmo de
volta ao seu estado inicial: terra, céu, mar e Tártaro, todos os espaços
são atingidos pelo fogo dos raios de Zeus, o que representa uma
recriação do mundo por meio da força. Não é por acaso que Abismo volta à
cena (700 e 814) e que as imagens e sons desse conflito cataclísmico
sejam amplificadas para o ouvinte por meio de uma imagem que remete à
união primordial entre Terra e Céu (700--5).\looseness=-1

Uma vez finalizada a guerra, o narrador nos narra, pela primeira vez,
como é a geografia das terras ínferas (721--819). Não que antes nada lá
houvesse. Com o aprisionamento dos Titãs, porém, à essa parte do cosmo é
conferida sua estabilidade e Zeus pode finalmente aparecer como o
organizador último de todos os espaços. É por essa razão que de deuses
como Sono e Morte e Noite e Dia, cujas funções cósmicas os ligam ao
Tártaro, finalmente se fala mais longamente, uma vez mais se mostrando
de que forma polos positivos e negativos da realidade estão
interligados. É precisamente por isso que também nesse momento do poema
descreve-se a função de Estige, ligada a uma jura divina que, quando
quebrada por um deus, o leva a uma morte virtual por dez anos. A ligação
entre Estige e Zeus mostra que também o juramento --- uma instituição
social fundamental também entre os homens --- é instituído pelo rei dos
deuses e homens para bem administrar o mundo divino onde conflitos não
são excepcionais.

\section{Zeus e Tifeu}

Curiosamente, porém, Zeus ainda terá que enfrentar mais um conflito
belicoso, a luta contra Tifeu (820--80). Por um lado, como nos dois
poemas épicos que conhecemos, a \textit{Ilíada} e a \textit{Odisseia}, o
maior herói se revela quando um derrota inimigo poderoso com suas
próprias mãos. Por outro lado, esse inimigo é, estranhamente, filho do
próprio Tártaro com Terra. Que a fertilidade exacerbada desta tenho
gerado um ser para destronar o novo senhor do cosmo, isto não
surpreende, pois a eminência parda feminina foi peça fundamental na
deposição de Céu e Crono; que aquele seja o pai, isto sim é curioso,
pois até este momento da narrativa dele apenas se falou como um espaço.
É como se, pela lógica da narrativa hesiódica, só agora ele tivesse
adquirido o estatuto pleno de divindade e precisasse se envolver em um
conflito que garanta que sua forma não se alterará.

Tifeu, por sua vez, adquire, devido à lógica da narrativa, o lugar de
filho de Zeus, pois todo rei anterior fora deposto por seu filho, sempre
ligado à Terra. O conflito contra os Titãs, porém, já mostrou que a
manipulação da astúcia e da força, no grau superlativo em que o faz
Zeus, não deixa espaço para a possibilidade de derrota, mesmo que o
adversário também seja muito forte --- Tifeu tem cabeças com olhos de
onde sai fogo --- e muito astuto --- suas cem cabeças produzem todo tipo
de som, sendo que a metamorfose é um elemento mítico típico do universo
da astúcia. Além disso, esse combate singular entre a criatura
monstruosa e Zeus também permite que Terra, derradeiramente, seja
derrotada e esterilizada. O fogo de Zeus como que a derrete: de criadora
de metal e artífice metalúrgica, Terra como que se transforma, graças ao
fogo aniquilador de Zeus, no metal que é manipulado por artesãos machos
(861--67).

\section{Zeus e suas esposas}

Uma vez derrotada a astuta Terra, que imediatamente se torna aliada de
Zeus (891), a primeira providência do soberano é casar com Astúcia e,
antes de essa parir seu primeiro filho, devorá-la, não esperando que
essa gerasse um deus macho mais forte que ele (886--900). Muellner (1996)
mostrou como esse episódio arremata todos os conflitos dinásticos
narrados até então: Zeus não devora seu primeiro filho, como Crono, ou
obriga que sua esposa o guarde no ventre, como Céu, mas assimila o
elemento feminino em si mesmo, Astúcia, e o gera como aliado, Atena.
Com isso, Zeus se torna um andrógino perfeito,\footnote{Do ponto de vista grego:
muito mais masculino que feminino.} e não um disforme emasculado, como
Céu. A astúcia revela-se mais uma vez essencialmente feminina, mas para
sempre assimilada pelo próprio rei. A filha produzida pelo rei não só
não é um macho --- e foram sempre jovens machos que derrotaram seus pais
---, mas é uma virgem, ou seja, uma deusa que não irá produzir uma ameaça
ao \textit{status quo}. Por fim, ao ingerir a esposa grávida do primeiro
filho, ele bloqueou a previsão de que, depois de Atena, Astúcia geraria
um filho mais forte que o pai. Pela primeira vez, o rei dos deuses
consegue \textit{desparir} de forma perfeita e acabada.

E somente agora nasce, de Zeus e várias de suas esposas, uma linhagem de
deuses responsáveis pelo que há de bom no cosmo propriamente humano, ou
seja, na sociedade (901--17): Norma, Decência, Justiça, Paz, as Musas,
Radiância, Alegria e Festa, notável prole antípoda aos filhos de Noite e
Briga. A última esposa de Zeus, Hera, é aquela que, de acordo com a
lógica do poema, representa a maior ameaça a Zeus, mas tanto o filho
mais perigoso que os dois têm juntos, quanto aquele que Hera, como que
emulando Zeus no caso de Atena, tem sozinha, Hefesto, não representam
adversários fortes o suficiente contra a filha que mais se assemelha ao
pai e está completamente alinhada com ele, Atena, senhora da guerra mas
também da astúcia (921--29).

É nesse sentido que se deve entender o longo catálogo que finaliza o
poema e que tem três partes: os casamentos de Zeus e os filhos deles
resultantes (901--29); um catálogo mais abrangente de casamentos divinos
(930--61), que revelam, de forma sumária, um panteão muito bem organizado
e potencialmente harmônico;\footnote{Como que servindo de epítome, o casamento
entre Ares e Afrodite produz, por um lado, os machos Terror e Pânico,
mas, por outro, Harmonia.} e finalmente um catálogo de deusas que se
uniram a mortais (962--1020). Ora, com as deusas fêmeas que se unem a
machos mortais, o princípio de ruptura que vigorara ao longo do poema
agora se desloca para o mundo dos homens, mais precisamente, o mundo dos
heróis: nesse mundo, filhos poderão ser mais fortes que os pais,
podendo, no limite, o que atesta Telégono, o filho de Circe e Odisseu,
matá-lo.

Para concluir, mencione-se que há uma discussão inconclusa sobre onde a
versão ``original'' da \textit{Teogonia} teria terminado. Autores como
Clay (2003) e Kelly (2007) mostraram que os catálogos tal como
analisados acima compõe um final muito adequado ao poema; assim,
provavelmente somente os quatro ou possivelmente os dois últimos versos
foram acrescentados ao poema em um certo momento de sua transmissão para
introduzir um outro poema atribuído a Hesíodo, o \textit{Catálogo das
mulheres}, que chegou a nós por meio de fragmentos, que procurava
dar uma visão geral da idade dos heróis a partir das mulheres que com
deuses dormiram por toda a Grécia, catálogo este que, possivelmente, era
concluído pelo catálogo de pretendentes de Helena, cujo casamento
redundou no grande cataclisma que foi a guerra de Troia, que
metonimicamente podia ser pensada, na Antiguidade, como o fim da época
dos heróis.

\section{Da tradução}

Para definir o texto grego aqui traduzido, cotejaram-se as seguintes
edições: West (1966), Most (2006) e Ricciardelli (2018). Também foram
muito úteis para se definir a opção por determinada leitura ou
interpretação, bem como para compor as notas, diversos textos citados na
bibliografia, especialmente Marg (1970), Verdenius (1972), Arrighetti
(2007), Pucci (2007) e Vergados (2020). Para a tradução, também foi
fundamental o léxico organizado por Snell \textit{et al.} (1955--2010).

Um dos principais problemas enfrentados pelo o tradutor da
\textit{Teogonia} diz respeito ao nomes das divindades. Não se buscou
nenhum tipo de padronização muito rígida, ou seja, ficou-se entre os
extremos de traduzir quase todos os nomes e quase nenhum nome. De
forma geral, os principais critérios foram o bom senso, o conhecimento
do leitor e a sonoridade. Além disso, as notas apresentam a
transliteração de todos os nomes, bem como explicitam algumas figuras
etimológicas.

Para facilitar a leitura, optou-se por seguir o que fazem a maioria dos
editores em sua forma de propor uma divisão do poema em partes
distintas. O recuo de parágrafo, ainda que estranho em um poema, deve
ser pensado como equivalente a um novo parágrafo em uma narrativa em
prosa. Não é possível saber, entretanto, se tais marcações são
equivalentes a pausa nas performances orais originais dos poemas.
Trata-se, portanto, de um recurso eminentemente didático.

Algumas soluções que adotei nas minhas traduções de Homero (2018a) e
(2018b) nortearam certas modificações nesta edição da tradução do poema
hesiódico. Uma delas é evitar excessos no uso da ordem sintática
indireta.

A numeração das notas de rodapé em forma de lemas segue o número que
indica um verso ou um conjunto de versos do poema.

Por fim, gostaria de agradecer àqueles que compartilharam comigo seu
conhecimento de Hesíodo, em especial, da \textit{Teogonia}, desde a 1ª
edição deste volume ou me apontaram o que nele poderia ser melhorado ou
corrigido: Camila Zanon, Thanassis Vergados, Jim Marks, Adrian Kelly,
Teodoro Assunção, André Malta, os membros da minha banca de
livre-docência --- Jaa Torrano, Zélia de Almeida Cardoso, Jacyntho L.
Brandão, Pedro Paulo Funari e Maria Beatriz Florenzano --- e
Antonio-Orlando Dourado Lopes.

\begin{bibliohedra}
\tit{allan}, W. Divine justice and cosmic order in early Greek Epic.
\textit{Journal of Hellenic Studies} v.\,126, 2006, p.\,1--35.

\tit{andersen}, Ø.; \textsc{haug}, D. T. T. (org.) \textit{Relative chronology in early
Greek epic poetry}. Cambridge: Cambridge University Press, 2012.

\tit{antunes}, C. L. B. 26 hinos homéricos. \textit{Cadernos de literatura em
tradução} v.\,15, p.\,13--23, 2015.

\tit{arnould}, D. Les noms des dieux dans la \textit{Théogonie} d'Hésiode:
étymologies et jeux de mots. \textit{Revue des études grecques} v.\,122,
2009, p.\,1--14.

\tit{arrighetti}, G. \textit{Esiodo opere}. Introdução, tradução e comentário.
Milano: Mondadori, 2007.

\tit{bakker}, E. J. Hesiod in performance. In: \textsc{loney}, A. C.; \textsc{scully}, S. (org.)
\textit{The Oxford Handbook of Hesiod}. Oxford: Oxford, 2018.

\tit{blaise}, F.; \textsc{judet de la combe}, P.; \textsc{rousseau}, P. (org.) \textit{Le métier
du mythe: lectures d' Hésiode}. Lille: Presses Universitaires du
Septentrion, 1996.

\tit{brandão}, J. L. \textit{Antiga Musa (arqueologia da ficção)}. 2ª edição.
Belo Horizonte: Relicário, 2015.

\tit{burkert}, W. \textit{The Orientalizing revolution: Near Eastern influence
on Greek culture in the early archaic age}. Cambridge, Mass.: Harvard
University Press, 1992.

\titidem. \textit{Religião grega na época clássica e arcaica}. Lisboa: Fundação
Calouste Gulbenkian, 1993.

\tit{cingano}, E. The Hesiodic corpus. In: \textsc{montanari}, F.; \textsc{rengakos}, A.;
\textsc{tsagalis}, C. (org.) \textit{Brill's companion to Hesiod}. Leiden/Boston:
Brill, 2009, p.\,91--130.

\tit{clay}, J. S. \textit{Hesiod's cosmos}. Cambridge: Cambridge University
Press, 2003.

\tit{colonna}, A. \textit{Opere di Esiodo}. Torino: Unione Tipografico-Editrice,
1977.

\tit{detienne}, M. \textit{Os mestres da verdade na Grécia} arcaica. Trad. A.
Daher. Rio de Janeiro: Jorge Zahar, 1988.

\tit{detienne}, M.; \textsc{vernant}, J.-P. \textit{Métis}: As astúcias da inteligência.
Trad. F. Hirata. São Paulo: Odysseus, 2008.

\tit{gagarin}, M. The poetry of justice: Hesiod and the origins of Greek law.
\textit{Ramus} v.\,21, 1992, p.\,61--78.

\tit{haubold}, J. \textit{Greece and Mesopotamia}: dialogues in literature.
Cambridge: Cambridge University Press, 2013.

\tit{janda}, M. \textit{Über `Stock und Stein': die indogermanischen Variationen
eines universalen Phraseologismus}. Röll: Dettelbach, 1997.

\tit{janko}, R. \textit{Homer, Hesiod and the Hymns:} diachronic development in
epic diction. Cambridge: Cambridge University Press, 1982.

\tit{kelly}, A. How to end an orally-derived epic poem? \textit{Transactions of
the American Philological Association} n.\,137, 2007, p.\,371--402.

\titidem. Gendrificando o mito de sucessão em Hesíodo e no antigo Oriente
Próximo. Trad.: C. A. Zanon. \textit{Classica} v.\,32, n.\,2, 2019, p.\,119--38.

\tit{koning}, H. \textit{Hesiod}: \textit{the other poet}: ancient reception of a
cultural icon. Leiden: Brill, 2010.

\tit{laks}, A. Le doublé du roi: remarques sur les antécédents hésiodiques du
philosophe-roi. In~: \textsc{blaise}, F.; \textsc{judet de la combe}, P.; \textsc{rousseau}, P.
(org.) \textit{Le métier du mythe}: lectures d' Hésiode. Lille: Presses
Universitaires du Septentrion, 1996.

\tit{lamberton}, R. \textit{Hesiod}. New Haven: Yale University Press, 1988.

\tit{leclerc}, M.-C. \textit{La parole chez Hésiode: à la recherche de
l'harmonie perdue}. Paris: Belles Lettres, 1993.

\tit{ledbetter}, G. M. \textit{Poetics before Plato: interpretation and
authority in early Greek theories of poetry}. Princeton: Princeton
University Press, 2003.

\tit{loney}, A. C. Hesiod's temporalities. In: \textsc{loney}, A. C.; \textsc{scully}, S. (org.)
\textit{The Oxford Handbook of Hesiod}. Oxford: Oxford, 2018.

\tit{macedo}, J. M. \textit{A palavra ofertada}: um estudo retórico dos hinos
gregos e indianos. Campinas: Edunicamp, 2010.

\tit{marg}, W. \textit{Hesiod}: Sämtliche Gedichte. Artemis: Zürich/Stuttgart,
1970.

\tit{martin}, R. P. Hesiod, Odysseus, and the instruction of princes.
\textit{Transactions of the American Philological Association} v.\,114,
1984, p.\,29--48.

\titidem. Hesiodic theology. In: \textsc{loney}, A. C.; \textsc{scully}, S. (org.) \textit{The
Oxford Handbook of Hesiod}. Oxford: Oxford, 2018.

\tit{meier-brügger}, M. Zu Hesiods Namen. \textit{Glotta} v.\,68, 1990, p.\,62--67.

\tit{moraes}, A. S. de. História e etnicidade: Homero à vizinhança do
pan-helenismo. \textit{Hélade} v.\,5, n.\,1, 2019, p.\,12--36.

\tit{most}, G. W. Hesiod and the textualization of personal temporality. In:
\textsc{montanari}, F.; \textsc{arrighetti}, G. (org.) \textit{La componente autobiografica
nella poesia greca e latina}. Pisa: Giardini, 1993, p.\,73--91.

\titidem. \textit{Hesiod}: \textit{Theogony, Works and Days, Testimonia}.
Cambridge, \textsc{ma}: Harvard University Press, 2006.

\tit{muellner}, L. C. \textit{The anger of Achilles}: mēnis \textit{in Greek
epic}. Ithaca: Cornell University Press, 1996.

\tit{murray}, P. Poetic inspiration in early Greece. \textit{Journal of Hellenic
Studies} v.\,101, 1981, p.\,87--100.

\tit{nagy}, G. Hesiod and the poetics of Pan-Hellenism. In: \line(1,0){25}. \textit{Greek
mythology and poetics}. Ithaca: Cornell Univesity Press, 1990, p.\,36--82.

\titidem. \textit{The best of the Achaeans: concepts of the hero in archaic Greek
poetry}. 2ª ed. Baltimore: Johns Hopkins University Press, 1999.

\tit{oliveira}, J. ``Áurea Afrodite'' e a ordem cósmica de Zeus na poesia
hesiódica. \textit{Codex} -- Revista de estudos clássicos. Rio de Janeiro,
v.\,7, n.\,2, 2019, p.\,69--80.

\titidem. A linhagem dos heróis na cosmologia hesiódica. \textit{Rónai} v.\,8, n.\,2, 2020, p.\,353--374.

\tit{pucci}, P. \textit{Hesiod and the language of poetry}. Baltimore: Johns
Hopkins University Press, 1977.

\titidem. \textit{Inno alle Muse (Esiodo}, Teogonia\textit{, 1--115): texto,
introduzione, traduzione e commento}. Pisa: Fabrizio Serra, 2007.

\titidem. The poetry of the \textit{Theogony}. In: \textsc{montanari}, F.; \textsc{rengakos}, A.;
\textsc{tsagalis}, C. (org.) \textit{Brill's Companion to Hesiod}. Leiden/Boston:
Brill, 2009, p.\,37--70.

\tit{ricciardelli}, G. \textit{Esiodo: Teogonia}. Milano: Fondazione Lorenzo
Valla / Mondadori, 2018.

\tit{rijksbaron}, A. Discourse cohesion in the proem of Hesiod's
\textit{Theogony}. In: \textsc{bakker}, S.; \textsc{wakker}, G. (org.) \textit{Discourse
cohesion in Ancient Greek}. Leiden: Brill, 2009.

\tit{ribeiro} Jr., W. A. \textit{et al.} \textit{Hinos homéricos: tradução, notas
e estudo}. São Paulo: Edunesp, 2011.

\tit{rosenfield}, K. H. \textit{Desenveredando Rosa: a obra de J. G. Rosa e
outros ensaios}. Rio de Janeiro: Topbooks, 2006.

\tit{rowe}, C. J. `Archaic thought' in Hesiod. \textit{Journal of Hellenic
Studies} v.\,103, p.\,124--35, 1983.

\tit{rutherford}, I. Hesiod and the literary traditions of the Near East. In:
\textsc{montanari}, F.; \textsc{rengakos}, A.; \textsc{tsagalis}, C. (org.) \textit{Brill's companion
to Hesiod}. Leiden: Brill, 2009.

\tit{scully}, S. \textit{Hesiod's~}Theogony\textit{:~from Near Eastern
creation myths to}~Paradise Lost\textit{.} Oxford and New York: Oxford
University Press,~2015.

\tit{snell}, B. O mundo dos deuses em Hesíodo. In: \line(1,0){25}. \textit{A cultura grega e
as origens do pensamento}. São Paulo: Perspectiva, 2001.

\tit{snodgrass}, A. M. \textit{The Dark Age of Greece}. Edinburgh: Edinburgh
University Press, 1971.

\tit{thalmann}, W. G. \textit{Conventions of form and thought in early Greek
epic}. Baltimore/ London: Johns Hopkins University Press, 1984.

\tit{torrano}, J. A. A. \textit{Hesíodo: Teogonia}. A origem dos deuses. Estudo
e tradução. 2\textsuperscript{a} edição. São Paulo: Iluminuras, 1992.

\titidem. \textit{O certame Homero-Hesíodo} (texto integral). \textit{Letras
clássicas} 9, p.\,215--24, 2005.

\tit{tsagalis}, C. Poetry and poetics in the Hesiodic corpus. In: \textsc{montanari},
F.; \textsc{rengakos}, A.; \textsc{tsagalis}, C. (org.) \textit{Brill's companion to
Hesiod}. Leiden: Brill, 2009, p.\,131--78.

\tit{verdenius}, W. J. Notes on the proem of Hesiod's \textit{Theogony}.
\textit{Mnemosyne} v.\,25, 1972, p.\,225--60.

\tit{vergados}, A. Stitching narratives: unity and episod in Hesiod. In:
\textsc{werner}, C.; \textsc{dourado-lopes}, A.; \textsc{werner}, E. (org.) \textit{Tecendo
narrativas}: unidade e episódio na literatura grega antiga. São Paulo:
Humanitas, 2015, p.\,29--54.

\titidem. \textit{Hesiod's verbal craft: studies in Hesiod's conception of
language and its ancient reception}. Oxford: Oxford University Press,
2020.

\tit{vernant}, J.-P. \textit{Mito e sociedade na Grécia antiga}. Rio de Janeiro:
José Olympio, 1992.

\titidem. \textit{Mito e pensamento entre os gregos}. Rio de Janeiro: Paz e
Terra, 2002.

\tit{versnel}, H. S. \textit{Coping with the gods: wayward readings in Greek
theology}. Leiden: Brill, 2011.

\tit{west}, M. L. \textit{Hesiod,} Theogony\textit{: edited with prolegomena and
commentary}. Oxford: Oxford University Press, 1966.

\titidem. \textit{The east face of Helicon: West Asiatic elements in Greek poetry
and myth}. Oxford: Oxford University Press, 1997.

\tit{woodward}, R. D. Hesiod and Greek myth. In: \line(1,0){25}. (org.) \textit{The Cambridge
companion to Greek mythology}. Cambridge: Cambridge University Press,
2007.

\tit{zanon}, C. A. \textit{Onde vivem os monstros: criaturas prodigiosas na
poesia de Homero e Hesíodo}. São Paulo: Humanitas, 2018.
\end{bibliohedra}


\endgroup




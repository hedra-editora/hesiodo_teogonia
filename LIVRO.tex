\documentclass[showtrims,		% Mostra as marcas de corte em cruz
			   %trimframe,		% Mostra as marcas de corte em linha, para conferência
			   11pt				% 8pt, 9pt, 10pt, 11pt, 12pt, 14pt, 17pt, 20pt 
			   ]{memoir}
\usepackage[brazilian,
			% english,
			% italian,
			% ngerman,
			% french,
			% russian,
			% polutonikogreek
			]{babel}
\usepackage{anyfontsize}			    % para tamanhos de fontes maiores que \Huge 
\usepackage{relsize}					% para aumentar ou diminuir fonte por pontos. Ex. \smaller[1]
\usepackage{fontspec}					% para rodar fontes do sistema
\usepackage{lipsum}						% para colocar textos lipsum
\usepackage{alltt}						% para colocar espaços duplos. Ex: verso livre
\usepackage{lettrine}					% para capsulares
\usepackage{comment}					% para comentar o código em bloco \begin{comment}...
\usepackage{adforn}						% para adornos & glyphs
\usepackage[babel]{microtype}			% para ajustes finos na mancha
\usepackage{enumerate,enumitem}			% para tipos diferentes de enumeração/formatação ver `edlab-extra.sty`
\usepackage{url}					% para citar sites \url
\usepackage{marginnote}					% para notas laterais
\usepackage{titlesec}					% para produzir os distanciamentos entre pontos no \dotfill

%\usepackage{makeidx} 					% para índice remissivo

\usepackage{edlab-penalties}
\usepackage{edlab-git}
\usepackage{edlab-toc}					% define sumário
\usepackage{edlab-extra}				% define epígrafe, quote
\usepackage[largepost]{edlab-margins}
\usepackage[%semcabeco, 				% para remover cabeço, sobe mancha e mantem estilos
			]{edlab-sections}			% define pagestyle (cabeço, rodapé e seções)
\usepackage[%
			% notasemlinha 			
			% notalinhalonga
			chicagofootnotes			% para notas com número e ponto cf. man. de Chicago
			]{edlab-footnotes}
\usepackage{reledmac}
\usepackage{reledpar}
\setRlineflag{}
\renewcommand{\numlabfont}{\normalfont\tiny}

% Medidas (ver: memoir p.11 fig.2.3)
\parindent=3ex			% Tamanho da indentação
\parskip=0pt			% Entre parágrafos
\marginparsep=1em		% Entre mancha e nota lateral 
\marginparwidth=4em		% Tamanho da caixa de texto da nota laterial

% Fontes
\newfontfamily\formular{Formular}
\newfontfamily\formularlight{Formular Light}
\newfontfamily\Porson{porson.ttf}
\setmainfont[Ligatures=TeX,Numbers=OldStyle]{Minion Pro}


% Estilos
\makeoddhead{baruch}{}{\scshape\MakeLowercase{\scshape\MakeTextLowercase{}}}{}
\makeevenhead{baruch}{}{\scshape\MakeLowercase{}}{}
\makeevenfoot{baruch}{}{\footnotesize\thepage}{}
\makeoddfoot{baruch}{}{\footnotesize\thepage}{}
\pagestyle{baruch}		
\headstyles{baruch}

% Testes


\begin{document}


%\input{MUSSUMIPSUM}  		% Teste da classe
% Este documento tem a ver com as partes do LIVRO. 

% Tamanhos
% \tiny
% \scriptsize
% \footnotesize
% \small 
% \normalsize
% \large 
% \Large 
% \LARGE 
% \huge
% \Huge

% Posicionamento
% \centering 
% \raggedright
% \raggedleft
% \vfill 
% \hfill 
% \vspace{Xcm}   % Colocar * caso esteja no começo de uma página. Ex: \vspace*{...}
% \hspace{Xcm}

% Estilo de página
% \thispagestyle{<<nosso>>}
% \thispagestyle{empty}
% \thispagestyle{plain}  (só número, sem cabeço)
% https://www.overleaf.com/learn/latex/Headers_and_footers

% Compilador que permite usar fonte de sistema: xelatex, lualatex
% Compilador que não permite usar fonte de sistema: latex, pdflatex

% Definindo fontes
% \setmainfont{Times New Roman}  % Todo o texto
% \newfontfamily\avenir{Avenir}  % Contexto

\begingroup\thispagestyle{empty}\vspace*{.05\textheight} 


              \formular
              \LARGE 
              \noindent
              \textsc{Teogonia}
              \smallskip
                      
              \large
              \noindent\textit{}
              \normalsize 
              \vspace{2em}
                    
\endgroup
\vfill
\pagebreak       % [Frontistício]
%\newcommand{\linhalayout}[2]{{\tiny\textbf{#1}\quad#2\par}}
\newcommand{\linha}[2]{\ifdef{#2}{\linhalayout{#1}{#2}}{}}

\begingroup\tiny
\parindent=0cm
\thispagestyle{empty}

\textbf{edição brasileira©}\quad			 {Hedra \the\year}\\
\textbf{introdução e tradução©}\quad		 {Christian Werner}\\
%\textbf{posfácio©}\quad			 		 {}\\
%\textbf{ilustração©}\quad			 		 {copyrightilustracao}\medskip

%\textbf{título original}\quad			 	 {}\\
%\textbf{edição consultada}\quad			 {edicaoconsultada}\\
%\textbf{primeira edição}\quad			 	 {}\\
%\textbf{agradecimentos}\quad			 	 {}\\
%\textbf{indicação}\quad			 		 {indicacao}\medskip

\textbf{edição}\quad			 			 {Jorge Sallum}\\
\textbf{coedição}\quad			 			 {Suzana Salama}\\
\textbf{assistência editorial}\quad			 {Paulo Henrique Pompermaier}\\
\textbf{revisão}\quad			 			 {Iuri Pereira}\\
%\textbf{preparação}\quad			 		 {preparacao}\\
%\textbf{iconografia}\quad			 		 {iconografia}\\
\textbf{capa}\quad			 				 {Lucas Kroëff}\\
%\textbf{imagem da capa}\quad			 	 {imagemcapa}\medskip

\textbf{\textsc{isbn}}\quad			 		 {978-65-89705-58-1}

\hspace{-5pt}\begin{tabular}{ll}
\textbf{conselho editorial} & Adriano Scatolin,  \\
							& Antonio Valverde,  \\
							& Caio Gagliardi,    \\
							& Jorge Sallum,      \\
							& Ricardo Valle,     \\
							& Tales Ab'Saber,    \\
							& Tâmis Parron      
\end{tabular}
 
\bigskip
\textit{Grafia atualizada segundo o Acordo Ortográfico da Língua\\
Portuguesa de 1990, em vigor no Brasil desde 2009.}\\

\vfill
\textit{Direitos reservados em língua\\ 
portuguesa somente para o Brasil}\\

\textsc{editora hedra ltda.}\\
R.~Fradique Coutinho, 1139 (subsolo)\\
05416--011 São Paulo \textsc{sp} Brasil\\
Telefone/Fax +55 11 3097 8304\\\smallskip
editora@hedra.com.br\\
www.hedra.com.br\\

Foi feito o depósito legal.

\endgroup
\pagebreak     % [Créditos]
% Tamanhos
% \tiny
% \scriptsize
% \footnotesize
% \small 
% \normalsize
% \large 
% \Large 
% \LARGE 
% \huge
% \Huge

% Posicionamento
% \centering 
% \raggedright
% \raggedleft
% \vfill 
% \hfill 
% \vspace{Xcm}   % Colocar * caso esteja no começo de uma página. Ex: \vspace*{...}
% \hspace{Xcm}

% Estilo de página
% \thispagestyle{<<nosso>>}
% \thispagestyle{empty}
% \thispagestyle{plain}  (só número, sem cabeço)
% https://www.overleaf.com/learn/latex/Headers_and_footers

% Compilador que permite usar fonte de sistema: xelatex, lualatex
% Compilador que não permite usar fonte de sistema: latex, pdflatex

% Definindo fontes
% \setmainfont{Times New Roman}  % Todo o texto
% \newfontfamily\avenir{Avenir}  % Contexto

\begingroup\thispagestyle{empty}\vspace*{.05\textheight} 

              \formular
              \Huge
              \noindent
              \textbf{Teogonia}
              \medskip  
              
              {\brabo\LARGE
              \noindent Hesíodo}
              \vspace{3cm}

              
              {\fontsize{30}{40}\selectfont \minion\small\noindent Christian Werner (\textit{introdução e tradução})}

              \noindent
              {\fontsize{30}{40}\selectfont\minion\small\noindent 2ª edição}

              \vfill

              \newfontfamily\timesnewroman{Times New Roman}
              {\noindent\fontsize{30}{40}\selectfont \timesnewroman hedra}

              \vspace{-0.5cm}
              {\selectfont\minion\small\noindent São Paulo \quad\the\year}

\endgroup
\pagebreak
	       % [folha de rosto]
% nothing			is level -3
% \book				is level -2
% \part				is level -1
% \chapter 			is level 0
% \section 			is level 1
% \subsection 		is level 2
% \subsubsection 	is level 3
% \paragraph 		is level 4
% \subparagraph 	is level 5
\setcounter{secnumdepth}{-2}
\setcounter{tocdepth}{0}

% \renewcommand{\contentsname}{Índex} 	% Trocar nome do sumário para 'Índex'
%\ifodd\thepage\relax\else\blankpage\fi 	% Verifica se página é par e coloca página branca
%\tableofcontents*

\pagebreak
\begingroup \footnotesize \parindent0pt \parskip 5pt \thispagestyle{empty} \vspace*{-0.5\textheight}\mbox{} \vfill
\baselineskip=.92\baselineskip
\textbf{Teogonia} (em grego \textit{theogonia}: \textit{theos} $=$ deus +
\textit{genea} $=$ origem) é um poema de 1022 versos hexâmetros datílicos que
descreve a origem e a genealogia dos deuses. Muito do que sabemos sobre os
antigos mitos gregos é graças a esse poema que, pela narração em primeira
pessoa do próprio poeta, sistematiza e organiza as histórias da criação do
mundo e do nascimento dos deuses, com ênfase especial a Zeus e às suas façanhas até chegar ao poder. A invocação das Musas, filhas da Memória, pelo aedo Hesíodo é o que lhe dá o conhecimento das coisas passadas e presentes e a possibilidade de cantar em celebração da imortalidade dos deuses; e é a partir daí que são narradas as peripécias que constituem o surgimento do universo e de seus deuses primordiais.  

\textbf{Hesíodo} foi um poeta grego arcaico e, assim como ocorre com Homero, não é possível provar que ele tenha realmente existido. Segundo certa tradição, porém, teria vivido por volta dos anos 750 e 650 a.C.  Supõe-se, a partir de passagens do poema \textit{Trabalhos e dias}, que o pai de Hesíodo tenha nascido no litoral da Ásia e viajado até a Beócia, para instalar-se num vilarejo chamado Ascra, onde teria nascido o poeta; supõe-se também que ele tenha tido um irmão, Perses, que teria tentado se apropriar, por meios ilegais, de uma parte maior da herança paterna do que a que lhe cabia, exigindo ainda ajuda de Hesíodo. Acredita-se que a única viagem que Hesíodo teria realizado tenha sido a Cálcis, com o objetivo de participar dos jogos funerários em honra de Anfidamas, dos quais teria sido o ganhador e recebido um tripé pelo desempenho na competição de cantos. Apenas três das obras atribuídas a Hesíodo resistiram ao tempo e chegaram às nossas mãos: são elas os \textit{Trabalhos e dias}, a \textit{Teogonia} e \textit{O escudo de Héracles}.


\textbf{Christian Werner} é professor livre-docente de língua e literatura grega na Faculdade de Letras da Universidade de São Paulo (\textsc{usp}). Publicou, entre outros, \textit{Duas tragédias gregas: Hécuba e Troianas} (Martins Fontes, 2004).




\endgroup

\pagebreak
{\begingroup\mbox{}\pagestyle{empty}
\pagestyle{empty} 
% \renewcommand{\contentsname}{Índex} 	% Trocar nome do sumário para 'Índex'
%\ifodd\thepage\relax\else\blankpage\fi 	% Verifica se página é par e coloca página branca
\addtocontents{toc}{\protect\thispagestyle{empty}}
\tableofcontents*\clearpage\endgroup}

\begingroup\widowpenalties = 3 10000 9000 6000
\chapter*{Introdução\smallskip\subtitulo{A linguagem e a narrativa\break desvelam o cosmo}}
\addcontentsline{toc}{chapter}{Introdução, \textit{por Christian Werner}}

\begin{flushright}
\textsc{christian werner}
\end{flushright}

\setlength{\epigraphwidth}{.65\textwidth}
\begin{epigraphs} 
\qitem{
Trepava ser o mais honesto de todos, ou o mais
danado, no tremeluz, conforme as quantas. Soava
no que falava, artes que falava, diferente
na autoridade, mas com uma autoridade muito veloz.}
{\textsc{joão guimarães rosa},\\
\textit{Grande sertão: Veredas}}
\end{epigraphs}

\noindent{}\textit{O mais honesto ou o mais danado} é como Riobaldo descreve Zé
Bebelo na parte inicial do romance. Trata-se de uma figura que ele
admira, pelas formas de sua astúcia e autoridade moderna, \textit{rápida}, em
contraste com aquela \textit{lenta} e arcaica de Joca Ramiro, o grande chefe
dos jagunços. Tal autoridade, porém, paulatinamente se revela fazer jus
ao mal que ecoa no nome Zé Bebelo, \textit{bellum}, ``guerra'', e belzebu,
cuja negatividade é contrária à justiça moderna desdobrada no discurso
da personagem. Dito de outra forma, Zé Bebelo move-se entre o arcaico e
o moderno, o mítico e o racional.\footnote{Minha interpretação de Zé
  Bebelo se apoia em Rosenfield (2006).}

\textit{Mutatis mutandis} pode-se dizer o mesmo do poema de Hesíodo e de
sua personagem central, Zeus. Também esse poema explora os meandros da
justiça e da soberania como idealizações dependentes da astúcia, \textit{mētis}, essa qualidade ou habilidade essencialmente múltipla e
imanente, focada no aqui e agora da experiência sempre cambiante.\footnote{``No
tremeluz\ldots{} muito veloz''.} E assim como em Rosa, mito e razão não
se revelam formas de pensamento opostas ou incompatíveis, em particular,
pela modo como, em Hesíodo, a linguagem e a narrativa desvelam o cosmo.

\section{Hesíodo: o poeta e sua época\protect\footnote{\MakeUppercase{A}baixo, procurei manter a
  indicação bibliográfica reduzida ao mínimo, sobretudo quando me apoio
  pontualmente no argumento de determinado autor. \MakeUppercase{P}ara uma bibliografia
  mais ampla, cf. a mencionada no final.}}

Diferentemente dos poemas de Homero, os de Hesíodo se associam, eles
próprios, a um poeta e a um lugar como espaço de sua gestação: o poeta
da \textit{Teogonia} se nomeia e se vincula ao entorno do monte Hélicon na
Beócia (22--23). Trata-se de uma região no centro da Grécia, cuja cidade
principal, no passado e hoje, é Tebas. Suas montanhas principais são o
Parnasso, junto a Delfos, o Citéron --- onde Édipo foi exposto --- e o
Hélicon, com sua fonte Hipocrene, ``Fonte do Cavalo'', estes dois
mencionados no início do poema em associação às Musas (1--8).

Indicações temporais, porém, estão virtualmente ausentes do poema, o que
permite reconstituições diversas, todas elas imprecisas e sujeitas a
críticas. Uma delas, feita pelos antigos, é associar Hesíodo a outros
poetas da tradição hexamétrica grega arcaica --- Museu, Orfeu e,
sobretudo, Homero --- e estabelecer uma cronologia relativa, para o que
um critério poderia ser a autoridade: a maior seria a do poeta mais
velho (Koning 2010). Modernamente, a cronologia relativa reaparece
fundamentada no exame linguístico-estatístico do \textit{corpus}
hexamétrico restante (Andersen \& Haug 2012). Assim, Janko (1982), um
trabalho seminal, definiu como sequência cronológica de composição
\textit{Ilíada}, \textit{Odisseia}, \textit{Teogonia} e \textit{Trabalhos e
dias}.

\textls[-10]{Outra forma de contextualizar os poemas no tempo está ligado a
tentativas de reconstituir os séculos \textsc{viii} e \textsc{vii} a.C. 
como a época na qual se sedimentaram uma série de fenômenos culturais e 
políticos que acabaram por definir as sociedades gregas, em especial o 
surgimento da \textit{polis} como principal organização política e social, 
o templo de Apolo e seu oráculo em Delfos como um santuário de todos os gregos,
festivais de cunho religioso, como os Jogos Olímpicos, que passaram a
atrair participantes de uma ampla gama de territórios grego, a
reintrodução da escrita, o culto aos heróis etc. Trata-se de fenômenos
que definem o que Gregory Nagy (1999), na esteira de Snodgrass (1971),
chama de pan-helenismo,\footnote{Moraes (2019, p.\,12) entende ``o
  pan-helenismo como um discurso político capaz de prover uma sensação
  de pertencimento às comunidades de língua grega, baseado em critérios
  simultaneamente culturais e políticos de caráter aglutinador e que
  atuou na produção e reprodução da identidade helênica''.} e do qual
faria parte a produção e recepção da \textit{Teogonia}.}\looseness=-1

A introdução paulatina, com adaptações, nos territórios gregos, nos
quais se falavam dialetos diversos, de um alfabeto de origem fenícia em
torno do século \textsc{viii} a.C. foi um dos responsáveis pela modificação
gradual de diversas práticas sociais, entre elas, a produção e recepção
de poesia. Os poemas podiam ser cantados ou recitados, e, quando
cantados por um coro (o que não é o caso da poesia épica como a
\textit{Teogonia}), esse produzia figuras de dança. É exatamente assim que
as Musas são representadas no início do poema (1--11), em contraste com o
cantor individual Hesíodo. Composições corais eram apresentadas em
ocasiões específicas, muitas vinculadas ao calendário religioso de
determinadas localidades. Quanto à poesia hesiódica, o contexto de
performance é desconhecido por nós. De fato, como se verá mais abaixo
por meio do nome de Hesíodo, é necessário tratar com cuidado os
elementos que parecem atar o poema à realidade.\looseness=-1

\section{Estrutura do poema}

Há diferentes maneiras de conceber a estrutura da \textit{Teogonia}. A de
Thalmann (1984, p.\,38--39), traduzida abaixo, tem a vantagem de
identificar em sua sequência de partes singulares e mais ou menos
independentes, uma moldura em anel (ainda que incompleta), marcada pela
repetição das letras em ordem inversa.

\begin{itemize}
\item \textsc{a.} 1--115\quad \textit{Proêmio}
\item \textsc{b.} 116--210\quad \textit{Os primeiros deuses e os Titãs; primeiro estágio do mito de sucessão}
\item \textsc{c.} 211--32\quad \textit{Prole de Noite}
\item \textsc{d-1.} 233--336\quad \textit{Prole de Mar, incluindo as Nereidas}
\item \textsc{d-2.} 337--70\quad \textit{Prole de Oceano, incluindo as Oceânides}
\item \textsc{e-1.} 371--403\quad \textit{Uniões de outros Titãs e o episódio de Estige}
\item \textsc{e-2.} 404--52\quad \textit{Uniões de outros Titãs e o episódio de Hécate}
\item \textsc{f-1.} 453--506\quad \textit{União de Reia e Crono; segundo estágio do mito de sucessão}
\item \textsc{g.} 507--616\quad \textit{Prole de Jápeto e o episódio de Prometeu}
\item \textsc{f-2.} 617--720\quad \textit{Batalha com os Titãs (Titanomaquia) e fim do segundo estágio}
\item \textsc{c.} 721--819\quad \textit{Descrição do Tártaro}
\item \textsc{b.} 820--80\quad \textit{Batalha com Tifeu, o último inimigo de Zeus}
\item \textsc{a.} 881--929\,(?)\quad \textit{Zeus torna-se rei e divide as honras; união com Astúcia e demais}\footnote{O ponto de interrogação indica que não há consenso que verso nos manuscritos do poema marcaria o fim da composição (Kelly 2007).} 
\end{itemize}

A estrutura em anel, na qual se retomam léxico e temas, é uma forma
retórica assaz trivial na poesia grega. Em Homero, por exemplo, o final
de um discurso pode retomar o tópico do início, indicando ao receptor
que o discurso está chegando ao fim.

Repare-se que proêmio do poema é longo se comparado com o início de
outras composições hexamétricas arcaicas identificado como tal. Nele,
\textit{grosso modo}, o aedo costuma estabelecer algum tipo de vínculo com
a Musa, a divindade da qual depende a performance de seu canto, e a
definir o tema geral do poema. Isto \textit{também} é feito na
\textit{Teogonia}, mas, de um modo bastante sofisticado, o tema principal
do poema --- a autoridade as ações de Zeus --- são interligadas àquelas
das Musas e do aedo.

Com isso, o corte entre o chamado proêmio e o restante do poema é bem
menos abrupto que aquele que se verifica na \textit{Ilíada} e na
\textit{Odisseia}: no proêmio nos podemos ver Zeus sendo celebrado como
deus supremo pelas Musas, e isso, de fato, é o que faz o poema como um
todo, pois, embora Zeus não seja o \textit{primeiro} deus, do ponto de
vista da sequência do poema, é como se ele fosse, já que nenhum deus é
tão poderoso ou merece ser tão celebrado como ele.

\section{Hino às Musas: o proêmio do poema}

Por certo é significativo que o narrador da \textit{Teogonia} --- ao
contrário do narrador dos poemas homéricos --- se nomeie no início do
poema\footnote{Mas apenas uma única vez.} no momento mesmo em que é narrado seu
encontro singular com a entidade religiosa tradicional que confere
autoridade a seu canto e garante a precisão de seu conteúdo, as Musas.
Os primeiros 115 versos do poema compõem um proêmio, no qual se celebram
essas divindades (1--103) e se demarca explicitamente o conteúdo do canto
a seguir (104--15). O trecho se assemelha a uma forma poético-religiosa
tradicional em várias sociedades antigas, o canto que celebra as
honrarias ou áreas de atuação, \textit{timē} no singular, de um deus e
que, mais tarde, passou a ser denominado ``hino'', \textit{humnos}.\footnote{O substantivo (que aparece uma vez na
  \textit{Odisseia}) e o verbo cognato, diversas vezes na
  \textit{Teogonia}, que traduzi por ``louvar'' ou ``cantar''
  (Torrano traduz consistentemente pelo neologismo ``hinear''), não
  parecem ser associados primordialmente a deuses nesses textos.} Com
efeito, tal tipo de canto ganhou na Grécia Antiga, em algum momento, uma
versão narrativa no contexto da tradição hexamétrica: são os hinos
homéricos longos ou médios (Ribeiro Antunes \textit{et al.} 2011; Antunes
2015). O que há de muito particular nesse hino da \textit{Teogonia},
porém, é que somente os gregos conheceram essas divindades coletivas
responsáveis por uma esfera cultural que podemos chamar de poesia, mas
que envolvia também música e dança.

Ao celebrar as Musas antes de apresentar o canto que elas propiciam, ou
seja, a cosmogonia e teogonia que começam no verso 116, o poeta também
fala da relação que há entre ele próprio e essas divindades, pois o
valor de verdade, ou seja, a autoridade do canto que apresenta depende
dessa relação. Como pode um mortal falar de eventos pretensamente reais
que não presenciou --- o surgimento do mundo conhecido e de todas as
divindades, bem como dos mortais que com elas dormiram --- se não
apresentar e fundamentar sua relação com certa autoridade transcendente,
já que não há uma tradição textual canônica e uniforme independente do
poema? Nesse sentido, não é mais possível, para nós, saber com certeza
se algum dia houve um poeta chamado Hesíodo e que foi o autor do poema
que conhecemos, ou se \textit{Hesíodo} teria sido uma autoridade
\textit{mítica} inseparável de certa tradição poética e que seria
reencarnada a cada apresentação do poema, um pouco como o ator que
reencarnaria, com uma máscara ritual, nas apresentações teatrais
atenienses no século \textsc{v} a.C., as figuras tradicionais do mito (Nagy
1990). Nesse diapasão, a iniciação no canto, conduzida pelas Musas, pela
qual teria passado o poeta Hesíodo (9--34) também faria parte desse
contexto mítico.

Isso pode ser exemplificado pelo nome \textit{Hesíodo}. Por certo não é
possível \textit{provar} que não tenha existido uma figura histórica com
esse nome responsável pela composição de um ou mais poemas associados ao
nome (Cingano 2009). Além disso, a etimologia do nome não é segura e tem
sido interpretada de diferentes modos (Most 2006, p.\,\textsc{xiv--xvi}).
Meier-Brügger (1990), por exemplo, rediscutiu todas as hipóteses e
defendeu que \textit{Hesíodo} significa ``aquele que se compraz com
caminhos'', o que pode ser interpretado metapoeticamente. Contudo, o
contexto imediato da única vez em que o nome é mencionado no poema
parece indicar que a expressão \textit{ossan hieisai}, ``voz emitindo'',
repetida diversas vezes no proêmio (10, 43, 65 e 67), seria uma glosa de
\textit{Hesíodo} (Nagy 1990, p.\,47--48; Vergados 2020, p.\,43--46), um exemplo
entre vários do que Vergados (2020) define como o pensamento etimológico do autor.

Outro elemento saliente no proêmio é Zeus. Na verdade, como soberano dos
deuses e dos homens, ou seja, como deus responsável pela estrutura
sociopolítica final do cosmo e, dessa forma, também pela manutenção de
sua dimensão física, não é raro Zeus desempenhar algum papel nos hinos
aos deuses que conhecemos, sobretudo, os hinos homéricos maiores. Sua
presença no proêmio da \textit{Teogonia}, porém, é ubíqua, e não apenas
como pai das Musas e seu público primeiro e principal,\footnote{Não nessa ordem
na sequência do poema.} mas também como o deus que, em vista do que
representa, é particularmente associado ao poder político exercido pelos
reis, \textit{basileus} no singular, no mundo humano. Não surpreende,
assim, que, no final do proêmio, as Musas sejam apresentadas como
sombremaneira ligadas não só aos poetas (94--103), mas também aos reis
(80--93), uma figura que, no contexto hesiódico, não representa um
monarca com amplos poderes, mas uma figura que, na esfera pública, age
sobretudo na função de um juiz (Gagarin 1992). O tipo de poder real
exercido por Zeus no poema --- o poder é absoluto e hereditário --- não é
homólogo àquele dos líderes políticos da época. O rei humano é antes de
tudo um aristocrata com prestígio local que participa da administração
da justiça. Que reis e poetas, porém, são figuras dissociáveis, isso
fica claro no destaque dado a Apolo nessa passagem; de qualquer forma, o
proêmio sugere que, entre os homens, poetas são figuras bastante
próximas dos reis (Laks 1996).\looseness=-1

\section{Abismo, «Khaos», e o início do cosmo}

Para chegar a Zeus e o modo como esse controla o cosmo, o tema central
do poema, Hesíodo inicia do começo, ou seja, de Abismo (116), um espaço
vazio cuja delimitação primeira surge na sequência, Terra, \textit{Gaia}.
Não se trata, porém, da Terra tal qual a conhecemos, mas de um espaço
físico ainda descaracterizado, ou melhor, marcado pela sua função
futura, ser o espaço de atuação dos deuses responsáveis pelo equilíbrio
cósmico, que vai, imageticamente, do Olimpo ao ínfero Tártaro. Antes de
Terra começar a gerar suas formas particulares, Montanhas e Mar, e das
divindades aparecerem, duas coisas fundamentais são necessárias, a
presença de Eros (120), o desejo sem o qual não há geração, e as
potências que permitem a sucessão temporal, Escuridão, Noite, Éter e Dia
(123--25).

Todos os deuses descendem de duas linhagens principais, a de Abismo e a
de Terra, mas entre elas não há nenhuma união. Os descendentes de Abismo
são, em sua maioria, potências cuja essência é negativa, como Noite, Morte,
Agonia etc.; várias delas, além disso, expressam ações e emoções que
permeiam os eventos violentos narrados na sucessão de gerações da
linhagem de Terra, como Briga, Disputas, Batalhas etc. A linhagem de
Abismo, portanto, através da descendência de Noite, \textit{Nux}, e Briga, \textit{Eris}, revela que a separação entre Terra e Abismo nunca é total\footnote{As ações e emoções representadas como descendência de Abismo são
executadas ou sentidas pelos descendentes de Terra.} e assim ilustra uma
constante no poema: o encadeamento das linhagens entre si e também delas
com as histórias que se sucedem mostram um poema no qual os catálogos
dos deuses nascentes e as narrativas nas quais os deuses estão
envolvidos não devem ser separados. Trata-se de uma articulação de
imagens, ações e ideias que pressupõe uma temporalidade própria --- ou
melhor, diversas temporalidades (Loney 2018) --- que revela uma mescla
entre o tempo da narrativa genealógica, o tempo da sucessão de um
deus-rei para o seguinte e o tempo da narração. É a partir disso que o
leitor deve entender, por exemplo, que um deus às vezes já apareça como
personagem no poema antes de o narrador mencionar seu nascimento
propriamente dito.

\section{Genealogias divinas}

No poema, teogonia e cosmogonia são inseparáveis à medida que o espaço
se constitui e as genealogias divinas se sucedem. As divindades que
passam pelo poema --- mais de 300 --- são de diversos tipos no que diz
respeito a cultos e mitos (West 1966): 

\begin{enumerate}
\item Os deuses do panteão --- sobretudo os Olímpicos, como Zeus, Apolo, Atena e Ártemis ---, cultuados pela Hélade mas de uma forma mais específica que aquela com que aparecem
no poema (por exemplo, vinculados a certo lugar ou templo específicos);

\item Deuses presentes nas histórias míticas, mas que provavelmente nunca
foram exatamente objetos de culto, como Atlas e, enquanto coletividade, provavelmente os Titãs; 

\item Partes do cosmo divinizados, como Terra, Noite, Montanhas; alguns eram cultuados;

\item Personificações. Elementos que,
para nós, são abstratos, mas não o eram para os gregos; 

\item Aqueles sobre os quais nada sabemos fora de Hesíodo, ou seja, podem ser parte de
um recurso típico dessa tradição, que permitiria a \textit{criação} de
divindades para compor catálogos ou expressar caraterísticas de uma
linhagem. Algo que não deve ser confundido com ficção nem com inovação.
\end{enumerate}

Essa tipologia, porém, não deve ser tomada como algo estático e
invariável. Eros, por exemplo, pode ser pensado como um deus de culto ou
não. Com efeito, o poema não pode ser om retrato de uma estrutura
religiosa fixa, pois essa não existia. Pelo contrário, ele e a tradição
da qual faz parte deveriam ser antes pensados como uma tentativa de
enquadrar, de dar certa forma a uma vivência religiosa que é
essencialmente plural no tempo e no espaço. O lance astuto incorporado
pela tradição --- ou pelo autor do poema --- é justamente procurar
apresentar como um sistema obviamente fixo algo que é necessariamente
variável. A isso está ligado seu sucesso pan-helênico.

\section{Afrodite}

Um dos modos do poeta expressar o que cada divindade tem de específico é
a derivação do seu nome e de seus epítetos. Uma das construções mais
desenvolvidas que exemplificam é a que trata do nascimento, a partir do
esperma de Céu, \textit{Ouranos}, de Afrodite (192--200):

\begin{verse}
\textit{
{[}\ldots{}{]} primeiro da numinosa Citera achegou-se,\\
e então de lá atingiu o oceânico Chipre.\\
E saiu a respeitada, bela deusa, e grama em volta\\
crescia sob os pés esbeltos: a ela Afrodite\\ %\num{195}
espumogênita e Citereia bela-coroa\\
chamam deuses e varões, porque na espuma\footnote{\textit{Aphros}.}\\
foi criada; Citereia, pois alcançou Citera;\\
cipriogênita, pois nasceu em Chipre cercado-de-\qb{}-mar;\\
e ama-sorriso,\footnote{\textit{Philommeidea}.} pois da genitália\footnote{\textit{Mēdōn}.} surgiu.
}
\end{verse}

Ora, à medida que o narrador, devido ao encontro que teve com as Musas,
garante estar falando a verdade, ao mostrar, por meio do próprio nome ---
aceito em toda a Hélade --- do deus que as histórias que ele conta como
que estão inscritas na identidade verbal mesma do deus, ele confronta
histórias de outras tradições que não revelariam o mesmo
conhecimento profundo e inequívoco da realidade por ele dominado. A
filiação da Afrodite de Homero --- ela é filha de Zeus e de Dione --- como
que sucumbe às \textit{provas} dadas na \textit{Teogonia}, cuja lógica só tem
espaço para uma Afrodite, a filha de Céu.

O surgimento de Afrodite é um dos nascimentos que marcam o fim da
supremacia de Céu sobre o cosmo incipiente, ou seja, um momento de crise
que antecede o equilíbrio cosmológico verificado ainda hoje pelos
ouvintes do poema no seu cotidiano. Depois de Céu, também Crono, seu
herdeiro como deus patriarca detentor do poder soberano, é derrotado;
somente Zeus, como rei dos deuses e homens, sempre tem sucesso nos
conflitos que enfrenta. No século \textsc{xx} percebeu-se que o chamado \textit{mito de
sucessão}, fundamental para o entendimento do poema, composto por três
gerações de deuses e seus \textit{patriarcas}, Céu, Crono e Zeus, e os
conflitos principais que cada uma enfrenta --- a castração de Céu, o
nascimento de Zeus possibilitado pelo truque da pedra aplicado por Reia
e o combate de Zeus contra os Titãs e, posteriormente, Tifeu --- guarda
semelhanças em graus diversos com mitos equivalentes transmitidos por
outras culturas antigas do Oriente, como a babilônia e hurro-hitita
(Rutherford 2009, Kelly 2019). O intercâmbio verificado entre essas
culturas problematiza, assim, a origem necessariamente nebulosa mas
certamente não helenocêntrica do poema, ou pelo menos de parte dele. A
maioria dos intérpretes concorda, hoje, que, de Homero e Hesíodo a
Platão, não deve ter havido nada parecido com um \textit{milagre grego},
ainda que não possamos sempre rastrear com precisão como teriam ocorrido
os diversos casos de intercâmbio entre as culturas orientais e a grega
(Burkert 1992, West 1997, Rutherford 2009, Haubold 2013).

\section{Astúcia «versus» força e criaturas prodigiosas}

Os eventos do mito de sucessão são permeados por um par de opostos
complementares fundamental na mitologia, vale dizer, na cultura grega,
\textit{astúcia} e \textit{força} (Detienne \& Vernant 2008). É ele, por exemplo,
que subjaz à oposição entre os heróis máximos dos dois poemas homéricos,
Odisseu e Aquiles, o primeiro, o astuto por excelência, o segundo, o
herói grego mais temido pelos troianos devido à sua força. Também é essa
oposição que mostra, em diversas fábulas, animais mais fracos
fisicamente derrotando os mais fortes ou velozes. No caso da
\textit{Teogonia}, desde o início a astúcia tem a particularidade de ser
uma característica essencialmente feminina. É de Terra o plano ardiloso
que permite a derrota de Céu; Farsa, \textit{Apatē}, é filha de Noite; e
Astúcia, \textit{Mētis} --- além de Persuasão, \textit{Peithō} ---, é uma das
dezenas de filhas de Oceano. No mito de sucesso, a divindade que usar
apenas uma das qualidades ou a usar de modo desproporcional em relação à
outra sempre sucumbe a adversários que combinam as duas de forma mais
eficaz.

Por outro lado, é a Terra que está ligada à geração dos seres
tradicionalmente chamados de monstros (270--335), Équidna, Hidra de
Lerna, Leão de Nemeia, Medusa, Pégaso, Cérbero, Quimera etc. O que
caracteriza tais criaturas como uma coletividade é que elas não se
assemelham nem aos deuses, nem aos homens, nem aos animais, mas são sempre
seres estranhamente mistos, dotados --- assim como sua ancestral primeira
--- de um inominável, enorme poder, algo que faz deles seres incapazes de
serem conquistados pelos mortais, ou seja, ``impossíveis'',
\textit{amêkhanos}. Nesse sentido, e tendo em vista a história do termo
\textit{monstro}, Zanon (2018) mostrou ser mais apropriado chamar essas
criaturas de \textit{prodígios}. Os únicos que as superaram foram certos
heróis, homens muito superiores em força e astúcia que os homens de hoje
e que, além disso, foram auxiliados por deuses.

\textls[-10]{Pela lógica da narrativa, as criaturas prodigiosas parecem ser uma
espécie de tentativa mal sucedida de continuar o desenvolvimento do
cosmo (Clay 2003), já que, em sua maioria, não têm função alguma salvo
contribuírem para a fama do herói que os derrotou. Além disso, por meio
delas se mostra que, assim como, no plano humano, mortais comuns se
opõem a heróis, no divino, deuses se opõem a monstros. Além disso, como
notou Pucci (2009), alguns deuses da geração de Zeus utilizam, eles
próprios, uma criatura para obter determinado fim pessoal, o que
sinaliza que o equilíbrio cósmico continua instável. Os monstros
presentes no poema indicam, para o leitor do presente, que, por ora, a
fertilidade feminina consubstanciada em Terra e que, na sua forma mais
frenética e disforme, gerou tais criaturas --- veja que nos versos 319 e
326 não fica claro quem é a mãe do respectivo monstro, o que parece
acentuar o desregramento ---, foi dominada e regrada por um elemento
masculino, mas esse não será, necessariamente, o fim da história. No
século \textsc{xx} e \textsc{xxi}, \textit{monstros} continuam a assombrar a fantasia humana,
seja na forma de ameaças espaciais ou da guerra atômica, seja como
consequência da forma com que o homem trata o planeta em que habita ---
ou seja, novamente é Gaia quem parece deter a palavra final e, desta
vez, inalienável.}

\section{Estige e Hécate}

Como que a contrabalançar o peso negativo dessas criaturas, na sequência
nascem duas coletividades benfazejas, os Rios e as Oceaninas (337--70),
e, entre essas últimas, destacam-se duas figuras femininas, Estige e
Hécate (383--452). Ambas aparecem na narrativa, de forma anacrônica, para
serem cooptadas por Zeus, cujo nascimento ainda não ocorreu. Isso se
deve, como já foi mencionado acima, pela lógica própria do poema. As
duas divindades femininas não só se opõem à negatividade essencialmente
feminina dos monstros, mas também preparam a narrativa por vir. Estige
ela mesma e seus filhos antecipam a vitória cósmica de Zeus e o novo
equilíbrio que ele vai instaurar e manter. Esse equilíbrio, porém, não é
resultado de uma tábula rasa, mas dá continuidade ao que já estivera
equilibrado durante a supremacia de Crono.\looseness=-1

Hécate, por sua vez, é a deusa que permite a primeira irrupção mais
substancial dos homens no poema. Como o objetivo do poema é revelar a
ordem do cosmo e as prerrogativas dos deuses e celebrá-los, é esperada a
posição absolutamente marginal que o gênero humano ocupa no poema (Clay
2003). Os homens e seu modo de vida são os protagonistas de outro poema
atribuído a Hesíodo, \textit{Trabalhos e dias}. Isso não significa, porém,
que, do ponto de vista dos próprios deuses, ou seja, em última análise,
da própria \textit{Teogonia}, as características da fronteira que separa
deuses e homens não sejam relevantes. Essas aparecem com clareza em dois
episódios que emolduram o nascimento de Zeus, a celebração de Hécate e a
história de Prometeu.

Se aos heróis --- esses humanos mortais que, vale assinalar, estão no
meio do caminho entre deuses e homens --- é dada uma razão de ser durante
o catálogo de monstros, a relação entre Zeus e Hécate, num momento do
poema em que se enfatiza o equilíbrio cósmico resultante das
responsabilidades diversas atribuídas a cada deus, revela que esse
equilíbrio é indissociável da presença, na terra, dos homens. Dito de
outro modo: para pensar-se, figurar-se o modo como os deuses são no
mundo por meio da sequência de eventos que levou à ordem presente,
utiliza-se também um retrato simplificado e razoavelmente genérico das
práticas cultuais humanas. Deuses, cosmo e homens não existem um sem o
outro. O trecho dedicado a Hécate, porém, revela também que a vida
humana, mais que marcada por certo equilíbrio, é permeada pelo
imponderável: por mais que os homens propiciem os deuses, nada garante
que serão auxiliados por eles.

Não possuímos nenhum testemunho histórico independente da
\textit{Teogonia} que aponte para a importância cultual, mesmo que apenas
local, de Hécate sugerida pelo destaque que lhe dado no poema. Isso é um
forte indício de que comentadores como Clay (2003) estão corretos ao
defender que a figura dessa deusa é usada para se falar de Zeus e da
relação entre os homens e os deuses inaugurada por ele. Menos certa é a
relação entre o nome de Hécate, a maneira como o poeta se refere ao seu
modo de atuação --- ``se ela quiser'' etc. --- e o acaso.

\section{Zeus e Prometeu}

O nascimento de Zeus narrado logo depois (453--91) é o evento que permite
a queda de Crono e a ascensão do terceiro soberano dos deuses. A astúcia
de Terra é a responsável pela castração de Céu, a libertação, ou
nascimento, de seus filhos, os Titãs, e a tomada de
poder por parte do filho mais novo, Crono. De forma homóloga, é a
astúcia da esposa de Crono, Reia, auxiliada pelos conselhos de Céu e
Terra, que permite que seus filhos vejam a luz do Sol e Zeus destrone o
pai. Desta vez, porém, há uma verdadeira competição entre astutos: como
todo bom rei, Crono é previdente, e, ao aprender parcialmente com o
erro de seu pai, decide engolir todos os filhos \textit{após} esses serem
paridos por sua esposa, com o que, porém, ainda exercita de uma forma
arbitrária sua força. Reia, porém, o ludibria no nascimento de Zeus, de
sorte que esse, por meio de uma série de manobras contadas rapidamente
no poema, pode ocupar a regência do cosmo. Ainda que, pelo menos em
parte, nesse momento da narrativa Zeus não seja representado como um
agente deliberando sozinho, seu poder é de pronto ligado às duas esferas
mencionadas acima, astúcia e força. Por enquanto, sua astúcia ainda é
aquela da mãe e da avó; sua força, porém, está ligada ao seu primeiro
ato como soberano --- do ponto de vista da lógica da narrativa: a
libertação dos Ciclopes (501--6), aqueles que lhe fornecerão os raios e o
trovão, atributos que, por certo, funcionam como armas mas também são
simbólicos, já que apontam para sua ligação com o céu.\looseness=-1

O primeiro conflito resolvido por Zeus, porém, envolve a astúcia
(507--616). Trata-se do momento em que deuses e homens se distinguiram,
se separaram em definitivo por ocasião de um banquete festivo para o
qual Prometeu separou a carne de um boi. Marcam esse evento a origem do
sacrifício, a conquista do fogo e a criação da mulher humana. O texto
não procura descrever detalhadamente a linhagem humana que não dominava
o fogo, ainda compartilhava da companhia dos deuses e não conhecia a
reprodução sexual; isso é feito, sob viés distinto, em \textit{Trabalhos e
dias}. Todavia, como o narrador deixa claro que Zeus aceita a repartição
da carne do boi feita por Prometeu para o banquete porque ele tinha em
mente males destinados \textit{aos homens mortais} (551--52), podemos supor
que, nesse momento de sua regência, quando Zeus ainda precisa consolidar
seu poder, os homens,\footnote{Os versos 50 e 185--87 talvez sugiram que esses
fossem guerreiros gigantes nascidos da terra, figuras que conhecemos de
outros relatos.} em conluio com Prometeu, representam uma ameaça que
precisa ser dominada antes que seja tarde demais. A previdência é um
atributo indispensável do soberano que quiser manter seu poder. Ao
contrário de Zeus, que antecipa o movimento do provável inimigo, Crono
falhou em sua tática de engolir os filhos: bastou que um escapasse para
ele ser destronado.

Outro momento fundamental da história de Prometeu é a criação da
primeira mulher. Ao contrário do que ocorre em \textit{Trabalhos e dias},
aqui o narrador não informa seu nome, que lá é Pandora. Como em todos os
eventos que marcam o episódio de Prometeu, bem e mal estão
indissociavelmente ligados (Vernant 1992 e 2002): nessa etiologia do
sacrifício, os ossos, que não podem ser digeridos (mal), são encobertos
pela gordura que solta delicioso aroma (bem), ao passo que a carne (bem)
é disfarçada sob o repelente estômago (mal). Assinale-se que o disfarce
--- e, consequentemente, a habilidade de reconhecer o que está disfarçado
--- também faz parte do domínio da astúcia: se Prometeu é astuto, Zeus o
é em ainda mais alto grau. Os ossos, bem como o aroma da gordura
queimada, são, por outro lado, sinais da imortalidade divina (bem), ao
passo que a carne deliciosa, o alimento perecível, comida pelos homens
aponta para sua mortalidade (mal). A a adoção da carne em sua dieta,
escondida no estômago do boi, deixa claro que os homens são escravos de
seu próprio estômago e precisam satisfazê-lo se não quiserem perecer.

No caso da Mulher, ela é dada aos homens em troca do fogo: ao passo que
o fogo permite que os homens sejam civilizados e não comam carne crua, a
mulher terá que ser por eles alimentada, caso queiram sobreviver por
intermédio de um herdeiro. De fato, fogo e mulher precisam ser
constantemente alimentados para que o homem não pereça. O sacrifício, o
fogo e a bela mulher, portanto, indicam que há elementos que apontam
para uma presença do divino no centro da vida humana, mas eles são tão
tênues como a fumaça que sobe do sacrifício para o céu e tão artificiais
quanto os enfeites da coroa da primeira mulher, contra a qual o homem
não tem defesa alguma.

\section{Titanomaquia}

Após essa separação entre deuses e homens levada a cabo por Zeus graças
à astúcia, a separação seguinte, entre os deuses da geração de seu pai,
os Titãs, e os da sua própria, os Olímpicos, é conseguida devido à
supremacia alcançada sobretudo por meio da força. O episódio conhecido
como Titanomaquia (617--720) mostra que o cosmo ficou mais complexo que
quando sobre ele regia Céu, pois se, para vencer seu pai, num primeiro
momento, Zeus contou com pelo menos dois ardis arquitetados pela mãe e
pela avó --- entregar a Crono uma pedra no lugar do bebê Zeus e,
posteriormente, fazê-lo vomitar todos os irmãos de Zeus que com ele por
fim lutariam contra os deuses mais velhos ---, num segundo momento, a
astúcia deixa de ser suficiente.

É de novo Terra quem aconselha ao neto libertar aqueles que haviam sido
presos por Céu e assim mantidos por Crono abaixo da terra, os
Cem-Braços. Trata-se de uma força descomunal que os dois soberanos
anteriores acharam por bem simplesmente manter paralisada, paralisia
homóloga àquela que tentaram, sem sucesso, implementar contra seus
filhos. Zeus, porém, consegue convencê-los a serem seus aliados e eles
se mostram decisivos no combate contra os Titãs, gratos por serem
trazidos de volta à luz.

Luz e trevas: essa polaridade marca toda a Titanomaquia, pois os Titãs,
uma vez vencidos, passam a ocupar o espaço subterrâneo onde antes
estiveram os Cem-Braços que, porém, agora tem uma honra, uma função no
cosmo, a de serem os eternos guardas dos deuses outrora poderosos, os
Titãs. Essa polaridade, ademais, também prepara o episódio seguinte,
pois o esforço de Zeus para vencer os Titãs como que traz o cosmo de
volta ao seu estado inicial: terra, céu, mar e Tártaro, todos os espaços
são atingidos pelo fogo dos raios de Zeus, o que representa uma
recriação do mundo por meio da força. Não é por acaso que Abismo volta à
cena (700 e 814) e que as imagens e sons desse conflito cataclísmico
sejam amplificadas para o ouvinte por meio de uma imagem que remete à
união primordial entre Terra e Céu (700--5).\looseness=-1

Uma vez finalizada a guerra, o narrador nos narra, pela primeira vez,
como é a geografia das terras ínferas (721--819). Não que antes nada lá
houvesse. Com o aprisionamento dos Titãs, porém, à essa parte do cosmo é
conferida sua estabilidade e Zeus pode finalmente aparecer como o
organizador último de todos os espaços. É por essa razão que de deuses
como Sono e Morte e Noite e Dia, cujas funções cósmicas os ligam ao
Tártaro, finalmente se fala mais longamente, uma vez mais se mostrando
de que forma polos positivos e negativos da realidade estão
interligados. É precisamente por isso que também nesse momento do poema
descreve-se a função de Estige, ligada a uma jura divina que, quando
quebrada por um deus, o leva a uma morte virtual por dez anos. A ligação
entre Estige e Zeus mostra que também o juramento --- uma instituição
social fundamental também entre os homens --- é instituído pelo rei dos
deuses e homens para bem administrar o mundo divino onde conflitos não
são excepcionais.

\section{Zeus e Tifeu}

Curiosamente, porém, Zeus ainda terá que enfrentar mais um conflito
belicoso, a luta contra Tifeu (820--80). Por um lado, como nos dois
poemas épicos que conhecemos, a \textit{Ilíada} e a \textit{Odisseia}, o
maior herói se revela quando um derrota inimigo poderoso com suas
próprias mãos. Por outro lado, esse inimigo é, estranhamente, filho do
próprio Tártaro com Terra. Que a fertilidade exacerbada desta tenho
gerado um ser para destronar o novo senhor do cosmo, isto não
surpreende, pois a eminência parda feminina foi peça fundamental na
deposição de Céu e Crono; que aquele seja o pai, isto sim é curioso,
pois até este momento da narrativa dele apenas se falou como um espaço.
É como se, pela lógica da narrativa hesiódica, só agora ele tivesse
adquirido o estatuto pleno de divindade e precisasse se envolver em um
conflito que garanta que sua forma não se alterará.

Tifeu, por sua vez, adquire, devido à lógica da narrativa, o lugar de
filho de Zeus, pois todo rei anterior fora deposto por seu filho, sempre
ligado à Terra. O conflito contra os Titãs, porém, já mostrou que a
manipulação da astúcia e da força, no grau superlativo em que o faz
Zeus, não deixa espaço para a possibilidade de derrota, mesmo que o
adversário também seja muito forte --- Tifeu tem cabeças com olhos de
onde sai fogo --- e muito astuto --- suas cem cabeças produzem todo tipo
de som, sendo que a metamorfose é um elemento mítico típico do universo
da astúcia. Além disso, esse combate singular entre a criatura
monstruosa e Zeus também permite que Terra, derradeiramente, seja
derrotada e esterilizada. O fogo de Zeus como que a derrete: de criadora
de metal e artífice metalúrgica, Terra como que se transforma, graças ao
fogo aniquilador de Zeus, no metal que é manipulado por artesãos machos
(861--67).

\section{Zeus e suas esposas}

Uma vez derrotada a astuta Terra, que imediatamente se torna aliada de
Zeus (891), a primeira providência do soberano é casar com Astúcia e,
antes de essa parir seu primeiro filho, devorá-la, não esperando que
essa gerasse um deus macho mais forte que ele (886--900). Muellner (1996)
mostrou como esse episódio arremata todos os conflitos dinásticos
narrados até então: Zeus não devora seu primeiro filho, como Crono, ou
obriga que sua esposa o guarde no ventre, como Céu, mas assimila o
elemento feminino em si mesmo, Astúcia, e o gera como aliado, Atena.
Com isso, Zeus se torna um andrógino perfeito,\footnote{Do ponto de vista grego:
muito mais masculino que feminino.} e não um disforme emasculado, como
Céu. A astúcia revela-se mais uma vez essencialmente feminina, mas para
sempre assimilada pelo próprio rei. A filha produzida pelo rei não só
não é um macho --- e foram sempre jovens machos que derrotaram seus pais
---, mas é uma virgem, ou seja, uma deusa que não irá produzir uma ameaça
ao \textit{status quo}. Por fim, ao ingerir a esposa grávida do primeiro
filho, ele bloqueou a previsão de que, depois de Atena, Astúcia geraria
um filho mais forte que o pai. Pela primeira vez, o rei dos deuses
consegue \textit{desparir} de forma perfeita e acabada.

E somente agora nasce, de Zeus e várias de suas esposas, uma linhagem de
deuses responsáveis pelo que há de bom no cosmo propriamente humano, ou
seja, na sociedade (901--17): Norma, Decência, Justiça, Paz, as Musas,
Radiância, Alegria e Festa, notável prole antípoda aos filhos de Noite e
Briga. A última esposa de Zeus, Hera, é aquela que, de acordo com a
lógica do poema, representa a maior ameaça a Zeus, mas tanto o filho
mais perigoso que os dois têm juntos, quanto aquele que Hera, como que
emulando Zeus no caso de Atena, tem sozinha, Hefesto, não representam
adversários fortes o suficiente contra a filha que mais se assemelha ao
pai e está completamente alinhada com ele, Atena, senhora da guerra mas
também da astúcia (921--29).

É nesse sentido que se deve entender o longo catálogo que finaliza o
poema e que tem três partes: os casamentos de Zeus e os filhos deles
resultantes (901--29); um catálogo mais abrangente de casamentos divinos
(930--61), que revelam, de forma sumária, um panteão muito bem organizado
e potencialmente harmônico;\footnote{Como que servindo de epítome, o casamento
entre Ares e Afrodite produz, por um lado, os machos Terror e Pânico,
mas, por outro, Harmonia.} e finalmente um catálogo de deusas que se
uniram a mortais (962--1020). Ora, com as deusas fêmeas que se unem a
machos mortais, o princípio de ruptura que vigorara ao longo do poema
agora se desloca para o mundo dos homens, mais precisamente, o mundo dos
heróis: nesse mundo, filhos poderão ser mais fortes que os pais,
podendo, no limite, o que atesta Telégono, o filho de Circe e Odisseu,
matá-lo.

Para concluir, mencione-se que há uma discussão inconclusa sobre onde a
versão ``original'' da \textit{Teogonia} teria terminado. Autores como
Clay (2003) e Kelly (2007) mostraram que os catálogos tal como
analisados acima compõe um final muito adequado ao poema; assim,
provavelmente somente os quatro ou possivelmente os dois últimos versos
foram acrescentados ao poema em um certo momento de sua transmissão para
introduzir um outro poema atribuído a Hesíodo, o \textit{Catálogo das
mulheres}, que chegou a nós por meio de fragmentos, que procurava
dar uma visão geral da idade dos heróis a partir das mulheres que com
deuses dormiram por toda a Grécia, catálogo este que, possivelmente, era
concluído pelo catálogo de pretendentes de Helena, cujo casamento
redundou no grande cataclisma que foi a guerra de Troia, que
metonimicamente podia ser pensada, na Antiguidade, como o fim da época
dos heróis.

\section{Da tradução}

Para definir o texto grego aqui traduzido, cotejaram-se as seguintes
edições: West (1966), Most (2006) e Ricciardelli (2018). Também foram
muito úteis para se definir a opção por determinada leitura ou
interpretação, bem como para compor as notas, diversos textos citados na
bibliografia, especialmente Marg (1970), Verdenius (1972), Arrighetti
(2007), Pucci (2007) e Vergados (2020). Para a tradução, também foi
fundamental o léxico organizado por Snell \textit{et al.} (1955--2010).

Um dos principais problemas enfrentados pelo o tradutor da
\textit{Teogonia} diz respeito ao nomes das divindades. Não se buscou
nenhum tipo de padronização muito rígida, ou seja, ficou-se entre os
extremos de traduzir quase todos os nomes e quase nenhum nome. De
forma geral, os principais critérios foram o bom senso, o conhecimento
do leitor e a sonoridade. Além disso, as notas apresentam a
transliteração de todos os nomes, bem como explicitam algumas figuras
etimológicas.

Para facilitar a leitura, optou-se por seguir o que fazem a maioria dos
editores em sua forma de propor uma divisão do poema em partes
distintas. O recuo de parágrafo, ainda que estranho em um poema, deve
ser pensado como equivalente a um novo parágrafo em uma narrativa em
prosa. Não é possível saber, entretanto, se tais marcações são
equivalentes a pausa nas performances orais originais dos poemas.
Trata-se, portanto, de um recurso eminentemente didático.

Algumas soluções que adotei nas minhas traduções de Homero (2018a) e
(2018b) nortearam certas modificações nesta edição da tradução do poema
hesiódico. Uma delas é evitar excessos no uso da ordem sintática
indireta.

A numeração das notas de rodapé em forma de lemas segue o número que
indica um verso ou um conjunto de versos do poema.

Por fim, gostaria de agradecer àqueles que compartilharam comigo seu
conhecimento de Hesíodo, em especial, da \textit{Teogonia}, desde a 1ª
edição deste volume ou me apontaram o que nele poderia ser melhorado ou
corrigido: Camila Zanon, Thanassis Vergados, Jim Marks, Adrian Kelly,
Teodoro Assunção, André Malta, os membros da minha banca de
livre-docência --- Jaa Torrano, Zélia de Almeida Cardoso, Jacyntho L.
Brandão, Pedro Paulo Funari e Maria Beatriz Florenzano --- e
Antonio-Orlando Dourado Lopes.

\begin{bibliohedra}
\tit{allan}, W. Divine justice and cosmic order in early Greek Epic.
\textit{Journal of Hellenic Studies} v.\,126, 2006, p.\,1--35.

\tit{andersen}, Ø.; \textsc{haug}, D. T. T. (org.) \textit{Relative chronology in early
Greek epic poetry}. Cambridge: Cambridge University Press, 2012.

\tit{antunes}, C. L. B. 26 hinos homéricos. \textit{Cadernos de literatura em
tradução} v.\,15, p.\,13--23, 2015.

\tit{arnould}, D. Les noms des dieux dans la \textit{Théogonie} d'Hésiode:
étymologies et jeux de mots. \textit{Revue des études grecques} v.\,122,
2009, p.\,1--14.

\tit{arrighetti}, G. \textit{Esiodo opere}. Introdução, tradução e comentário.
Milano: Mondadori, 2007.

\tit{bakker}, E. J. Hesiod in performance. In: \textsc{loney}, A. C.; \textsc{scully}, S. (org.)
\textit{The Oxford Handbook of Hesiod}. Oxford: Oxford, 2018.

\tit{blaise}, F.; \textsc{judet de la combe}, P.; \textsc{rousseau}, P. (org.) \textit{Le métier
du mythe: lectures d' Hésiode}. Lille: Presses Universitaires du
Septentrion, 1996.

\tit{brandão}, J. L. \textit{Antiga Musa (arqueologia da ficção)}. 2ª edição.
Belo Horizonte: Relicário, 2015.

\tit{burkert}, W. \textit{The Orientalizing revolution: Near Eastern influence
on Greek culture in the early archaic age}. Cambridge, Mass.: Harvard
University Press, 1992.

\titidem. \textit{Religião grega na época clássica e arcaica}. Lisboa: Fundação
Calouste Gulbenkian, 1993.

\tit{cingano}, E. The Hesiodic corpus. In: \textsc{montanari}, F.; \textsc{rengakos}, A.;
\textsc{tsagalis}, C. (org.) \textit{Brill's companion to Hesiod}. Leiden/Boston:
Brill, 2009, p.\,91--130.

\tit{clay}, J. S. \textit{Hesiod's cosmos}. Cambridge: Cambridge University
Press, 2003.

\tit{colonna}, A. \textit{Opere di Esiodo}. Torino: Unione Tipografico-Editrice,
1977.

\tit{detienne}, M. \textit{Os mestres da verdade na Grécia} arcaica. Trad. A.
Daher. Rio de Janeiro: Jorge Zahar, 1988.

\tit{detienne}, M.; \textsc{vernant}, J.-P. \textit{Métis}: As astúcias da inteligência.
Trad. F. Hirata. São Paulo: Odysseus, 2008.

\tit{gagarin}, M. The poetry of justice: Hesiod and the origins of Greek law.
\textit{Ramus} v.\,21, 1992, p.\,61--78.

\tit{haubold}, J. \textit{Greece and Mesopotamia}: dialogues in literature.
Cambridge: Cambridge University Press, 2013.

\tit{janda}, M. \textit{Über `Stock und Stein': die indogermanischen Variationen
eines universalen Phraseologismus}. Röll: Dettelbach, 1997.

\tit{janko}, R. \textit{Homer, Hesiod and the Hymns:} diachronic development in
epic diction. Cambridge: Cambridge University Press, 1982.

\tit{kelly}, A. How to end an orally-derived epic poem? \textit{Transactions of
the American Philological Association} n.\,137, 2007, p.\,371--402.

\titidem. Gendrificando o mito de sucessão em Hesíodo e no antigo Oriente
Próximo. Trad.: C. A. Zanon. \textit{Classica} v.\,32, n.\,2, 2019, p.\,119--38.

\tit{koning}, H. \textit{Hesiod}: \textit{the other poet}: ancient reception of a
cultural icon. Leiden: Brill, 2010.

\tit{laks}, A. Le doublé du roi: remarques sur les antécédents hésiodiques du
philosophe-roi. In~: \textsc{blaise}, F.; \textsc{judet de la combe}, P.; \textsc{rousseau}, P.
(org.) \textit{Le métier du mythe}: lectures d' Hésiode. Lille: Presses
Universitaires du Septentrion, 1996.

\tit{lamberton}, R. \textit{Hesiod}. New Haven: Yale University Press, 1988.

\tit{leclerc}, M.-C. \textit{La parole chez Hésiode: à la recherche de
l'harmonie perdue}. Paris: Belles Lettres, 1993.

\tit{ledbetter}, G. M. \textit{Poetics before Plato: interpretation and
authority in early Greek theories of poetry}. Princeton: Princeton
University Press, 2003.

\tit{loney}, A. C. Hesiod's temporalities. In: \textsc{loney}, A. C.; \textsc{scully}, S. (org.)
\textit{The Oxford Handbook of Hesiod}. Oxford: Oxford, 2018.

\tit{macedo}, J. M. \textit{A palavra ofertada}: um estudo retórico dos hinos
gregos e indianos. Campinas: Edunicamp, 2010.

\tit{marg}, W. \textit{Hesiod}: Sämtliche Gedichte. Artemis: Zürich/Stuttgart,
1970.

\tit{martin}, R. P. Hesiod, Odysseus, and the instruction of princes.
\textit{Transactions of the American Philological Association} v.\,114,
1984, p.\,29--48.

\titidem. Hesiodic theology. In: \textsc{loney}, A. C.; \textsc{scully}, S. (org.) \textit{The
Oxford Handbook of Hesiod}. Oxford: Oxford, 2018.

\tit{meier-brügger}, M. Zu Hesiods Namen. \textit{Glotta} v.\,68, 1990, p.\,62--67.

\tit{moraes}, A. S. de. História e etnicidade: Homero à vizinhança do
pan-helenismo. \textit{Hélade} v.\,5, n.\,1, 2019, p.\,12--36.

\tit{most}, G. W. Hesiod and the textualization of personal temporality. In:
\textsc{montanari}, F.; \textsc{arrighetti}, G. (org.) \textit{La componente autobiografica
nella poesia greca e latina}. Pisa: Giardini, 1993, p.\,73--91.

\titidem. \textit{Hesiod}: \textit{Theogony, Works and Days, Testimonia}.
Cambridge, \textsc{ma}: Harvard University Press, 2006.

\tit{muellner}, L. C. \textit{The anger of Achilles}: mēnis \textit{in Greek
epic}. Ithaca: Cornell University Press, 1996.

\tit{murray}, P. Poetic inspiration in early Greece. \textit{Journal of Hellenic
Studies} v.\,101, 1981, p.\,87--100.

\tit{nagy}, G. Hesiod and the poetics of Pan-Hellenism. In: \line(1,0){25}. \textit{Greek
mythology and poetics}. Ithaca: Cornell Univesity Press, 1990, p.\,36--82.

\titidem. \textit{The best of the Achaeans: concepts of the hero in archaic Greek
poetry}. 2ª ed. Baltimore: Johns Hopkins University Press, 1999.

\tit{oliveira}, J. ``Áurea Afrodite'' e a ordem cósmica de Zeus na poesia
hesiódica. \textit{Codex} -- Revista de estudos clássicos. Rio de Janeiro,
v.\,7, n.\,2, 2019, p.\,69--80.

\titidem. A linhagem dos heróis na cosmologia hesiódica. \textit{Rónai} v.\,8, n.\,2, 2020, p.\,353--374.

\tit{pucci}, P. \textit{Hesiod and the language of poetry}. Baltimore: Johns
Hopkins University Press, 1977.

\titidem. \textit{Inno alle Muse (Esiodo}, Teogonia\textit{, 1--115): texto,
introduzione, traduzione e commento}. Pisa: Fabrizio Serra, 2007.

\titidem. The poetry of the \textit{Theogony}. In: \textsc{montanari}, F.; \textsc{rengakos}, A.;
\textsc{tsagalis}, C. (org.) \textit{Brill's Companion to Hesiod}. Leiden/Boston:
Brill, 2009, p.\,37--70.

\tit{ricciardelli}, G. \textit{Esiodo: Teogonia}. Milano: Fondazione Lorenzo
Valla / Mondadori, 2018.

\tit{rijksbaron}, A. Discourse cohesion in the proem of Hesiod's
\textit{Theogony}. In: \textsc{bakker}, S.; \textsc{wakker}, G. (org.) \textit{Discourse
cohesion in Ancient Greek}. Leiden: Brill, 2009.

\tit{ribeiro} Jr., W. A. \textit{et al.} \textit{Hinos homéricos: tradução, notas
e estudo}. São Paulo: Edunesp, 2011.

\tit{rosenfield}, K. H. \textit{Desenveredando Rosa: a obra de J. G. Rosa e
outros ensaios}. Rio de Janeiro: Topbooks, 2006.

\tit{rowe}, C. J. `Archaic thought' in Hesiod. \textit{Journal of Hellenic
Studies} v.\,103, p.\,124--35, 1983.

\tit{rutherford}, I. Hesiod and the literary traditions of the Near East. In:
\textsc{montanari}, F.; \textsc{rengakos}, A.; \textsc{tsagalis}, C. (org.) \textit{Brill's companion
to Hesiod}. Leiden: Brill, 2009.

\tit{scully}, S. \textit{Hesiod's~}Theogony\textit{:~from Near Eastern
creation myths to}~Paradise Lost\textit{.} Oxford and New York: Oxford
University Press,~2015.

\tit{snell}, B. O mundo dos deuses em Hesíodo. In: \line(1,0){25}. \textit{A cultura grega e
as origens do pensamento}. São Paulo: Perspectiva, 2001.

\tit{snodgrass}, A. M. \textit{The Dark Age of Greece}. Edinburgh: Edinburgh
University Press, 1971.

\tit{thalmann}, W. G. \textit{Conventions of form and thought in early Greek
epic}. Baltimore/ London: Johns Hopkins University Press, 1984.

\tit{torrano}, J. A. A. \textit{Hesíodo: Teogonia}. A origem dos deuses. Estudo
e tradução. 2\textsuperscript{a} edição. São Paulo: Iluminuras, 1992.

\titidem. \textit{O certame Homero-Hesíodo} (texto integral). \textit{Letras
clássicas} 9, p.\,215--24, 2005.

\tit{tsagalis}, C. Poetry and poetics in the Hesiodic corpus. In: \textsc{montanari},
F.; \textsc{rengakos}, A.; \textsc{tsagalis}, C. (org.) \textit{Brill's companion to
Hesiod}. Leiden: Brill, 2009, p.\,131--78.

\tit{verdenius}, W. J. Notes on the proem of Hesiod's \textit{Theogony}.
\textit{Mnemosyne} v.\,25, 1972, p.\,225--60.

\tit{vergados}, A. Stitching narratives: unity and episod in Hesiod. In:
\textsc{werner}, C.; \textsc{dourado-lopes}, A.; \textsc{werner}, E. (org.) \textit{Tecendo
narrativas}: unidade e episódio na literatura grega antiga. São Paulo:
Humanitas, 2015, p.\,29--54.

\titidem. \textit{Hesiod's verbal craft: studies in Hesiod's conception of
language and its ancient reception}. Oxford: Oxford University Press,
2020.

\tit{vernant}, J.-P. \textit{Mito e sociedade na Grécia antiga}. Rio de Janeiro:
José Olympio, 1992.

\titidem. \textit{Mito e pensamento entre os gregos}. Rio de Janeiro: Paz e
Terra, 2002.

\tit{versnel}, H. S. \textit{Coping with the gods: wayward readings in Greek
theology}. Leiden: Brill, 2011.

\tit{west}, M. L. \textit{Hesiod,} Theogony\textit{: edited with prolegomena and
commentary}. Oxford: Oxford University Press, 1966.

\titidem. \textit{The east face of Helicon: West Asiatic elements in Greek poetry
and myth}. Oxford: Oxford University Press, 1997.

\tit{woodward}, R. D. Hesiod and Greek myth. In: \line(1,0){25}. (org.) \textit{The Cambridge
companion to Greek mythology}. Cambridge: Cambridge University Press,
2007.

\tit{zanon}, C. A. \textit{Onde vivem os monstros: criaturas prodigiosas na
poesia de Homero e Hesíodo}. São Paulo: Humanitas, 2018.
\end{bibliohedra}


\endgroup




%\openany
\part{Teogonia}


\begin{pages}
    \begin{Leftside}
        \beginnumbering
            \pstart
				\Porson
Μουσάων Ἑλικωνιάδων ἀρχώμεθ' ἀείδειν,\\
αἵ θ' Ἑλικῶνος ἔχουσιν ὄρος μέγα τε ζάθεόν τε, \\
καί τε περὶ κρήνην ἰοειδέα πόσσ' ἁπαλοῖσιν\\
ὀρχεῦνται καὶ βωμὸν ἐρισθενέος Κρονίωνος·\\
καί τε λοεσσάμεναι τέρενα χρόα Περμησσοῖο \\
ἠ' Ἵππου κρήνης ἠ' Ὀλμειοῦ ζαθέοιο\\
ἀκροτάτῳ Ἑλικῶνι χοροὺς ἐνεποιήσαντο,\\
καλοὺς ἱμερόεντας, ἐπερρώσαντο δὲ ποσσίν.\\

\quad{}ἔνθεν ἀπορνύμεναι κεκαλυμμέναι ἠέρι πολλῷ \\
ἐννύχιαι στεῖχον περικαλλέα ὄσσαν ἱεῖσαι, \\
ὑμνεῦσαι Δία τ' αἰγίοχον καὶ πότνιαν Ἥρην\\
Ἀργείην, χρυσέοισι πεδίλοις ἐμβεβαυῖαν, \\
κούρην τ' αἰγιόχοιο Διὸς γλαυκῶπιν Ἀθήνην\\
Φοῖβόν τ' Ἀπόλλωνα καὶ Ἄρτεμιν ἰοχέαιραν\\
ἠδὲ Ποσειδάωνα γαιήοχον ἐννοσίγαιον \\
καὶ Θέμιν αἰδοίην ἑλικοβλέφαρόν τ' Ἀφροδίτην\\
Ἥβην τε χρυσοστέφανον καλήν τε Διώνην\\
Λητώ τ' Ἰαπετόν τε ἰδὲ Κρόνον ἀγκυλομήτην\\
Ἠῶ τ' Ἠέλιόν τε μέγαν λαμπράν τε Σελήνην\\
Γαῖάν τ' Ὠκεανόν τε μέγαν καὶ Νύκτα μέλαιναν \\
ἄλλων τ' ἀθανάτων ἱερὸν γένος αἰὲν ἐόντων.\\

\quad{}αἵ νύ ποθ' Ἡσίοδον καλὴν ἐδίδαξαν ἀοιδήν,\\
ἄρνας ποιμαίνονθ' Ἑλικῶνος ὕπο ζαθέοιο.\\
τόνδε δέ με πρώτιστα θεαὶ πρὸς μῦθον ἔειπον,\\
Μοῦσαι Ὀλυμπιάδες, κοῦραι Διὸς αἰγιόχοιο·  \\
``ποιμένες ἄγραυλοι, κάκ' ἐλέγχεα, γαστέρες οἶον,\\
ἴδμεν ψεύδεα πολλὰ λέγειν ἐτύμοισιν ὁμοῖα,\\
ἴδμεν δ' εὖτ' ἐθέλωμεν ἀληθέα γηρύσασθαι.''\\
ὣς ἔφασαν κοῦραι μεγάλου Διὸς ἀρτιέπειαι,\\
καί μοι σκῆπτρον ἔδον δάφνης ἐριθηλέος ὄζον \\
δρέψασαι, θηητόν· ἐνέπνευσαν δέ μοι αὐδὴν \\
θέσπιν, ἵνα κλείοιμι τά τ' ἐσσόμενα πρό τ' ἐόντα, \\
καί μ' ἐκέλονθ' ὑμνεῖν μακάρων γένος αἰὲν ἐόντων,\\
σφᾶς δ' αὐτὰς πρῶτόν τε καὶ ὕστατον αἰὲν ἀείδειν.\\

\quad{}ἀλλὰ τίη μοι ταῦτα περὶ δρῦν ἢ περὶ πέτρην;  \\
τύνη, Μουσάων ἀρχώμεθα, ταὶ Διὶ πατρὶ\\
ὑμνεῦσαι τέρπουσι μέγαν νόον ἐντὸς Ὀλύμπου,\\
εἴρευσαι τά τ' ἐόντα τά τ' ἐσσόμενα πρό τ' ἐόντα,\\
φωνῇ ὁμηρεῦσαι, τῶν δ' ἀκάματος ῥέει αὐδὴ \\
ἐκ στομάτων ἡδεῖα· γελᾷ δέ τε δώματα πατρὸς  \\
Ζηνὸς ἐριγδούποιο θεᾶν ὀπὶ λειριοέσσῃ\\
σκιδναμένῃ, ἠχεῖ δὲ κάρη νιφόεντος Ὀλύμπου \\
δώματά τ' ἀθανάτων· αἱ δ' ἄμβροτον ὄσσαν ἱεῖσαι \\
θεῶν γένος αἰδοῖον πρῶτον κλείουσιν ἀοιδῇ\\
ἐξ ἀρχῆς, οὓς Γαῖα καὶ Οὐρανὸς εὐρὺς ἔτικτεν, \\
οἵ τ' ἐκ τῶν ἐγένοντο, θεοὶ δωτῆρες ἐάων· \\
δεύτερον αὖτε Ζῆνα θεῶν πατέρ' ἠδὲ καὶ ἀνδρῶν, \\
ἀρχόμεναί θ' ὑμνεῦσι καὶ ἐκλήγουσαι ἀοιδῆς,\\
ὅσσον φέρτατός ἐστι θεῶν κάρτει τε μέγιστος·\\
αὖτις δ' ἀνθρώπων τε γένος κρατερῶν τε Γιγάντων \\
ὑμνεῦσαι τέρπουσι Διὸς νόον ἐντὸς Ὀλύμπου\\
Μοῦσαι Ὀλυμπιάδες, κοῦραι Διὸς αἰγιόχοιο.\\

\quad{}τὰς ἐν Πιερίῃ Κρονίδῃ τέκε πατρὶ μιγεῖσα\\
Μνημοσύνη, γουνοῖσιν Ἐλευθῆρος μεδέουσα,\\
λησμοσύνην τε κακῶν ἄμπαυμά τε μερμηράων. \\
ἐννέα γάρ οἱ νύκτας ἐμίσγετο μητίετα Ζεὺς\\
νόσφιν ἀπ' ἀθανάτων ἱερὸν λέχος εἰσαναβαίνων·\\
ἀλλ' ὅτε δή ῥ' ἐνιαυτὸς ἔην, περὶ δ' ἔτραπον ὧραι\\
μηνῶν φθινόντων, περὶ δ' ἤματα πόλλ' ἐτελέσθη,\\
ἡ δ' ἔτεκ' ἐννέα κούρας, ὁμόφρονας, ᾗσιν ἀοιδὴ  \\
μέμβλεται ἐν στήθεσσιν, ἀκηδέα θυμὸν ἐχούσαις,\\
τυτθὸν ἀπ' ἀκροτάτης κορυφῆς νιφόεντος Ὀλύμπου·\\
ἔνθά σφιν λιπαροί τε χοροὶ καὶ δώματα καλά,\\
πὰρ δ' αὐτῇς Χάριτές τε καὶ Ἵμερος οἰκί' ἔχουσιν\\
ἐν θαλίῃς· ἐρατὴν δὲ διὰ στόμα ὄσσαν ἱεῖσαι  \\
μέλπονται, πάντων τε νόμους καὶ ἤθεα κεδνὰ \\
ἀθανάτων κλείουσιν, ἐπήρατον ὄσσαν ἱεῖσαι.\\
αἳ τότ' ἴσαν πρὸς Ὄλυμπον, ἀγαλλόμεναι ὀπὶ καλῇ, \\
ἀμβροσίῃ μολπῇ· περὶ δ' ἴαχε γαῖα μέλαινα \\
ὑμνεύσαις, ἐρατὸς δὲ ποδῶν ὕπο δοῦπος ὀρώρει \\
νισομένων πατέρ' εἰς ὅν· ὁ δ' οὐρανῷ ἐμβασιλεύει, \\
αὐτὸς ἔχων βροντὴν ἠδ' αἰθαλόεντα κεραυνόν,\\
κάρτει νικήσας πατέρα Κρόνον· εὖ δὲ ἕκαστα \\
ἀθανάτοις διέταξε ὁμῶς καὶ ἐπέφραδε τιμάς.\\

\quad{}ταῦτ' ἄρα Μοῦσαι ἄειδον Ὀλύμπια δώματ' ἔχουσαι, \\
ἐννέα θυγατέρες μεγάλου Διὸς ἐκγεγαυῖαι,\\
Κλειώ τ' Εὐτέρπη τε Θάλειά τε Μελπομένη τε\\
Τερψιχόρη τ' Ἐρατώ τε Πολύμνιά τ' Οὐρανίη τε\\
Καλλιόπη θ'· ἡ δὲ προφερεστάτη ἐστὶν ἁπασέων. \\
ἡ γὰρ καὶ βασιλεῦσιν ἅμ' αἰδοίοισιν ὀπηδεῖ.  \\
ὅντινα τιμήσουσι Διὸς κοῦραι μεγάλοιο\\
γεινόμενόν τε ἴδωσι διοτρεφέων βασιλήων,\\
τῷ μὲν ἐπὶ γλώσσῃ γλυκερὴν χείουσιν ἐέρσην,\\
τοῦ δ' ἔπε' ἐκ στόματος ῥεῖ μείλιχα· οἱ δέ νυ λαοὶ \\
πάντες ἐς αὐτὸν ὁρῶσι διακρίνοντα θέμιστας \\
ἰθείῃσι δίκῃσιν· ὁ δ' ἀσφαλέως ἀγορεύων\\
αἶψά τι καὶ μέγα νεῖκος ἐπισταμένως κατέπαυσε· \\
τούνεκα γὰρ βασιλῆες ἐχέφρονες, οὕνεκα λαοῖς \\
βλαπτομένοις ἀγορῆφι μετάτροπα ἔργα τελεῦσι\\
ῥηιδίως, μαλακοῖσι παραιφάμενοι ἐπέεσσιν·  \\
ἐρχόμενον δ' ἀν' ἀγῶνα θεὸν ὣς ἱλάσκονται\\
αἰδοῖ μειλιχίῃ, μετὰ δὲ πρέπει ἀγρομένοισι. \\

\quad{}τοίη Μουσάων ἱερὴ δόσις ἀνθρώποισιν.\\
ἐκ γάρ τοι Μουσέων καὶ ἑκηβόλου Ἀπόλλωνος\\
ἄνδρες ἀοιδοὶ ἔασιν ἐπὶ χθόνα καὶ κιθαρισταί, \\
ἐκ δὲ Διὸς βασιλῆες· ὁ δ' ὄλβιος, ὅντινα Μοῦσαι\\
φίλωνται· γλυκερή οἱ ἀπὸ στόματος ῥέει αὐδή. \\
εἰ γάρ τις καὶ πένθος ἔχων νεοκηδέι θυμῷ\\
ἄζηται κραδίην ἀκαχήμενος, αὐτὰρ ἀοιδὸς\\
Μουσάων θεράπων κλεῖα προτέρων ἀνθρώπων  \\
ὑμνήσει μάκαράς τε θεοὺς οἳ Ὄλυμπον ἔχουσιν, \\
αἶψ' ὅ γε δυσφροσυνέων ἐπιλήθεται οὐδέ τι κηδέων\\
μέμνηται· ταχέως δὲ παρέτραπε δῶρα θεάων. \\

\quad{}χαίρετε τέκνα Διός, δότε δ' ἱμερόεσσαν ἀοιδήν· \\
κλείετε δ' ἀθανάτων ἱερὸν γένος αἰὲν ἐόντων, \\
οἳ Γῆς ἐξεγένοντο καὶ Οὐρανοῦ ἀστερόεντος,\\
Νυκτός τε δνοφερῆς, οὕς θ' ἁλμυρὸς ἔτρεφε Πόντος.\\
εἴπατε δ' ὡς τὰ πρῶτα θεοὶ καὶ γαῖα γένοντο \\
καὶ ποταμοὶ καὶ πόντος ἀπείριτος οἴδματι θυίων \\
ἄστρά τε λαμπετόωντα καὶ οὐρανὸς εὐρὺς ὕπερθεν·  \\
οἵ τ' ἐκ τῶν ἐγένοντο, θεοὶ δωτῆρες ἐάων· \\
ὥς τ' ἄφενος δάσσαντο καὶ ὡς τιμὰς διέλοντο, \\
ἠδὲ καὶ ὡς τὰ πρῶτα πολύπτυχον ἔσχον Ὄλυμπον.\\
ταῦτά μοι ἔσπετε Μοῦσαι Ὀλύμπια δώματ' ἔχουσαι \\
ἐξ ἀρχῆς, καὶ εἴπαθ', ὅτι πρῶτον γένετ' αὐτῶν. \\

\quad{}ἤτοι μὲν πρώτιστα Χάος γένετ'· αὐτὰρ ἔπειτα\\
Γαῖ' εὐρύστερνος, πάντων ἕδος ἀσφαλὲς αἰεὶ\\
ἀθανάτων οἳ ἔχουσι κάρη νιφόεντος Ὀλύμπου \\
Τάρταρά τ' ἠερόεντα μυχῷ χθονὸς εὐρυοδείης,\\
ἠδ' Ἔρος, ὃς κάλλιστος ἐν ἀθανάτοισι θεοῖσι, \\
λυσιμελής, πάντων τε θεῶν πάντων τ' ἀνθρώπων\\
δάμναται ἐν στήθεσσι νόον καὶ ἐπίφρονα βουλήν.\\

{\minion\Para}
ἐκ Χάεος δ' Ἔρεβός τε μέλαινά τε Νὺξ ἐγένοντο· \\
Νυκτὸς δ' αὖτ' Αἰθήρ τε καὶ Ἡμέρη ἐξεγένοντο,\\
οὓς τέκε κυσαμένη Ἐρέβει φιλότητι μιγεῖσα. \\

{\minion\Para}
Γαῖα δέ τοι πρῶτον μὲν ἐγείνατο ἶσον ἑωυτῇ\\
Οὐρανὸν ἀστερόενθ', ἵνα μιν περὶ πάντα καλύπτοι,\\
ὄφρ' εἴη μακάρεσσι θεοῖς ἕδος ἀσφαλὲς αἰεί,\\
γείνατο δ' οὔρεα μακρά, θεᾶν χαρίεντας ἐναύλους \\
Νυμφέων, αἳ ναίουσιν ἀν' οὔρεα βησσήεντα,  \\
ἠδὲ καὶ ἀτρύγετον πέλαγος τέκεν οἴδματι θυῖον,\\
Πόντον, ἄτερ φιλότητος ἐφιμέρου· αὐτὰρ ἔπειτα\\
Οὐρανῷ εὐνηθεῖσα τέκ' Ὠκεανὸν βαθυδίνην \\
Κοῖόν τε Κρεῖόν θ' Ὑπερίονά τ' Ἰαπετόν τε\\
Θείαν τε Ῥείαν τε Θέμιν τε Μνημοσύνην τε \\
Φοίβην τε χρυσοστέφανον Τηθύν τ' ἐρατεινήν.\\
τοὺς δὲ μέθ' ὁπλότατος γένετο Κρόνος ἀγκυλομήτης,\\
δεινότατος παίδων, θαλερὸν δ' ἤχθηρε τοκῆα. \\

\quad{}γείνατο δ' αὖ Κύκλωπας ὑπέρβιον ἦτορ ἔχοντας,\\
Βρόντην τε Στερόπην τε καὶ Ἄργην ὀβριμόθυμον, \\
οἳ Ζηνὶ βροντήν τ' ἔδοσαν τεῦξάν τε κεραυνόν.\\
οἱ δ' ἤτοι τὰ μὲν ἄλλα θεοῖς ἐναλίγκιοι ἦσαν,\\
μοῦνος δ' ὀφθαλμὸς μέσσῳ ἐνέκειτο μετώπῳ· \\
Κύκλωπες δ' ὄνομ' ἦσαν ἐπώνυμον, οὕνεκ' ἄρά σφεων\\
κυκλοτερὴς ὀφθαλμὸς ἕεις ἐνέκειτο μετώπῳ· \\
ἰσχὺς δ' ἠδὲ βίη καὶ μηχαναὶ ἦσαν ἐπ' ἔργοις.\\

\quad{}ἄλλοι δ' αὖ Γαίης τε καὶ Οὐρανοῦ ἐξεγένοντο\\
τρεῖς παῖδες μεγάλοι \textless{}τε\textgreater{} καὶ ὄβριμοι, οὐκ ὀνομαστοί, \\
Κόττος τε Βριάρεώς τε Γύγης θ', ὑπερήφανα τέκνα. \\
τῶν ἑκατὸν μὲν χεῖρες ἀπ' ὤμων ἀίσσοντο, \\
ἄπλαστοι, κεφαλαὶ δὲ ἑκάστῳ πεντήκοντα\\
ἐξ ὤμων ἐπέφυκον ἐπὶ στιβαροῖσι μέλεσσιν· \\
ἰσχὺς δ' ἄπλητος κρατερὴ μεγάλῳ ἐπὶ εἴδει.\\

\quad{}ὅσσοι γὰρ Γαίης τε καὶ Οὐρανοῦ ἐξεγένοντο,\\
δεινότατοι παίδων, σφετέρῳ δ' ἤχθοντο τοκῆι \\
ἐξ ἀρχῆς· καὶ τῶν μὲν ὅπως τις πρῶτα γένοιτο, \\
πάντας ἀποκρύπτασκε καὶ ἐς φάος οὐκ ἀνίεσκε\\
Γαίης ἐν κευθμῶνι, κακῷ δ' ἐπετέρπετο ἔργῳ, \\
Οὐρανός· ἡ δ' ἐντὸς στοναχίζετο Γαῖα πελώρη\\
στεινομένη, δολίην δὲ κακὴν ἐπεφράσσατο τέχνην. \\
αἶψα δὲ ποιήσασα γένος πολιοῦ ἀδάμαντος\\
τεῦξε μέγα δρέπανον καὶ ἐπέφραδε παισὶ φίλοισιν· \\

\quad{}εἶπε δὲ θαρσύνουσα, φίλον τετιημένη ἦτορ·\\
``παῖδες ἐμοὶ καὶ πατρὸς ἀτασθάλου, αἴ κ' ἐθέλητε\\
πείθεσθαι· πατρός κε κακὴν τεισαίμεθα λώβην \\
ὑμετέρου· πρότερος γὰρ ἀεικέα μήσατο ἔργα.'' \\
ὣς φάτο· τοὺς δ' ἄρα πάντας ἕλεν δέος, οὐδέ τις αὐτῶν\\

\quad{}φθέγξατο. θαρσήσας δὲ μέγας Κρόνος ἀγκυλομήτης\\
αἶψ' αὖτις μύθοισι προσηύδα μητέρα κεδνήν·\\
``μῆτερ, ἐγώ κεν τοῦτό γ' ὑποσχόμενος τελέσαιμι  \\
ἔργον, ἐπεὶ πατρός γε δυσωνύμου οὐκ ἀλεγίζω\\
ἡμετέρου· πρότερος γὰρ ἀεικέα μήσατο ἔργα.''\\

\quad{}ὣς φάτο· γήθησεν δὲ μέγα φρεσὶ Γαῖα πελώρη· \\
εἷσε δέ μιν κρύψασα λόχῳ, ἐνέθηκε δὲ χερσὶν \\
ἅρπην καρχαρόδοντα, δόλον δ' ὑπεθήκατο πάντα.  \\
ἦλθε δὲ νύκτ' ἐπάγων μέγας Οὐρανός, ἀμφὶ δὲ Γαίῃ\\
ἱμείρων φιλότητος ἐπέσχετο, καί ῥ' ἐτανύσθη\\
πάντῃ· ὁ δ' ἐκ λοχέοιο πάις ὠρέξατο χειρὶ\\
σκαιῇ, δεξιτερῇ δὲ πελώριον ἔλλαβεν ἅρπην,\\
μακρὴν καρχαρόδοντα, φίλου δ' ἀπὸ μήδεα πατρὸς \\
ἐσσυμένως ἤμησε, πάλιν δ' ἔρριψε φέρεσθαι\\
ἐξοπίσω. τὰ μὲν οὔ τι ἐτώσια ἔκφυγε χειρός· \\
ὅσσαι γὰρ ῥαθάμιγγες ἀπέσσυθεν αἱματόεσσαι,\\
πάσας δέξατο Γαῖα· περιπλομένων δ' ἐνιαυτῶν \\
γείνατ' Ἐρινῦς τε κρατερὰς μεγάλους τε Γίγαντας, \\
τεύχεσι λαμπομένους, δολίχ' ἔγχεα χερσὶν ἔχοντας,\\
Νύμφας θ' ἃς Μελίας καλέουσ' ἐπ' ἀπείρονα γαῖαν. \\
μήδεα δ' ὡς τὸ πρῶτον ἀποτμήξας ἀδάμαντι\\
κάββαλ' ἀπ' ἠπείροιο πολυκλύστῳ ἐνὶ πόντῳ,\\
ὣς φέρετ' ἂμ πέλαγος πουλὺν χρόνον, ἀμφὶ δὲ λευκὸς \\
ἀφρὸς ἀπ' ἀθανάτου χροὸς ὤρνυτο· τῷ δ' ἔνι κούρη \\
ἐθρέφθη· πρῶτον δὲ Κυθήροισι ζαθέοισιν \\
ἔπλητ', ἔνθεν ἔπειτα περίρρυτον ἵκετο Κύπρον.\\
ἐκ δ' ἔβη αἰδοίη καλὴ θεός, ἀμφὶ δὲ ποίη\\
ποσσὶν ὕπο ῥαδινοῖσιν ἀέξετο· τὴν δ' Ἀφροδίτην  \\
ἀφρογενέα τε θεὰν καὶ ἐυστέφανον Κυθέρειαν\\
κικλήσκουσι θεοί τε καὶ ἀνέρες, οὕνεκ' ἐν ἀφρῷ\\
θρέφθη· ἀτὰρ Κυθέρειαν, ὅτι προσέκυρσε Κυθήροις· \\
Κυπρογενέα δ', ὅτι γέντο περικλύστῳ ἐνὶ Κύπρῳ·\\
ἠδὲ φιλομμειδέα, ὅτι μηδέων ἐξεφαάνθη. \\
τῇ δ' Ἔρος ὡμάρτησε καὶ Ἵμερος ἔσπετο καλὸς\\
γεινομένῃ τὰ πρῶτα θεῶν τ' ἐς φῦλον ἰούσῃ· \\
ταύτην δ' ἐξ ἀρχῆς τιμὴν ἔχει ἠδὲ λέλογχε\\
μοῖραν ἐν ἀνθρώποισι καὶ ἀθανάτοισι θεοῖσι,\\
παρθενίους τ' ὀάρους μειδήματά τ' ἐξαπάτας τε \\
τέρψίν τε γλυκερὴν φιλότητά τε μειλιχίην τε.\\

\quad{}τοὺς δὲ πατὴρ Τιτῆνας ἐπίκλησιν καλέεσκε\\
παῖδας νεικείων μέγας Οὐρανός, οὓς τέκεν αὐτός· \\
φάσκε δὲ τιταίνοντας ἀτασθαλίῃ μέγα ῥέξαι\\
ἔργον, τοῖο δ' ἔπειτα τίσιν μετόπισθεν ἔσεσθαι. \\

\quad{}Νὺξ δ' ἔτεκε στυγερόν τε Μόρον καὶ Κῆρα μέλαιναν \\
καὶ Θάνατον, τέκε δ' Ὕπνον, ἔτικτε δὲ φῦλον Ὀνείρων. \\
οὔ τινι κοιμηθεῖσα θεῶν τέκε Νὺξ ἐρεβεννή. \\
δεύτερον αὖ Μῶμον καὶ Ὀιζὺν ἀλγινόεσσαν\\
Ἑσπερίδας θ', αἷς μῆλα πέρην κλυτοῦ Ὠκεανοῖο  \\
χρύσεα καλὰ μέλουσι φέροντά τε δένδρεα καρπόν·\\
καὶ Μοίρας καὶ Κῆρας ἐγείνατο νηλεοποίνους,\\
{[}Κλωθώ τε Λάχεσίν τε καὶ Ἄτροπον, αἵ τε βροτοῖσι \\
γεινομένοισι διδοῦσιν ἔχειν ἀγαθόν τε κακόν τε,{]} \\
αἵ τ' ἀνδρῶν τε θεῶν τε παραιβασίας ἐφέπουσιν,  \\
οὐδέ ποτε λήγουσι θεαὶ δεινοῖο χόλοιο,\\
πρίν γ' ἀπὸ τῷ δώωσι κακὴν ὄπιν, ὅστις ἁμάρτῃ.\\
τίκτε δὲ καὶ Νέμεσιν πῆμα θνητοῖσι βροτοῖσι \\
Νὺξ ὀλοή· μετὰ τὴν δ' Ἀπάτην τέκε καὶ Φιλότητα \\
Γῆράς τ' οὐλόμενον, καὶ Ἔριν τέκε καρτερόθυμον. \\

\quad{}αὐτὰρ Ἔρις στυγερὴ τέκε μὲν Πόνον ἀλγινόεντα\\
Λήθην τε Λιμόν τε καὶ Ἄλγεα δακρυόεντα\\
Ὑσμίνας τε Μάχας τε Φόνους τ' Ἀνδροκτασίας τε\\
Νείκεά τε Ψεύδεά τε Λόγους τ' Ἀμφιλλογίας τε \\
Δυσνομίην τ' Ἄτην τε, συνήθεας ἀλλήλῃσιν, \\
Ὅρκόν θ', ὃς δὴ πλεῖστον ἐπιχθονίους ἀνθρώπους\\
πημαίνει, ὅτε κέν τις ἑκὼν ἐπίορκον ὀμόσσῃ· \\

{\minion\Para}
Νηρέα δ' ἀψευδέα καὶ ἀληθέα γείνατο Πόντος \\
πρεσβύτατον παίδων· αὐτὰρ καλέουσι γέροντα, \\
οὕνεκα νημερτής τε καὶ ἤπιος, οὐδὲ θεμίστων \\
λήθεται, ἀλλὰ δίκαια καὶ ἤπια δήνεα οἶδεν· \\
αὖτις δ' αὖ Θαύμαντα μέγαν καὶ ἀγήνορα Φόρκυν\\
Γαίῃ μισγόμενος καὶ Κητὼ καλλιπάρηον \\
Εὐρυβίην τ' ἀδάμαντος ἐνὶ φρεσὶ θυμὸν ἔχουσαν.\\
Νηρῆος δ' ἐγένοντο μεγήριτα τέκνα θεάων \\
πόντῳ ἐν ἀτρυγέτῳ καὶ Δωρίδος ἠυκόμοιο,\\
κούρης Ὠκεανοῖο τελήεντος ποταμοῖο, \\
Πρωθώ τ' Εὐκράντη τε Σαώ τ' Ἀμφιτρίτη τε \\
Εὐδώρη τε Θέτις τε Γαλήνη τε Γλαύκη τε,\\
Κυμοθόη Σπειώ τε θοὴ Θαλίη τ' ἐρόεσσα \\
Πασιθέη τ' Ἐρατώ τε καὶ Εὐνίκη ῥοδόπηχυς\\
καὶ Μελίτη χαρίεσσα καὶ Εὐλιμένη καὶ Ἀγαυὴ\\
Δωτώ τε Πρωτώ τε Φέρουσά τε Δυναμένη τε\\
Νησαίη τε καὶ Ἀκταίη καὶ Πρωτομέδεια,\\
Δωρὶς καὶ Πανόπη καὶ εὐειδὴς Γαλάτεια  \\
Ἱπποθόη τ' ἐρόεσσα καὶ Ἱππονόη ῥοδόπηχυς\\
Κυμοδόκη θ', ἣ κύματ' ἐν ἠεροειδέι πόντῳ\\
πνοιάς τε ζαέων ἀνέμων σὺν Κυματολήγῃ\\
ῥεῖα πρηΰνει καὶ ἐυσφύρῳ Ἀμφιτρίτῃ,\\
Κυμώ τ' Ἠιόνη τε ἐυστέφανός θ' Ἁλιμήδη \\
Γλαυκονόμη τε φιλομμειδὴς καὶ Ποντοπόρεια\\
Λειαγόρη τε καὶ Εὐαγόρη καὶ Λαομέδεια \\
Πουλυνόη τε καὶ Αὐτονόη καὶ Λυσιάνασσα\\
Εὐάρνη τε φυὴν ἐρατὴ καὶ εἶδος ἄμωμος\\
καὶ Ψαμάθη χαρίεσσα δέμας δίη τε Μενίππη \\
Νησώ τ' Εὐπόμπη τε Θεμιστώ τε Προνόη τε\\
Νημερτής θ', ἣ πατρὸς ἔχει νόον ἀθανάτοιο.\\
αὗται μὲν Νηρῆος ἀμύμονος ἐξεγένοντο\\
κοῦραι πεντήκοντα, ἀμύμονα ἔργ' εἰδυῖαι· \\

\quad{}Θαύμας δ' Ὠκεανοῖο βαθυρρείταο θύγατρα \\
ἠγάγετ' Ἠλέκτρην· ἡ δ' ὠκεῖαν τέκεν Ἶριν \\
ἠυκόμους θ' Ἁρπυίας, Ἀελλώ τ' Ὠκυπέτην τε, \\
αἵ ῥ' ἀνέμων πνοιῇσι καὶ οἰωνοῖς ἅμ' ἕπονται\\
ὠκείῃς πτερύγεσσι· μεταχρόνιαι γὰρ ἴαλλον. \\

{\minion\Para}
Φόρκυι δ' αὖ Κητὼ γραίας τέκε καλλιπαρήους \\
ἐκ γενετῆς πολιάς, τὰς δὴ Γραίας καλέουσιν\\
ἀθάνατοί τε θεοὶ χαμαὶ ἐρχόμενοί τ' ἄνθρωποι,\\
Πεμφρηδώ τ' εὔπεπλον Ἐνυώ τε κροκόπεπλον,\\
Γοργούς θ', αἳ ναίουσι πέρην κλυτοῦ Ὠκεανοῖο\\
ἐσχατιῇ πρὸς νυκτός, ἵν' Ἑσπερίδες λιγύφωνοι, \\
Σθεννώ τ' Εὐρυάλη τε Μέδουσά τε λυγρὰ παθοῦσα· \\
ἡ μὲν ἔην θνητή, αἱ δ' ἀθάνατοι καὶ ἀγήρῳ, \\
αἱ δύο· τῇ δὲ μιῇ παρελέξατο Κυανοχαίτης \\
ἐν μαλακῷ λειμῶνι καὶ ἄνθεσιν εἰαρινοῖσι. \\
τῆς ὅτε δὴ Περσεὺς κεφαλὴν ἀπεδειροτόμησεν,  \\
ἐξέθορε Χρυσάωρ τε μέγας καὶ Πήγασος ἵππος.\\
τῷ μὲν ἐπώνυμον ἦν, ὅτ' ἄρ' Ὠκεανοῦ παρὰ πηγὰς\\
γένθ', ὁ δ' ἄορ χρύσειον ἔχων μετὰ χερσὶ φίλῃσι. \\
χὠ μὲν ἀποπτάμενος, προλιπὼν χθόνα μητέρα μήλων,\\
ἵκετ' ἐς ἀθανάτους· Ζηνὸς δ' ἐν δώμασι ναίει  \\
βροντήν τε στεροπήν τε φέρων Διὶ μητιόεντι·\\
Χρυσάωρ δ' ἔτεκε τρικέφαλον Γηρυονῆα\\
μιχθεὶς Καλλιρόῃ κούρῃ κλυτοῦ Ὠκεανοῖο· \\
τὸν μὲν ἄρ' ἐξενάριξε βίη Ἡρακληείη\\
βουσὶ πάρ' εἰλιπόδεσσι περιρρύτῳ εἰν Ἐρυθείῃ \\
ἤματι τῷ, ὅτε περ βοῦς ἤλασεν εὐρυμετώπους \\
Τίρυνθ' εἰς ἱερήν, διαβὰς πόρον Ὠκεανοῖο, \\
Ὄρθόν τε κτείνας καὶ βουκόλον Εὐρυτίωνα\\
σταθμῷ ἐν ἠερόεντι πέρην κλυτοῦ Ὠκεανοῖο.\\

\quad{}ἡ δ' ἔτεκ' ἄλλο πέλωρον ἀμήχανον, οὐδὲν ἐοικὸς  \\
θνητοῖς ἀνθρώποις οὐδ' ἀθανάτοισι θεοῖσι, \\
σπῆι ἔνι γλαφυρῷ, θείην κρατερόφρον' Ἔχιδναν, \\
ἥμισυ μὲν νύμφην ἑλικώπιδα καλλιπάρηον, \\
ἥμισυ δ' αὖτε πέλωρον ὄφιν δεινόν τε μέγαν τε\\
αἰόλον ὠμηστήν, ζαθέης ὑπὸ κεύθεσι γαίης.  \\
ἔνθα δέ οἱ σπέος ἐστὶ κάτω κοίλῃ ὑπὸ πέτρῃ\\
τηλοῦ ἀπ' ἀθανάτων τε θεῶν θνητῶν τ' ἀνθρώπων,\\
ἔνθ' ἄρα οἱ δάσσαντο θεοὶ κλυτὰ δώματα ναίειν.\\

\quad{}ἡ δ' ἔρυτ' εἰν Ἀρίμοισιν ὑπὸ χθόνα λυγρὴ Ἔχιδνα, \\
ἀθάνατος νύμφη καὶ ἀγήραος ἤματα πάντα. \\
τῇ δὲ Τυφάονά φασι μιγήμεναι ἐν φιλότητι\\
δεινόν θ' ὑβριστήν τ' ἄνομόν θ' ἑλικώπιδι κούρῃ· \\
ἡ δ' ὑποκυσαμένη τέκετο κρατερόφρονα τέκνα.\\
Ὄρθον μὲν πρῶτον κύνα γείνατο Γηρυονῆι· \\
δεύτερον αὖτις ἔτικτεν ἀμήχανον, οὔ τι φατειόν,  \\
Κέρβερον ὠμηστήν, Ἀίδεω κύνα χαλκεόφωνον,\\
πεντηκοντακέφαλον, ἀναιδέα τε κρατερόν τε· \\
τὸ τρίτον Ὕδρην αὖτις ἐγείνατο λύγρ' εἰδυῖαν \\
Λερναίην, ἣν θρέψε θεὰ λευκώλενος Ἥρη\\
ἄπλητον κοτέουσα βίῃ Ἡρακληείῃ. \\
καὶ τὴν μὲν Διὸς υἱὸς ἐνήρατο νηλέι χαλκῷ\\
Ἀμφιτρυωνιάδης σὺν ἀρηιφίλῳ Ἰολάῳ\\
Ἡρακλέης βουλῇσιν Ἀθηναίης ἀγελείης·\\
ἡ δὲ Χίμαιραν ἔτικτε πνέουσαν ἀμαιμάκετον πῦρ,\\
δεινήν τε μεγάλην τε ποδώκεά τε κρατερήν τε.  \\
τῆς ἦν τρεῖς κεφαλαί· μία μὲν χαροποῖο λέοντος,\\
ἡ δὲ χιμαίρης, ἡ δ' ὄφιος κρατεροῖο δράκοντος.\\
πρόσθε λέων, ὄπιθεν δὲ δράκων, μέσση δὲ χίμαιρα,\\
δεινὸν ἀποπνείουσα πυρὸς μένος αἰθομένοιο.\\
τὴν μὲν Πήγασος εἷλε καὶ ἐσθλὸς Βελλεροφόντης. \\
ἡ δ' ἄρα Φῖκ' ὀλοὴν τέκε Καδμείοισιν ὄλεθρον,\\
Ὄρθῳ ὑποδμηθεῖσα, Νεμειαῖόν τε λέοντα, \\
τόν ῥ' Ἥρη θρέψασα Διὸς κυδρὴ παράκοιτις\\
γουνοῖσιν κατένασσε Νεμείης, πῆμ' ἀνθρώποις.\\
ἔνθ' ἄρ' ὅ γ' οἰκείων ἐλεφαίρετο φῦλ' ἀνθρώπων, \\
κοιρανέων Τρητοῖο Νεμείης ἠδ' Ἀπέσαντος· \\
ἀλλά ἑ ἲς ἐδάμασσε βίης Ἡρακληείης.\\

\quad{}Κητὼ δ' ὁπλότατον Φόρκυι φιλότητι μιγεῖσα\\
γείνατο δεινὸν ὄφιν, ὃς ἐρεμνῆς κεύθεσι γαίης\\
πείρασιν ἐν μεγάλοις παγχρύσεα μῆλα φυλάσσει. \\

{\minion\Para}
τοῦτο μὲν ἐκ Κητοῦς καὶ Φόρκυνος γένος ἐστί. \\
Τηθὺς δ' Ὠκεανῷ ποταμοὺς τέκε δινήεντας,\\
Νεῖλόν τ' Ἀλφειόν τε καὶ Ἠριδανὸν βαθυδίνην,\\
Στρυμόνα Μαίανδρόν τε καὶ Ἴστρον καλλιρέεθρον\\
Φᾶσίν τε Ῥῆσόν τ' Ἀχελῷόν τ' ἀργυροδίνην \\
Νέσσόν τε Ῥοδίον θ' Ἁλιάκμονά θ' Ἑπτάπορόν τε\\
Γρήνικόν τε καὶ Αἴσηπον θεῖόν τε Σιμοῦντα\\
Πηνειόν τε καὶ Ἕρμον ἐυρρείτην τε Κάικον\\
Σαγγάριόν τε μέγαν Λάδωνά τε Παρθένιόν τε\\
Εὔηνόν τε καὶ Ἀλδῆσκον θεῖόν τε Σκάμανδρον· \\
τίκτε δὲ θυγατέρων ἱερὸν γένος, αἳ κατὰ γαῖαν\\
ἄνδρας κουρίζουσι σὺν Ἀπόλλωνι ἄνακτι\\
καὶ ποταμοῖς, ταύτην δὲ Διὸς πάρα μοῖραν ἔχουσι,\\
Πειθώ τ' Ἀδμήτη τε Ἰάνθη τ' Ἠλέκτρη τε\\
Δωρίς τε Πρυμνώ τε καὶ Οὐρανίη θεοειδὴς \\
Ἱππώ τε Κλυμένη τε Ῥόδειά τε Καλλιρόη τε\\
Ζευξώ τε Κλυτίη τε Ἰδυῖά τε Πασιθόη τε\\
Πληξαύρη τε Γαλαξαύρη τ' ἐρατή τε Διώνη\\
Μηλόβοσίς τε Θόη τε καὶ εὐειδὴς Πολυδώρη \\
Κερκηίς τε φυὴν ἐρατὴ Πλουτώ τε βοῶπις \\
Περσηίς τ' Ἰάνειρά τ' Ἀκάστη τε Ξάνθη τε\\
Πετραίη τ' ἐρόεσσα Μενεσθώ τ' Εὐρώπη τε\\
Μῆτίς τ' Εὐρυνόμη τε Τελεστώ τε κροκόπεπλος\\
Χρυσηίς τ' Ἀσίη τε καὶ ἱμερόεσσα Καλυψὼ\\
Εὐδώρη τε Τύχη τε καὶ Ἀμφιρὼ Ὠκυρόη τε \\
καὶ Στύξ, ἣ δή σφεων προφερεστάτη ἐστὶν ἁπασέων.\\
αὗται ἄρ' Ὠκεανοῦ καὶ Τηθύος ἐξεγένοντο\\
πρεσβύταται κοῦραι· πολλαί γε μέν εἰσι καὶ ἄλλαι·\\
τρὶς γὰρ χίλιαί εἰσι τανίσφυροι Ὠκεανῖναι,\\
αἵ ῥα πολυσπερέες γαῖαν καὶ βένθεα λίμνης \\
πάντῃ ὁμῶς ἐφέπουσι, θεάων ἀγλαὰ τέκνα. \\
τόσσοι δ' αὖθ' ἕτεροι ποταμοὶ καναχηδὰ ῥέοντες,\\
υἱέες Ὠκεανοῦ, τοὺς γείνατο πότνια Τηθύς· \\
τῶν ὄνομ' ἀργαλέον πάντων βροτὸν ἄνδρα ἐνισπεῖν, \\
οἱ δὲ ἕκαστοι ἴσασιν, ὅσοι περιναιετάουσι.  \\

\quad{}Θεία δ' Ἠέλιόν τε μέγαν λαμπράν τε Σελήνην \\
Ἠῶ θ', ἣ πάντεσσιν ἐπιχθονίοισι φαείνει\\
ἀθανάτοις τε θεοῖσι τοὶ οὐρανὸν εὐρὺν ἔχουσι,\\
γείναθ' ὑποδμηθεῖσ' Ὑπερίονος ἐν φιλότητι. \\
Κρείῳ δ' Εὐρυβίη τέκεν ἐν φιλότητι μιγεῖσα  \\
Ἀστραῖόν τε μέγαν Πάλλαντά τε δῖα θεάων\\
Πέρσην θ', ὃς καὶ πᾶσι μετέπρεπεν ἰδμοσύνῃσιν.\\
Ἀστραίῳ δ' Ἠὼς ἀνέμους τέκε καρτεροθύμους,\\
ἀργεστὴν Ζέφυρον Βορέην τ' αἰψηροκέλευθον \\
καὶ Νότον, ἐν φιλότητι θεὰ θεῷ εὐνηθεῖσα. \\
τοὺς δὲ μέτ' ἀστέρα τίκτεν Ἑωσφόρον Ἠριγένεια\\
ἄστρά τε λαμπετόωντα, τά τ' οὐρανὸς ἐστεφάνωται. \\

\quad{}Στὺξ δ' ἔτεκ' Ὠκεανοῦ θυγάτηρ Πάλλαντι μιγεῖσα\\
Ζῆλον καὶ Νίκην καλλίσφυρον ἐν μεγάροισι \\
καὶ Κράτος ἠδὲ Βίην ἀριδείκετα γείνατο τέκνα.  \\
τῶν οὐκ ἔστ' ἀπάνευθε Διὸς δόμος, οὐδέ τις ἕδρη,\\
οὐδ' ὁδός, ὅππῃ μὴ κείνοις θεὸς ἡγεμονεύει,\\
ἀλλ' αἰεὶ πὰρ Ζηνὶ βαρυκτύπῳ ἑδριόωνται.\\
ὣς γὰρ ἐβούλευσε Στὺξ ἄφθιτος Ὠκεανίνη\\
ἤματι τῷ, ὅτε πάντας Ὀλύμπιος ἀστεροπητὴς \\
ἀθανάτους ἐκάλεσσε θεοὺς ἐς μακρὸν Ὄλυμπον,\\
εἶπε δ', ὃς ἂν μετὰ εἷο θεῶν Τιτῆσι μάχοιτο,\\
μή τιν' ἀπορραίσειν γεράων, τιμὴν δὲ ἕκαστον\\
ἑξέμεν, ἣν τὸ πάρος γε μετ' ἀθανάτοισι θεοῖσι.\\
τὸν δ' ἔφαθ', ὅστις ἄτιμος ὑπὸ Κρόνου ἠδ' ἀγέραστος, \\
τιμῆς καὶ γεράων ἐπιβησέμεν, ἣ θέμις ἐστίν.\\
ἦλθε δ' ἄρα πρώτη Στὺξ ἄφθιτος Οὔλυμπόνδε\\
σὺν σφοῖσιν παίδεσσι φίλου διὰ μήδεα πατρός· \\
τὴν δὲ Ζεὺς τίμησε, περισσὰ δὲ δῶρα ἔδωκεν.\\
αὐτὴν μὲν γὰρ ἔθηκε θεῶν μέγαν ἔμμεναι ὅρκον, \\
παῖδας δ' ἤματα πάντα ἑοῦ μεταναιέτας εἶναι.\\
ὣς δ' αὔτως πάντεσσι διαμπερές, ὥς περ ὑπέστη,\\
ἐξετέλεσσ'· αὐτὸς δὲ μέγα κρατεῖ ἠδὲ ἀνάσσει. \\

\quad{}Φοίβη δ' αὖ Κοίου πολυήρατον ἦλθεν ἐς εὐνήν· \\
κυσαμένη δἤπειτα θεὰ θεοῦ ἐν φιλότητι  \\
Λητὼ κυανόπεπλον ἐγείνατο, μείλιχον αἰεί,\\
ἤπιον ἀνθρώποισι καὶ ἀθανάτοισι θεοῖσι, \\
μείλιχον ἐξ ἀρχῆς, ἀγανώτατον ἐντὸς Ὀλύμπου.\\
γείνατο δ' Ἀστερίην εὐώνυμον, ἥν ποτε Πέρσης\\
ἠγάγετ' ἐς μέγα δῶμα φίλην κεκλῆσθαι ἄκοιτιν. \\

\quad{}ἡ δ' ὑποκυσαμένη Ἑκάτην τέκε, τὴν περὶ πάντων \\
Ζεὺς Κρονίδης τίμησε· πόρεν δέ οἱ ἀγλαὰ δῶρα, \\
μοῖραν ἔχειν γαίης τε καὶ ἀτρυγέτοιο θαλάσσης.\\
ἡ δὲ καὶ ἀστερόεντος ἀπ' οὐρανοῦ ἔμμορε τιμῆς, \\
ἀθανάτοις τε θεοῖσι τετιμένη ἐστὶ μάλιστα. \\
καὶ γὰρ νῦν, ὅτε πού τις ἐπιχθονίων ἀνθρώπων\\
ἔρδων ἱερὰ καλὰ κατὰ νόμον ἱλάσκηται,\\
κικλήσκει Ἑκάτην· πολλή τέ οἱ ἔσπετο τιμὴ \\
ῥεῖα μάλ', ᾧ πρόφρων γε θεὰ ὑποδέξεται εὐχάς,\\
καί τέ οἱ ὄλβον ὀπάζει, ἐπεὶ δύναμίς γε πάρεστιν. \\
ὅσσοι γὰρ Γαίης τε καὶ Οὐρανοῦ ἐξεγένοντο\\
καὶ τιμὴν ἔλαχον, τούτων ἔχει αἶσαν ἁπάντων· \\
οὐδέ τί μιν Κρονίδης ἐβιήσατο οὐδέ τ' ἀπηύρα,\\
ὅσσ' ἔλαχεν Τιτῆσι μέτα προτέροισι θεοῖσιν, \\
ἀλλ' ἔχει, ὡς τὸ πρῶτον ἀπ' ἀρχῆς ἔπλετο δασμός.  \\
οὐδ', ὅτι μουνογενής, ἧσσον θεὰ ἔμμορε τιμῆς \\
καὶ γεράων γαίῃ τε καὶ οὐρανῷ ἠδὲ θαλάσσῃ, \\
ἀλλ' ἔτι καὶ πολὺ μᾶλλον, ἐπεὶ Ζεὺς τίεται αὐτήν.\\
ᾧ δ' ἐθέλῃ, μεγάλως παραγίνεται ἠδ' ὀνίνησιν· \\
ἔν τ' ἀγορῇ λαοῖσι μεταπρέπει, ὅν κ' ἐθέλῃσιν·  \\
ἠδ' ὁπότ' ἐς πόλεμον φθισήνορα θωρήσσωνται\\
ἀνέρες, ἔνθα θεὰ παραγίνεται, οἷς κ' ἐθέλῃσι\\
νίκην προφρονέως ὀπάσαι καὶ κῦδος ὀρέξαι.\\
ἔν τε δίκῃ βασιλεῦσι παρ' αἰδοίοισι καθίζει,\\
ἐσθλὴ δ' αὖθ' ὁπότ' ἄνδρες ἀεθλεύωσ' ἐν ἀγῶνι· \\
ἔνθα θεὰ καὶ τοῖς παραγίνεται ἠδ' ὀνίνησι· \\
νικήσας δὲ βίῃ καὶ κάρτει, καλὸν ἄεθλον \\
ῥεῖα φέρει χαίρων τε, τοκεῦσι δὲ κῦδος ὀπάζει.\\
ἐσθλὴ δ' ἱππήεσσι παρεστάμεν, οἷς κ' ἐθέλῃσιν·\\
καὶ τοῖς, οἳ γλαυκὴν δυσπέμφελον ἐργάζονται, \\
εὔχονται δ' Ἑκάτῃ καὶ ἐρικτύπῳ Ἐννοσιγαίῳ,\\
ῥηιδίως ἄγρην κυδρὴ θεὸς ὤπασε πολλήν,\\
ῥεῖα δ' ἀφείλετο φαινομένην, ἐθέλουσά γε θυμῷ.\\
ἐσθλὴ δ' ἐν σταθμοῖσι σὺν Ἑρμῇ ληίδ' ἀέξειν· \\
βουκολίας δὲ βοῶν τε καὶ αἰπόλια πλατέ' αἰγῶν \\
ποίμνας τ' εἰροπόκων ὀίων, θυμῷ γ' ἐθέλουσα,\\
ἐξ ὀλίγων βριάει κἀκ πολλῶν μείονα θῆκεν.\\
οὕτω τοι καὶ μουνογενὴς ἐκ μητρὸς ἐοῦσα\\
πᾶσι μετ' ἀθανάτοισι τετίμηται γεράεσσι. \\
θῆκε δέ μιν Κρονίδης κουροτρόφον, οἳ μετ' ἐκείνην\\
ὀφθαλμοῖσιν ἴδοντο φάος πολυδερκέος Ἠοῦς.\\
οὕτως ἐξ ἀρχῆς κουροτρόφος, αἳ δέ τε τιμαί.\\

{\minion\Para}
Ῥείη δὲ δμηθεῖσα Κρόνῳ τέκε φαίδιμα τέκνα, \\
Ἱστίην Δήμητρα καὶ Ἥρην χρυσοπέδιλον, \\
ἴφθιμόν τ' Ἀίδην, ὃς ὑπὸ χθονὶ δώματα ναίει \\
νηλεὲς ἦτορ ἔχων, καὶ ἐρίκτυπον Ἐννοσίγαιον,\\
Ζῆνά τε μητιόεντα, θεῶν πατέρ' ἠδὲ καὶ ἀνδρῶν,\\
τοῦ καὶ ὑπὸ βροντῆς πελεμίζεται εὐρεῖα χθών.\\
καὶ τοὺς μὲν κατέπινε μέγας Κρόνος, ὥς τις ἕκαστος\\
νηδύος ἐξ ἱερῆς μητρὸς πρὸς γούναθ' ἵκοιτο, \\
τὰ φρονέων, ἵνα μή τις ἀγαυῶν Οὐρανιώνων\\
ἄλλος ἐν ἀθανάτοισιν ἔχοι βασιληίδα τιμήν.\\
πεύθετο γὰρ Γαίης τε καὶ Οὐρανοῦ ἀστερόεντος \\
οὕνεκά οἱ πέπρωτο ἑῷ ὑπὸ παιδὶ δαμῆναι, \\
καὶ κρατερῷ περ ἐόντι, Διὸς μεγάλου διὰ βουλάς.  \\
τῷ ὅ γ' ἄρ' οὐκ ἀλαοσκοπιὴν ἔχεν, ἀλλὰ δοκεύων \\
παῖδας ἑοὺς κατέπινε· Ῥέην δ' ἔχε πένθος ἄλαστον. \\
ἀλλ' ὅτε δὴ Δί' ἔμελλε θεῶν πατέρ' ἠδὲ καὶ ἀνδρῶν\\
τέξεσθαι, τότ' ἔπειτα φίλους λιτάνευε τοκῆας\\
τοὺς αὐτῆς, Γαῖάν τε καὶ Οὐρανὸν ἀστερόεντα, \\
μῆτιν συμφράσσασθαι, ὅπως λελάθοιτο τεκοῦσα\\
παῖδα φίλον, τείσαιτο δ' ἐρινῦς πατρὸς ἑοῖο \\
παίδων \textless{}θ'\textgreater{} οὓς κατέπινε μέγας Κρόνος ἀγκυλομήτης. \\
οἱ δὲ θυγατρὶ φίλῃ μάλα μὲν κλύον ἠδ' ἐπίθοντο, \\
καί οἱ πεφραδέτην, ὅσα περ πέπρωτο γενέσθαι \\
ἀμφὶ Κρόνῳ βασιλῆι καὶ υἱέι καρτεροθύμῳ· \\
πέμψαν δ' ἐς Λύκτον, Κρήτης ἐς πίονα δῆμον,\\
ὁππότ' ἄρ' ὁπλότατον παίδων ἤμελλε τεκέσθαι, \\
Ζῆνα μέγαν· τὸν μέν οἱ ἐδέξατο Γαῖα πελώρη \\
Κρήτῃ ἐν εὐρείῃ τρεφέμεν ἀτιταλλέμεναί τε.  \\
ἔνθά μιν ἷκτο φέρουσα θοὴν διὰ νύκτα μέλαιναν, \\
πρώτην ἐς Λύκτον· κρύψεν δέ ἑ χερσὶ λαβοῦσα \\
ἄντρῳ ἐν ἠλιβάτῳ, ζαθέης ὑπὸ κεύθεσι γαίης,\\
Αἰγαίῳ ἐν ὄρει πεπυκασμένῳ ὑλήεντι.\\
τῷ δὲ σπαργανίσασα μέγαν λίθον ἐγγυάλιξεν \\
Οὐρανίδῃ μέγ' ἄνακτι, θεῶν προτέρων βασιλῆι.\\
τὸν τόθ' ἑλὼν χείρεσσιν ἑὴν ἐσκάτθετο νηδύν, \\
σχέτλιος, οὐδ' ἐνόησε μετὰ φρεσίν, ὥς οἱ ὀπίσσω \\
ἀντὶ λίθου ἑὸς υἱὸς ἀνίκητος καὶ ἀκηδὴς\\
λείπεθ', ὅ μιν τάχ' ἔμελλε βίῃ καὶ χερσὶ δαμάσσας \\
τιμῆς ἐξελάαν, ὁ δ' ἐν ἀθανάτοισιν ἀνάξειν. \\

\quad{}καρπαλίμως δ' ἄρ' ἔπειτα μένος καὶ φαίδιμα γυῖα\\
ηὔξετο τοῖο ἄνακτος· ἐπιπλομένου δ' ἐνιαυτοῦ, \\
Γαίης ἐννεσίῃσι πολυφραδέεσσι δολωθείς, \\
ὃν γόνον ἂψ ἀνέηκε μέγας Κρόνος ἀγκυλομήτης,  \\
νικηθεὶς τέχνῃσι βίηφί τε παιδὸς ἑοῖο.\\
πρῶτον δ' ἐξήμησε λίθον, πύματον καταπίνων·\\
τὸν μὲν Ζεὺς στήριξε κατὰ χθονὸς εὐρυοδείης\\
Πυθοῖ ἐν ἠγαθέῃ, γυάλοις ὕπο Παρνησσοῖο, \\
σῆμ' ἔμεν ἐξοπίσω, θαῦμα θνητοῖσι βροτοῖσι.  \\

\quad{}λῦσε δὲ πατροκασιγνήτους ὀλοῶν ὑπὸ δεσμῶν, \\
Οὐρανίδας, οὓς δῆσε πατὴρ ἀεσιφροσύνῃσιν· \\
οἵ οἱ ἀπεμνήσαντο χάριν εὐεργεσιάων,\\
δῶκαν δὲ βροντὴν ἠδ' αἰθαλόεντα κεραυνὸν\\
καὶ στεροπήν· τὸ πρὶν δὲ πελώρη Γαῖα κεκεύθει·  \\
τοῖς πίσυνος θνητοῖσι καὶ ἀθανάτοισιν ἀνάσσει.\\

\quad{}κούρην δ' Ἰαπετὸς καλλίσφυρον Ὠκεανίνην\\
ἠγάγετο Κλυμένην καὶ ὁμὸν λέχος εἰσανέβαινεν.\\
ἡ δέ οἱ Ἄτλαντα κρατερόφρονα γείνατο παῖδα,\\
τίκτε δ' ὑπερκύδαντα Μενοίτιον ἠδὲ Προμηθέα, \\
ποικίλον αἰολόμητιν, ἁμαρτίνοόν τ' Ἐπιμηθέα· \\
ὃς κακὸν ἐξ ἀρχῆς γένετ' ἀνδράσιν ἀλφηστῇσι· \\
πρῶτος γάρ ῥα Διὸς πλαστὴν ὑπέδεκτο γυναῖκα\\
παρθένον. ὑβριστὴν δὲ Μενοίτιον εὐρύοπα Ζεὺς\\
εἰς ἔρεβος κατέπεμψε βαλὼν ψολόεντι κεραυνῷ  \\
εἵνεκ' ἀτασθαλίης τε καὶ ἠνορέης ὑπερόπλου.\\
Ἄτλας δ' οὐρανὸν εὐρὺν ἔχει κρατερῆς ὑπ' ἀνάγκης, \\
πείρασιν ἐν γαίης πρόπαρ' Ἑσπερίδων λιγυφώνων \\
ἑστηώς, κεφαλῇ τε καὶ ἀκαμάτῃσι χέρεσσι· \\
ταύτην γάρ οἱ μοῖραν ἐδάσσατο μητίετα Ζεύς. \\
δῆσε δ' ἀλυκτοπέδῃσι Προμηθέα ποικιλόβουλον,\\
δεσμοῖς ἀργαλέοισι, μέσον διὰ κίον' ἐλάσσας· \\
καί οἱ ἐπ' αἰετὸν ὦρσε τανύπτερον· αὐτὰρ ὅ γ' ἧπαρ \\
ἤσθιεν ἀθάνατον, τὸ δ' ἀέξετο ἶσον ἁπάντῃ \\
νυκτός, ὅσον πρόπαν ἦμαρ ἔδοι τανυσίπτερος ὄρνις.  \\
τὸν μὲν ἄρ' Ἀλκμήνης καλλισφύρου ἄλκιμος υἱὸς\\
Ἡρακλέης ἔκτεινε, κακὴν δ' ἀπὸ νοῦσον ἄλαλκεν\\
Ἰαπετιονίδῃ καὶ ἐλύσατο δυσφροσυνάων, \\
οὐκ ἀέκητι Ζηνὸς Ὀλυμπίου ὕψι μέδοντος,\\
ὄφρ' Ἡρακλῆος Θηβαγενέος κλέος εἴη \\
πλεῖον ἔτ' ἢ τὸ πάροιθεν ἐπὶ χθόνα πουλυβότειραν.\\
ταῦτ' ἄρα ἁζόμενος τίμα ἀριδείκετον υἱόν·\\
καί περ χωόμενος παύθη χόλου, ὃν πρὶν ἔχεσκεν,\\
οὕνεκ' ἐρίζετο βουλὰς ὑπερμενέι Κρονίωνι.\\

\quad{}καὶ γὰρ ὅτ' ἐκρίνοντο θεοὶ θνητοί τ' ἄνθρωποι \\
Μηκώνῃ, τότ' ἔπειτα μέγαν βοῦν πρόφρονι θυμῷ\\
δασσάμενος προύθηκε, Διὸς νόον ἐξαπαφίσκων. \\
τῷ μὲν γὰρ σάρκάς τε καὶ ἔγκατα πίονα δημῷ \\
ἐν ῥινῷ κατέθηκε, καλύψας γαστρὶ βοείῃ,\\
τοῖς δ' αὖτ' ὀστέα λευκὰ βοὸς δολίῃ ἐπὶ τέχνῃ  \\
εὐθετίσας κατέθηκε, καλύψας ἀργέτι δημῷ. \\

\quad{}δὴ τότε μιν προσέειπε πατὴρ ἀνδρῶν τε θεῶν τε· \\
``Ἰαπετιονίδη, πάντων ἀριδείκετ' ἀνάκτων, \\
ὦ πέπον, ὡς ἑτεροζήλως διεδάσσαο μοίρας.'' \\

\quad{}ὣς φάτο κερτομέων Ζεὺς ἄφθιτα μήδεα εἰδώς·  \\
τὸν δ' αὖτε προσέειπε Προμηθεὺς ἀγκυλομήτης,\\
ἦκ' ἐπιμειδήσας, δολίης δ' οὐ λήθετο τέχνης· \\
``Ζεῦ κύδιστε μέγιστε θεῶν αἰειγενετάων, \\
τῶν δ' ἕλευ ὁπποτέρην σε ἐνὶ φρεσὶ θυμὸς ἀνώγει.'' \\

\quad{}φῆ ῥα δολοφρονέων· Ζεὺς δ' ἄφθιτα μήδεα εἰδὼς  \\
γνῶ ῥ' οὐδ' ἠγνοίησε δόλον· κακὰ δ' ὄσσετο θυμῷ \\
θνητοῖς ἀνθρώποισι, τὰ καὶ τελέεσθαι ἔμελλε.\\
χερσὶ δ' ὅ γ' ἀμφοτέρῃσιν ἀνείλετο λευκὸν ἄλειφαρ, \\
χώσατο δὲ φρένας ἀμφί, χόλος δέ μιν ἵκετο θυμόν,\\
ὡς ἴδεν ὀστέα λευκὰ βοὸς δολίῃ ἐπὶ τέχῃ. \\
ἐκ τοῦ δ' ἀθανάτοισιν ἐπὶ χθονὶ φῦλ' ἀνθρώπων\\
καίουσ' ὀστέα λευκὰ θυηέντων ἐπὶ βωμῶν. \\

\quad{}τὸν δὲ μέγ' ὀχθήσας προσέφη νεφεληγερέτα Ζεύς· \\
``Ἰαπετιονίδη, πάντων πέρι μήδεα εἰδώς, \\
ὦ πέπον, οὐκ ἄρα πω δολίης ἐπελήθεο τέχνης.''  \\

\quad{}ὣς φάτο χωόμενος Ζεὺς ἄφθιτα μήδεα εἰδώς. \\
ἐκ τούτου δἤπειτα χόλου μεμνημένος αἰεὶ \\
οὐκ ἐδίδου μελίῃσι πυρὸς μένος ἀκαμάτοιο\\
θνητοῖς ἀνθρώποις οἳ ἐπὶ χθονὶ ναιετάουσιν· \\
ἀλλά μιν ἐξαπάτησεν ἐὺς πάις Ἰαπετοῖο \\
κλέψας ἀκαμάτοιο πυρὸς τηλέσκοπον αὐγὴν\\
ἐν κοίλῳ νάρθηκι· δάκεν δ' ἄρα νειόθι θυμὸν\\
Ζῆν' ὑψιβρεμέτην, ἐχόλωσε δέ μιν φίλον ἦτορ,\\
ὡς ἴδ' ἐν ἀνθρώποισι πυρὸς τηλέσκοπον αὐγήν.\\
αὐτίκα δ' ἀντὶ πυρὸς τεῦξεν κακὸν ἀνθρώποισι·  \\
γαίης γὰρ σύμπλασσε περικλυτὸς Ἀμφιγυήεις\\
παρθένῳ αἰδοίῃ ἴκελον Κρονίδεω διὰ βουλάς· \\
ζῶσε δὲ καὶ κόσμησε θεὰ γλαυκῶπις Ἀθήνη\\
ἀργυφέῃ ἐσθῆτι· κατὰ κρῆθεν δὲ καλύπτρην \\
δαιδαλέην χείρεσσι κατέσχεθε, θαῦμα ἰδέσθαι·  \\
ἀμφὶ δέ οἱ στεφάνους νεοθηλέας, ἄνθεα ποίης,\\
ἱμερτοὺς περίθηκε καρήατι Παλλὰς Ἀθήνη· \\
ἀμφὶ δέ οἱ στεφάνην χρυσέην κεφαλῆφιν ἔθηκε,\\
τὴν αὐτὸς ποίησε περικλυτὸς Ἀμφιγυήεις\\
ἀσκήσας παλάμῃσι, χαριζόμενος Διὶ πατρί. \\
τῇ δ' ἔνι δαίδαλα πολλὰ τετεύχατο, θαῦμα ἰδέσθαι,\\
κνώδαλ' ὅσ' ἤπειρος δεινὰ τρέφει ἠδὲ θάλασσα·\\
τῶν ὅ γε πόλλ' ἐνέθηκε, χάρις δ' ἐπὶ πᾶσιν ἄητο,\\
θαυμάσια, ζωοῖσιν ἐοικότα φωνήεσσιν.\\

\quad{}αὐτὰρ ἐπεὶ δὴ τεῦξε καλὸν κακὸν ἀντ' ἀγαθοῖο,  \\
ἐξάγαγ' ἔνθά περ ἄλλοι ἔσαν θεοὶ ἠδ' ἄνθρωποι,\\
κόσμῳ ἀγαλλομένην γλαυκώπιδος Ὀβριμοπάτρης· \\
θαῦμα δ' ἔχ' ἀθανάτους τε θεοὺς θνητούς τ' ἀνθρώπους,\\
ὡς εἶδον δόλον αἰπύν, ἀμήχανον ἀνθρώποισιν.\\
ἐκ τῆς γὰρ γένος ἐστὶ γυναικῶν θηλυτεράων, \\
τῆς γὰρ ὀλοίιόν ἐστι γένος καὶ φῦλα γυναικῶν,\\
πῆμα μέγα θνητοῖσι, σὺν ἀνδράσι ναιετάουσαι,\\
οὐλομένης Πενίης οὐ σύμφοροι, ἀλλὰ Κόροιο.\\
ὡς δ' ὁπότ' ἐν σμήνεσσι κατηρεφέεσσι μέλισσαι\\
κηφῆνας βόσκωσι, κακῶν ξυνήονας ἔργων· \\
αἱ μέν τε πρόπαν ἦμαρ ἐς ἠέλιον καταδύντα\\
ἠμάτιαι σπεύδουσι τιθεῖσί τε κηρία λευκά,\\
οἱ δ' ἔντοσθε μένοντες ἐπηρεφέας κατὰ σίμβλους \\
ἀλλότριον κάματον σφετέρην ἐς γαστέρ' ἀμῶνται· \\
ὣς δ' αὔτως ἄνδρεσσι κακὸν θνητοῖσι γυναῖκας \\
Ζεὺς ὑψιβρεμέτης θῆκε, ξυνήονας ἔργων\\
ἀργαλέων. ἕτερον δὲ πόρεν κακὸν ἀντ' ἀγαθοῖο,\\
ὅς κε γάμον φεύγων καὶ μέρμερα ἔργα γυναικῶν\\
μὴ γῆμαι ἐθέλῃ, ὀλοὸν δ' ἐπὶ γῆρας ἵκηται\\
χήτει γηροκόμοιο· ὁ δ' οὐ βιότου γ' ἐπιδευὴς  \\
ζώει, ἀποφθιμένου δὲ διὰ ζωὴν δατέονται\\
χηρωσταί. ᾧ δ' αὖτε γάμου μετὰ μοῖρα γένηται, \\
κεδνὴν δ' ἔσχεν ἄκοιτιν, ἀρηρυῖαν πραπίδεσσι, \\
τῷ δέ τ' ἀπ' αἰῶνος κακὸν ἐσθλῷ ἀντιφερίζει\\
ἐμμενές· ὃς δέ κε τέτμῃ ἀταρτηροῖο γενέθλης,  \\
ζώει ἐνὶ στήθεσσιν ἔχων ἀλίαστον ἀνίην\\
θυμῷ καὶ κραδίῃ, καὶ ἀνήκεστον κακόν ἐστιν.\\

\quad{}ὣς οὐκ ἔστι Διὸς κλέψαι νόον οὐδὲ παρελθεῖν.\\
οὐδὲ γὰρ Ἰαπετιονίδης ἀκάκητα Προμηθεὺς\\
τοῖό γ' ὑπεξήλυξε βαρὺν χόλον, ἀλλ' ὑπ' ἀνάγκης \\
καὶ πολύιδριν ἐόντα μέγας κατὰ δεσμὸς ἐρύκει.\\

{\minion\Para}
Ὀβριάρεῳ δ' ὡς πρῶτα πατὴρ ὠδύσσατο θυμῷ \\
Κόττῳ τ' ἠδὲ Γύγῃ, δῆσε κρατερῷ ἐνὶ δεσμῷ, \\
ἠνορέην ὑπέροπλον ἀγώμενος ἠδὲ καὶ εἶδος\\
καὶ μέγεθος· κατένασσε δ' ὑπὸ χθονὸς εὐρυοδείης.  \\
ἔνθ' οἵ γ' ἄλγε' ἔχοντες ὑπὸ χθονὶ ναιετάοντες\\
εἵατ' ἐπ' ἐσχατιῇ μεγάλης ἐν πείρασι γαίης \\
δηθὰ μάλ' ἀχνύμενοι, κραδίῃ μέγα πένθος ἔχοντες.\\
ἀλλά σφεας Κρονίδης τε καὶ ἀθάνατοι θεοὶ ἄλλοι \\
οὓς τέκεν ἠύκομος Ῥείη Κρόνου ἐν φιλότητι  \\
Γαίης φραδμοσύνῃσιν ἀνήγαγον ἐς φάος αὖτις· \\
αὐτὴ γάρ σφιν ἅπαντα διηνεκέως κατέλεξε, \\
σὺν κείνοις νίκην τε καὶ ἀγλαὸν εὖχος ἀρέσθαι.\\
δηρὸν γὰρ μάρναντο πόνον θυμαλγέ' ἔχοντες\\
ἀντίον ἀλλήλοισι διὰ κρατερὰς ὑσμίνας   \\
Τιτῆνές τε θεοὶ καὶ ὅσοι Κρόνου ἐξεγένοντο, \\
οἱ μὲν ἀφ' ὑψηλῆς Ὄθρυος Τιτῆνες ἀγαυοί,  \\
οἱ δ' ἄρ' ἀπ' Οὐλύμποιο θεοὶ δωτῆρες ἐάων \\
οὓς τέκεν ἠύκομος Ῥείη Κρόνῳ εὐνηθεῖσα.\\
οἵ ῥα τότ' ἀλλήλοισι πόνον θυμαλγέ' ἔχοντες  \\
συνεχέως ἐμάχοντο δέκα πλείους ἐνιαυτούς· \\
οὐδέ τις ἦν ἔριδος χαλεπῆς λύσις οὐδὲ τελευτὴ\\
οὐδετέροις, ἶσον δὲ τέλος τέτατο πτολέμοιο.\\

\quad{}ἀλλ' ὅτε δὴ κείνοισι παρέσχεθεν ἄρμενα πάντα,\\
νέκταρ τ' ἀμβροσίην τε, τά περ θεοὶ αὐτοὶ ἔδουσι, \\
πάντων \textless{}τ'\textgreater{} ἐν στήθεσσιν ἀέξετο θυμὸς ἀγήνωρ,\\
ὡς νέκταρ τ' ἐπάσαντο καὶ ἀμβροσίην ἐρατεινήν,\\
δὴ τότε τοῖς μετέειπε πατὴρ ἀνδρῶν τε θεῶν τε·\\
``κέκλυτέ μευ Γαίης τε καὶ Οὐρανοῦ ἀγλαὰ τέκνα, \\
ὄφρ' εἴπω τά με θυμὸς ἐνὶ στήθεσσι κελεύει.  \\
ἤδη γὰρ μάλα δηρὸν ἐναντίοι ἀλλήλοισι\\
νίκης καὶ κάρτευς πέρι μαρνάμεθ' ἤματα πάντα, \\
Τιτῆνές τε θεοὶ καὶ ὅσοι Κρόνου ἐκγενόμεσθα.\\
ὑμεῖς δὲ μεγάλην τε βίην καὶ χεῖρας ἀάπτους\\
φαίνετε Τιτήνεσσιν ἐναντίον ἐν δαῒ λυγρῇ, \\
μνησάμενοι φιλότητος ἐνηέος, ὅσσα παθόντες\\
ἐς φάος ἂψ ἀφίκεσθε δυσηλεγέος ὑπὸ δεσμοῦ\\
ἡμετέρας διὰ βουλὰς ὑπὸ ζόφου ἠερόεντος.''\\

\quad{}ὣς φάτο· τὸν δ' αἶψ' αὖτις ἀμείβετο Κόττος ἀμύμων· \\
``δαιμόνι', οὐκ ἀδάητα πιφαύσκεαι, ἀλλὰ καὶ αὐτοὶ  \\
ἴδμεν ὅ τοι περὶ μὲν πραπίδες, περὶ δ' ἐστὶ νόημα,\\
ἀλκτὴρ δ' ἀθανάτοισιν ἀρῆς γένεο κρυεροῖο, \\
σῇσι δ' ἐπιφροσύνῃσιν ὑπὸ ζόφου ἠερόεντος\\
ἄψορρον ἐξαῦτις ἀμειλίκτων ὑπὸ δεσμῶν\\
ἠλύθομεν, Κρόνου υἱὲ ἄναξ, ἀνάελπτα παθόντες. \\
τῷ καὶ νῦν ἀτενεῖ τε νόῳ καὶ πρόφρονι θυμῷ\\
ῥυσόμεθα κράτος ὑμὸν ἐν αἰνῇ δηιοτῆτι, \\
μαρνάμενοι Τιτῆσιν ἀνὰ κρατερὰς ὑσμίνας.'' \\
ὣς φάτ'· ἐπῄνησαν δὲ θεοὶ δωτῆρες ἐάων \\
μῦθον ἀκούσαντες· πολέμου δ' ἐλιλαίετο θυμὸς  \\
μᾶλλον ἔτ' ἢ τὸ πάροιθε· μάχην δ' ἀμέγαρτον ἔγειραν \\
πάντες, θήλειαί τε καὶ ἄρσενες, ἤματι κείνῳ,\\
Τιτῆνές τε θεοὶ καὶ ὅσοι Κρόνου ἐξεγένοντο,\\
οὕς τε Ζεὺς ἐρέβεσφιν ὑπὸ χθονὸς ἧκε φόωσδε,\\
δεινοί τε κρατεροί τε, βίην ὑπέροπλον ἔχοντες. \\
τῶν ἑκατὸν μὲν χεῖρες ἀπ' ὤμων ἀίσσοντο\\
πᾶσιν ὁμῶς, κεφαλαὶ δὲ ἑκάστῳ πεντήκοντα\\
ἐξ ὤμων ἐπέφυκον ἐπὶ στιβαροῖσι μέλεσσιν.\\
οἳ τότε Τιτήνεσσι κατέσταθεν ἐν δαῒ λυγρῇ\\
πέτρας ἠλιβάτους στιβαρῇς ἐν χερσὶν ἔχοντες·  \\
Τιτῆνες δ' ἑτέρωθεν ἐκαρτύναντο φάλαγγας\\
προφρονέως· χειρῶν τε βίης θ' ἅμα ἔργον ἔφαινον \\
ἀμφότεροι, δεινὸν δὲ περίαχε πόντος ἀπείρων,\\
γῆ δὲ μέγ' ἐσμαράγησεν, ἐπέστενε δ' οὐρανὸς εὐρὺς\\
σειόμενος, πεδόθεν δὲ τινάσσετο μακρὸς Ὄλυμπος \\
ῥιπῇ ὕπ' ἀθανάτων, ἔνοσις δ' ἵκανε βαρεῖα\\
τάρταρον ἠερόεντα ποδῶν αἰπεῖά τ' ἰωὴ\\
ἀσπέτου ἰωχμοῖο βολάων τε κρατεράων. \\
ὣς ἄρ' ἐπ' ἀλλήλοις ἵεσαν βέλεα στονόεντα· \\
φωνὴ δ' ἀμφοτέρων ἵκετ' οὐρανὸν ἀστερόεντα \\
κεκλομένων· οἱ δὲ ξύνισαν μεγάλῳ ἀλαλητῷ. \\

\quad{}οὐδ' ἄρ' ἔτι Ζεὺς ἴσχεν ἑὸν μένος, ἀλλά νυ τοῦ γε\\
εἶθαρ μὲν μένεος πλῆντο φρένες, ἐκ δέ τε πᾶσαν\\
φαῖνε βίην· ἄμυδις δ' ἄρ' ἀπ' οὐρανοῦ ἠδ' ἀπ' Ὀλύμπου \\
ἀστράπτων ἔστειχε συνωχαδόν, οἱ δὲ κεραυνοὶ  \\
ἴκταρ ἅμα βροντῇ τε καὶ ἀστεροπῇ ποτέοντο\\
χειρὸς ἄπο στιβαρῆς, ἱερὴν φλόγα εἰλυφόωντες, \\
ταρφέες· ἀμφὶ δὲ γαῖα φερέσβιος ἐσμαράγιζε \\
καιομένη, λάκε δ' ἀμφὶ περὶ μεγάλ' ἄσπετος ὕλη·\\
ἔζεε δὲ χθὼν πᾶσα καὶ Ὠκεανοῖο ῥέεθρα \\
πόντός τ' ἀτρύγετος· τοὺς δ' ἄμφεπε θερμὸς ἀυτμὴ \\
Τιτῆνας χθονίους, φλὸξ δ' αἰθέρα δῖαν ἵκανεν\\
ἄσπετος, ὄσσε δ' ἄμερδε καὶ ἰφθίμων περ ἐόντων\\
αὐγὴ μαρμαίρουσα κεραυνοῦ τε στεροπῆς τε.\\
καῦμα δὲ θεσπέσιον κάτεχεν χάος· εἴσατο δ' ἄντα  \\
ὀφθαλμοῖσιν ἰδεῖν ἠδ' οὔασιν ὄσσαν ἀκοῦσαι\\
αὔτως, ὡς ὅτε γαῖα καὶ οὐρανὸς εὐρὺς ὕπερθε \\
πίλνατο· τοῖος γάρ κε μέγας ὑπὸ δοῦπος ὀρώρει, \\
τῆς μὲν ἐρειπομένης, τοῦ δ' ὑψόθεν ἐξεριπόντος· \\
τόσσος δοῦπος ἔγεντο θεῶν ἔριδι ξυνιόντων. \\
σὺν δ' ἄνεμοι ἔνοσίν τε κονίην τ' ἐσφαράγιζον\\
βροντήν τε στεροπήν τε καὶ αἰθαλόεντα κεραυνόν,\\
κῆλα Διὸς μεγάλοιο, φέρον δ' ἰαχήν τ' ἐνοπήν τε\\
ἐς μέσον ἀμφοτέρων· ὄτοβος δ' ἄπλητος ὀρώρει \\
σμερδαλέης ἔριδος, κάρτευς δ' ἀνεφαίνετο ἔργον. \\

\quad{}ἐκλίνθη δὲ μάχη· πρὶν δ' ἀλλήλοις ἐπέχοντες \\
ἐμμενέως ἐμάχοντο διὰ κρατερὰς ὑσμίνας.\\
οἱ δ' ἄρ' ἐνὶ πρώτοισι μάχην δριμεῖαν ἔγειραν, \\
Κόττος τε Βριάρεώς τε Γύγης τ' ἄατος πολέμοιο·\\
οἵ ῥα τριηκοσίας πέτρας στιβαρέων ἀπὸ χειρῶν  \\
πέμπον ἐπασσυτέρας, κατὰ δ' ἐσκίασαν βελέεσσι\\
Τιτῆνας· καὶ τοὺς μὲν ὑπὸ χθονὸς εὐρυοδείης \\
πέμψαν καὶ δεσμοῖσιν ἐν ἀργαλέοισιν ἔδησαν,\\
νικήσαντες χερσὶν ὑπερθύμους περ ἐόντας, \\
τόσσον ἔνερθ' ὑπὸ γῆς ὅσον οὐρανός ἐστ' ἀπὸ γαίης·  \\

{\minion\Para}
τόσσον γάρ τ' ἀπὸ γῆς ἐς τάρταρον ἠερόεντα.\\
ἐννέα γὰρ νύκτας τε καὶ ἤματα χάλκεος ἄκμων\\
οὐρανόθεν κατιών, δεκάτῃ κ' ἐς γαῖαν ἵκοιτο· \\
\skipnumbering{[}ἶσον δ' αὖτ' ἀπὸ γῆς ἐς τάρταρον ἠερόεντα·{]}\\
ἐννέα δ' αὖ νύκτας τε καὶ ἤματα χάλκεος ἄκμων\\
ἐκ γαίης κατιών, δεκάτῃ κ' ἐς τάρταρον ἵκοι.  \\
τὸν πέρι χάλκεον ἕρκος ἐλήλαται· ἀμφὶ δέ μιν νὺξ \\
τριστοιχὶ κέχυται περὶ δειρήν· αὐτὰρ ὕπερθε\\
γῆς ῥίζαι πεφύασι καὶ ἀτρυγέτοιο θαλάσσης.\\

\quad{}ἔνθα θεοὶ Τιτῆνες ὑπὸ ζόφῳ ἠερόεντι\\
κεκρύφαται βουλῇσι Διὸς νεφεληγερέταο,  \\
χώρῳ ἐν εὐρώεντι, πελώρης ἔσχατα γαίης.\\
τοῖς οὐκ ἐξιτόν ἐστι, θύρας δ' ἐπέθηκε Ποσειδέων \\
χαλκείας, τεῖχος δ' ἐπελήλαται ἀμφοτέρωθεν.\\

\quad{}ἔνθα Γύγης Κόττος τε καὶ Ὀβριάρεως μεγάθυμος \\
ναίουσιν, φύλακες πιστοὶ Διὸς αἰγιόχοιο. \\

\quad{}ἔνθα δὲ γῆς δνοφερῆς καὶ ταρτάρου ἠερόεντος \\
πόντου τ' ἀτρυγέτοιο καὶ οὐρανοῦ ἀστερόεντος\\
ἑξείης πάντων πηγαὶ καὶ πείρατ' ἔασιν,\\
ἀργαλέ' εὐρώεντα, τά τε στυγέουσι θεοί περ·\\
χάσμα μέγ', οὐδέ κε πάντα τελεσφόρον εἰς ἐνιαυτὸν \\
οὖδας ἵκοιτ', εἰ πρῶτα πυλέων ἔντοσθε γένοιτο,\\
ἀλλά κεν ἔνθα καὶ ἔνθα φέροι πρὸ θύελλα θυέλλης \\
ἀργαλέη· δεινὸν δὲ καὶ ἀθανάτοισι θεοῖσι \\
τοῦτο τέρας· καὶ Νυκτὸς ἐρεμνῆς οἰκία δεινὰ\\
ἕστηκεν νεφέλῃς κεκαλυμμένα κυανέῃσι.  \\

\quad{}τῶν πρόσθ' Ἰαπετοῖο πάις ἔχει οὐρανὸν εὐρὺν\\
ἑστηὼς κεφαλῇ τε καὶ ἀκαμάτῃσι χέρεσσιν\\
ἀστεμφέως, ὅθι Νύξ τε καὶ Ἡμέρη ἆσσον ἰοῦσαι\\
ἀλλήλας προσέειπον ἀμειβόμεναι μέγαν οὐδὸν \\
χάλκεον· ἡ μὲν ἔσω καταβήσεται, ἡ δὲ θύραζε  \\
ἔρχεται, οὐδέ ποτ' ἀμφοτέρας δόμος ἐντὸς ἐέργει,\\
ἀλλ' αἰεὶ ἑτέρη γε δόμων ἔκτοσθεν ἐοῦσα\\
γαῖαν ἐπιστρέφεται, ἡ δ' αὖ δόμου ἐντὸς ἐοῦσα \\
μίμνει τὴν αὐτῆς ὥρην ὁδοῦ, ἔστ' ἂν ἵκηται· \\
ἡ μὲν ἐπιχθονίοισι φάος πολυδερκὲς ἔχουσα,  \\
ἡ δ' Ὕπνον μετὰ χερσί, κασίγνητον Θανάτοιο, \\
Νὺξ ὀλοή, νεφέλῃ κεκαλυμμένη ἠεροειδεῖ.\\

\quad{}ἔνθα δὲ Νυκτὸς παῖδες ἐρεμνῆς οἰκί' ἔχουσιν,\\
Ὕπνος καὶ Θάνατος, δεινοὶ θεοί· οὐδέ ποτ' αὐτοὺς \\
Ἠέλιος φαέθων ἐπιδέρκεται ἀκτίνεσσιν \\
οὐρανὸν εἰσανιὼν οὐδ' οὐρανόθεν καταβαίνων. \\
τῶν ἕτερος μὲν γῆν τε καὶ εὐρέα νῶτα θαλάσσης \\
ἥσυχος ἀνστρέφεται καὶ μείλιχος ἀνθρώποισι,\\
τοῦ δὲ σιδηρέη μὲν κραδίη, χάλκεον δέ οἱ ἦτορ\\
νηλεὲς ἐν στήθεσσιν· ἔχει δ' ὃν πρῶτα λάβῃσιν  \\
ἀνθρώπων· ἐχθρὸς δὲ καὶ ἀθανάτοισι θεοῖσιν. \\

\quad{}ἔνθα θεοῦ χθονίου πρόσθεν δόμοι ἠχήεντες\\
ἰφθίμου τ' Ἀίδεω καὶ ἐπαινῆς Περσεφονείης\\
ἑστᾶσιν, δεινὸς δὲ κύων προπάροιθε φυλάσσει, \\
νηλειής, τέχνην δὲ κακὴν ἔχει· ἐς μὲν ἰόντας  \\
σαίνει ὁμῶς οὐρῇ τε καὶ οὔασιν ἀμφοτέροισιν,\\
ἐξελθεῖν δ' οὐκ αὖτις ἐᾷ πάλιν, ἀλλὰ δοκεύων\\
ἐσθίει, ὅν κε λάβῃσι πυλέων ἔκτοσθεν ἰόντα.\\
ἰφθίμου τ' Ἀίδεω καὶ ἐπαινῆς Περσεφονείης.\\

\quad{}ἔνθα δὲ ναιετάει στυγερὴ θεὸς ἀθανάτοισι, \\
δεινὴ Στύξ, θυγάτηρ ἀψορρόου Ὠκεανοῖο\\
πρεσβυτάτη· νόσφιν δὲ θεῶν κλυτὰ δώματα ναίει \\
μακρῇσιν πέτρῃσι κατηρεφέ'· ἀμφὶ δὲ πάντῃ \\
κίοσιν ἀργυρέοισι πρὸς οὐρανὸν ἐστήρικται.\\
παῦρα δὲ Θαύμαντος θυγάτηρ πόδας ὠκέα Ἶρις \\
ἀγγελίῃ πωλεῖται ἐπ' εὐρέα νῶτα θαλάσσης. \\
ὁππότ' ἔρις καὶ νεῖκος ἐν ἀθανάτοισιν ὄρηται, \\
καί ῥ' ὅστις ψεύδηται Ὀλύμπια δώματ' ἐχόντων,\\
Ζεὺς δέ τε Ἶριν ἔπεμψε θεῶν μέγαν ὅρκον ἐνεῖκαι\\
τηλόθεν ἐν χρυσέῃ προχόῳ πολυώνυμον ὕδωρ,  \\
ψυχρόν, ὅ τ' ἐκ πέτρης καταλείβεται ἠλιβάτοιο \\
ὑψηλῆς· πολλὸν δὲ ὑπὸ χθονὸς εὐρυοδείης \\
ἐξ ἱεροῦ ποταμοῖο ῥέει διὰ νύκτα μέλαιναν· \\
Ὠκεανοῖο κέρας, δεκάτη δ' ἐπὶ μοῖρα δέδασται· \\
ἐννέα μὲν περὶ γῆν τε καὶ εὐρέα νῶτα θαλάσσης \\
δίνῃς ἀργυρέῃς εἱλιγμένος εἰς ἅλα πίπτει,\\
ἡ δὲ μί' ἐκ πέτρης προρέει, μέγα πῆμα θεοῖσιν. \\
ὅς κεν τὴν ἐπίορκον ἀπολλείψας ἐπομόσσῃ\\
ἀθανάτων οἳ ἔχουσι κάρη νιφόεντος Ὀλύμπου,\\
κεῖται νήυτμος τετελεσμένον εἰς ἐνιαυτόν·  \\
οὐδέ ποτ' ἀμβροσίης καὶ νέκταρος ἔρχεται ἆσσον\\
βρώσιος, ἀλλά τε κεῖται ἀνάπνευστος καὶ ἄναυδος\\
στρωτοῖς ἐν λεχέεσσι, κακὸν δ' ἐπὶ κῶμα καλύπτει.\\
αὐτὰρ ἐπὴν νοῦσον τελέσει μέγαν εἰς ἐνιαυτόν,\\
ἄλλος δ' ἐξ ἄλλου δέχεται χαλεπώτερος ἆθλος·  \\
εἰνάετες δὲ θεῶν ἀπαμείρεται αἰὲν ἐόντων,\\
οὐδέ ποτ' ἐς βουλὴν ἐπιμίσγεται οὐδ' ἐπὶ δαῖτας\\
ἐννέα πάντ' ἔτεα· δεκάτῳ δ' ἐπιμίσγεται αὖτις \\
εἴρας ἐς ἀθανάτων οἳ Ὀλύμπια δώματ' ἔχουσι. \\
τοῖον ἄρ' ὅρκον ἔθεντο θεοὶ Στυγὸς ἄφθιτον ὕδωρ,  \\
ὠγύγιον· τὸ δ' ἵησι καταστυφέλου διὰ χώρου. \\

\quad{}ἔνθα δὲ γῆς δνοφερῆς καὶ ταρτάρου ἠερόεντος \\
πόντου τ' ἀτρυγέτοιο καὶ οὐρανοῦ ἀστερόεντος\\
ἑξείης πάντων πηγαὶ καὶ πείρατ' ἔασιν, \\
ἀργαλέ' εὐρώεντα, τά τε στυγέουσι θεοί περ. \\

\quad{}ἔνθα δὲ μαρμάρεαί τε πύλαι καὶ χάλκεος οὐδός, \\
ἀστεμφὲς ῥίζῃσι διηνεκέεσσιν ἀρηρώς, \\
αὐτοφυής· πρόσθεν δὲ θεῶν ἔκτοσθεν ἁπάντων \\
Τιτῆνες ναίουσι, πέρην χάεος ζοφεροῖο. \\
αὐτὰρ ἐρισμαράγοιο Διὸς κλειτοὶ ἐπίκουροι \\
δώματα ναιετάουσιν ἐπ' Ὠκεανοῖο θεμέθλοις,\\
Κόττος τ' ἠδὲ Γύγης· Βριάρεών γε μὲν ἠὺν ἐόντα \\
γαμβρὸν ἑὸν ποίησε βαρύκτυπος Ἐννοσίγαιος,\\
δῶκε δὲ Κυμοπόλειαν ὀπυίειν, θυγατέρα ἥν.\\


{\minion\Para}
αὐτὰρ ἐπεὶ Τιτῆνας ἀπ' οὐρανοῦ ἐξέλασε Ζεύς,  \\
ὁπλότατον τέκε παῖδα Τυφωέα Γαῖα πελώρη\\
Ταρτάρου ἐν φιλότητι διὰ χρυσῆν Ἀφροδίτην· \\
οὗ χεῖρες †μὲν ἔασιν ἐπ' ἰσχύι ἔργματ' ἔχουσαι,†\\
καὶ πόδες ἀκάματοι κρατεροῦ θεοῦ· ἐκ δέ οἱ ὤμων \\
ἦν ἑκατὸν κεφαλαὶ ὄφιος δεινοῖο δράκοντος,  \\
γλώσσῃσι δνοφερῇσι λελιχμότες· ἐκ δέ οἱ ὄσσων \\
θεσπεσίῃς κεφαλῇσιν ὑπ' ὀφρύσι πῦρ ἀμάρυσσεν· \\
πασέων δ' ἐκ κεφαλέων πῦρ καίετο δερκομένοιο· \\
φωναὶ δ' ἐν πάσῃσιν ἔσαν δεινῇς κεφαλῇσι,\\
παντοίην ὄπ' ἰεῖσαι ἀθέσφατον· ἄλλοτε μὲν γὰρ  \\
φθέγγονθ' ὥς τε θεοῖσι συνιέμεν, ἄλλοτε δ' αὖτε\\
ταύρου ἐριβρύχεω μένος ἀσχέτου ὄσσαν ἀγαύρου, \\
ἄλλοτε δ' αὖτε λέοντος ἀναιδέα θυμὸν ἔχοντος,\\
ἄλλοτε δ' αὖ σκυλάκεσσιν ἐοικότα, θαύματ' ἀκοῦσαι,\\
ἄλλοτε δ' αὖ ῥοίζεσχ', ὑπὸ δ' ἤχεεν οὔρεα μακρά. \\
καί νύ κεν ἔπλετο ἔργον ἀμήχανον ἤματι κείνῳ,\\
καί κεν ὅ γε θνητοῖσι καὶ ἀθανάτοισιν ἄναξεν,\\
εἰ μὴ ἄρ' ὀξὺ νόησε πατὴρ ἀνδρῶν τε θεῶν τε· \\
σκληρὸν δ' ἐβρόντησε καὶ ὄβριμον, ἀμφὶ δὲ γαῖα\\
σμερδαλέον κονάβησε καὶ οὐρανὸς εὐρὺς ὕπερθε \\
πόντός τ' Ὠκεανοῦ τε ῥοαὶ καὶ Τάρταρα γαίης. \\
ποσσὶ δ' ὕπ' ἀθανάτοισι μέγας πελεμίζετ' Ὄλυμπος\\
ὀρνυμένοιο ἄνακτος· ἐπεστονάχιζε δὲ γαῖα.\\
καῦμα δ' ὑπ' ἀμφοτέρων κάτεχεν ἰοειδέα πόντον\\
βροντῆς τε στεροπῆς τε πυρός τ' ἀπὸ τοῖο πελώρου  \\
πρηστήρων ἀνέμων τε κεραυνοῦ τε φλεγέθοντος· \\
ἔζεε δὲ χθὼν πᾶσα καὶ οὐρανὸς ἠδὲ θάλασσα· \\
θυῖε δ' ἄρ' ἀμφ' ἀκτὰς περί τ' ἀμφί τε κύματα μακρὰ\\
ῥιπῇ ὕπ' ἀθανάτων, ἔνοσις δ' ἄσβεστος ὀρώρει· \\
τρέε δ' Ἀίδης ἐνέροισι καταφθιμένοισιν ἀνάσσων \\
Τιτῆνές θ' ὑποταρτάριοι Κρόνον ἀμφὶς ἐόντες \\
ἀσβέστου κελάδοιο καὶ αἰνῆς δηιοτῆτος. \\

\quad{}Ζεὺς δ' ἐπεὶ οὖν κόρθυνεν ἑὸν μένος, εἵλετο δ' ὅπλα,\\
βροντήν τε στεροπήν τε καὶ αἰθαλόεντα κεραυνόν,\\
πλῆξεν ἀπ' Οὐλύμποιο ἐπάλμενος· ἀμφὶ δὲ πάσας \\
ἔπρεσε θεσπεσίας κεφαλὰς δεινοῖο πελώρου.\\
αὐτὰρ ἐπεὶ δή μιν δάμασε πληγῇσιν ἱμάσσας,\\
ἤριπε γυιωθείς, στονάχιζε δὲ γαῖα πελώρη· \\
φλὸξ δὲ κεραυνωθέντος ἀπέσσυτο τοῖο ἄνακτος\\
οὔρεος ἐν βήσσῃσιν ἀιδνῆς παιπαλοέσσης \\
πληγέντος, πολλὴ δὲ πελώρη καίετο γαῖα \\
αὐτμῇ θεσπεσίῃ, καὶ ἐτήκετο κασσίτερος ὣς\\
τέχνῃ ὑπ' αἰζηῶν ἐν ἐυτρήτοις χοάνοισι\\
θαλφθείς, ἠὲ σίδηρος, ὅ περ κρατερώτατός ἐστιν, \\
οὔρεος ἐν βήσσῃσι δαμαζόμενος πυρὶ κηλέῳ \\
τήκεται ἐν χθονὶ δίῃ ὑφ' Ἡφαίστου παλάμῃσιν· \\
ὣς ἄρα τήκετο γαῖα σέλαι πυρὸς αἰθομένοιο.\\
ῥῖψε δέ μιν θυμῷ ἀκαχὼν ἐς τάρταρον εὐρύν. \\

\quad{}ἐκ δὲ Τυφωέος ἔστ' ἀνέμων μένος ὑγρὸν ἀέντων,\\
νόσφι Νότου Βορέω τε καὶ ἀργεστέω Ζεφύροιο· \\
οἵ γε μὲν ἐκ θεόφιν γενεήν, θνητοῖς μέγ' ὄνειαρ.\\
αἱ δ' ἄλλαι μὰψ αὖραι ἐπιπνείουσι θάλασσαν·\\
αἳ δή τοι πίπτουσαι ἐς ἠεροειδέα πόντον,\\
πῆμα μέγα θνητοῖσι, κακῇ θυίουσιν ἀέλλῃ· \\
ἄλλοτε δ' ἄλλαι ἄεισι διασκιδνᾶσί τε νῆας \\
ναύτας τε φθείρουσι· κακοῦ δ' οὐ γίνεται ἀλκὴ \\
ἀνδράσιν, οἳ κείνῃσι συνάντωνται κατὰ πόντον. \\
αἱ δ' αὖ καὶ κατὰ γαῖαν ἀπείριτον ἀνθεμόεσσαν\\
ἔργ' ἐρατὰ φθείρουσι χαμαιγενέων ἀνθρώπων, \\
πιμπλεῖσαι κόνιός τε καὶ ἀργαλέου κολοσυρτοῦ. \\

{\minion\Para}
αὐτὰρ ἐπεί ῥα πόνον μάκαρες θεοὶ ἐξετέλεσσαν, \\
Τιτήνεσσι δὲ τιμάων κρίναντο βίηφι,\\
δή ῥα τότ' ὤτρυνον βασιλευέμεν ἠδὲ ἀνάσσειν\\
Γαίης φραδμοσύνῃσιν Ὀλύμπιον εὐρύοπα Ζῆν\\
ἀθανάτων· ὁ δὲ τοῖσιν ἐὺ διεδάσσατο τιμάς. \\

\quad{}Ζεὺς δὲ θεῶν βασιλεὺς πρώτην ἄλοχον θέτο Μῆτιν, \\
πλεῖστα θεῶν εἰδυῖαν ἰδὲ θνητῶν ἀνθρώπων. \\
ἀλλ' ὅτε δὴ ἄρ' ἔμελλε θεὰν γλαυκῶπιν Ἀθήνην\\
τέξεσθαι, τότ' ἔπειτα δόλῳ φρένας ἐξαπατήσας\\
αἱμυλίοισι λόγοισιν ἑὴν ἐσκάτθετο νηδύν,  \\
Γαίης φραδμοσύνῃσι καὶ Οὐρανοῦ ἀστερόεντος· \\
τὼς γάρ οἱ φρασάτην, ἵνα μὴ βασιληίδα τιμὴν\\
ἄλλος ἔχοι Διὸς ἀντὶ θεῶν αἰειγενετάων.\\
ἐκ γὰρ τῆς εἵμαρτο περίφρονα τέκνα γενέσθαι· \\
πρώτην μὲν κούρην γλαυκώπιδα Τριτογένειαν, \\
ἶσον ἔχουσαν πατρὶ μένος καὶ ἐπίφρονα βουλήν,\\
αὐτὰρ ἔπειτ' ἄρα παῖδα θεῶν βασιλῆα καὶ ἀνδρῶν\\
ἤμελλεν τέξεσθαι, ὑπέρβιον ἦτορ ἔχοντα·\\
ἀλλ' ἄρα μιν Ζεὺς πρόσθεν ἑὴν ἐσκάτθετο νηδύν,\\
ὥς οἱ συμφράσσαιτο θεὰ ἀγαθόν τε κακόν τε. \\

\quad{}δεύτερον ἠγάγετο λιπαρὴν Θέμιν, ἣ τέκεν Ὥρας,\\
Εὐνομίην τε Δίκην τε καὶ Εἰρήνην τεθαλυῖαν,\\
αἵ τ' ἔργ' ὠρεύουσι καταθνητοῖσι βροτοῖσι,\\
Μοίρας θ', ᾗς πλείστην τιμὴν πόρε μητίετα Ζεύς, \\
Κλωθώ τε Λάχεσίν τε καὶ Ἄτροπον, αἵ τε διδοῦσι  \\
θνητοῖς ἀνθρώποισιν ἔχειν ἀγαθόν τε κακόν τε.\\

\quad{}τρεῖς δέ οἱ Εὐρυνόμη Χάριτας τέκε καλλιπαρήους,\\
Ὠκεανοῦ κούρη πολυήρατον εἶδος ἔχουσα,\\
Ἀγλαΐην τε καὶ Εὐφροσύνην Θαλίην τ' ἐρατεινήν·\\
τῶν καὶ ἀπὸ βλεφάρων ἔρος εἴβετο δερκομενάων \\
λυσιμελής· καλὸν δέ θ' ὑπ' ὀφρύσι δερκιόωνται. \\

\quad{}αὐτὰρ ὁ Δήμητρος πολυφόρβης ἐς λέχος ἦλθεν· \\
ἣ τέκε Περσεφόνην λευκώλενον, ἣν Ἀιδωνεὺς\\
ἥρπασεν ἧς παρὰ μητρός, ἔδωκε δὲ μητίετα Ζεύς. \\

\quad{}Μνημοσύνης δ' ἐξαῦτις ἐράσσατο καλλικόμοιο,  \\
ἐξ ἧς οἱ Μοῦσαι χρυσάμπυκες ἐξεγένοντο\\
ἐννέα, τῇσιν ἅδον θαλίαι καὶ τέρψις ἀοιδῆς.\\

\quad{}Λητὼ δ' Ἀπόλλωνα καὶ Ἄρτεμιν ἰοχέαιραν \\
ἱμερόεντα γόνον περὶ πάντων Οὐρανιώνων\\
γείνατ' ἄρ' αἰγιόχοιο Διὸς φιλότητι μιγεῖσα. \\

\quad{}λοισθοτάτην δ' Ἥρην θαλερὴν ποιήσατ' ἄκοιτιν· \\
ἡ δ' Ἥβην καὶ Ἄρηα καὶ Εἰλείθυιαν ἔτικτε \\
μιχθεῖσ' ἐν φιλότητι θεῶν βασιλῆι καὶ ἀνδρῶν. \\

\quad{}αὐτὸς δ' ἐκ κεφαλῆς γλαυκώπιδα γείνατ' Ἀθήνην, \\
δεινὴν ἐγρεκύδοιμον ἀγέστρατον ἀτρυτώνην,  \\
πότνιαν, ᾗ κέλαδοί τε ἅδον πόλεμοί τε μάχαι τε· \\
Ἥρη δ' Ἥφαιστον κλυτὸν οὐ φιλότητι μιγεῖσα\\
γείνατο, καὶ ζαμένησε καὶ ἤρισεν ᾧ παρακοίτῃ,\\
ἐκ πάντων τέχνῃσι κεκασμένον Οὐρανιώνων.\\

\quad{}ἐκ δ' Ἀμφιτρίτης καὶ ἐρικτύπου Ἐννοσιγαίου \\
Τρίτων εὐρυβίης γένετο μέγας, ὅς τε θαλάσσης \\
πυθμέν' ἔχων παρὰ μητρὶ φίλῃ καὶ πατρὶ ἄνακτι\\
ναίει χρύσεα δῶ, δεινὸς θεός. αὐτὰρ Ἄρηι\\
ῥινοτόρῳ Κυθέρεια Φόβον καὶ Δεῖμον ἔτικτε, \\
δεινούς, οἵ τ' ἀνδρῶν πυκινὰς κλονέουσι φάλαγγας \\
ἐν πολέμῳ κρυόεντι σὺν Ἄρηι πτολιπόρθῳ,\\
Ἁρμονίην θ', ἣν Κάδμος ὑπέρθυμος θέτ' ἄκοιτιν. \\

\quad{}Ζηνὶ δ' ἄρ' Ἀτλαντὶς Μαίη τέκε κύδιμον Ἑρμῆν,\\
κήρυκ' ἀθανάτων, ἱερὸν λέχος εἰσαναβᾶσα.\\

\quad{}Καδμηὶς δ' ἄρα οἱ Σεμέλη τέκε φαίδιμον υἱὸν  \\
μιχθεῖσ' ἐν φιλότητι, Διώνυσον πολυγηθέα, \\
ἀθάνατον θνητή· νῦν δ' ἀμφότεροι θεοί εἰσιν. \\

\quad{}Ἀλκμήνη δ' ἄρ' ἔτικτε βίην Ἡρακληείην\\
μιχθεῖσ' ἐν φιλότητι Διὸς νεφεληγερέταο. \\

\quad{}Ἀγλαΐην δ' Ἥφαιστος ἀγακλυτὸς ἀμφιγυήεις  \\
ὁπλοτάτην Χαρίτων θαλερὴν ποιήσατ' ἄκοιτιν.\\

\quad{}χρυσοκόμης δὲ Διώνυσος ξανθὴν Ἀριάδνην,\\
κούρην Μίνωος, θαλερὴν ποιήσατ' ἄκοιτιν· \\
τὴν δέ οἱ ἀθάνατον καὶ ἀγήρων θῆκε Κρονίων.\\

\quad{}Ἥβην δ' Ἀλκμήνης καλλισφύρου ἄλκιμος υἱός,  \\
ἲς Ἡρακλῆος, τελέσας στονόεντας ἀέθλους,\\
παῖδα Διὸς μεγάλοιο καὶ Ἥρης χρυσοπεδίλου,\\
αἰδοίην θέτ' ἄκοιτιν ἐν Οὐλύμπῳ νιφόεντι· \\
ὄλβιος, ὃς μέγα ἔργον ἐν ἀθανάτοισιν ἀνύσσας\\
ναίει ἀπήμαντος καὶ ἀγήραος ἤματα πάντα. \\

\quad{}Ἠελίῳ δ' ἀκάμαντι τέκε κλυτὸς Ὠκεανίνη \\
Περσηὶς Κίρκην τε καὶ Αἰήτην βασιλῆα.\\
Αἰήτης δ' υἱὸς φαεσιμβρότου Ἠελίοιο\\
κούρην Ὠκεανοῖο τελήεντος ποταμοῖο\\
γῆμε θεῶν βουλῇσιν, Ἰδυῖαν καλλιπάρηον· \\
ἣ δή οἱ Μήδειαν ἐύσφυρον ἐν φιλότητι\\
γείναθ' ὑποδμηθεῖσα διὰ χρυσῆν Ἀφροδίτην. \\

{\minion\Para}
ὑμεῖς μὲν νῦν χαίρετ', Ὀλύμπια δώματ' ἔχοντες, \\
νῆσοί τ' ἤπειροί τε καὶ ἁλμυρὸς ἔνδοθι πόντος· \\
νῦν δὲ θεάων φῦλον ἀείσατε, ἡδυέπειαι \\
Μοῦσαι Ὀλυμπιάδες, κοῦραι Διὸς αἰγιόχοιο,\\
ὅσσαι δὴ θνητοῖσι παρ' ἀνδράσιν εὐνηθεῖσαι\\
ἀθάναται γείναντο θεοῖς ἐπιείκελα τέκνα.\\

\quad{}Δημήτηρ μὲν Πλοῦτον ἐγείνατο δῖα θεάων,\\
Ἰασίῳ ἥρωι μιγεῖσ' ἐρατῇ φιλότητι  \\
νειῷ ἔνι τριπόλῳ, Κρήτης ἐν πίονι δήμῳ,\\
ἐσθλόν, ὃς εἶσ' ἐπὶ γῆν τε καὶ εὐρέα νῶτα θαλάσσης\\
πᾶσαν· τῷ δὲ τυχόντι καὶ οὗ κ' ἐς χεῖρας ἵκηται, \\
τὸν δὴ ἀφνειὸν ἔθηκε, πολὺν δέ οἱ ὤπασεν ὄλβον.\\

\quad{}Κάδμῳ δ' Ἁρμονίη, θυγάτηρ χρυσῆς Ἀφροδίτης, \\
Ἰνὼ καὶ Σεμέλην καὶ Ἀγαυὴν καλλιπάρηον \\
Αὐτονόην θ', ἣν γῆμεν Ἀρισταῖος βαθυχαίτης,\\
γείνατο καὶ Πολύδωρον ἐυστεφάνῳ ἐνὶ Θήβῃ.\\
κούρη δ' Ὠκεανοῦ Χρυσάορι καρτεροθύμῳ\\

\quad{}μιχθεῖσ' ἐν φιλότητι πολυχρύσου Ἀφροδίτης \\
Καλλιρόη τέκε παῖδα βροτῶν κάρτιστον ἁπάντων,\\
Γηρυονέα, τὸν κτεῖνε βίη Ἡρακληείη\\
βοῶν ἕνεκ' εἰλιπόδων ἀμφιρρύτῳ εἰν Ἐρυθείῃ.\\
Τιθωνῷ δ' Ἠὼς τέκε Μέμνονα χαλκοκορυστήν,\\

\quad{}Αἰθιόπων βασιλῆα, καὶ Ἠμαθίωνα ἄνακτα. \\
αὐτάρ τοι Κεφάλῳ φιτύσατο φαίδιμον υἱόν, \\
ἴφθιμον Φαέθοντα, θεοῖς ἐπιείκελον ἄνδρα· \\
τόν ῥα νέον τέρεν ἄνθος ἔχοντ' ἐρικυδέος ἥβης\\
παῖδ' ἀταλὰ φρονέοντα φιλομμειδὴς Ἀφροδίτη\\
ὦρτ' ἀνερειψαμένη, καί μιν ζαθέοις ἐνὶ νηοῖς  \\
νηοπόλον μύχιον ποιήσατο, δαίμονα δῖον. \\
κούρην δ' Αἰήταο διοτρεφέος βασιλῆος\\

\quad{}Αἰσονίδης βουλῇσι θεῶν αἰειγενετάων\\
ἦγε παρ' Αἰήτεω, τελέσας στονόεντας ἀέθλους,\\
τοὺς πολλοὺς ἐπέτελλε μέγας βασιλεὺς ὑπερήνωρ, \\
ὑβριστὴς Πελίης καὶ ἀτάσθαλος ὀβριμοεργός· \\
τοὺς τελέσας ἐς Ἰωλκὸν ἀφίκετο πολλὰ μογήσας\\
ὠκείης ἐπὶ νηὸς ἄγων ἑλικώπιδα κούρην\\
Αἰσονίδης, καί μιν θαλερὴν ποιήσατ' ἄκοιτιν.\\
καί ῥ' ἥ γε δμηθεῖσ' ὑπ' Ἰήσονι ποιμένι λαῶν  \\
Μήδειον τέκε παῖδα, τὸν οὔρεσιν ἔτρεφε Χείρων\\
Φιλλυρίδης· μεγάλου δὲ Διὸς νόος ἐξετελεῖτο. \\
αὐτὰρ Νηρῆος κοῦραι ἁλίοιο γέροντος,\\

\quad{}ἤτοι μὲν Φῶκον Ψαμάθη τέκε δῖα θεάων\\
Αἰακοῦ ἐν φιλότητι διὰ χρυσῆν Ἀφροδίτην·  \\
Πηλεῖ δὲ δμηθεῖσα θεὰ Θέτις ἀργυρόπεζα\\
γείνατ' Ἀχιλλῆα ῥηξήνορα θυμολέοντα.\\
Αἰνείαν δ' ἄρ' ἔτικτεν ἐυστέφανος Κυθέρεια,\\

\quad{}Ἀγχίσῃ ἥρωι μιγεῖσ' ἐρατῇ φιλότητι \\
Ἴδης ἐν κορυφῇσι πολυπτύχου ἠνεμοέσσης. \\
Κίρκη δ' Ἠελίου θυγάτηρ Ὑπεριονίδαο\\

\quad{}γείνατ' Ὀδυσσῆος ταλασίφρονος ἐν φιλότητι\\
Ἄγριον ἠδὲ Λατῖνον ἀμύμονά τε κρατερόν τε· \\
{[}Τηλέγονον δὲ ἔτικτε διὰ χρυσῆν Ἀφροδίτην·{]}\\
οἳ δή τοι μάλα τῆλε μυχῷ νήσων ἱεράων \\ %Aqui tem um verso a menos do que a tradução
πᾶσιν Τυρσηνοῖσιν ἀγακλειτοῖσιν ἄνασσον.\\

\quad{}Ναυσίθοον δ' Ὀδυσῆι Καλυψὼ δῖα θεάων\\
γείνατο Ναυσίνοόν τε μιγεῖσ' ἐρατῇ φιλότητι. \\

\quad{}αὗται μὲν θνητοῖσι παρ' ἀνδράσιν εὐνηθεῖσαι\\
ἀθάναται γείναντο θεοῖς ἐπιείκελα τέκνα. \\
{[}νῦν δὲ γυναικῶν φῦλον ἀείσατε, ἡδυέπειαι\\
Μοῦσαι Ὀλυμπιάδες, κοῦραι Διὸς αἰγιόχοιο.{]}

            \pend
         \endnumbering
    \end{Leftside}
    %
    \begin{Rightside}
        \beginnumbering
            \pstart
\noindent{}Pelas Musas do \edtext{Hélicon}{\nota{montanha próxima ao vilarejo de Ascra, na Beócia, mencionado
em \emph{Trabalhos e dias} como a localidade para onde emigrara o pai do
poeta.}} comecemos a cantar, \\
elas que o Hélicon ocupam, monte grande e numinoso,\\
e em volta de fonte violácea com pés macios\\
\edtext{dançam,}{\nota{As Musas dançam em conjunto como um coro feminino, prática
músico-ritual comum em várias ocasiões sócio-religiosas específicas nas
comunidades gregas arcaicas.}} e do altar do possante Cronida;\\
tendo a pele delicada no Permesso banhado,\\
na \edtext{Fonte do Cavalo}{\nota{traduz Hipocrene.}} ou no Olmeio numinoso,\\
no cimo do Hélicon compõem danças corais\\
belas, desejáveis, e fluem com os pés.\\

\quad{}De lá se lançando, ocultas por densa neblina,\\
de noite avançavam, belíssima voz emitindo,\\
\edtext{louvando}{\nota{``Louvar'' (e às vezes ``cantar'') traduz o verbo grego
\emph{humnein}, de etimologia desconhecida, que, se guarda alguma
especificidade nas tradições hexamétricas gregas, essa não é mais
recuperável.}} Zeus porta-égide, a soberana Hera\\
argiva, que pisa com douradas sandálias,\\
a filha de Zeus porta-égide, \edtext{Atena olhos-de-coruja,}{\nota{Olhos-de-coruja é provavelmente o sentido cultual original desse
epíteto, que, na época histórica, em algum momento passou a ser
reinterpretado como ``com olhar brilhante (glauco)''.}}\\
\edtext{Febo Apolo}{\nota{Epíteto de Apolo de origem desconhecida, talvez ligado à luz
ou à pureza.}} e Ártemis verte-setas,\\
Posêidon, \edtext{Treme-Solo sustém-terra,}{\nota{Epítetos de Posêidon.}}\\
respeitada \edtext{Norma}{\nota{(ou Regra) = \emph{Themis}.}} e \edtext{Afrodite olhar-vibrante,}{\nota{Não há segurança sobre o sentido do epíteto de Afrodite; ``de olhos negros'' é outra possibilidade.}}\\
\edtext{Juventude}{\nota{\emph{Hēbē}.}} coroa-dourada e a bela \edtext{Dione,}{\nota{em Homero, é a mãe de Afrodite, mas não em Hesíodo.}}\\
Leto, Jápeto e Crono curva-astúcia,\\
\edtext{Aurora,}{\nota{\emph{Ēōs}.}} o grande \edtext{Sol}{\nota{\emph{Ēelios}.}} e a reluzente \edtext{Lua,}{\nota{\emph{Selēnē}.}}\\
\edtext{Terra,}{\nota{\emph{Gaia}.}} o grande Oceano e a negra \edtext{Noite,}{\nota{\emph{Nux}.}} \\
e a sacra linhagem dos outros imortais sempre vivos.\\

\quad{}Sim, então essas a Hesíodo \edtext{o belo canto ensinaram,}{\nota{A arte de cantar em geral e não um canto específico.}}\\
quando apascentava cordeiros sob o Hélicon numinoso.\\
Este discurso, primeiríssimo ato, dirigiram-me as deusas,\\
as Musas do Olimpo, filhas de Zeus porta-égide: \\
``Pastores rústicos, infâmias vis, ventres somente,\\
sabemos falar muito fato enganoso como genuíno,\\
e sabemos, quando queremos, proclamar verdades''.\\
Assim falaram as filhas do grande Zeus, as \edtext{palavra-ajustada,}{\nota{traduz \emph{artiepēs}, que na \emph{Ilíada}
22.281 tem sentido negativo (Aquiles censura a ladina manipulação
discursiva de Heitor).}}\\
e me deram o \edtext{cetro,}{\nota{O cetro costuma ser associado a Zeus e a reis, mas, como aqui é de
um loureiro, o vínculo com Apolo também é possível.}} galho vicejante de louro, \\
após o colher, admirável; e sopraram-me voz\\
inspirada para eu glorificar o que será e foi,\\
pedindo que louvasse a linhagem dos ditosos sempre vivos\\
e a elas mesmas primeiro e por último sempre cantasse.\\

\quad{}Mas \edtext{por que disso falo em torno do carvalho e da pedra?}{\nota{O uso que Hesíodo faz dessa expressão é controverso; independente do
contexto (poético), uma análise comparativa (indo-europeia) propõe que o
sentido da fórmula utilizado aqui é ``de forma geral, de tudo um
pouco''. Hesíodo, portanto, se perguntaria: ``por que divago''?}} \\
Ei tu, comecemos pelas Musas, que para Zeus pai\\
cantam e deleitam sua grande mente no Olimpo,\\
dizendo o que é, o que será e o que foi antes,\\
harmonizando com o som, e, incansável, flui sua voz\\
das bocas, doce; sorri a morada do pai \\
Zeus altissoante com a voz de lírio das deusas,\\
irradiante; ressoam o cume do Olimpo nevado\\
e as casas dos imortais: elas, imorredoura voz emitindo,\\
dos deuses a respeitada linhagem primo glorificam no canto\\
dês o início, estes que Terra e amplo \edtext{Céu}{\nota{\emph{Ouranos}.}} pariram, \\
e estes que deles nasceram, os deuses oferentes de bens;\\
na sequência, a Zeus, pai de deuses e homens,\\
que elas \edtext{louvam ao iniciar e cessar o canto,}{\nota{O texto tal como transmitido pelos manuscritos tem problemas, e sua
tradução seria ``as deussas cantam, ao iniciar e cessar o canto''; a
maioria dos filólogos opta por deletá-lo. Seguindo-se Colonna e Pucci,
adotou-se uma correção de A. Ludwich no 2º hemistíquio.}}\\
pois é o mais forte dos deuses e supremo em poder;\\
depois, a linhagem dos homens e dos poderosos Gigantes \\
louvando, deleitam a mente de Zeus no Olimpo\\
as Musas do Olimpo, filhas de Zeus porta-égide.\\

\quad{}Pariu-lhes na \edtext{Piéria,}{\nota{região logo ao norte do Olimpo.}} após se unir ao pai, o Cronida,\\
\edtext{Memória,}{\nota{\emph{Mnēmosunē}.}} regente das ladeiras de Eleuteros,\\
como \edtext{esquecimento}{\nota{Em grego, o par memória \emph{versus} esquecimento é marcado fonicamente (\emph{mnēmosunēe} x \emph{lēsmosunēō}).}} de males e suspensão de afãs. \\
Por nove noites com ela se uniu o astucioso Zeus\\
longe dos imortais, subindo no sacro leito;\\
mas quando o ano chegou, e as estações deram a volta,\\
os meses finando, e muitos dias passaram,\\
ela gerou nove filhas concordes, que do canto \\
no peito se ocupam com ânimo sereno,\\
perto do mais alto pico do Olimpo nevado:\\
lá têm reluzentes pistas de dança e belas moradas,\\
e junto delas as \edtext{Graças}{\nota{\emph{Kharites} (sing. \emph{Kharis}).}} e \edtext{Desejo}{\nota{\emph{Himeros}.}} habitam\\
em festas; pela boca amável voz emitindo, \\
cantam e dançam e os costumes e usos sábios de todos\\
os imortais glorificam, amável voz emitindo.\\
Nisso iam ao Olimpo, gozando a bela voz,\\
com música imortal; rugia a negra terra em volta\\
ao cantarem, e amável ressoo subia dos pés \\
ao retornarem a seu pai: ele reina no céu,\\
ele mesmo segurando trovão e raio chamejante,\\
pois no poder venceu o pai Crono; bem cada coisa\\
apontou aos imortais por igual e indicou suas honrarias.\\

\quad{}Isso cantavam as Musas, que têm morada olímpia, \\
as nove filhas geradas do grande Zeus,\\
\edtext{Glória,}{\nota{\emph{Klio}.}} \edtext{Aprazível,}{\nota{\emph{Euterpē}.}} \edtext{Festa,}{\nota{\emph{Thaleia}.}} \edtext{Cantarina,}{\nota{\emph{Melpomenē}.}}\\
\edtext{Dançapraz,}{\nota{\emph{Terpsikhorē}.}} \edtext{Saudosa,}{\nota{\emph{Eratō}.}} \edtext{Muitacanção,}{\nota{\emph{Polumnia}.}} \edtext{Celeste}{\nota{\emph{Ouraniē}.}}\\
e \edtext{Belavoz:}{\nota{\emph{Kalliopē}.}} essa é a superior entre todas.\\
Pois essa também a respeitados reis acompanha. \\
Quem quer que honrem as filhas do grande Zeus\\
e o veem ao nascer, um dos reis criados por Zeus,\\
para ele, sobre a língua, vertem doce orvalho,\\
e da boca dele fluem palavras amáveis; as gentes\\
todas o miram quando decide entre sentenças \\
com retos juízos: falando com segurança,\\
de pronto até disputa grande interrompe destramente;\\
por isso reis são sensatos, pois às gentes\\
prejudicadas completam na ágora ações reparatórias\\
fácil, induzindo com palavras macias; \\
ao se mover na praça, como um deus o propiciam\\
com respeito amável, e destaca-se na multidão.\\

\quad{}Tal é a sacra dádiva das Musas aos homens.\\
Pois das Musas, vê, e de Apolo lança-de-longe\\
vêm os varão cantores sobre a terra e os citaredos, \\
e de Zeus, os reis: este é afortunado, quem as Musas\\
amam; doce é a voz que flui de sua boca.\\
Pois se alguém, com pesar no ânimo recém-afligido,\\
seca no coração, angustiado, mas um cantor,\\
assistente das Musas, glórias de homens de antanho \\
e deuses ditosos, que ocupam o Olimpo, cantar,\\
de pronto ele esquece as tristezas e de aflição alguma\\
se lembra: rápido as desviam os dons das deusas.\\

\quad{}Felicidades, filhas de Zeus, e dai canto desejável;\\
glorificai a sacra linhagem dos imortais sempre vivos, \\
os que de Terra nasceram, do estrelado Céu\\
e da escura Noite, e esses que criou o salso \edtext{Mar.}{\nota{\emph{Pontos}.}}\\
Dizei como no início os deuses e Terra nasceram,\\
os Rios e o Mar sem-fim, furioso nas ondas,\\
os Astros fulgentes e o amplo Céu acima, \\
e esses que deles nasceram, os deuses oferentes de bens;\\
como dividiram a abastança, repartiram as honrarias,\\
e também como no início ocuparam o Olimpo de muitos vales.\\
Disso me narrem, Musas que têm morada olímpia,\\
do princípio, e dizei qual deles primeiro nasceu. \\

\quad{}Bem no início, \edtext{Abismo}{\nota{\emph{Khaos}, segundo a interpretação mais aceita, um vazio sem forma, e não uma matéria indistinta.}} nasceu; depois,\\
\edtext{Terra largo-peito, de todos assento sempre firme,\\
dos imortais que possuem o pico do Olimpo nevado\\
e o Tártaro brumoso no recesso da terra largas-rotas,\\
e Eros,}{\lemma{Terra \ldots{} Eros}{\nota{A leitura mais aceita é que Terra e Eros são divindades, e o
Tártaro, um espaço abaixo da superfície terrestre. Alguns optam pelo
Tártaro, nesta passagem, como uma divindade, colocando uma vírgula no
final do verso 118.}}} que é o mais belo entre os deuses imortais,\\
o solta-membros, e de todos os deuses e todos os homens\\
subjuga, no peito, mente e desígnio refletido.\\

\Para
De Abismo nasceram \edtext{Escuridão}{\nota{\emph{Erebos}, lugar escuro, amiúde associado ao Hades.}} e a negra Noite;\\
de Noite, então, Éter e Dia nasceram,\\
que gerou, grávida, após com Escuridão unir-se em amor.\\

\Para
Terra primeiro gerou, igual a ela,\\
o estrelado Céu, a fim de encobri-la por inteiro\\
para ser, dos deuses venturosos, assento sempre firme;\\
gerou as enormes Montanhas, refúgios graciosos de deusas,\\
as Ninfas, que habitam montanhas matosas; \\
pariu também o ruidoso pélago, furioso nas ondas,\\
Mar, sem amor desejante; e então\\
deitou-se com Céu e pariu Oceano fundo-redemunho,\\
Coio, Creio, \edtext{Hipérion,}{\nota{Na poesia grega arcaica, Hipérion sempre aparece em conexão com o Sol.}} Jápeto,\\
Teia, Reia, Norma, Memória, \\
Febe coroa-dourada e a atraente Tetís.\\
Depois deles, o mais novo nasceu, Crono curva-astúcia,\\
o mais fero dos filhos; e odiou o viçoso pai.\\

\quad{}Então gerou os Ciclopes, que têm brutal coração,\\
\edtext{Trovão,}{\nota{\emph{Brontē}.}} \edtext{Relâmpago}{\nota{\emph{Steropē}.}} e \edtext{Clarão}{\nota{\emph{Argēs}.}} ânimo-ponderoso, \\
eles que o trovão deram a Zeus e fabricaram o raio.\\
Quanto a eles, de resto assemelhavam-se aos deuses,\\
mas um só olho no meio da fronte jazia;\\
\edtext{Ciclopes eram seu nome epônimo, porque deles\\
circular o olho,}{\lemma{Ciclopes \ldots{} circular o olho}{\nota{No grego, o jogo etimológico é ainda mais saliente (\emph{Kuklōpes} --- \emph{kukloterēs}).}}} um só, que na fronte jazia;\\
energia, força e engenho havia em seus feitos.\\

\quad{}E outros então de Terra e Céu nasceram,\\
três filhos grandes e ponderosos, inomináveis,\\
Coto, Briareu e Giges, rebentos insolentes.\\
Cem braços de seus ombros se lançavam, \\
inabordáveis, e cabeças, em cada um, cinquenta\\
dos ombros nasceram sobre os membros robustos;\\
a energia imensa era brutal na grande figura.\\

\quad{}Pois tantos quantos de Terra e Céu nasceram,\\
os mais feros dos filhos, por seu pai foram odiados \\
desde o princípio: assim que nascesse um deles,\\
a todos ocultava, não os deixava à luz subir,\\
no recesso de Terra, e com o feito vil se regozijava\\
Céu; ela dentro gemia, a portentosa Terra,\\
constrita, e planejou ardiloso, nocivo estratagema. \\
De pronto criou a espécie do cinzento adamanto,\\
fabricou grande foice e mostrou-a aos caros filhos.\\

\quad{}Atiçando-os, disse, agastada no caro coração:\\
``Filhos meus e de pai iníquo, caso quiserdes,\\
obedecei: nos vingaríamos da vil ofensa do pai \\
vosso, o primeiro a armar feitos ultrajantes''.\\

\quad{}Assim falou; e o medo pegou a todos, e nenhum deles\\
falou. Com audácia, o grande Crono curva-astúcia\\
de pronto com um discurso respondeu à mãe devotada:\\
``Mãe, isso sob promessa eu cumpriria, \\
o feito, pois desconsidero o inominável pai\\
nosso, o primeiro a armar feitos ultrajantes''.\\

\quad{}Assim falou; muito alegrou-se no juízo a portentosa Terra.\\
Escondeu-o numa tocaia, colocou em suas mãos\\
a foice serridêntea e instruiu-o em todo o ardil. \\
Veio, trazendo a noite, o grande Céu, e em torno de Terra\\
estendeu-se, desejoso de amor, e estirou-se em toda\\
direção. O outro, o filho, da tocaia a mão esticou,\\
a esquerda, e com a direita pegou a foice portentosa,\\
grande, serridêntea, os genitais do caro pai \\
com avidez ceifou e lançou para trás, que fossem\\
embora. Mas, ao escapar da mão, não ficaram sem efeito:\\
tantas gotas de sangue quantas escapuliram,\\
Terra a todas recebeu; após os anos volverem-se,\\
gerou as Erínias brutais e os grandes Gigantes, \\
luzidios em armas, com longas lanças nas mãos,\\
e as Ninfas que chamam \edtext{Mélias}{\nota{ninfas ligadas a árvores.}} na terra sem-fim.\\
Os genitais, quando primeiro os cortou com adamanto,\\
lançou-os para baixo, da costa ao mar encapelado,\\
levou-os o pélago muito tempo, e em volta, branca \\
espuma lançou-se da carne imortal; e nela moça\\
foi criada: primeiro da numinosa \edtext{Citera}{\nota{Em Citera, ilha na ponta sudoeste do Peloponeso, ficava um templo de Afrodite.}} achegou-se,\\
e então de lá atingiu o oceânico \edtext{Chipre.}{\nota{É em Chipre que os gregos costumavam representar a origem de Afrodite; lá ficavam seus centros cultuais mais importantes.}}\\
E saiu a respeitada, bela deusa, e grama em volta\\
crescia sob os pés esbeltos: a ela Afrodite \\
espumogênita e Citereia bela-coroa\\
chamam deuses e varões, porque na \edtext{espuma}{\nota{Jogo etimológico entre \emph{Aphrodite} e \emph{aphros} (``espuma'').}}\\
foi criada; Citereia, pois alcançou Citera;\\
cipriogênita, pois nasceu em Chipre cercado-de-mar;\\
e \edtext{ama-sorriso, pois da genitália surgiu.}{\nota{Jogo etimológico entre \emph{philommeidēs} (``ama-sorriso'') e \emph{mēdea} (``genitália masculina''), homófono de um termo que
significa ``planos ardilosos'', cujo radical é o mesmo do verbo ``armar'' (v. 166).}} \\
Eros acompanhou-a e Desejo a seguiu, belo,\\
quando ela nasceu e dirigiu-se à tribo dos deuses.\\
Tem esta honra desde o início e granjeou\\
quinhão entre homens e deuses imortais,\\
flertes de meninas, sorrisos e farsas, \\
delicioso prazer, amor e afeto.\\

\quad{}Àqueles o pai chamava, por apelido, \edtext{Titãs,\\
o grande Céu brigando com filhos que ele mesmo gerou;\\
dizia que, iníquos, se esticaram para efetuar enorme\\
feito, pelo qual haveria vingança}{\lemma{Titãs \ldots{} vingança}{\nota{Jogo etimológico entre \emph{Titēnas} (``Titãs''), \emph{titainontas} (de \emph{titainein}, ``estender, esticar'') e
\emph{tisis} (``vingança'').}}} depois no futuro. \\

\quad{}E Noite pariu a medonha \edtext{Sina,}{\nota{\emph{Moros}.}} \edtext{Perdição}{\nota{\emph{Kēr}.}} negra\\
e \edtext{Morte,}{\nota{\emph{Thanatos}.}} e pariu \edtext{Sono,}{\nota{\emph{Hupnos}.}} e pariu a tribo de \edtext{Sonhos;}{\nota{\emph{Oneiros}.}}\\
sem se deitar com um deus, pariu a \edtext{escura Noite.}{\nota{``Escura'' (\emph{erebennē}) parece remeter a Escuridão (\emph{Erebos}), parceiro sexual de Noite no início da cosmogonia.}}\\
Em seguida, \edtext{Pecha}{\nota{\emph{Momos}.}} e aflitiva \edtext{Agonia,}{\nota{\emph{Oizus}.}}\\
e Hespérides, que, para lá do glorioso Oceano, de belas \\
maçãs de ouro cuidam e das árvores que trazem o fruto;\\
e gerou as \edtext{Moiras}{\nota{Destino, Quinhão.}} e Perdições castigo-implacável,\\
\edtext{Fiandeira,}{\nota{\emph{Klotō}.}} \edtext{Sorteadora}{\nota{\emph{Lakhesis}.}} e \edtext{Inflexível,}{\nota{\emph{Atropos}.}} elas que aos \edtext{mortais,\\
ao nascerem, lhes concedem bem e mal como seus,\\
e elas que alcançam transgressões de homens e deuses \\
e nunca desistem, as deusas, da raiva assombrosa\\
até retribuir com maligna punição àquele que errar.}{\lemma{Fiandeira \ldots{} errar}{\nota{A maioria dos críticos considera os versos 218--19
(905--6) como interpolados. Preferi considerar que
218--19 referem-se às Moiras, e 220--22, às Perdições.}}}\\
Também pariu \edtext{Indignação,}{\nota{\emph{Nemesis}.}} desgraça aos humanos mortais,\\
a ruinosa Noite; depois pariu \edtext{Farsa}{\nota{\emph{Apatē}.}} e \edtext{Amor}{\nota{\emph{Philotēs}.}}\\
e a destrutiva \edtext{Velhice,}{\nota{\emph{Geras}.}} e pariu \edtext{Briga}{\nota{\emph{Eris}.}} ânimo-potente. \\

\quad{}E a odiosa Briga pariu o aflitivo \edtext{Labor,}{\nota{\emph{Ponos}.}}\\
\edtext{Esquecimento,}{\nota{\emph{Lēthē}.}} \edtext{Fome,}{\nota{\emph{Limos}.}} \edtext{Aflições}{\nota{\emph{Algos}.}} lacrimosas,\\
\edtext{Batalhas,}{\nota{\emph{Husminē}.}} \edtext{Combates,}{\nota{\emph{Makhē}.}} \edtext{Matanças,}{\nota{\emph{Phonos}.}} \edtext{Carnificinas,}{\nota{\emph{Androktasia}.}}\\
\edtext{Disputas,}{\nota{\emph{Neikos}.}} \edtext{Embustes,}{\nota{\emph{Pseudos}.}} \edtext{Contos,}{\nota{\emph{Logos}.}} \edtext{Contendas,}{\nota{\emph{Amphillogia}.}}\\
\edtext{Má-Norma}{\nota{\emph{Dusnomia}.}} e \edtext{Desastre,}{\nota{\emph{Atē}.}} vizinhas recíprocas, \\
e \edtext{Jura,}{\nota{\emph{Horkos}.}} ela que demais aos homens mortais\\
desgraça se alguém, de bom grado, perjura.\\

\Para
A Nereu, probo e verdadeiro, gerou Mar,\\
ao mais velho dos filhos: chamam-no ``ancião''\\
porque é veraz e gentil e das regras \\
não se esquece, mas planos justos e gentis conhece;\\
e então ao grande Taumas e ao orgulhoso Fórcis,\\
a Terra unido, e a Cetó bela-face\\
e \edtext{Amplaforça}{\nota{\emph{Eurubiē}.}} com ânimo de adamanto no íntimo.\\
E de Nereu nasceram numerosas filhas de deusas, \\
no mar ruidoso, com \edtext{Dóris}{\nota{o seu nome também remete à raiz verbal de ``dar'', elemento presente em algumas de suas filhas.}} belas-tranças,\\
filha do circular rio Oceano:\\
\edtext{Propele,}{\nota{\emph{Prothō}.}} \edtext{Completriz,}{\nota{\emph{Eukrantē}.}} \edtext{Salva,}{\nota{\emph{Saō}.}} Anfitrite,\\
Tétis, \edtext{Dadivosa,}{\nota{\emph{Eudōrē}.}} \edtext{Calmaria,}{\nota{\emph{Galēnē}.}} \edtext{Azúlis,}{\nota{\emph{Glaukē}.}}\\
\edtext{Ondacélere,}{\nota{\emph{Kumothoē}.}} a veloz \edtext{Gruta,}{\nota{\emph{Speiē}.}} a desejável \edtext{Festa,}{\nota{\emph{Thalia}. Alguns críticos (Mazon, Ricciardelli) defendem, para a segunda metade do verso, ``\ldots{} Gruta, Veloz e a desejável \emph{Marinha}''.}} \\
\edtext{Admiradíssima,}{\nota{\emph{Pasiteē}.}} \edtext{Saudosa,}{\nota{\emph{Eratō}.}} \edtext{Belarrixa}{\nota{\emph{Eunikē}.}} braço-róseo,\\
a graciosa \edtext{Amelada,}{\nota{\emph{Melitē}.}} \edtext{Enseada,}{\nota{\emph{Eulimenē}.}} \edtext{Resplende,}{\nota{\emph{Agauē}.}}\\
\edtext{Doadora,}{\nota{\emph{Dōtō}.}} \edtext{Inicia,}{\nota{\emph{Prōtō}.}} \edtext{Levadora,}{\nota{\emph{Pherousa}.}} \edtext{Poderosa,}{\nota{\emph{Dunamenē}.}}\\
\edtext{Ilhoa,}{\nota{\emph{Nēsaiē}.}} \edtext{Costeira,}{\nota{\emph{Aktaiē}.}} \edtext{Primazia,}{\nota{\emph{Prōtomedeia}.}}\\
Dóris, \edtext{Tudovê,}{\nota{\emph{Panopē}.}} a benfeita Galateia, \\
a desejável \edtext{Hipocorre,}{\nota{\emph{Hippotoē}.}} \edtext{Hipomente}{\nota{
\emph{Hipponoē}.}} braço-róseo,\\
\edtext{Seguronda,}{\nota{\emph{Kumodokē}.}} que ondas no mar embaciado\\
e rajadas de ventos bravios com \edtext{Cessonda}{\nota{\emph{Kumatolēgē}.}}\\
e Anfitrite de belo tornozelo fácil apazigua,\\
\edtext{Ondina,}{\nota{\emph{Kumō}.}} \edtext{Praiana,}{\nota{\emph{Eionō}.}} \edtext{Mandamar}{\nota{\emph{Halimēdē}.}} bela-coroa, \\
\edtext{Partilhazúlis}{\nota{\emph{Glaukonomē}.}} ama-sorriso, \edtext{Viajamar,}{\nota{\emph{Pontoporeia}.}}\\
\edtext{Juntapovo,}{\nota{\emph{Leiagorē}.}} \edtext{Juntabem,}{\nota{\emph{Euagorē}.}} \edtext{Cuidapovo,}{\nota{\emph{Laomedeia}.}}\\
\edtext{Espirituosa,}{\nota{\emph{Poulunoē}.}} \edtext{Cônscia,}{\nota{\emph{Autonoē}.}} \edtext{Compensadora,}{\nota{\emph{Lusianassa}.}}\\
\edtext{Rebanhosa,}{\nota{\emph{Euarnē}.}} desejável no físico, impecável na forma,\\
\edtext{Areiana,}{\nota{\emph{Psamathē}.}} graciosa de corpo, a divina \edtext{Forcequina,}{\nota{\emph{Menippē}.}} \\
\edtext{Ilheia,}{\nota{\emph{Nēsō}.}} \edtext{Benconduz,}{\nota{\emph{Eupompē}.}} \edtext{Normativa,}{\nota{\emph{Themistō}.}} \edtext{Previdente}{\nota{\emph{Pronoē}.}}\\
e \edtext{Veraz,}{\nota{\emph{Nēmertēs}.}} que tem o espírito do pai imortal.\\
Essas nasceram do impecável Nereu,\\
cinquenta filhas, peritas em impecáveis trabalhos.\\

\quad{}E Taumas a filha de Oceano funda-corrente \\
desposou, \edtext{Brilhante;}{\nota{\emph{Elektrē}.}} e ela pariu Íris veloz\\
e as Hárpias belas-tranças, \edtext{Tempesta}{\nota{\emph{Aellō}.}} e \edtext{Voaveloz,}{\nota{\emph{Okupetēs}.}}\\
que, com rajadas de ventos e aves, junto seguem\\
com asas velozes, pois disparam, altaneiras.\\

\Para
E Cetó pariu para Fórcis velhas bela-face,\\
grisalhas de nascença, que chamam \edtext{Velhas}{\nota{\emph{Graiai}.}}\\
os deuses imortais e homens que andam na terra,\\
Penfredó belo-peplo, Enió peplo-açafrão\\
e as Górgonas, que habitam para lá do glorioso Oceano\\
no limite, rumo à noite, onde estão as Hespérides clara-voz --- \\
Estenó, Euríale e Medusa, que sofreu o funesto:\\
esta era mortal, as outras, imortais e sem velhice,\\
as duas; e só junto a ela deitou-se \edtext{Juba-Cobalto}{\nota{é Posêidon.}}\\
num prado macio com flores primaveris.\\
Dela, quando Perseu a cabeça cortou do pescoço, \\
p'ra fora pularam o grande \edtext{Espadouro}{\nota{\emph{Khrusaōr}.}} e o cavalo Pégaso.\\
Ele tinha esse epônimo pois pegado às \edtext{fontes}{\nota{O nome é ligado a \emph{pēgas}, ``fontes''.}} de Oceano\\
nasceu, e aquele, com espada de ouro nas caras mãos.\\
Pégaso alçou vôo, após deixar a terra, mãe de ovelhas,\\
e dirigiu-se aos imortais; a casa de Zeus habita \\
e leva trovão e raio ao astucioso Zeus.\\
E Espadouro gerou Gerioneu três-cabeças,\\
unido a Bonflux, filha do famoso Oceano:\\
eis que a esse matou \edtext{a força de Héracles,}{\nota{o vigor de Héracles, (v. 951).}}\\
junto a bois passo-arrastado na oceânica Eriteia \\
naquele dia em que tangeu os bois fronte-larga\\
até a sacra Tirinto, após cruzar o estreito de Oceano\\
e ter matado Orto e o pastor Euritíon\\
na quinta brumosa p'ra lá do famoso Oceano.\\

\quad{}Ela gerou outro ser portentoso, impossível, nem parecido \\
com homens mortais nem com deuses imortais,\\
em cava gruta, a divina Équidna juízo-forte,\\
metade moça olhar-luzente, bela-face,\\
metade serpente portentosa, terrível e grande,\\
dardejante come-cru, sob os confins da numinosa terra. \\
Lá fica sua caverna, para baixo, sob cava pedra,\\
longe de deuses imortais e homens mortais,\\
onde os deuses lhe atribuíram casa gloriosa p'ra morar.\\

\quad{}Ela fica nos \edtext{Arimos}{\nota{Não se sabe o que são (cadeia de montanhas? povo?) nem onde ficavam.}} sob a terra, a funesta Équidna,\\
moça imortal e sem velhice para todos os dias. \\
Com ela, dizem, Tifeu uniu-se em amor,\\
o violento, terrível e ímpio com a moça olhar-luzente;\\
ela, após engravidar, gerou rebentos juízo-forte.\\
Orto primeiro ela gerou, um cão para Gerioneu;\\
depois pariu o impossível, de todo impronunciável, \\
Cérbero come-cru, o cão bronzissonante de Hades,\\
cinquenta-cabeças, insolente e brutal;\\
como terceiro, gerou Hidra, versada no funesto,\\
de Lerna, a quem nutriu a divina Hera alvo-braço,\\
com imenso rancor da força de Héracles. \\
A ela matou o filho de Zeus com bronze impiedoso,\\
o filho de Anfitríon, com Iolau caro-a-Ares ---\\
Héracles --- pelos desígnios de Atena guia-tropa.\\
\edtext{E ela}{\nota{Não fica claro quem é ``ela'', Cetó, Hidra ou Équidna. ``Quimera'',
em grego, é ``cabra''.}} pariu Quimera, que fogo indômito soprava,\\
terrível, grande, pé-ligeiro, brutal.\\ 
Tinha três cabeças: uma, de leão olhar-cobiçoso,\\
outra, de cabra, a terceira, de cobra, brutal serpente.\\
\edtext{Na frente, leão, atrás, serpente, no meio, cabra,\\
soprando o fero ímpeto do fogo chamejante.}{\lemma{Na frente \ldots{} chamejante}{\nota{Como esses versos repetem dois versos da \emph{Ilíada} e estão
--- ou parecem estar --- em contradição com os dois versos anteriores, são
deletados por diversos editores.}}}\\
A ela pegaram Pégaso e o valoroso Belerofonte. \\
\edtext{E ela}{\nota{Não fica claro quem é ``ela'', Cetó, Quimera ou Équidna.}} pariu a ruinosa Esfinge, ruína dos cadmeus,\\
após ser subjugada por Orto, e o leão de Nemeia,\\
do qual Hera cuidou, a majestosa consorte de Zeus,\\
e o alocou nos morros de Nemeia, desgraça dos homens.\\
Ele, lá habitando, encurralava a linhagem de homens, \\
dominando Tretos, na Nemeia, e Apesas;\\
mas a ele subjugou o vigor da força de Héracles.\\

\quad{}Cetó, unida em amor a Fórcis, como o mais jovem\\
gerou terrível serpente, que nos confins da terra lúgubre,\\
nos grandes limites, guarda um rebanho todo de ouro.\\

\Para
E essa é a linhagem de Ceto e Fórcis.\\
E Tetís para Oceano pariu rios vertiginosos,\\
Nilo, Alfeios e Eridanos fundo-redemunho,\\
Estrímon, Maiandros e Istros bela-corrente,\\
Fásis, Resos e Aquelôo argênteo-redemunho, \\
Nessos, Ródios, Haliácmon, Heptaporos,\\
Grenicos, Esepos e o divino Simoente,\\
Peneios, Hermos e Caícos bem-fluente,\\
grande Sangarios, Ládon e Partênios,\\
Euenos, Aldescos e o divino Escamandro. \\
E pariu sacra linhagem de moças, que, pela terra,\\
a meninos tornam varões com o senhor Apolo\\
e com os rios, e de Zeus tem esse quinhão,\\
\edtext{Persuasão,}{\nota{\emph{Peithō}.}} \edtext{Indomada,}{\nota{\emph{Admētē}.}} \edtext{Violeta}{\nota{\emph{Ianthēe}.}} e \edtext{Brilhante,}{\nota{\emph{Elektrē}.}}\\
Dóris, \edtext{Sopé}{\nota{\emph{Prumnō}.}} e a divinal \edtext{Celeste,}{\nota{\emph{Ourania}.}} \\
\edtext{Equina,}{\nota{\emph{Hippō}.}} \edtext{Famosa,}{\nota{\emph{Klumenē}.}} \edtext{Rósea }{\nota{Rósea = \emph{Rhodeia}.}} e \edtext{Bonflux,}{\nota{\emph{Kalliroē}.}}\\
Zeuxó, \edtext{Gloriosa,}{\nota{\emph{Klutiē}.}} \edtext{Sapiente}{\nota{\emph{Iduia}.}} e \edtext{Admiradíssima,}{\nota{\emph{Pasithoē}.}}\\
Plexaure, Galaxaure e a encantadora Dione,\\
\edtext{Ovelheira,}{\nota{\emph{Melobosis}.}} \edtext{Veloz}{\nota{\emph{Thoē}.}} e \edtext{Muitadádiva}{\nota{\emph{Poludōrē}.}} bela-aparência,\\
a atraente \edtext{Lançadeira}{\nota{\emph{Kerkēis}.}} e \edtext{Riqueza}{\nota{\emph{Ploutō}.}} olho-bovino,\\ 
Perseís, Iáneira, Acaste e \edtext{Loira,}{\nota{\emph{Xanthē}.}}\\
a apaixonante \edtext{Pétrea,}{\nota{\emph{Petraiē}.}} \edtext{Potência}{\nota{\emph{Menesthō}.}} e Europa,\\
\edtext{Astúcia,}{\nota{\emph{Mētis}.}} Eurínome e \edtext{Círcula}{\nota{\emph{Telestō}.}} peplo-açafrão,\\
Criseís, Ásia e a desejável \edtext{Calipso,}{\nota{transliteração de \emph{Kalipso}, algo como ``Encobre''.}}\\
\edtext{Beladádiva,}{\nota{\emph{Eudōrē}.}} \edtext{Fortuna,}{\nota{\emph{Tukhē}.}} \edtext{Tornoflux}{\nota{\emph{Amphirō}.}} e \edtext{Celereflux,}{\nota{\emph{Okuroē}.}}\\
e Estige, essa que é a superior entre todas.\\
Essas nasceram de Oceano e Tetís,\\
as moças mais velhas. Também muitas outras há:\\
três mil são as Oceaninas tornozelo-fino,\\
elas que, bem-espalhadas, terra e profundas do mar, \\
todo lugar por igual, frequentam, filhas radiantes de deusas.\\
E tantos e distintos os rios que fluem estrepitantes,\\
filhos de Oceano, aos quais gerou a senhora Tetís:\\
deles, o nome de todos custa ao varão mortal narrar,\\
e estes o respectivo conhecem, os que moram no entorno. \\

\quad{}E Teia ao grande Sol, à fúlgida Lua\\
e à Aurora, que brilha para todos os mortais\\
e aos deuses imortais que do amplo céu dispõem,\\
gerou-os, subjugada em amor por Hipérion.\\
E para Creio Euribie pariu, unida em amor, \\
diva entre as deusas, o grande \edtext{Estrelado,}{\nota{\emph{Astraios}.}} Palas\\
e Perses, que entre todos sobressaía pela sapiência.\\
Para Estrelado Aurora pariu ventos ânimo-potente,\\
o clareante Zéfiro, Bóreas rota-ligeira\\
e Noto, em amor a deusa com o deus deitada. \\
Depois deles, \edtext{Nasce-Cedo}{\nota{Aurora.}} pariu \edtext{Estrela da Manhã}{\nota{\emph{Heōsphoros}, ``traz-aurora''.}}\\
e astros fulgentes, com os quais o céu se coroa.\\

\quad{}E Estige, filha de Oceano, pariu, unida a Palas,\\
\edtext{Emulação}{\nota{\emph{Zēlos}.}} e \edtext{Vitória}{\nota{\emph{Nikē}.}} linda-canela no palácio\\
e \edtext{Poder}{\nota{\emph{Kratos}.}} e \edtext{Força}{\nota{\emph{Biē}.}} gerou, filhos insignes. \\
Não fica longe de Zeus nem sua casa nem seu assento,\\
nem via por onde o deus na frente deles não vá,\\
mas sempre junto a Zeus grave-ressoo se assentam.\\
Pois assim Estige planejou, a Oceanina eterna,\\
no dia em que o relampejante olímpico a todos \\
os deuses imortais chamou ao grande Olimpo,\\
e disse que todo deus que com ele combatesse os Titãs,\\
desse não arrancaria suas mercês, e cada um a honra\\
teria tal como antes entre os deuses imortais.\\
Disse que quem não obtivera honra e mercê devido a Crono, \\
esse receberia honra e mercês, como é a norma.\\
Eis que veio por primeiro ao Olimpo a eterna Estige\\
com seus filhos devido aos projetos do caro pai;\\
a ela Zeus honrou e deu-lhe dons prodigiosos.\\
Pois dela fez a grande jura dos deuses, \\
e a seus filhos, por todos os dias, tornou seus coabitantes.\\
Assim como prometera para todos, sem exceção,\\
realizou; e ele mesmo tem grande poder e rege.\\

\quad{}E dirigiu-se Foibe ao desejável leito de Coio;\\
então engravidou a deusa em amor pelo deus \\
e gerou Leto peplo-negro, sempre amável,\\
gentil para com os homens e deuses imortais,\\
amável dês o início, a mais suave dentro do Olimpo.\\
E gerou a auspiciosa Astéria bom-nome, que um dia Perses\\
fez conduzir à grande casa para ser chamada sua esposa. \\

\quad{}Ela engravidou e pariu Hécate, a quem, mais que a todos,\\
Zeus Cronida honrou; e deu-lhe dádivas radiantes\\
para ela ter porção da terra e do mar ruidoso.\\
Ela também partilhou a honra do céu estrelado,\\
e pelos deuses imortais é sumamente honrada. \\
Também agora, quando um homem mortal\\
faz belos sacrifícios regrados para os propiciar,\\
invoca Hécate: bastante honra segue aquele,\\
fácil, de quem, benévola, a deusa aceita preces,\\
e a ele oferta fortuna, pois a potência está a seu lado. \\
Tantos quantos de Terra e Céu nasceram\\
e granjearam honraria, de todos ela tem uma porção\\
e com ela o Cronida em nada foi violento nem usurpou\\
daquilo que granjeou entre os Titãs, primevos deuses,\\
mas possui como foi, dês o início, a divisão original. \\
Nem, sendo filha única, tem menor porção de honra\\
e de mercês na terra, no céu e no mar,\\
mas ainda muito mais, pois Zeus a honra.\\
Para quem ela quiser, magnificente, fica ao lado e favorece:\\
na assembleia, entre o povo se destaca quem ela quiser; \\
e quando rumo à batalha aniquiladora se armam\\
os varões, a deusa ao lado fica daquele a quem quer,\\
benevolente, vitória ofertar e glória estender.\\
Num julgamento senta-se junto a reis respeitáveis,\\
e valorosa é sempre que varões disputam uma prova: \\
aí a deusa também fica ao lado deles e os favorece,\\
e, tendo vencido pela força e vigor, belo prêmio\\
ele fácil leva, alegre, e aos pais oferta a glória.\\
É valorosa ao se por junto a cavaleiros, aos que quer,\\
e para estes que trabalham o glauco encrespado \\
e fazem prece a Hécate e a \edtext{Treme-Solo ressoa-alto,}{\nota{``Treme-Solo'' e ``ressoa-alto'' são epítetos de Posêidon e geralmente identificam o deus neste poema.}}\\
fácil a deusa majestosa oferta muita presa,\\
e fácil a tira quando aparece, se no ânimo quiser.\\
Valorosa é com Hermes, nas quintas, no aumentar os bens:\\
rebanhos de gado, amplos rebanhos de cabras, \\
rebanhos de ovelhas lanosas, se ela no ânimo quiser,\\
de poucos, os fortalece, e de muitos, torna menores.\\
Assim, embora sendo filha única da mãe,\\
entre todos os imortais é honrada com mercês.\\
O Cronida tornou-a nutre-jovem dos que, depois dela, \\
com os olhos veem a luz de Aurora muito-observa.\\
Assim, dês o início é nutre-jovem, e essas, as honras.\\

\Para
E Reia, subjugada por Crono, pariu filhos insignes,\\
Héstia, Deméter e Hera sandália-dourada,\\
e o altivo Hades, que sob a terra habita sua casa \\
com coração impiedoso, e Treme-Solo ressoa-alto,\\
e o astuto Zeus, pai de deuses e homens,\\
cujo raio sacode a ampla terra.\\
A esses engolia o grande Crono, quando cada um\\
se dirigisse do sacro ventre aos joelhos da mãe, \\
pensando isso para nenhum ilustre celeste,\\
um outro entre os imortais, obter a honraria real.\\
Pois escutara de Terra e do estrelado Céu\\
que lhe estava destinado ser subjugado por seu filho ---\\
embora mais poderoso, pelos desígnios do grande Zeus. \\
Por isso não mantinha vigia cega, mas, observador,\\
engolia seus filhos; e a Reia dominava aflição inesquecível.\\
Mas quando iria a Zeus, pai de deuses e homens,\\
parir, nisso ela então suplicou aos caros genitores,\\
aos seus próprios, Terra e Céu estrelado, \\
com ela planejarem ardil para, sem ser notada, parir\\
o caro filho e fazer Crono pagar às \edtext{erínias}{\nota{espíritos de vingança.}} do pai\\
e dos filhos que ele engolia, o grande Crono curva-astúcia.\\
Eles à cara filha ouviram bem e obedeceram\\
e lhe apontaram tudo destinado a ocorrer \\
acerca do rei Crono e do filho ânimo-potente.\\
Enviaram-na a Lictos, à fértil região de Creta,\\
quando iria parir o mais novo dos filhos,\\
o grande Zeus; a esse recebeu a portentosa Terra\\
na ampla Creta para criar e alimentar. \\
Lá ela chegou, levando-o pela negra noite veloz,\\
primeiro a Lictos; pegou-o nos braços e escondeu
em gruta rochosa, sob os recessos numinosos da terra,\\
na montanha Egeia, coberta de mato cerrado.\\
Em grande pedra pôs um cueiro e àquele o estendeu, \\
ao grande senhor filho de Céu, rei dos deuses primevos.\\
Pegou-a então com as mãos e em seu ventre depositou,\\
o terrível, e não notou no juízo que para ele, no futuro,\\
ao invés da pedra seu filho invencível e sereno\\
ficou, quem logo o iria subjugar com força e braços, \\
o despojaria de sua honra e entre os imortais regeria.\\

\quad{}Eis que celeremente ímpeto e membros insignes\\
do senhor cresceram; e após um ano passar,\\
ludibriado pela sugestão refletida de Terra,\\
sua prole regurgitou o grande Crono curva-astúcia, \\
vencido pela arte e força do próprio filho.\\
Primeiro vomitou a pedra, que por último engolira;\\
a ela Zeus fixou na terra largas-rotas\\
na divina Pitó, embaixo nas \edtext{reentrâncias do Parnasso,}{\nota{Ou seja, em Delfos.}}\\
sinal aos vindouros, assombro aos homens mortais. \\

\quad{}E soltou os irmãos do pai de seus laços ruinosos,\\
filhos de Céu, que prendera o pai devido a cego juízo;\\
eles, pela boa ação, retribuíram com um favor,\\
e deram-lhe trovão, raio chamejante\\
e relâmpago: antes a portentosa Terra os mantivera ocultos; \\
com o apoio desses, ele rege sobre mortais e imortais.\\

\quad{}E Jápeto a moça linda-canela, a Oceanina\\
Famosa, fez ser conduzida e subiu no leito comum.\\
Ela gerou-lhe, como filho, Atlas juízo-forte\\
e pariu Menoitio super-majestoso, Prometeu, \\
o variegado astúcia-cintilante, e o equivocado Epimeteu;\\
um mal foi esse, dês o início, aos homens come-grão:\\
recebeu originalmente, modelada, uma mulher\\
moça. E ao violento Menoitio Zeus ampla-visão\\
à escuridão abaixo enviou ao acertá-lo com raio fumoso \\
por causa de iniquidade e insolente virilidade.\\
Atlas sustém o amplo céu, sob imperiosa necessidade,\\
nos limites da terra ante as Hespérides clara-voz\\
parado, com a cabeça e incansáveis braços:\\
esse quinhão lhe atribuiu o astuto Zeus. \\
Prendeu a grilhões Prometeu desígnio-variegado,\\
a laços aflitivos, pelo meio puxando um pilar.\\
Contra ele instigou águia asa-longa; essa ao fígado\\
imortal comia, e esse crescia por completo, igual,\\
de noite, o que de dia comeria a ave asa-longa. \\
Eis que a ela o bravo filho de Alcmena linda-canela,\\
Héracles, matou, e afastou a praga vil\\
do filho de Jápeto e libertou-o das amarguras\\
não contra o olímpico Zeus que do alto rege,\\
para que o tebano Héracles tivesse fama \\
ainda mais que no passado sobre o solo nutre-muitos.\\
Assim, respeitando-o, Zeus honrava o insigne filho;\\
embora irado, cessou a raiva que antes tinha,\\
pois desafiara os desígnios do impetuoso Cronida.\\

\quad{}De fato, quando deuses e homens mortais se distinguiam \\
em Mecone, nisso grande boi, com ânimo resoluto,\\
Prometeu dividiu e dispôs, tentando enganar o espírito de Zeus.\\
Pois, para um, carne e entranhas fartas em gordura\\
na pele colocou, escondendo no ventre bovino;\\
para os outros, brancos ossos do boi com arte ardilosa \\
arrumou e dispôs, escondendo com branca gordura.\\

\quad{}Então lhe disse o pai de varões e deuses:\\
``Filho de Jápeto, insigne entre todos os senhores,\\
meu caro, que modo parcial de dividir as porções''.\\

\quad{}Assim provocou- Zeus, mestre em ideias imperecíveis; \\
e a ele retrucou Prometeu curva-astúcia,\\
de leve sorriu e não esqueceu a arte ardilosa:\\
``Majestoso Zeus, maior dos deuses sempiternos,\\
dessas escolhe a que no íntimo o ânimo te ordena''.\\

\quad{}Falou ardilosamente; Zeus, mestre em ideias imperecíveis, \\
atentou, não desatento ao ardil; olhou com males no ânimo\\
contra os homens mortais, os quais iriam se cumprir.\\
Com ambas as mãos, pegou a gordura branca\\
e irou-se no juízo, e raiva alcançou seu ânimo\\
quando viu brancos os ossos do boi, fruto da arte ardilosa. \\
Daí, aos imortais as tribos de homens sobre a terra\\
queimam brancos ossos sobre altares fragrantes.\\

\quad{}Muito perturbado, disse-lhe Zeus junta-nuvens:\\
``Filho de Jápeto, supremo mestre em planos,\\
meu caro, pois não esqueceste a arte ardilosa''. \\

\quad{}Assim falou, irado, Zeus, mestre em ideias imperecíveis.\\
Depois disso, então, da raiva sempre se lembrando,\\
\edtext{não dava aos freixos o ímpeto do fogo incansável\\
para os homens mortais,}{\lemma{não dava \ldots{} homens mortais}{\nota{versos problemáticos; uma pequena alteração poderia redundar em
``não dava o ímpeto do fogo incansável para os homens mortais (nascidos
das ninfas) dos freixos''.}}} que sobre a terra habitam.\\
Mas a ele enganou o brioso filho de Jápeto \\
ao roubar o clarão visto-ao-longe do fogo incansável\\
em cavo funcho-gigante: isso mordeu o ânimo\\
de Zeus troveja-no-alto, e enraiveceu-se em seu coração\\
ao fitar entre os homens o clarão visto-ao-longe do fogo.\\
De pronto, pelo fogo fabricou um mal para os homens: \\
da terra modelou o gloriosíssimo \edtext{Duas-Curvas,}{\nota{Epíteto que identifica Hefesto.}}\\
pelos desígnios do Cronida, a imagem de uma moça respeitada.\\
A ela cinturou e adornou a deusa, Atena olhos-de-coruja,\\
com veste argêntea; cabeça abaixo um véu\\
adornado, com as mãos, fez pender, assombro à visão; \\
\edtext{em volta dela, coroas broto-novo de flores do prado,\\
desejáveis, pôs Palas Atena em sua a cabeça.}{\lemma{em volta dela \ldots{} a cabeça}{\nota{versos deletados por muitos editores (Marg, West); Most os mantém.}}}\\
Em volta dela, pôs coroa dourada na cabeça,\\
que ele próprio fizera, o gloriosíssimo Duas-Curvas,\\
ao labutar com as palmas, comprazendo ao pai Zeus. \\
Nela muito adorno foi fabricado, assombro à visão,\\
tantos animais terríveis quantos nutrem terra e mar;\\
muitos desses nela pôs, e graça sobre todos soprou,\\
admiráveis, semelhantes a criaturas com voz.\\

\quad{}E após fabricar o belo mal pelo bem, \\
levou-a aonde estavam os outros deuses e homens,\\
ela feliz com o adorno da \edtext{Olhos-de-Coruja de pai ponderoso.}{\nota{Dois epítetos comuns de Atena, filha de Zeus.}}\\
Assombro tomou os deuses imortais e os homens mortais\\
quando viram o íngreme ardil, impossível para os homens.\\
\edtext{Pois dela vem a linhagem das bem femininas mulheres, \\
pois é dela a linhagem ruinosa, as tribos de mulheres,}{\lemma{Pois dela \ldots{} tribos de mulheres}{\nota{versos muito parecidos, o que faz a maioria dos editores optar por um ou outro.}}}\\
grande desgraça aos mortais, morando com varões,\\
camaradas não da ruinosa Pobreza, mas de Abundância.\\
Como quando abelhas, em colmeias arqueadas,\\
alimentam zangões, parceiros de feitos vis: \\
elas, o dia inteiro até o sol se pôr,\\
todo dia se apressam e favos luzidios depositam,\\
e eles ficam dentro nas colmeias salientes\\
e a faina alheia para o próprio ventre recolhem ---\\
bem assim as mulheres, mal aos homens mortais, \\
Zeus troveja-no-alto impôs, parceiras de feitos\
aflitivos. E outro mal forneceu pelo bem:\\
quem das bodas fugir e dos feitos \edtext{devastadores}{\nota{``Devastadores'' busca traduzir \emph{mermera}, um termo de sentido algo incerto.}} das mulheres\\
e não quiser casar, atingirá velhice ruinosa\\
carente de quem o cuide; não privado de sustento \\
vive, mas, ao perecer, dividem seus recursos\\
parentes distantes. Já quem partilhar do casamento\\
e obtiver consorte devotada, ajustada em suas ideias,\\
para ele, dês a juventude, o mal contrabalança o bem\\
sempre; e quem encontrar espécie insultante, \\
vive com irritação incessante no íntimo,\\
no ânimo e no coração, e o mal é incurável.\\

\quad{}Assim não se pode lograr nem ultrapassar a mente de Zeus.\\
Pois nem o filho de Jápeto, o \edtext{benéfico Prometeu,}{\nota{O sentido do epíteto grego traduzido por ``benéfico'' é, na verdade, obscuro.}}\\
se esquivou de sua raiva pesada, mas, sob coação, \\
embora multi-perspicaz, grande laço o subjuga.\\

\Para

Assim que o pai teve ódio no ânimo por Obriareu,\\
Coto e Giges, prendeu-os em laço forte,\\
irritado com a virilidade insolente, a aparência\\
e a altura; e alocou-os embaixo da terra largas-rotas. \\
Lá eles, que sofriam habitando sob a terra,\\
estavam sentados na ponta, nos limites da grande terra,\\
há muito angustiados com grande pesar no coração.\\
Mas a eles o Cronida e outros deuses imortais,\\
os que Reia belas-tranças pariu em amor por Crono, \\
graças ao plano de Terra, levaram de volta à luz:\\
ela tudo lhes contara, do início ao fim,\\
como com aqueles obter vitória e triunfo radiante.\\
Pois muito tempo lutaram em pugna aflige-ânimo,\\
uns contra os outros em batalhas brutais, \\
os deuses Titãs e todos os que nasceram de Crono, \\
aqueles a partir do alto Otris, os ilustres Titãs, \\
estes a partir do Olimpo, os deuses oferentes de bens,\\
os que pariu Reia belas-tranças deitada com Crono.\\
Eles então entre si, em pugna aflige-ânimo, \\
sem parar pelejaram dez anos inteiros;\\
solução não havia \edtext{para a dura briga,}{\nota{A saber, Coto, Obriareu e Giges.}} nem fim\\
para lado algum, e o remate da guerra se equilibrava.\\

\quad{}Mas quando, vê, ofertou-lhes tudo que é adequado,\\
néctar e ambrosia, o que comem os próprios deuses, \\
e no íntimo de todos avolumou-se o ânimo arrogante\\
\edtext{quando comeram o néctar e a desejável ambrosia,}{\nota{Diversos editores deletam o verso.}}\\
nisso então entre eles falou o pai de deuses e homens:\\
``Ouvi-me, filhos radiantes de Terra e Céu,\\
para eu dizer o que o ânimo no peito me ordena. \\
Já muito tempo uns contra os outros\\
pela vitória e poder combatemos todo dia,\\
os deuses Titãs e todos os que nascemos de Crono.\\
Vós grande força e mãos intocáveis\\
mostrai em oposição aos Titãs no prélio funesto \\
ao se lembrar da amizade afável, quanto sofreram\\
e de novo a luz alcançaram, soltos do laço tenebroso\\
graças a nossos desígnios, vindos das trevas brumosas''.\\

\quad{}Assim falou; logo lhe respondeu o impecável Coto:\\
``Honorável, não anuncias algo ignoto, mas também nós \\
sabemos que sobressais no discernimento e na ideia,\\
e te tornaste protetor dos imortais contra dano gelado,\\
e com tua sagacidade, vindos das trevas brumosas,\\
de volta de novo, dos laços inamáveis,\\
viemos, senhor Cronida, após sofrer o inesperado. \\
Assim também agora, com ideia tenaz e ânimo resoluto,\\
protegeremos vosso poder na refrega terrível,\\
combatendo os Titãs nas batalhas brutais''.\\
Assim falou; e aprovaram os deuses oferentes de bens\\
o discurso após o ouvir: à peleja almejou seu ânimo \\
mais ainda que antes; e à luta não invejável acordaram\\
todos, fêmeas e machos, naquele dia,\\
os deuses Titãs e todos os que nasceram de Crono,\\
e os que Zeus da escuridão, sob a terra, à luz enviou,\\
terríveis e brutais, com força insolente. \\
De seus ombros cem braços se lançavam,\\
igual para todos, e cabeças, em cada um, cinquenta\\
nasceram dos ombros sobre os membros robustos.\\
Contra os Titãs então se postaram no prélio funesto\\
com rochas alcantiladas nas mãos robustas; \\
os Titãs, do outro lado, revigoraram suas falanges\\
com afã: ação conjunta de braços e de força mostraram\\
ambos, e o mar sem-fim em volta rugia, terrível,\\
e a terra, alto, ribombava, e gemia o amplo céu\\
sacudido, e tremia do fundo o enorme Olimpo \\
com o arremesso dos imortais, e tremor atingia, pesado,\\
dos pés, o Tártaro brumoso, bem como agudo zunido\\
do fragor indizível e dos arremessos brutais.\\
Assim uns nos outros lançavam projéteis desoladores;\\
alcançava o céu estrelado o som de ambas as partes, \\
das exortações; e se chocaram com grande algaraviada.\\

\quad{}E Zeus não mais conteve seu ímpeto, mas dele agora\\
de pronto o peito se encheu de ímpeto, e toda\\
a força mostrou. Ao mesmo tempo, do céu e do Olimpo\\
relampejando, progrediu sem parar, e os raios \\
em profusão, com trovão e relâmpago, voavam\\
de sua mão robusta, revolvendo a sagrada chama,\\
em massa. Em volta, ribombava a terra traz-víveres,\\
queimando, e, no entorno, alto chiava mato incontável.\\
Todo o solo fervia, as correntes de Oceano \\
e o mar ruidoso; a eles rodeava o bafo quente,\\
aos terrestres Titãs, e chama alcançou a bruma divina,\\
indizível, e aos olhos deles, embora altivos, cegou\\
a luz cintilante do raio e do relâmpago.\\
Prodigiosa queimada ocupou o \edtext{abismo;}{\nota{Ou ``Abismo''.}} parecia, em face\\
olhando-se com olhos e com ouvidos ouvindo-se o rumor,\\
assim como quando Terra e o amplo Céu acima\\
se reuniram: tal ressoo, enorme, subiu,\\
ela pressionada e ele, do alto, pressionando ---\\
tamanho baque quando os deuses se chocaram na briga. \\
Junto, ventos engrossavam o tremor, a poeira,\\
trovão, raio e relâmpago em fogo,\\
setas do grande Zeus, e levavam grito e assuada\\
ao meio de ambas as partes: veio imenso clangor\\
da briga aterrorizante, e o feito do poder se mostrou. \\

\quad{}E a batalha se inclinou; antes, com avanços recíprocos,\\
pelejavam sem cessar em batalhas audazes.\\
Estes, entre os da frente, acordaram peleja lancinante,\\
Coto, Briareu e Giges, insaciável na guerra:\\
eles trezentas pedras de suas mãos robustas \\
enviavam em sucessão, e com os projéteis sombrearam\\
os Titãs; e a eles para baixo da terra largas-rotas\\
enviaram e com laços aflitivos prenderam,\\
após vencê-los no braço, embora autoconfiantes,\\
tão longe abaixo da terra quanto o céu está da terra. \\

\Para
Tal a distância da terra até o Tártaro brumoso.\\
Pois por nove noites e dias bigorna de bronze,\\
caindo do céu, no décimo a terra alcançaria;\\
\skipnumbering{[}\edtext{por sua vez, igual da terra até o Tártaro brumoso.}{\nota{A maioria dos editores rejeita esse verso.}}{]}\\   %\num{723a}
De novo, por nove noites e dias bigorna de bronze,\\
da terra caindo, no décimo o Tártaro alcançaria. \\
Em volta dele, corre muro de bronze; no entorno, noite\\
camada-tripla derrama-se em volta da garganta; acima,\\
crescem as raízes da terra e do mar ruidoso.\\

\quad{}Para lá os deuses Titãs, sob brumosa escuridão,\\
foram removidos pelos desígnios de Zeus junta-nuvem, \\
em região bolorenta, extremos da terra portentosa.\\
É-lhes impossível sair, Posêidon fixou portões\\
de bronze, e muralha corre para os dois lados.\\

\quad{}Lá Giges, Coto e o animoso Obriareu\\
habitam, fiéis guardiões de Zeus porta-égide. \\

\quad{}Lá da terra escura, do Tártaro brumoso,\\
do mar ruidoso e do céu estrelado\\
as fontes e limites, de tudo, em ordem estão,\\
aflitivos, bolorentos, aos quais até os deuses odeiam;\\
grande fenda, e nem no ciclo de um ano inteiro \\
alguém atingiria o chão, os portões uma vez cruzados,\\
mas p'ra lá e p'ra cá o levaria rajada após rajada,\\
aflitiva: assombroso é também para deuses imortais\\
esse prodígio; e a morada assombrosa de Noite\\
está de pé, escondida em nuvem cobalto. \\

\quad{}Na frente, o filho de Jápeto sustém o amplo céu,\\
parado, com a cabeça e braços incansáveis,\\
imóvel, onde Noite e Dia passam perto\\
e falam entre si ao cruzarem o grande umbral\\
de bronze: uma entra e a outra pela porta \\
vai, e nunca a ambas a casa dentro encerra,\\
mas sempre uma delas deixa a casa\\
e à terra se dirige, e a outra na casa fica\\
e, até aquela chegar, aguarda a sua hora de ir.\\
Uma, para os mortais na terra, tem luz muito-observa; \\
a outra tem nas mãos Sono, irmão de Morte,\\
a ruinosa Noite, escondida em nuvem embaçada.\\

\quad{}Lá habitam os filhos da lúgubre Noite,\\
Sono e Morte, deuses terríveis; nunca a eles\\
Sol, alumiando, observa com os raios \\
quando sobe ao céu nem quando desce do céu.\\
Deles, um à terra e ao largo dorso do mar,\\
calmo, se dirige, amável para os homens,\\
e do outro o ânimo é de ferro, e de bronze, seu coração\\
impiedoso no peito: segura assim que pega algum \\
dos homens; é odioso até aos deuses imortais.\\

\quad{}Lá na frente, a morada ruidosa do deus terrestre,\\
\edtext{o altivo Hades, e da atroz Perséfone\\
está de pé, e terrível cão vigia na frente,\\
impiedoso, com arte vil: para quem entra, \\
abana por igual o rabo e as duas orelhas\\
e não permite que de volta saia, mas, ao perceber,\\
come quem pegar saindo pelos portões\\
do altivo Hades e da atroz Perséfone.}{\lemma{o altivo \ldots{} Perséfone}{\nota{versos iguais; ambos são prováveis interpolações.}}}\\

\quad{}Lá habita a deusa, \edtext{estigma}{\nota{``Estigma'' procura reproduzir a sugestão poética de que ``Estige'' (\emph{Stux}) derivaria de ``odioso'' (\emph{stugeros}); no
grego, ``odioso para os imortais''.}} para os imortais, \\
a terrível Estige, filha de Oceano flui-de-volta,\\
primogênita: longe dos deuses, habita casa gloriosa\\
com abóboda de grandes pedras; em todo seu entorno,\\
colunas de prata a sustentam rumo ao céu.\\
Raramente a filha de Taumas, a velocípede Íris, \\
vem com mensagem sobre o largo dorso do mar.\\
Quando briga e disputa se instaura entre imortais,\\
e se mente um dos que têm morada olímpia,\\
Zeus envia Íris para trazer a grande jura dos deuses\\
de longe, em jarra de ouro, a renomada água, \\
gelada, que goteja de rocha alcantilada,\\
elevada: do fundo da terra largas-rotas, muito\\
flui do sacro rio através da negra noite ---\\
braço de Oceano, e a décima parte a ela foi atribuída;\\
nove partes, em torno da terra e do largo dorso do mar, \\
com remoinho prateado ele gira e cai no mar,\\
e ela, uma só, da rocha flui, grande aflição dos deuses.\\
Quem, com ela tendo libado, jurar em falso,\\
um imortal dos que possuem o pico do Olimpo nevado,\\
esse jaz sem respirar até um ano se completar; \\
nunca de ambrosia e néctar se aproxima\\
quanto à comida, mas jaz sem fôlego e sem voz\\
num leito estendido, e sono vil o encobre.\\
Após cumprir a praga no grande dia ao fim do ciclo,\\
a essa prova segue outra ainda mais cruel: \\
por nove anos, é privado dos deuses sempre vivos,\\
e nunca se junta a eles em conselho ou banquete\\
por nove anos inteiros; no décimo, se junta de novo\\
às reuniões dos imortais que têm morada olímpia.\\
Tal jura os deuses fizeram da água eterna de Estige, \\
primeva; e ela flui através da terra escarpada.\\

\quad{}Lá da terra escura, do Tártaro brumoso,\\
do mar ruidoso e do céu estrelado\\
as raízes e limites, de tudo, em ordem estão,\\
aflitivos, bolorentos, aos quais até os deuses odeiam. \\

\quad{}Lá ficam os portões luzidios e o umbral de bronze,\\
ajustados, imóveis, com raízes contínuas,\\
naturais; na frente, longe de todos os deuses,\\
habitam os Titãs, para lá do abismo penumbroso.\\
E os gloriosos aliados de Zeus troveja-alto \\
habitam casas nos fundamentos de Oceano,\\
Coto e Giges; quanto a Briareu, sendo valoroso,\\
fez dele seu genro Agita-a-Terra grave-ressoo,\\
e deu-lhe \edtext{Flanonda,}{\nota{\emph{Kumopoleia}.}} sua filha, para desposar.\\

\Para
Mas depois que Zeus expulsou os Titãs do céu, \\
pariu Tifeu, o filho mais novo, a portentosa Terra\\
em amor por Tártaro devido à dourada Afrodite:\\
dele, os braços \dagger{}\edtext{façanhas seguram sobre a energia}{\nota{Verso corrupto.}}\dagger{},\\
e são incansáveis os pés do deus brutal; de seus ombros\\
havia cem cabeças de cobra, brutal serpente, \\
movendo escuras línguas; de seus olhos,\\
nas cabeças prodigiosas, fogo sob as celhas luzia,\\
e de toda a cabeça fogo queimava ao fixar o olhar.\\
Vozes havia em toda cabeça assombrosa,\\
som de todo tipo emitindo, ilimitado: ora \\
soavam como se para deuses entenderem, ora\\
voz de touro guincho-alto, ímpeto incontido, altivo,\\
ora, por sua vez, a de leão de ânimo insolente,\\
ora semelhante a cachorrinhos, assombro de se ouvir,\\
ora sibilava, e, abaixo, grandes montanhas ecoavam. \\
Feito impossível teria havido naquele dia,\\
e ele de mortais e imortais teria se tornado senhor,\\
se não tivesse notado, arguto, o pai de varões e deuses:\\
trovejou de forma dura e ponderosa, em torno a terra\\
ecoou, aterrorizante, e também, acima, o amplo céu, \\
o mar, as correntes de Oceano e o Tártaro da terra.\\
Sob os pés imortais, o grande Olimpo foi sacudido\\
quando o senhor se lançou; e a terra gemia em resposta.\\
Queimada abaixo dos dois tomou conta do mar violeta\\
vinda do trovão, do raio e do fogo desse portento, \\
\edtext{dos ventos de ígneos tornados}{\nota{Sintaxe ambígua; ``dos ventos de ígneos tornados'' pode referir-se às armas de Zeus ou ao modo de combater de Tifeu.}} e do relâmpago ardente;\\
todo o solo fervia, e o céu e o mar:\\
grandes ondas grassavam no entorno das praias\\
com o jato dos imortais, e tremor inextinguível se fez;\\
Hades, que rege os ínferos finados, amedrontou-se, \\
e os Titãs, embaixo no Tártaro, em volta de Crono,\\
com o inextinguível zunido e a refrega apavorante.\\

\quad{}Zeus, após rematar seu ímpeto, pegou as armas,\\
trovão, raio e o chamejante relâmpago,\\
e golpeou-o arremetendo do Olimpo; em volta, todas \\
as cabeças prodigiosas do terrível portento queimou.\\
Após subjugá-lo, tendo-o com golpes fustigado,\\
o outro tombou, aleijado, e gemeu a portentosa Terra;\\
\edtext{e a chama fugiu desse senhor, relampejado,\\
nos vales da montanha escura, escarpada, \\
ao ser atingido, e a valer queimou a terra portentosa\\
com o bafo prodigioso, e fundiu-se como estanho,\\
em cadinhos bem furados, com arte por varões\\
aquecido, ou ferro, que é a coisa mais forte,\\
nos vales de montanha subjugado por fogo ardente \\
funde-se em solo divino pelas mãos de Hefesto ---}{\lemma{e a chama \ldots{} Hefesto ---}{\nota{manteve-se na tradução certa obscuridade da sintaxe arrevesada do original. Na comparação, estanho e ferro são coordenados: a terra
fundiu-se como o estanho trabalhado por jovens metalúrgicos ou o ferro
fundido por Hefesto.}}}\\
assim \edtext{fundiu-se}{\nota{Pucci (2009) nota que o verbo ``fundir'', nos versos 862 e 867, guarda paralelos sonoros com o verbo ``parir'' (v. 821) que abre o
episódio, respecitivamente, (\emph{e})\emph{tēketo} e \emph{teke}.}} a terra com a fulgência do fogo chamejante.\\
E arremessou-o, atormentado no ânimo, no largo Tártaro.\\

\quad{}De Tifeu é o ímpeto dos ventos de úmido sopro,\\
exceto Noto, Bóreas e o clareante Zéfiro, \\
que são de cepa divina, de grande valia aos mortais.\\
As outras brisas à toa sopram no oceano;\\
quanto à elas, caindo no mar embaçado,\\
grande desgraça aos mortais, correm com rajada má:\\
sopram p'ra cá depois p'ra lá, despedaçam naus \\
e nautas destroem; contra o mal não há defesa\\
para homens que com elas se deparam no mar.\\
Essas também, na terra sem-fim, florida,\\
lavouras amadas destroem dos homens na terra nascidos,\\
enchendo-as de poeira e confusão aflitiva. \\

\Para
Mas após a pugna cumprirem os deuses venturosos\\
e com os Titãs as honrarias separarem à força,\\
então instigaram a ser rei e senhor,\\
pelo plano de Terra, ao olímpico Zeus ampla-visão ---\\
dos imortais; e ele bem distribuiu suas honrarias.\\

\quad{}Zeus, rei dos deuses, fez de \edtext{Astúcia}{\nota{\emph{Mētis}.}} a primeira esposa,\\
a mais inteligente entre os deuses e homens mortais.\\
Mas quando ela iria à deusa, Atena olhos-de-coruja,\\
parir, nisso, com um truque, ele enganou seu juízo\\
e com contos solertes depositou-a em seu ventre \\
graças ao plano de Terra e do estrelado Céu:\\
assim lhe aconselharam, para a honraria real\\
outro dos deuses sempiternos, salvo Zeus, não ter.\\
Pois dela foi-lhe destinado gerar filhos bem-ajuizados:\\
primeiro a filha olhos-de-coruja, a \edtext{Tritogênia,}{\nota{termo de significado desconhecido, possivelmente aludindo a um lugar (mítico?) onde Atena teria nascido.}} \\
com ímpeto igual ao do pai e desígnio refletido,\\
e eis que então um filho, rei dos deuses e varões,\\
possuindo brutal coração, iria gerar;\\
mas Zeus depositou-a antes em seu ventre\\
para a deusa lhe aconselhar sobre o bem e o mal. \\

\quad{}A segunda, fez conduzir a luzidia Norma, mãe das \edtext{Estações,}{\nota{\emph{Hōrai} (sing. \emph{Hōra}).}}\\
\edtext{Decência,}{\nota{\emph{Eunomiē}.}} \edtext{Justiça}{\nota{\emph{Dikē}.}} e a luxuriante \edtext{Paz,}{\nota{\emph{Eirēnē}.}}\\
elas que \edtext{zelam}{\nota{``Zelar'' (\emph{ōrein}) ecoa \emph{Hōra} (``estação'').}}
pelos \edtext{trabalhos}{\nota{\emph{erga}, aqui traduzido por ``trabalhos'', também pode se referir a ``lavouras'', como no verso 879. O conjunto --- trabalho agrícola e
virtudes cívicas --- é como que uma síntese das ideias desenvolvidas por
Hesíodo em \emph{Trabalhos e dias}.}} dos homens mortais,\\
e as \edtext{Moiras,}{\nota{As Moiras também são filhas da Noite; a dupla origem parece indicar que as ações das deusas podiam ser pensadas de formas distintas e/ou
remeter a tradições (locais) diversas.}} a quem deu suma honraria o astuto Zeus,\\
Fiandeira, Sorteadora e Inflexível, que concedem \\
aos homens mortais bem e mal como seus.\\

\quad{}Três Graças bela-face lhe pariu Eurínome,\\
a filha de Oceano, com aparência desejável,\\
\edtext{Radiância,}{\nota{\emph{Aglaiē}.}} \edtext{Alegria}{\nota{\emph{Euphrosunē}.}} e a atraente \edtext{Festa:}{\nota{\emph{Thaliē}.}}\\
de suas pálpebras, quando olham, pinga desejo \\
solta-membros; belo é o olhar sob as celhas.\\

\quad{}E dirigiu-se ao leito de Deméter multinutriz:\\
ela pariu Perséfone alvos-braços, que \edtext{Aidoneu}{\nota{Aidoneu é Hades.}}\\
raptou de junto da mãe, e deu-lha o astuto Zeus.\\

\quad{}Por Memória então se enamorou, a belas-tranças, \\
e dela as Musas faixa-dourada lhe nasceram,\\
nove, às quais agradam festas e o prazer do canto.\\

\quad{}E Leto a Apolo e Ártemis verte-setas,\\
prole desejável mais que todos os Celestes,\\
gerou, após unir-se em amor com Zeus porta-égide. \\

\quad{}Como última, de Hera fez sua viçosa consorte:\\
ela pariu \edtext{Juventude,}{\nota{\emph{Hēbē}.}} Ares e Eilêitia,\\
unida em amor com o rei dos deuses e homens.\\

\quad{}Ele próprio da cabeça gerou Atena olhos-de-coruja,\\
terrível atiça-peleja, conduz-exército, \edtext{infatigável,}{\nota{Embora aqui traduzido por ``infatigável'', o sentido original do
adjetivo \emph{atrutonē}, utilizado somente para Atena, é desconhecido.
``Infatigável'' e ``invencível'' eram as glosas mais comuns na
Antiguidade.}}\\ 
senhora a quem agradam gritaria, guerras e combates.\\
E Hera ao glorioso Hefesto, não unida em amor,\\
\edtext{gerou, pois,}{\nota{Um caso de \emph{husteron proteron}, ou seja, o recurso
estilístico-narrativo no qual o que acontece antes é mencionado em
segundo lugar. A conjunção ``pois'' não está em grego; é acrescentada
para não tornar a frase incompreensível para o leitor da tradução.}} enfurecida, brigou com seu marido:\\
aquele nas artes supera todos os Celestes.\\

\quad{}E de Anfitrite e de Treme-Solo ressoa-alto \\
nasceu o grande Tríton ampla-força, que do mar\\
a base ocupa e junto à cara mãe e ao senhor pai\\
habita casa dourada, o deus terrível. \edtext{E para Ares\\
\edtext{fura-pele}{\nota{pode dizer respeito à pele do herói ferido ou ao
couro do escudo.}} Citereia pariu}{\lemma{E para Ares \ldots{} Pânico}{\nota{Na \emph{Odisseia}, Afrodite é representada como amante de Ares, mas casada com Hefesto, que, por sua vez, na \emph{Teogonia} e em outros textos, é representado casado com uma Graça.}}} \edtext{Terror}{\nota{\emph{Phobos}.}} e \edtext{Pânico,}{\nota{\emph{Deimos}.}}\\
terríveis, que tumultuam cerradas falanges de varões \\
com Ares arrasa-urbe em sinistra batalha,\\
e \edtext{Harmonia,}{\nota{Harmonia é um termo grego.}} a quem o autoconfiante Cadmo desposou.\\

\quad{}Para Zeus a filha de Atlas, Maia, pariu o glorioso Hermes,\\
arauto dos deuses, após subir no sacro leito.\\

\quad{}E a filha de Cadmo, Semele, gerou-lhe filho insigne, \\
unida em amor, Dioniso muito-júbilo,\\
a mortal ao imortal: ambos agora são deuses.\\

\quad{}E Alcmena pariu a força de Héracles,\\
unida em amor com Zeus junta-nuvem.\\

\quad{}E de Radiância o esplêndido Hefesto duas-curvas, \\
da mais nova das Graças, fez sua viçosa consorte.\\

\quad{}E Dioniso juba-dourada da loira Ariadne,\\
a filha de Minos, fez sua viçosa consorte:\\
a ela, para ele, imortal e sem velhice tornou o Cronida.\\

\quad{}E de Juventude o bravo filho de Alcmena linda-canela, \\
o vigor de Héracles, após findar tristes provas,\\
da filha do grande Zeus e de Hera sandália-dourada\\
fez sua esposa, respeitada no Olimpo nevado:\\
afortunado, que grande feito realizou entre os imortais,\\
e habita sem miséria e velhice por todos os dias. \\

\quad{}A gloriosa filha de Oceano pariu ao incansável Sol\\
Perseís, Circe e o rei Eetes.\\
Eetes, o filho de Sol ilumina-mortal,\\
à filha do circular rio Oceano\\
desposou, Sapiente bela-face, pelos desígnios dos deuses: \\
ela gerou-lhe Medeia belo-tornozelo,\\
em amor subjugada devido à dourada Afrodite.\\

\Para
Agora, felicidades, vós que tendes moradas olímpia,\\
ilhas, continentes e, no interior, o salso mar;\\
mas agora a tribo das deusas cantai, doce-palavra \\
Musas do Olimpo, filhas de Zeus porta-égide,\\
tantas quantas junto a varões mortais deitaram\\
e, imortais, geraram filhos semelhantes a deuses.\\

\quad{}Deméter a \edtext{Pluto}{\nota{\emph{Ploutos}, ``riqueza''.}} gerou, diva entre as deusas,\\
unida ao herói Iasíon em desejável amor, \\
em pousio com três sulcos, na fértil região de Creta,\\
ao valoroso, que vai pelas amplas costas do mar e terra\\
inteira: a quem ao acaso topa e alcança suas mãos,\\
a esse torna rico e lhe dá grande fortuna.\\

\quad{}Para Cadmo Harmonia, filha de dourada Afrodite, \\
a Ino, Semele, Agave bela-face,\\
Autônoe, a quem desposou Aristaio cabeleira-farta,\\
e também Polidoro gerou em Tebas \edtext{bem-coroada.}{\nota{referência às famosas muralhas da cidade.}}\\

\quad{}A filha de Oceano, após ao destemido Espadouro \qb{}ânimo-potente\\
unir-se em amor de Afrodite muito-ouro, \\
Bonflux, pariu o filho mais vigoroso de todos os mortais,\\
Gerioneu, a quem matou a força de Héracles\\
pelos bois passo-arrastado na oceânica Eriteia.\\

\quad{}E para Títono Aurora gerou Mêmnon elmo-brônzeo,\\
rei dos \edtext{etíopes,}{\nota{tribo mítica ainda não associada à região
posteriormente conhecida como Etiópia; diz respeito ao norte da África
de forma geral.}} e o senhor Emátion. \\
E para Céfalo gerou um filho insigne,\\
o altivo Faéton, varão semelhante a deuses:\\
ao jovem na suave flor da gloriosa juventude,\\
garoto imaturo, Afrodite ama-sorriso\\
lançou-se e o carregou, e de seus templos numinosos \\
fez dele o servo bem no fundo, divo espírito.\\

\quad{}E à filha de Eetes o rei criado-por-Zeus,\\
o Esonida, pelos desígnios dos deuses \edtext{sempiternos,}{\nota{Trata-se de Jasão e Medeia}}\\
levou de junto de Eetes, após findar tristes provas,\\
muitas, que lhe impôs o grande rei arrogante, \\
o violento e iníquo Pélias ação-ponderosa:\\
quando as findou, chegou a Iolco, após muito sofrer,\\
sobre rápida nau levando a jovem olhar-luzente\\
o Esonida, e dela fez sua viçosa consorte.\\
E ela, subjugada por Jasão, pastor de tropa, \\
gerou o filho Medeio, de quem Quíron cuidou nos morros,\\
o filho de Filira; e a ideia do grande Zeus foi completada.\\

\quad{}E as filhas de Nereu, o velho do mar,\\
a Focos, por um lado, Areiana pariu, diva entre as deusas,\\
em amor por Eaco devido à dourada Afrodite; \\
e a Peleu subjugada, a deusa Tétis pés-de-prata\\
gerou Aquiles rompe-batalhão, de ânimo leonino.\\

\quad{}E a Eneias pariu Citereia bela-coroa,\\
após ao herói Anquises se unir em desejável amor\\
nos picos do ventoso Ida muito-vale. \\

\quad{}E Circe, a filha do Hiperionida Sol,\\
gerou, em amor por Odisseu juizo-paciente,\\
Ágrio e Latino, impecável e forte;\\
e a \edtext{Telégono}{\nota{O nome Telégono --- ``filho (nascido) longe'' --- remete ao outro filho de Odisseu, Telêmaco.}} pariu devido à dourada Afrodite:\\
quanto a eles, bem longe, no recesso de sacras ilhas, \\
regiam todos os esplêndidos tirrenos.\\

\quad{}E \edtext{Nauveloz}{\nota{\emph{Nausithoos}.}} para Odisseu Calipso, diva entre as deusas,\\
e \edtext{Náutico}{\nota{\emph{Nausinoos}.}} gerou, unida em desejável amor.\\

\quad{}Essas deitaram junto a varões mortais\\
e, imortais, geraram filhos semelhantes a deuses. \\
Agora cantai a tribo das mulheres, doce-palavra\\
Musas do Olimpo, filhas de Zeus porta-égide.            
            \pend
         \endnumbering
    \end{Rightside}
\end{pages}
\Pages

\pagebreak
\blankpage
%\blankAteven
\pagestyle{empty}
\begingroup
\fontsize{7}{8}\selectfont

{\large\textsc{coleção «hedra edições»}}

\begin{enumerate}
\setlength\parskip{4.2pt}
\setlength\itemsep{-1.4mm}
%\item \textit{Poemas da cabana montanhesa}, Saigy\=o
\item \textit{A arte da guerra}, Maquiavel
\item \textit{A conjuração de Catilina}, Salústio
\item \textit{A cruzada das crianças/ Vidas imaginárias}, Marcel Schwob
\item \textit{A filosofia na era trágica dos gregos}, Friedrich Nietzsche
\item \textit{A fábrica de robôs}, Karel Tchápek 
\item \textit{A história trágica do Doutor Fausto}, Christopher Marlowe
\item \textit{A metamorfose}, Franz Kafka
\item \textit{A monadologia e outros textos}, Gottfried Leibniz
\item \textit{A morte de Ivan Ilitch}, Lev Tolstói 
\item \textit{A velha Izerguil e outros contos}, Maksim Górki
\item \textit{A vida é sonho}, Calderón de la Barca
\item \textit{A volta do parafuso}, Henry James
\item \textit{A voz dos botequins e outros poemas}, Paul Verlaine 
\item \textit{A vênus das peles}, Leopold von Sacher{}-Masoch
\item \textit{A última folha e outros contos}, O.\,Henry
\item \textit{Americanismo e fordismo}, Antonio Gramsci
\item \textit{Anarquia pela educação}, Élisée Reclus 
\item \textit{Apologia de Galileu}, Tommaso Campanella 
\item \textit{Arcana C\oe lestia} e \textit{Apocalipsis revelata}, Emanuel Swedenborg
\item \textit{As bacantes}, Eurípides
\item \textit{Autobiografia de uma pulga}, [Stanislas de Rhodes]
\item \textit{Ação direta e outros escritos}, Voltairine de Cleyre
\item \textit{Balada dos enforcados e outros poemas}, François Villon
\item \textit{Carmilla, a vampira de Karnstein}, Sheridan Le Fanu
\item \textit{Carta sobre a tolerância}, John Locke
\item \textit{Contos clássicos de vampiro}, L.\,Byron, B.\,Stoker \& outros
\item \textit{Contos de amor, de loucura e de morte}, Horacio Quiroga
\item \textit{Contos indianos}, Stéphane Mallarmé
\item \textit{Cultura estética e liberdade}, Friedrich von Schiller
\item \textit{Cântico dos cânticos}, [Salomão]
\item \textit{Dao De Jing}, Lao Zi
\item \textit{Discursos ímpios}, Marquês de Sade
\item \textit{Dissertação sobre as paixões}, David Hume
\item \textit{Diário de um escritor (1873)}, Fiódor Dostoiévski
\item \textit{Diário parisiense e outros escritos}, Walter Benjamin
\item \textit{Diários de Adão e Eva}, Mark Twain
\item \textit{Don Juan}, Molière
\item \textit{Dos novos sistemas na arte}, Kazimir Maliévitch
\item \textit{Educação e sociologia}, Émile Durkheim
\item \textit{Édipo Rei}, Sófocles
\item \textit{Elogio da loucura}, Erasmo de Rotterdam
\item \textit{Émile e Sophie ou os solitários}, Jean-Jacques Rousseau 
\item \textit{Emília Galotti}, Gotthold Ephraim Lessing
\item \textit{Entre camponeses}, Errico Malatesta
\item \textit{Ernestine ou o nascimento do amor}, Stendhal
\item \textit{Escritos revolucionários}, Errico Malatesta
\item \textit{Escritos sobre arte}, Charles Baudelaire
\item \textit{Escritos sobre literatura}, Sigmund Freud
\item \textit{Eu acuso!}, Zola/\,\textit{O processo do capitão Dreyfus}, Rui Barbosa
\item \textit{Explosão: romance da etnologia}, Hubert Fichte
\item \textit{Fedro}, Platão
\item \textit{Feitiço de amor e outros contos}, Ludwig Tieck
\item \textit{Flossie, a Vênus de quinze anos}, [Swinburne]
\item \textit{Fábula de Polifemo e Galateia e outros poemas}, Góngora
\item \textit{Fé e saber}, Georg W.\,F.\,Hegel
\item \textit{Gente de Hemsö}, August Strindberg 
\item \textit{Hawthorne e seus musgos}, Melville
\item \textit{Hino a Afrodite e outros poemas}, Safo de Lesbos 
\item \textit{História da anarquia (vol.\,\textsc{ii})}, Max Nettlau
\item \textit{História da anarquia (vol.\,\textsc{i})}, Max Nettlau
\item \textit{Imitação de Cristo}, Tomás de Kempis
\item \textit{Incidentes da vida de uma escrava}, Harriet Jacobs
\item \textit{Inferno}, August Strindberg
\item \textit{Investigação sobre o entendimento humano}, David Hume
\item \textit{Jazz rural}, Mário de Andrade
\item \textit{Jerusalém}, William Blake
\item \textit{Joana d'Arc}, Jules Michelet
\item \textit{Lira gregra}, Giuliana Ragusa (org.)
\item \textit{Lisístrata}, Aristófanes 
\item \textit{Ludwig Feuerbach e o fim da filosofia clássica alemã}, Friederich Engels
\item \textit{Manifesto comunista}, Karl Marx e Friederich Engels
\item \textit{Memórias do subsolo}, Fiódor Dostoiévski
\item \textit{Metamorfoses}, Ovídio
\item \textit{Micromegas e outros contos}, Voltaire
\item \textit{Narrativa de William W.\,Brown, escravo fugitivo}, William Wells Brown
\item \textit{Nascidos na escravidão: depoimentos norte-americanos}, \textsc{wpa}
\item \textit{No coração das trevas}, Joseph Conrad
\item \textit{Noites egípcias e outros contos}, Aleksandr Púchkin
\item \textit{O casamento do Céu e do Inferno}, William Blake
\item \textit{O cego e outros contos}, \textsc{d.\,h}.\,Lawrence
\item \textit{O chamado de Cthulhu}, \textsc{h.\,p.}\,lovecraft
\item \textit{O contador de histórias e outros textos}, Walter Benjamin
\item \textit{O corno de si próprio e outros contos}, Marquês de Sade
\item \textit{O destino do erudito}, Johann Fichte
\item \textit{O estranho caso do dr.\,Jekyll e Mr. Hyde}, Robert Louis Stevenson
\item \textit{O fim do ciúme e outros contos}, Marcel Proust
\item \textit{O indivíduo, a sociedade e o Estado, e outros ensaios}, Emma Goldman
\item \textit{O ladrão honesto e outros contos}, Fiódor Dostoiévski
\item \textit{O livro de Monelle}, Marcel Schwob
\item \textit{O mundo ou tratado da luz}, René Descartes
\item \textit{O novo Epicuro: as delícias do sexo}, Edward Sellon
\item \textit{O pequeno Zacarias, chamado Cinábrio}, \textsc{e.\,t.\,a.}\,Hoffmann
\item \textit{O primeiro Hamlet}, William Shakespeare
\item \textit{O princípio anarquista e outros ensaios}, Piotr Kropotkin
\item \textit{O princípio do Estado e outros ensaios}, Mikhail Bakunin
\item \textit{O príncipe}, Maquiavel
\item \textit{O que eu vi, o que nós veremos}, Santos-Dumont
\item \textit{O retrato de Dorian Gray}, Oscar Wilde
\item \textit{O sobrinho de Rameau}, Diderot
\item \textit{Ode ao Vento Oeste e outros poemas}, \textsc{p.\,b.}\,Shelley
\item \textit{Ode sobre a melancolia e outros poemas}, John Keats
\item \textit{Odisseia}, Homero
\item \textit{Oliver Twist}, Charles Dickens
\item \textit{Origem do drama barroco}, Walter Benjamin
\item \textit{Os sofrimentos do jovem Werther}, Goethe
\item \textit{Os sovietes traídos pelos bolcheviques}, Rudolf Rocker
\item \textit{Para serem lidas à noite}, Ion Minulescu
\item \textit{Pensamento político de Maquiavel}, Johann Fichte
\item \textit{Pequeno-burgueses}, Maksim Górki
\item \textit{Pequenos poemas em prosa}, Charles Baudelaire
\item \textit{Perversão: a forma erótica do ódio}, Robert Stoller
\item \textit{Poemas}, Lord Byron
\item \textit{Poesia basca: das origens à Guerra Civil} 
\item \textit{Poesia catalã: das origens à Guerra Civil} 
\item \textit{Poesia espanhola: das origens à Guerra Civil} 
\item \textit{Poesia galega: das origens à Guerra Civil} 
\item \textit{Pr\ae terita}, John Ruskin
\item \textit{Primeiro livro dos Amores}, Ovídio
\item \textit{Rashômon e outros contos}, Ryūnosuke Akutagawa
\item \textit{Revolução e liberdade: cartas de 1845 a 1875}, Mikhail Bakunin
\item \textit{Robinson Crusoé}, Daniel Defoe
\item \textit{Romanceiro cigano}, Federico García Lorca
\item \textit{Sagas}, August Strindberg
\item \textit{Sobre a amizade e outros diálogos}, Jorge Luis Borges e Osvaldo Ferrari
\item \textit{Sobre a filosofia e outros diálogos}, Jorge Luis Borges e Osvaldo Ferrari
\item \textit{Sobre a filosofia e seu método (Parerga e paralipomena)} (v.\textsc{ii}, t.\textsc{i}), Arthur Schopenhauer 
\item \textit{Sobre a liberdade}, Stuart Mill
\item \textit{Sobre a utilidade e a desvantagem da histório para a vida}, Friedrich Nietzsche
\item \textit{Sobre a ética (Parerga e paralipomena)} (v.\textsc{ii}, t.\textsc{ii}), Arthur Schopenhauer 
\item \textit{Sobre anarquismo, sexo e casamento}, Emma Goldman
\item \textit{Sobre o riso e a loucura}, [Hipócrates]
\item \textit{Sobre os sonhos e outros diálogos}, Jorge Luis Borges e Osvaldo Ferrari
\item \textit{Sobre verdade e mentira}, Friedrich Nietzsche
\item \textit{Sonetos}, William Shakespeare
\item \textit{Sátiras, fábulas, aforismos e profecias}, Leonardo da Vinci
\item \textit{Teleny, ou o reverso da medalha}, Oscar Wilde
\item \textit{Teogonia}, Hesíodo
\item \textit{Trabalhos e dias}, Hesíodo
\item \textit{Triunfos}, Petrarca
\item \textit{Um anarquista e outros contos}, Joseph Conrad
\item \textit{Viagem aos Estados Unidos}, Alexis de Tocqueville
\item \textit{Viagem em volta do meu quarto}, Xavier de Maistre 
\item \textit{Viagem sentimental}, Laurence Sterne
\end{enumerate}

{\large\textsc{coleção «metabiblioteca»}}

\begin{enumerate}
\setlength\parskip{4.2pt}
\setlength\itemsep{-1.4mm}
\item \textit{A carteira de meu tio}, Joaquim Manuel de Macedo
\item \textit{A cidade e as serras}, Eça de Queirós
\item \textit{A escrava}, Maria Firmina dos Reis
\item \textit{A família Medeiros}, Júlia Lopes de Almeida 
\item \textit{A pele do lobo e outras peças}, Artur Azevedo
\item \textit{Auto da barca do inferno}, Gil Vicente
\item \textit{Bom crioulo}, Adolfo Caminha
\item \textit{Cartas a favor da escravidão}, José de Alencar
\item \textit{Contos e novelas}, Júlia Lopes de Almeida
\item \textit{Crime}, Luiz Gama
\item \textit{Democracia}, Luiz Gama
\item \textit{Direito}, Luiz Gama
\item \textit{Elixir do pajé: poemas de humor, sátira e escatologia}, Bernardo Guimarães
\item \textit{Eu}, Augusto dos Anjos
\item \textit{Farsa de Inês Pereira}, Gil Vicente
\item \textit{Helianto}, Orides Fontela
\item \textit{História da província Santa Cruz}, Gandavo
\item \textit{Iracema}, José de Alencar
\item \textit{Liberdade}, Luiz Gama
\item \textit{Mensagem}, Fernando Pessoa
\item \textit{Meridiano 55}, Flávio de Carvalho
\item \textit{O Ateneu}, Raul Pompeia
\item \textit{O cortiço}, Aluísio Azevedo
\item \textit{O desertor}, Silva Alvarenga
\item \textit{Oração aos moços}, Rui Barbosa
\item \textit{Pai contra mãe e outros contos}, Machado de Assis
\item \textit{Poemas completos de Alberto Caeiro}, Fernando Pessoa
\item \textit{Teatro de êxtase}, Fernando Pessoa
\item \textit{Transposição}, Orides Fontela
\item \textit{Tratado descritivo do Brasil em 1587}, Gabriel Soares de Sousa
\item \textit{Tratados da terra e gente do Brasil}, Fernão Cardim 
\item \textit{Utopia Brasil}, Darcy Ribeiro
\item \textit{Índice das coisas mais notáveis}, Antônio Vieira
\end{enumerate}

\medskip
{\large\textsc{coleção «que horas são?»}}

\begin{enumerate}
\setlength\parskip{4.2pt}
\setlength\itemsep{-1.4mm}
\item \textit{8/1: A rebelião dos manés}, Pedro Fiori Arantes, Fernando Frias e Maria Luiza Meneses
\item \textit{A linguagem fascista}, Carlos Piovezani \& Emilio Gentile
\item \textit{A sociedade de controle}, J.\,Souza; R.\,Avelino; S.\,Amadeu (orgs.)
\item \textit{Ativismo digital hoje}, R.\,Segurado; C.\,Penteado; S.\,Amadeu (orgs.)
\item \textit{Crédito à morte}, Anselm Jappe
\item \textit{Descobrindo o Islã no Brasil}, Karla Lima
\item \textit{Desinformação e democracia}, Rosemary Segurado
\item \textit{Dilma Rousseff e o ódio político}, Tales Ab'Sáber
\item \textit{Labirintos do fascismo} (v.\textsc{iii}), João Bernardo
\item \textit{Labirintos do fascismo} (v.\textsc{ii}), João Bernardo
\item \textit{Labirintos do fascismo} (v.\textsc{iv}), João Bernardo
\item \textit{Labirintos do fascismo} (v.\textsc{i}), João Bernardo
\item \textit{Labirintos do fascismo} (v.\textsc{vi}), João Bernardo
\item \textit{Labirintos do fascismo} (v.\textsc{v}), João Bernardo
\item \textit{Lugar de negro, lugar de branco?}, Douglas Rodrigues Barros
\item \textit{Lulismo, carisma pop e cultura anticrítica}, Tales Ab'Sáber
\item \textit{Machismo, racismo, capitalismo identitário}, Pablo Polese
\item \textit{Michel Temer e o fascismo comum}, Tales Ab'Sáber
\item \textit{O quarto poder: uma outra história}, Paulo Henrique Amorim
\item \textit{Universidade, cidade e cidadania}, Franklin Leopoldo e Silva
\end{enumerate}

\medskip
{\large\textsc{coleção «mundo indígena»}}

\begin{enumerate}
\setlength\parskip{4.2pt}
\setlength\itemsep{-1.4mm}
\item \textit{A folha divina}, Timóteo Verá Tupã Popygua
\item \textit{A mulher que virou tatu}, Eliane Camargo
\item \textit{A terra uma só}, Timóteo Verá Tupã Popygua
\item \textit{A árvore dos cantos}, Pajés Parahiteri
\item \textit{Cantos dos animais primordiais}, Ava Ñomoandyja Atanásio Teixeira
\item \textit{Crônicas de caça e criação}, Uirá Garcia
\item \textit{Círculos de coca e fumaça}, Danilo Paiva Ramos
\item \textit{Nas redes guarani}, Valéria Macedo \& Dominique Tilkin-Gallois
\item \textit{Não havia mais homens}, Luciana Storto
\item \textit{O surgimento da noite}, Pajés Parahiteri
\item \textit{O surgimento dos pássaros}, Pajés Parahiteri
\item \textit{Os Aruaques}, Max Schmidt
\item \textit{Os cantos do homem-sombra}, Patience Epps e Danilo Paiva Ramos
\item \textit{Os comedores de terra}, Pajés Parahiteri
\item \textit{Xamanismos ameríndios}, A.\,Barcelos Neto, L.\,Pérez Gil \& D.\,Paiva Ramos
\end{enumerate}

\medskip
{\large\textsc{coleção «ecopolítica»}}

\begin{enumerate}
\setlength\parskip{4.2pt}
\setlength\itemsep{-1.4mm}
\item \textit{Anarquistas na América do Sul}, E.\,Passetti, S.\,Gallo; A.\,Augusto  (orgs.)
\item \textit{Ecopolítica}, E.\,Passetti; A.\,Augusto; B.\,Carneiro; S.\,Oliveira, T.\,Rodrigues  (orgs.)
\item \textit{Pandemia e anarquia}, E.\,Passetti; J.\,da Mata; J.\,Ferreira  (orgs.)
\end{enumerate}

% \medskip
% {\large\textsc{coleção <<anarc>>}}

% \begin{enumerate}
% \setlength\parskip{4.2pt}
% \setlength\itemsep{-1.4mm}
% \item \textit{Anarquia pela educação}, Élisée Reclus 
% \item \textit{Ação direta e outros escritos}, Voltairine de Cleyre
% \item \textit{Entre camponeses}, Errico Malatesta
% \item \textit{Escritos revolucionários}, Errico Malatesta
% \item \textit{História da anarquia (vol.\,\textsc{ii})}, Max Nettlau
% \item \textit{História da anarquia (vol.\,\textsc{i})}, Max Nettlau
% \item \textit{O indivíduo, a sociedade e o Estado, e outros ensaios}, Emma Goldman
% \item \textit{O princípio anarquista e outros ensaios}, Piotr Kropotkin
% \item \textit{O princípio do Estado e outros ensaios}, Mikhail Bakunin
% \item \textit{Os sovietes traídos pelos bolcheviques}, Rudolf Rocker
% \item \textit{Revolução e liberdade: cartas de 1845 a 1875}, Mikhail Bakunin
% \item \textit{Sobre anarquismo, sexo e casamento}, Emma Goldman
% \end{enumerate}

% \medskip
% {\large\textsc{coleção <<narrativas da escravidão>>}}

% \begin{enumerate}
% \setlength\parskip{4.2pt}
% \setlength\itemsep{-1.4mm}
% \item \textit{Incidentes da vida de uma escrava}, Harriet Jacobs
% \item \textit{Narrativa de William W.\,Brown, escravo fugitivo}, William Wells Brown
% \item \textit{Nascidos na escravidão: depoimentos norte-americanos}, \textsc{wpa}
% \end{enumerate}

\pagebreak	   % [lista de livros publicados]
\pagebreak

\ifodd\thepage\blankpage\fi

\parindent=0pt
\footnotesize\thispagestyle{empty}

% \noindent\textbf{Dados Internacionais de Catalogação na Publicação -- CIP}\\
% \noindent\textbf{(Câmara Brasileira do Livro, SP, Brasil)}\\

% \dotfill\\

% \hspace{20pt}ISBN 978-65-86238-31-0 (Livro do Estudante)

% \hspace{20pt}ISBN 978-65-86238-30-3 (Manual do Professor)\\[6pt]

% \hspace{20pt}\parbox{190pt}{1. Crônicas Brasileira. 2. Contos Brasileiro. 3. Rosa, Alexandre. I. Título.}\\[6pt]

% \hspace{188pt}\textsc{cdd}-B869.8

% \dotfill

% \noindent{}Elaborado por Regina Célia Paiva da Silva CRB -- 1051\\

\mbox{}\vfill
\begin{center}
		\begin{minipage}{.7\textwidth}\tiny\noindent{}
		\centering\tiny
		Adverte-se aos curiosos que se imprimiu este 
		livro na gráfica Meta Brasil, 
		na data de \today, em papel pólen soft, composto em tipologia Minion Pro e Formular, 
		com diversos sofwares livres, 
		dentre eles Lua\LaTeX e git.\\ 
		\ifdef{\RevisionInfo{}}{\par(v.\,\RevisionInfo)}{}\medskip\\\
		\adforn{64}
		\end{minipage}
\end{center}		   % [colofon]

\checkandfixthelayout
\end{document}

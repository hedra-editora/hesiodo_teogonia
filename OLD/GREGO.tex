Μουσάων Ἑλικωνιάδων ἀρχώμεθ' ἀείδειν,

αἵ θ' Ἑλικῶνος ἔχουσιν ὄρος μέγα τε ζάθεόν τε, 

καί τε περὶ κρήνην ἰοειδέα πόσσ' ἁπαλοῖσιν

ὀρχεῦνται καὶ βωμὸν ἐρισθενέος Κρονίωνος·

καί τε λοεσσάμεναι τέρενα χρόα Περμησσοῖο \num{5}

ἠ' Ἵππου κρήνης ἠ' Ὀλμειοῦ ζαθέοιο

ἀκροτάτῳ Ἑλικῶνι χοροὺς ἐνεποιήσαντο,

καλοὺς ἱμερόεντας, ἐπερρώσαντο δὲ ποσσίν.

ἔνθεν ἀπορνύμεναι κεκαλυμμέναι ἠέρι πολλῷ 

ἐννύχιαι στεῖχον περικαλλέα ὄσσαν ἱεῖσαι, \num{10}

ὑμνεῦσαι Δία τ' αἰγίοχον καὶ πότνιαν Ἥρην

Ἀργείην, χρυσέοισι πεδίλοις ἐμβεβαυῖαν, 

κούρην τ' αἰγιόχοιο Διὸς γλαυκῶπιν Ἀθήνην

Φοῖβόν τ' Ἀπόλλωνα καὶ Ἄρτεμιν ἰοχέαιραν

ἠδὲ Ποσειδάωνα γαιήοχον ἐννοσίγαιον \num{15} 

καὶ Θέμιν αἰδοίην ἑλικοβλέφαρόν τ' Ἀφροδίτην

Ἥβην τε χρυσοστέφανον καλήν τε Διώνην

Λητώ τ' Ἰαπετόν τε ἰδὲ Κρόνον ἀγκυλομήτην

Ἠῶ τ' Ἠέλιόν τε μέγαν λαμπράν τε Σελήνην

Γαῖάν τ' Ὠκεανόν τε μέγαν καὶ Νύκτα μέλαιναν \num{20}

ἄλλων τ' ἀθανάτων ἱερὸν γένος αἰὲν ἐόντων.

αἵ νύ ποθ' Ἡσίοδον καλὴν ἐδίδαξαν ἀοιδήν,

ἄρνας ποιμαίνονθ' Ἑλικῶνος ὕπο ζαθέοιο.

τόνδε δέ με πρώτιστα θεαὶ πρὸς μῦθον ἔειπον,

Μοῦσαι Ὀλυμπιάδες, κοῦραι Διὸς αἰγιόχοιο· \num{25} 

``ποιμένες ἄγραυλοι, κάκ' ἐλέγχεα, γαστέρες οἶον,

ἴδμεν ψεύδεα πολλὰ λέγειν ἐτύμοισιν ὁμοῖα,

ἴδμεν δ' εὖτ' ἐθέλωμεν ἀληθέα γηρύσασθαι.''

ὣς ἔφασαν κοῦραι μεγάλου Διὸς ἀρτιέπειαι,

καί μοι σκῆπτρον ἔδον δάφνης ἐριθηλέος ὄζον \num{30}

δρέψασαι, θηητόν· ἐνέπνευσαν δέ μοι αὐδὴν 

θέσπιν, ἵνα κλείοιμι τά τ' ἐσσόμενα πρό τ' ἐόντα, 

καί μ' ἐκέλονθ' ὑμνεῖν μακάρων γένος αἰὲν ἐόντων,

σφᾶς δ' αὐτὰς πρῶτόν τε καὶ ὕστατον αἰὲν ἀείδειν.

ἀλλὰ τίη μοι ταῦτα περὶ δρῦν ἢ περὶ πέτρην; \num{35} 

τύνη, Μουσάων ἀρχώμεθα, ταὶ Διὶ πατρὶ

ὑμνεῦσαι τέρπουσι μέγαν νόον ἐντὸς Ὀλύμπου,

εἴρευσαι τά τ' ἐόντα τά τ' ἐσσόμενα πρό τ' ἐόντα,

φωνῇ ὁμηρεῦσαι, τῶν δ' ἀκάματος ῥέει αὐδὴ 

ἐκ στομάτων ἡδεῖα· γελᾷ δέ τε δώματα πατρὸς \num{40} 

Ζηνὸς ἐριγδούποιο θεᾶν ὀπὶ λειριοέσσῃ

σκιδναμένῃ, ἠχεῖ δὲ κάρη νιφόεντος Ὀλύμπου 

δώματά τ' ἀθανάτων· αἱ δ' ἄμβροτον ὄσσαν ἱεῖσαι 

θεῶν γένος αἰδοῖον πρῶτον κλείουσιν ἀοιδῇ

ἐξ ἀρχῆς, οὓς Γαῖα καὶ Οὐρανὸς εὐρὺς ἔτικτεν, \num{45}

οἵ τ' ἐκ τῶν ἐγένοντο, θεοὶ δωτῆρες ἐάων· 

δεύτερον αὖτε Ζῆνα θεῶν πατέρ' ἠδὲ καὶ ἀνδρῶν, 

ἀρχόμεναί θ' ὑμνεῦσι καὶ ἐκλήγουσαι ἀοιδῆς,

ὅσσον φέρτατός ἐστι θεῶν κάρτει τε μέγιστος·

αὖτις δ' ἀνθρώπων τε γένος κρατερῶν τε Γιγάντων \num{50}

ὑμνεῦσαι τέρπουσι Διὸς νόον ἐντὸς Ὀλύμπου

Μοῦσαι Ὀλυμπιάδες, κοῦραι Διὸς αἰγιόχοιο.

τὰς ἐν Πιερίῃ Κρονίδῃ τέκε πατρὶ μιγεῖσα

Μνημοσύνη, γουνοῖσιν Ἐλευθῆρος μεδέουσα,

λησμοσύνην τε κακῶν ἄμπαυμά τε μερμηράων. \num{55}

ἐννέα γάρ οἱ νύκτας ἐμίσγετο μητίετα Ζεὺς

νόσφιν ἀπ' ἀθανάτων ἱερὸν λέχος εἰσαναβαίνων·

ἀλλ' ὅτε δή ῥ' ἐνιαυτὸς ἔην, περὶ δ' ἔτραπον ὧραι

μηνῶν φθινόντων, περὶ δ' ἤματα πόλλ' ἐτελέσθη,

ἡ δ' ἔτεκ' ἐννέα κούρας, ὁμόφρονας, ᾗσιν ἀοιδὴ \num{60} 

μέμβλεται ἐν στήθεσσιν, ἀκηδέα θυμὸν ἐχούσαις,

τυτθὸν ἀπ' ἀκροτάτης κορυφῆς νιφόεντος Ὀλύμπου·

ἔνθά σφιν λιπαροί τε χοροὶ καὶ δώματα καλά,

πὰρ δ' αὐτῇς Χάριτές τε καὶ Ἵμερος οἰκί' ἔχουσιν

ἐν θαλίῃς· ἐρατὴν δὲ διὰ στόμα ὄσσαν ἱεῖσαι \num{65} 

μέλπονται, πάντων τε νόμους καὶ ἤθεα κεδνὰ 

ἀθανάτων κλείουσιν, ἐπήρατον ὄσσαν ἱεῖσαι.

αἳ τότ' ἴσαν πρὸς Ὄλυμπον, ἀγαλλόμεναι ὀπὶ καλῇ, 

ἀμβροσίῃ μολπῇ· περὶ δ' ἴαχε γαῖα μέλαινα 

ὑμνεύσαις, ἐρατὸς δὲ ποδῶν ὕπο δοῦπος ὀρώρει \num{70}

νισομένων πατέρ' εἰς ὅν· ὁ δ' οὐρανῷ ἐμβασιλεύει, 

αὐτὸς ἔχων βροντὴν ἠδ' αἰθαλόεντα κεραυνόν,

κάρτει νικήσας πατέρα Κρόνον· εὖ δὲ ἕκαστα 

ἀθανάτοις διέταξε ὁμῶς καὶ ἐπέφραδε τιμάς.

ταῦτ' ἄρα Μοῦσαι ἄειδον Ὀλύμπια δώματ' ἔχουσαι,\num{75} 

ἐννέα θυγατέρες μεγάλου Διὸς ἐκγεγαυῖαι,

Κλειώ τ' Εὐτέρπη τε Θάλειά τε Μελπομένη τε

Τερψιχόρη τ' Ἐρατώ τε Πολύμνιά τ' Οὐρανίη τε

Καλλιόπη θ'· ἡ δὲ προφερεστάτη ἐστὶν ἁπασέων. 

ἡ γὰρ καὶ βασιλεῦσιν ἅμ' αἰδοίοισιν ὀπηδεῖ. \num{80} 

ὅντινα τιμήσουσι Διὸς κοῦραι μεγάλοιο

γεινόμενόν τε ἴδωσι διοτρεφέων βασιλήων,

τῷ μὲν ἐπὶ γλώσσῃ γλυκερὴν χείουσιν ἐέρσην,

τοῦ δ' ἔπε' ἐκ στόματος ῥεῖ μείλιχα· οἱ δέ νυ λαοὶ 

πάντες ἐς αὐτὸν ὁρῶσι διακρίνοντα θέμιστας \num{85}

ἰθείῃσι δίκῃσιν· ὁ δ' ἀσφαλέως ἀγορεύων

αἶψά τι καὶ μέγα νεῖκος ἐπισταμένως κατέπαυσε· 

τούνεκα γὰρ βασιλῆες ἐχέφρονες, οὕνεκα λαοῖς 

βλαπτομένοις ἀγορῆφι μετάτροπα ἔργα τελεῦσι

ῥηιδίως, μαλακοῖσι παραιφάμενοι ἐπέεσσιν· \num{90} 

ἐρχόμενον δ' ἀν' ἀγῶνα θεὸν ὣς ἱλάσκονται

αἰδοῖ μειλιχίῃ, μετὰ δὲ πρέπει ἀγρομένοισι. 

τοίη Μουσάων ἱερὴ δόσις ἀνθρώποισιν.

ἐκ γάρ τοι Μουσέων καὶ ἑκηβόλου Ἀπόλλωνος

ἄνδρες ἀοιδοὶ ἔασιν ἐπὶ χθόνα καὶ κιθαρισταί, \num{95}

ἐκ δὲ Διὸς βασιλῆες· ὁ δ' ὄλβιος, ὅντινα Μοῦσαι

φίλωνται· γλυκερή οἱ ἀπὸ στόματος ῥέει αὐδή. 

εἰ γάρ τις καὶ πένθος ἔχων νεοκηδέι θυμῷ

ἄζηται κραδίην ἀκαχήμενος, αὐτὰρ ἀοιδὸς

Μουσάων θεράπων κλεῖα προτέρων ἀνθρώπων \num{100} 

ὑμνήσει μάκαράς τε θεοὺς οἳ Ὄλυμπον ἔχουσιν, 

αἶψ' ὅ γε δυσφροσυνέων ἐπιλήθεται οὐδέ τι κηδέων

μέμνηται· ταχέως δὲ παρέτραπε δῶρα θεάων. 

χαίρετε τέκνα Διός, δότε δ' ἱμερόεσσαν ἀοιδήν· 

κλείετε δ' ἀθανάτων ἱερὸν γένος αἰὲν ἐόντων, \num{105}

οἳ Γῆς ἐξεγένοντο καὶ Οὐρανοῦ ἀστερόεντος,

Νυκτός τε δνοφερῆς, οὕς θ' ἁλμυρὸς ἔτρεφε Πόντος.

εἴπατε δ' ὡς τὰ πρῶτα θεοὶ καὶ γαῖα γένοντο 

καὶ ποταμοὶ καὶ πόντος ἀπείριτος οἴδματι θυίων 

ἄστρά τε λαμπετόωντα καὶ οὐρανὸς εὐρὺς ὕπερθεν· \num{110} 

οἵ τ' ἐκ τῶν ἐγένοντο, θεοὶ δωτῆρες ἐάων· 

ὥς τ' ἄφενος δάσσαντο καὶ ὡς τιμὰς διέλοντο, 

ἠδὲ καὶ ὡς τὰ πρῶτα πολύπτυχον ἔσχον Ὄλυμπον.

ταῦτά μοι ἔσπετε Μοῦσαι Ὀλύμπια δώματ' ἔχουσαι 

ἐξ ἀρχῆς, καὶ εἴπαθ', ὅτι πρῶτον γένετ' αὐτῶν. \num{115}

ἤτοι μὲν πρώτιστα Χάος γένετ'· αὐτὰρ ἔπειτα

Γαῖ' εὐρύστερνος, πάντων ἕδος ἀσφαλὲς αἰεὶ

ἀθανάτων οἳ ἔχουσι κάρη νιφόεντος Ὀλύμπου 

Τάρταρά τ' ἠερόεντα μυχῷ χθονὸς εὐρυοδείης,

ἠδ' Ἔρος, ὃς κάλλιστος ἐν ἀθανάτοισι θεοῖσι, \num{120}

λυσιμελής, πάντων τε θεῶν πάντων τ' ἀνθρώπων

δάμναται ἐν στήθεσσι νόον καὶ ἐπίφρονα βουλήν.

ἐκ Χάεος δ' Ἔρεβός τε μέλαινά τε Νὺξ ἐγένοντο· 

Νυκτὸς δ' αὖτ' Αἰθήρ τε καὶ Ἡμέρη ἐξεγένοντο,

οὓς τέκε κυσαμένη Ἐρέβει φιλότητι μιγεῖσα. \num{125}

Γαῖα δέ τοι πρῶτον μὲν ἐγείνατο ἶσον ἑωυτῇ

Οὐρανὸν ἀστερόενθ', ἵνα μιν περὶ πάντα καλύπτοι,

ὄφρ' εἴη μακάρεσσι θεοῖς ἕδος ἀσφαλὲς αἰεί,

γείνατο δ' οὔρεα μακρά, θεᾶν χαρίεντας ἐναύλους 

Νυμφέων, αἳ ναίουσιν ἀν' οὔρεα βησσήεντα, \num{130} 

ἠδὲ καὶ ἀτρύγετον πέλαγος τέκεν οἴδματι θυῖον,

Πόντον, ἄτερ φιλότητος ἐφιμέρου· αὐτὰρ ἔπειτα

Οὐρανῷ εὐνηθεῖσα τέκ' Ὠκεανὸν βαθυδίνην 

Κοῖόν τε Κρεῖόν θ' Ὑπερίονά τ' Ἰαπετόν τε

Θείαν τε Ῥείαν τε Θέμιν τε Μνημοσύνην τε \num{135}

Φοίβην τε χρυσοστέφανον Τηθύν τ' ἐρατεινήν.

τοὺς δὲ μέθ' ὁπλότατος γένετο Κρόνος ἀγκυλομήτης,

δεινότατος παίδων, θαλερὸν δ' ἤχθηρε τοκῆα. 

γείνατο δ' αὖ Κύκλωπας ὑπέρβιον ἦτορ ἔχοντας,

Βρόντην τε Στερόπην τε καὶ Ἄργην ὀβριμόθυμον, \num{140}

οἳ Ζηνὶ βροντήν τ' ἔδοσαν τεῦξάν τε κεραυνόν.

οἱ δ' ἤτοι τὰ μὲν ἄλλα θεοῖς ἐναλίγκιοι ἦσαν,

μοῦνος δ' ὀφθαλμὸς μέσσῳ ἐνέκειτο μετώπῳ· 

Κύκλωπες δ' ὄνομ' ἦσαν ἐπώνυμον, οὕνεκ' ἄρά σφεων

κυκλοτερὴς ὀφθαλμὸς ἕεις ἐνέκειτο μετώπῳ· \num{145}

ἰσχὺς δ' ἠδὲ βίη καὶ μηχαναὶ ἦσαν ἐπ' ἔργοις.

ἄλλοι δ' αὖ Γαίης τε καὶ Οὐρανοῦ ἐξεγένοντο

τρεῖς παῖδες μεγάλοι \textless{}τε\textgreater{} καὶ ὄβριμοι, οὐκ
ὀνομαστοί, 

Κόττος τε Βριάρεώς τε Γύγης θ', ὑπερήφανα τέκνα. 

τῶν ἑκατὸν μὲν χεῖρες ἀπ' ὤμων ἀίσσοντο, \num{150}

ἄπλαστοι, κεφαλαὶ δὲ ἑκάστῳ πεντήκοντα

ἐξ ὤμων ἐπέφυκον ἐπὶ στιβαροῖσι μέλεσσιν· 

ἰσχὺς δ' ἄπλητος κρατερὴ μεγάλῳ ἐπὶ εἴδει.

ὅσσοι γὰρ Γαίης τε καὶ Οὐρανοῦ ἐξεγένοντο,

δεινότατοι παίδων, σφετέρῳ δ' ἤχθοντο τοκῆι \num{155}

ἐξ ἀρχῆς· καὶ τῶν μὲν ὅπως τις πρῶτα γένοιτο, 

πάντας ἀποκρύπτασκε καὶ ἐς φάος οὐκ ἀνίεσκε

Γαίης ἐν κευθμῶνι, κακῷ δ' ἐπετέρπετο ἔργῳ, 

Οὐρανός· ἡ δ' ἐντὸς στοναχίζετο Γαῖα πελώρη

στεινομένη, δολίην δὲ κακὴν ἐπεφράσσατο τέχνην. \num{160}

αἶψα δὲ ποιήσασα γένος πολιοῦ ἀδάμαντος

τεῦξε μέγα δρέπανον καὶ ἐπέφραδε παισὶ φίλοισιν· 

εἶπε δὲ θαρσύνουσα, φίλον τετιημένη ἦτορ·

``παῖδες ἐμοὶ καὶ πατρὸς ἀτασθάλου, αἴ κ' ἐθέλητε

πείθεσθαι· πατρός κε κακὴν τεισαίμεθα λώβην \num{165}

ὑμετέρου· πρότερος γὰρ ἀεικέα μήσατο ἔργα.'' 

ὣς φάτο· τοὺς δ' ἄρα πάντας ἕλεν δέος, οὐδέ τις αὐτῶν

φθέγξατο. θαρσήσας δὲ μέγας Κρόνος ἀγκυλομήτης

αἶψ' αὖτις μύθοισι προσηύδα μητέρα κεδνήν·

``μῆτερ, ἐγώ κεν τοῦτό γ' ὑποσχόμενος τελέσαιμι \num{170} 

ἔργον, ἐπεὶ πατρός γε δυσωνύμου οὐκ ἀλεγίζω

ἡμετέρου· πρότερος γὰρ ἀεικέα μήσατο ἔργα.''

ὣς φάτο· γήθησεν δὲ μέγα φρεσὶ Γαῖα πελώρη· 

εἷσε δέ μιν κρύψασα λόχῳ, ἐνέθηκε δὲ χερσὶν 

ἅρπην καρχαρόδοντα, δόλον δ' ὑπεθήκατο πάντα. \num{175} 

ἦλθε δὲ νύκτ' ἐπάγων μέγας Οὐρανός, ἀμφὶ δὲ Γαίῃ

ἱμείρων φιλότητος ἐπέσχετο, καί ῥ' ἐτανύσθη

πάντῃ· ὁ δ' ἐκ λοχέοιο πάις ὠρέξατο χειρὶ

σκαιῇ, δεξιτερῇ δὲ πελώριον ἔλλαβεν ἅρπην,

μακρὴν καρχαρόδοντα, φίλου δ' ἀπὸ μήδεα πατρὸς \num{180}

ἐσσυμένως ἤμησε, πάλιν δ' ἔρριψε φέρεσθαι

ἐξοπίσω. τὰ μὲν οὔ τι ἐτώσια ἔκφυγε χειρός· 

ὅσσαι γὰρ ῥαθάμιγγες ἀπέσσυθεν αἱματόεσσαι,

πάσας δέξατο Γαῖα· περιπλομένων δ' ἐνιαυτῶν 

γείνατ' Ἐρινῦς τε κρατερὰς μεγάλους τε Γίγαντας, \num{185}

τεύχεσι λαμπομένους, δολίχ' ἔγχεα χερσὶν ἔχοντας,

Νύμφας θ' ἃς Μελίας καλέουσ' ἐπ' ἀπείρονα γαῖαν. 

μήδεα δ' ὡς τὸ πρῶτον ἀποτμήξας ἀδάμαντι

κάββαλ' ἀπ' ἠπείροιο πολυκλύστῳ ἐνὶ πόντῳ,

ὣς φέρετ' ἂμ πέλαγος πουλὺν χρόνον, ἀμφὶ δὲ λευκὸς \num{190}

ἀφρὸς ἀπ' ἀθανάτου χροὸς ὤρνυτο· τῷ δ' ἔνι κούρη 

ἐθρέφθη· πρῶτον δὲ Κυθήροισι ζαθέοισιν 

ἔπλητ', ἔνθεν ἔπειτα περίρρυτον ἵκετο Κύπρον.

ἐκ δ' ἔβη αἰδοίη καλὴ θεός, ἀμφὶ δὲ ποίη

ποσσὶν ὕπο ῥαδινοῖσιν ἀέξετο· τὴν δ' Ἀφροδίτην \num{195} 

ἀφρογενέα τε θεὰν καὶ ἐυστέφανον Κυθέρειαν

κικλήσκουσι θεοί τε καὶ ἀνέρες, οὕνεκ' ἐν ἀφρῷ

θρέφθη· ἀτὰρ Κυθέρειαν, ὅτι προσέκυρσε Κυθήροις· 

Κυπρογενέα δ', ὅτι γέντο περικλύστῳ ἐνὶ Κύπρῳ·

ἠδὲ φιλομμειδέα, ὅτι μηδέων ἐξεφαάνθη. \num{200}

τῇ δ' Ἔρος ὡμάρτησε καὶ Ἵμερος ἔσπετο καλὸς

γεινομένῃ τὰ πρῶτα θεῶν τ' ἐς φῦλον ἰούσῃ· 

ταύτην δ' ἐξ ἀρχῆς τιμὴν ἔχει ἠδὲ λέλογχε

μοῖραν ἐν ἀνθρώποισι καὶ ἀθανάτοισι θεοῖσι,

παρθενίους τ' ὀάρους μειδήματά τ' ἐξαπάτας τε \num{205}

τέρψίν τε γλυκερὴν φιλότητά τε μειλιχίην τε.

τοὺς δὲ πατὴρ Τιτῆνας ἐπίκλησιν καλέεσκε

παῖδας νεικείων μέγας Οὐρανός, οὓς τέκεν αὐτός· 

φάσκε δὲ τιταίνοντας ἀτασθαλίῃ μέγα ῥέξαι

ἔργον, τοῖο δ' ἔπειτα τίσιν μετόπισθεν ἔσεσθαι. \num{210}

Νὺξ δ' ἔτεκε στυγερόν τε Μόρον καὶ Κῆρα μέλαιναν 

καὶ Θάνατον, τέκε δ' Ὕπνον, ἔτικτε δὲ φῦλον Ὀνείρων. 

οὔ τινι κοιμηθεῖσα θεῶν τέκε Νὺξ ἐρεβεννή. 

δεύτερον αὖ Μῶμον καὶ Ὀιζὺν ἀλγινόεσσαν

Ἑσπερίδας θ', αἷς μῆλα πέρην κλυτοῦ Ὠκεανοῖο \num{215} 

χρύσεα καλὰ μέλουσι φέροντά τε δένδρεα καρπόν·

καὶ Μοίρας καὶ Κῆρας ἐγείνατο νηλεοποίνους,

{[}Κλωθώ τε Λάχεσίν τε καὶ Ἄτροπον, αἵ τε βροτοῖσι 

γεινομένοισι διδοῦσιν ἔχειν ἀγαθόν τε κακόν τε,{]} 

αἵ τ' ἀνδρῶν τε θεῶν τε παραιβασίας ἐφέπουσιν, \num{220} 

οὐδέ ποτε λήγουσι θεαὶ δεινοῖο χόλοιο,

πρίν γ' ἀπὸ τῷ δώωσι κακὴν ὄπιν, ὅστις ἁμάρτῃ.

τίκτε δὲ καὶ Νέμεσιν πῆμα θνητοῖσι βροτοῖσι 

Νὺξ ὀλοή· μετὰ τὴν δ' Ἀπάτην τέκε καὶ Φιλότητα 

Γῆράς τ' οὐλόμενον, καὶ Ἔριν τέκε καρτερόθυμον. \num{225}

αὐτὰρ Ἔρις στυγερὴ τέκε μὲν Πόνον ἀλγινόεντα

Λήθην τε Λιμόν τε καὶ Ἄλγεα δακρυόεντα

Ὑσμίνας τε Μάχας τε Φόνους τ' Ἀνδροκτασίας τε

Νείκεά τε Ψεύδεά τε Λόγους τ' Ἀμφιλλογίας τε 

Δυσνομίην τ' Ἄτην τε, συνήθεας ἀλλήλῃσιν, \num{230}

Ὅρκόν θ', ὃς δὴ πλεῖστον ἐπιχθονίους ἀνθρώπους

πημαίνει, ὅτε κέν τις ἑκὼν ἐπίορκον ὀμόσσῃ· 

Νηρέα δ' ἀψευδέα καὶ ἀληθέα γείνατο Πόντος 

πρεσβύτατον παίδων· αὐτὰρ καλέουσι γέροντα, 

οὕνεκα νημερτής τε καὶ ἤπιος, οὐδὲ θεμίστων \num{235}

λήθεται, ἀλλὰ δίκαια καὶ ἤπια δήνεα οἶδεν· 

αὖτις δ' αὖ Θαύμαντα μέγαν καὶ ἀγήνορα Φόρκυν

Γαίῃ μισγόμενος καὶ Κητὼ καλλιπάρηον 

Εὐρυβίην τ' ἀδάμαντος ἐνὶ φρεσὶ θυμὸν ἔχουσαν.

Νηρῆος δ' ἐγένοντο μεγήριτα τέκνα θεάων \num{240}

πόντῳ ἐν ἀτρυγέτῳ καὶ Δωρίδος ἠυκόμοιο,

κούρης Ὠκεανοῖο τελήεντος ποταμοῖο, 

Πρωθώ τ' Εὐκράντη τε Σαώ τ' Ἀμφιτρίτη τε 

Εὐδώρη τε Θέτις τε Γαλήνη τε Γλαύκη τε,

Κυμοθόη Σπειώ τε θοὴ Θαλίη τ' ἐρόεσσα \num{245}

Πασιθέη τ' Ἐρατώ τε καὶ Εὐνίκη ῥοδόπηχυς

καὶ Μελίτη χαρίεσσα καὶ Εὐλιμένη καὶ Ἀγαυὴ

Δωτώ τε Πρωτώ τε Φέρουσά τε Δυναμένη τε

Νησαίη τε καὶ Ἀκταίη καὶ Πρωτομέδεια,

Δωρὶς καὶ Πανόπη καὶ εὐειδὴς Γαλάτεια \num{250} 

Ἱπποθόη τ' ἐρόεσσα καὶ Ἱππονόη ῥοδόπηχυς

Κυμοδόκη θ', ἣ κύματ' ἐν ἠεροειδέι πόντῳ

πνοιάς τε ζαέων ἀνέμων σὺν Κυματολήγῃ

ῥεῖα πρηΰνει καὶ ἐυσφύρῳ Ἀμφιτρίτῃ,

Κυμώ τ' Ἠιόνη τε ἐυστέφανός θ' Ἁλιμήδη \num{255}

Γλαυκονόμη τε φιλομμειδὴς καὶ Ποντοπόρεια

Λειαγόρη τε καὶ Εὐαγόρη καὶ Λαομέδεια 

Πουλυνόη τε καὶ Αὐτονόη καὶ Λυσιάνασσα

Εὐάρνη τε φυὴν ἐρατὴ καὶ εἶδος ἄμωμος

καὶ Ψαμάθη χαρίεσσα δέμας δίη τε Μενίππη \num{260}

Νησώ τ' Εὐπόμπη τε Θεμιστώ τε Προνόη τε

Νημερτής θ', ἣ πατρὸς ἔχει νόον ἀθανάτοιο.

αὗται μὲν Νηρῆος ἀμύμονος ἐξεγένοντο

κοῦραι πεντήκοντα, ἀμύμονα ἔργ' εἰδυῖαι· 

Θαύμας δ' Ὠκεανοῖο βαθυρρείταο θύγατρα \num{265}

ἠγάγετ' Ἠλέκτρην· ἡ δ' ὠκεῖαν τέκεν Ἶριν 

ἠυκόμους θ' Ἁρπυίας, Ἀελλώ τ' Ὠκυπέτην τε, 

αἵ ῥ' ἀνέμων πνοιῇσι καὶ οἰωνοῖς ἅμ' ἕπονται

ὠκείῃς πτερύγεσσι· μεταχρόνιαι γὰρ ἴαλλον. 

Φόρκυι δ' αὖ Κητὼ γραίας τέκε καλλιπαρήους \num{270}

ἐκ γενετῆς πολιάς, τὰς δὴ Γραίας καλέουσιν

ἀθάνατοί τε θεοὶ χαμαὶ ἐρχόμενοί τ' ἄνθρωποι,

Πεμφρηδώ τ' εὔπεπλον Ἐνυώ τε κροκόπεπλον,

Γοργούς θ', αἳ ναίουσι πέρην κλυτοῦ Ὠκεανοῖο

ἐσχατιῇ πρὸς νυκτός, ἵν' Ἑσπερίδες λιγύφωνοι, \num{275}

Σθεννώ τ' Εὐρυάλη τε Μέδουσά τε λυγρὰ παθοῦσα· 

ἡ μὲν ἔην θνητή, αἱ δ' ἀθάνατοι καὶ ἀγήρῳ, 

αἱ δύο· τῇ δὲ μιῇ παρελέξατο Κυανοχαίτης 

ἐν μαλακῷ λειμῶνι καὶ ἄνθεσιν εἰαρινοῖσι. 

τῆς ὅτε δὴ Περσεὺς κεφαλὴν ἀπεδειροτόμησεν, \num{280} 

ἐξέθορε Χρυσάωρ τε μέγας καὶ Πήγασος ἵππος.

τῷ μὲν ἐπώνυμον ἦν, ὅτ' ἄρ' Ὠκεανοῦ παρὰ πηγὰς

γένθ', ὁ δ' ἄορ χρύσειον ἔχων μετὰ χερσὶ φίλῃσι. 

χὠ μὲν ἀποπτάμενος, προλιπὼν χθόνα μητέρα μήλων,

ἵκετ' ἐς ἀθανάτους· Ζηνὸς δ' ἐν δώμασι ναίει \num{285} 

βροντήν τε στεροπήν τε φέρων Διὶ μητιόεντι·

Χρυσάωρ δ' ἔτεκε τρικέφαλον Γηρυονῆα

μιχθεὶς Καλλιρόῃ κούρῃ κλυτοῦ Ὠκεανοῖο· 

τὸν μὲν ἄρ' ἐξενάριξε βίη Ἡρακληείη

βουσὶ πάρ' εἰλιπόδεσσι περιρρύτῳ εἰν Ἐρυθείῃ \num{290}

ἤματι τῷ, ὅτε περ βοῦς ἤλασεν εὐρυμετώπους 

Τίρυνθ' εἰς ἱερήν, διαβὰς πόρον Ὠκεανοῖο, 

Ὄρθόν τε κτείνας καὶ βουκόλον Εὐρυτίωνα

σταθμῷ ἐν ἠερόεντι πέρην κλυτοῦ Ὠκεανοῖο.

ἡ δ' ἔτεκ' ἄλλο πέλωρον ἀμήχανον, οὐδὲν ἐοικὸς \num{295} 

θνητοῖς ἀνθρώποις οὐδ' ἀθανάτοισι θεοῖσι, 

σπῆι ἔνι γλαφυρῷ, θείην κρατερόφρον' Ἔχιδναν, 

ἥμισυ μὲν νύμφην ἑλικώπιδα καλλιπάρηον, 

ἥμισυ δ' αὖτε πέλωρον ὄφιν δεινόν τε μέγαν τε

αἰόλον ὠμηστήν, ζαθέης ὑπὸ κεύθεσι γαίης. \num{300} 

ἔνθα δέ οἱ σπέος ἐστὶ κάτω κοίλῃ ὑπὸ πέτρῃ

τηλοῦ ἀπ' ἀθανάτων τε θεῶν θνητῶν τ' ἀνθρώπων,

ἔνθ' ἄρα οἱ δάσσαντο θεοὶ κλυτὰ δώματα ναίειν.

ἡ δ' ἔρυτ' εἰν Ἀρίμοισιν ὑπὸ χθόνα λυγρὴ Ἔχιδνα, 

ἀθάνατος νύμφη καὶ ἀγήραος ἤματα πάντα. \num{305}

τῇ δὲ Τυφάονά φασι μιγήμεναι ἐν φιλότητι

δεινόν θ' ὑβριστήν τ' ἄνομόν θ' ἑλικώπιδι κούρῃ· 

ἡ δ' ὑποκυσαμένη τέκετο κρατερόφρονα τέκνα.

Ὄρθον μὲν πρῶτον κύνα γείνατο Γηρυονῆι· 

δεύτερον αὖτις ἔτικτεν ἀμήχανον, οὔ τι φατειόν, \num{310} 

Κέρβερον ὠμηστήν, Ἀίδεω κύνα χαλκεόφωνον,

πεντηκοντακέφαλον, ἀναιδέα τε κρατερόν τε· 

τὸ τρίτον Ὕδρην αὖτις ἐγείνατο λύγρ' εἰδυῖαν 

Λερναίην, ἣν θρέψε θεὰ λευκώλενος Ἥρη

ἄπλητον κοτέουσα βίῃ Ἡρακληείῃ. \num{315}

καὶ τὴν μὲν Διὸς υἱὸς ἐνήρατο νηλέι χαλκῷ

Ἀμφιτρυωνιάδης σὺν ἀρηιφίλῳ Ἰολάῳ

Ἡρακλέης βουλῇσιν Ἀθηναίης ἀγελείης·

ἡ δὲ Χίμαιραν ἔτικτε πνέουσαν ἀμαιμάκετον πῦρ,

δεινήν τε μεγάλην τε ποδώκεά τε κρατερήν τε. \num{320} 

τῆς ἦν τρεῖς κεφαλαί· μία μὲν χαροποῖο λέοντος,

ἡ δὲ χιμαίρης, ἡ δ' ὄφιος κρατεροῖο δράκοντος.

πρόσθε λέων, ὄπιθεν δὲ δράκων, μέσση δὲ χίμαιρα,

δεινὸν ἀποπνείουσα πυρὸς μένος αἰθομένοιο.

τὴν μὲν Πήγασος εἷλε καὶ ἐσθλὸς Βελλεροφόντης. \num{325}

ἡ δ' ἄρα Φῖκ' ὀλοὴν τέκε Καδμείοισιν ὄλεθρον,

Ὄρθῳ ὑποδμηθεῖσα, Νεμειαῖόν τε λέοντα, 

τόν ῥ' Ἥρη θρέψασα Διὸς κυδρὴ παράκοιτις

γουνοῖσιν κατένασσε Νεμείης, πῆμ' ἀνθρώποις.

ἔνθ' ἄρ' ὅ γ' οἰκείων ἐλεφαίρετο φῦλ' ἀνθρώπων, \num{330}

κοιρανέων Τρητοῖο Νεμείης ἠδ' Ἀπέσαντος· 

ἀλλά ἑ ἲς ἐδάμασσε βίης Ἡρακληείης.

Κητὼ δ' ὁπλότατον Φόρκυι φιλότητι μιγεῖσα

γείνατο δεινὸν ὄφιν, ὃς ἐρεμνῆς κεύθεσι γαίης

πείρασιν ἐν μεγάλοις παγχρύσεα μῆλα φυλάσσει. \num{335}

τοῦτο μὲν ἐκ Κητοῦς καὶ Φόρκυνος γένος ἐστί. 

Τηθὺς δ' Ὠκεανῷ ποταμοὺς τέκε δινήεντας,

Νεῖλόν τ' Ἀλφειόν τε καὶ Ἠριδανὸν βαθυδίνην,

Στρυμόνα Μαίανδρόν τε καὶ Ἴστρον καλλιρέεθρον

Φᾶσίν τε Ῥῆσόν τ' Ἀχελῷόν τ' ἀργυροδίνην \num{340}

Νέσσόν τε Ῥοδίον θ' Ἁλιάκμονά θ' Ἑπτάπορόν τε

Γρήνικόν τε καὶ Αἴσηπον θεῖόν τε Σιμοῦντα

Πηνειόν τε καὶ Ἕρμον ἐυρρείτην τε Κάικον

Σαγγάριόν τε μέγαν Λάδωνά τε Παρθένιόν τε

Εὔηνόν τε καὶ Ἀλδῆσκον θεῖόν τε Σκάμανδρον· \num{345}

τίκτε δὲ θυγατέρων ἱερὸν γένος, αἳ κατὰ γαῖαν

ἄνδρας κουρίζουσι σὺν Ἀπόλλωνι ἄνακτι

καὶ ποταμοῖς, ταύτην δὲ Διὸς πάρα μοῖραν ἔχουσι,

Πειθώ τ' Ἀδμήτη τε Ἰάνθη τ' Ἠλέκτρη τε

Δωρίς τε Πρυμνώ τε καὶ Οὐρανίη θεοειδὴς \num{350}

Ἱππώ τε Κλυμένη τε Ῥόδειά τε Καλλιρόη τε

Ζευξώ τε Κλυτίη τε Ἰδυῖά τε Πασιθόη τε

Πληξαύρη τε Γαλαξαύρη τ' ἐρατή τε Διώνη

Μηλόβοσίς τε Θόη τε καὶ εὐειδὴς Πολυδώρη 

Κερκηίς τε φυὴν ἐρατὴ Πλουτώ τε βοῶπις \num{355}

Περσηίς τ' Ἰάνειρά τ' Ἀκάστη τε Ξάνθη τε

Πετραίη τ' ἐρόεσσα Μενεσθώ τ' Εὐρώπη τε

Μῆτίς τ' Εὐρυνόμη τε Τελεστώ τε κροκόπεπλος

Χρυσηίς τ' Ἀσίη τε καὶ ἱμερόεσσα Καλυψὼ

Εὐδώρη τε Τύχη τε καὶ Ἀμφιρὼ Ὠκυρόη τε \num{360}

καὶ Στύξ, ἣ δή σφεων προφερεστάτη ἐστὶν ἁπασέων.

αὗται ἄρ' Ὠκεανοῦ καὶ Τηθύος ἐξεγένοντο

πρεσβύταται κοῦραι· πολλαί γε μέν εἰσι καὶ ἄλλαι·

τρὶς γὰρ χίλιαί εἰσι τανίσφυροι Ὠκεανῖναι,

αἵ ῥα πολυσπερέες γαῖαν καὶ βένθεα λίμνης \num{365}

πάντῃ ὁμῶς ἐφέπουσι, θεάων ἀγλαὰ τέκνα. 

τόσσοι δ' αὖθ' ἕτεροι ποταμοὶ καναχηδὰ ῥέοντες,

υἱέες Ὠκεανοῦ, τοὺς γείνατο πότνια Τηθύς· 

τῶν ὄνομ' ἀργαλέον πάντων βροτὸν ἄνδρα ἐνισπεῖν, 

οἱ δὲ ἕκαστοι ἴσασιν, ὅσοι περιναιετάουσι. \num{370} 

Θεία δ' Ἠέλιόν τε μέγαν λαμπράν τε Σελήνην 

Ἠῶ θ', ἣ πάντεσσιν ἐπιχθονίοισι φαείνει

ἀθανάτοις τε θεοῖσι τοὶ οὐρανὸν εὐρὺν ἔχουσι,

γείναθ' ὑποδμηθεῖσ' Ὑπερίονος ἐν φιλότητι. 

Κρείῳ δ' Εὐρυβίη τέκεν ἐν φιλότητι μιγεῖσα \num{375} 

Ἀστραῖόν τε μέγαν Πάλλαντά τε δῖα θεάων

Πέρσην θ', ὃς καὶ πᾶσι μετέπρεπεν ἰδμοσύνῃσιν.

Ἀστραίῳ δ' Ἠὼς ἀνέμους τέκε καρτεροθύμους,

ἀργεστὴν Ζέφυρον Βορέην τ' αἰψηροκέλευθον 

καὶ Νότον, ἐν φιλότητι θεὰ θεῷ εὐνηθεῖσα. \num{380}

τοὺς δὲ μέτ' ἀστέρα τίκτεν Ἑωσφόρον Ἠριγένεια

ἄστρά τε λαμπετόωντα, τά τ' οὐρανὸς ἐστεφάνωται. 

Στὺξ δ' ἔτεκ' Ὠκεανοῦ θυγάτηρ Πάλλαντι μιγεῖσα

Ζῆλον καὶ Νίκην καλλίσφυρον ἐν μεγάροισι 

καὶ Κράτος ἠδὲ Βίην ἀριδείκετα γείνατο τέκνα. \num{385} 

τῶν οὐκ ἔστ' ἀπάνευθε Διὸς δόμος, οὐδέ τις ἕδρη,

οὐδ' ὁδός, ὅππῃ μὴ κείνοις θεὸς ἡγεμονεύει,

ἀλλ' αἰεὶ πὰρ Ζηνὶ βαρυκτύπῳ ἑδριόωνται.

ὣς γὰρ ἐβούλευσε Στὺξ ἄφθιτος Ὠκεανίνη

ἤματι τῷ, ὅτε πάντας Ὀλύμπιος ἀστεροπητὴς \num{390}

ἀθανάτους ἐκάλεσσε θεοὺς ἐς μακρὸν Ὄλυμπον,

εἶπε δ', ὃς ἂν μετὰ εἷο θεῶν Τιτῆσι μάχοιτο,

μή τιν' ἀπορραίσειν γεράων, τιμὴν δὲ ἕκαστον

ἑξέμεν, ἣν τὸ πάρος γε μετ' ἀθανάτοισι θεοῖσι.

τὸν δ' ἔφαθ', ὅστις ἄτιμος ὑπὸ Κρόνου ἠδ' ἀγέραστος, \num{395}

τιμῆς καὶ γεράων ἐπιβησέμεν, ἣ θέμις ἐστίν.

ἦλθε δ' ἄρα πρώτη Στὺξ ἄφθιτος Οὔλυμπόνδε

σὺν σφοῖσιν παίδεσσι φίλου διὰ μήδεα πατρός· 

τὴν δὲ Ζεὺς τίμησε, περισσὰ δὲ δῶρα ἔδωκεν.

αὐτὴν μὲν γὰρ ἔθηκε θεῶν μέγαν ἔμμεναι ὅρκον, \num{400}

παῖδας δ' ἤματα πάντα ἑοῦ μεταναιέτας εἶναι.

ὣς δ' αὔτως πάντεσσι διαμπερές, ὥς περ ὑπέστη,

ἐξετέλεσσ'· αὐτὸς δὲ μέγα κρατεῖ ἠδὲ ἀνάσσει. 

Φοίβη δ' αὖ Κοίου πολυήρατον ἦλθεν ἐς εὐνήν· 

κυσαμένη δἤπειτα θεὰ θεοῦ ἐν φιλότητι \num{405} 

Λητὼ κυανόπεπλον ἐγείνατο, μείλιχον αἰεί,

ἤπιον ἀνθρώποισι καὶ ἀθανάτοισι θεοῖσι, 

μείλιχον ἐξ ἀρχῆς, ἀγανώτατον ἐντὸς Ὀλύμπου.

γείνατο δ' Ἀστερίην εὐώνυμον, ἥν ποτε Πέρσης

ἠγάγετ' ἐς μέγα δῶμα φίλην κεκλῆσθαι ἄκοιτιν. \num{410}

ἡ δ' ὑποκυσαμένη Ἑκάτην τέκε, τὴν περὶ πάντων 

Ζεὺς Κρονίδης τίμησε· πόρεν δέ οἱ ἀγλαὰ δῶρα, 

μοῖραν ἔχειν γαίης τε καὶ ἀτρυγέτοιο θαλάσσης.

ἡ δὲ καὶ ἀστερόεντος ἀπ' οὐρανοῦ ἔμμορε τιμῆς, 

ἀθανάτοις τε θεοῖσι τετιμένη ἐστὶ μάλιστα. \num{415}

καὶ γὰρ νῦν, ὅτε πού τις ἐπιχθονίων ἀνθρώπων

ἔρδων ἱερὰ καλὰ κατὰ νόμον ἱλάσκηται,

κικλήσκει Ἑκάτην· πολλή τέ οἱ ἔσπετο τιμὴ 

ῥεῖα μάλ', ᾧ πρόφρων γε θεὰ ὑποδέξεται εὐχάς,

καί τέ οἱ ὄλβον ὀπάζει, ἐπεὶ δύναμίς γε πάρεστιν. \num{420}

ὅσσοι γὰρ Γαίης τε καὶ Οὐρανοῦ ἐξεγένοντο

καὶ τιμὴν ἔλαχον, τούτων ἔχει αἶσαν ἁπάντων· 

οὐδέ τί μιν Κρονίδης ἐβιήσατο οὐδέ τ' ἀπηύρα,

ὅσσ' ἔλαχεν Τιτῆσι μέτα προτέροισι θεοῖσιν, 

ἀλλ' ἔχει, ὡς τὸ πρῶτον ἀπ' ἀρχῆς ἔπλετο δασμός. \num{425} 

οὐδ', ὅτι μουνογενής, ἧσσον θεὰ ἔμμορε τιμῆς 

καὶ γεράων γαίῃ τε καὶ οὐρανῷ ἠδὲ θαλάσσῃ, 

ἀλλ' ἔτι καὶ πολὺ μᾶλλον, ἐπεὶ Ζεὺς τίεται αὐτήν.

ᾧ δ' ἐθέλῃ, μεγάλως παραγίνεται ἠδ' ὀνίνησιν· 

ἔν τ' ἀγορῇ λαοῖσι μεταπρέπει, ὅν κ' ἐθέλῃσιν· \num{430} 

ἠδ' ὁπότ' ἐς πόλεμον φθισήνορα θωρήσσωνται

ἀνέρες, ἔνθα θεὰ παραγίνεται, οἷς κ' ἐθέλῃσι

νίκην προφρονέως ὀπάσαι καὶ κῦδος ὀρέξαι.

ἔν τε δίκῃ βασιλεῦσι παρ' αἰδοίοισι καθίζει,

ἐσθλὴ δ' αὖθ' ὁπότ' ἄνδρες ἀεθλεύωσ' ἐν ἀγῶνι· \num{435}

ἔνθα θεὰ καὶ τοῖς παραγίνεται ἠδ' ὀνίνησι· 

νικήσας δὲ βίῃ καὶ κάρτει, καλὸν ἄεθλον 

ῥεῖα φέρει χαίρων τε, τοκεῦσι δὲ κῦδος ὀπάζει.

ἐσθλὴ δ' ἱππήεσσι παρεστάμεν, οἷς κ' ἐθέλῃσιν·

καὶ τοῖς, οἳ γλαυκὴν δυσπέμφελον ἐργάζονται, \num{440}

εὔχονται δ' Ἑκάτῃ καὶ ἐρικτύπῳ Ἐννοσιγαίῳ,

ῥηιδίως ἄγρην κυδρὴ θεὸς ὤπασε πολλήν,

ῥεῖα δ' ἀφείλετο φαινομένην, ἐθέλουσά γε θυμῷ.

ἐσθλὴ δ' ἐν σταθμοῖσι σὺν Ἑρμῇ ληίδ' ἀέξειν· 

βουκολίας δὲ βοῶν τε καὶ αἰπόλια πλατέ' αἰγῶν \num{445}

ποίμνας τ' εἰροπόκων ὀίων, θυμῷ γ' ἐθέλουσα,

ἐξ ὀλίγων βριάει κἀκ πολλῶν μείονα θῆκεν.

οὕτω τοι καὶ μουνογενὴς ἐκ μητρὸς ἐοῦσα

πᾶσι μετ' ἀθανάτοισι τετίμηται γεράεσσι. 

θῆκε δέ μιν Κρονίδης κουροτρόφον, οἳ μετ' ἐκείνην\num{450}

ὀφθαλμοῖσιν ἴδοντο φάος πολυδερκέος Ἠοῦς.

οὕτως ἐξ ἀρχῆς κουροτρόφος, αἳ δέ τε τιμαί.

Ῥείη δὲ δμηθεῖσα Κρόνῳ τέκε φαίδιμα τέκνα, 

Ἱστίην Δήμητρα καὶ Ἥρην χρυσοπέδιλον, 

ἴφθιμόν τ' Ἀίδην, ὃς ὑπὸ χθονὶ δώματα ναίει \num{455}

νηλεὲς ἦτορ ἔχων, καὶ ἐρίκτυπον Ἐννοσίγαιον,

Ζῆνά τε μητιόεντα, θεῶν πατέρ' ἠδὲ καὶ ἀνδρῶν,

τοῦ καὶ ὑπὸ βροντῆς πελεμίζεται εὐρεῖα χθών.

καὶ τοὺς μὲν κατέπινε μέγας Κρόνος, ὥς τις ἕκαστος

νηδύος ἐξ ἱερῆς μητρὸς πρὸς γούναθ' ἵκοιτο, \num{460}

τὰ φρονέων, ἵνα μή τις ἀγαυῶν Οὐρανιώνων

ἄλλος ἐν ἀθανάτοισιν ἔχοι βασιληίδα τιμήν.

πεύθετο γὰρ Γαίης τε καὶ Οὐρανοῦ ἀστερόεντος 

οὕνεκά οἱ πέπρωτο ἑῷ ὑπὸ παιδὶ δαμῆναι, 

καὶ κρατερῷ περ ἐόντι, Διὸς μεγάλου διὰ βουλάς. \num{465} 

τῷ ὅ γ' ἄρ' οὐκ ἀλαοσκοπιὴν ἔχεν, ἀλλὰ δοκεύων 

παῖδας ἑοὺς κατέπινε· Ῥέην δ' ἔχε πένθος ἄλαστον. 

ἀλλ' ὅτε δὴ Δί' ἔμελλε θεῶν πατέρ' ἠδὲ καὶ ἀνδρῶν

τέξεσθαι, τότ' ἔπειτα φίλους λιτάνευε τοκῆας

τοὺς αὐτῆς, Γαῖάν τε καὶ Οὐρανὸν ἀστερόεντα, \num{470}

μῆτιν συμφράσσασθαι, ὅπως λελάθοιτο τεκοῦσα

παῖδα φίλον, τείσαιτο δ' ἐρινῦς πατρὸς ἑοῖο 

παίδων \textless{}θ'\textgreater{} οὓς κατέπινε μέγας Κρόνος
ἀγκυλομήτης. 

οἱ δὲ θυγατρὶ φίλῃ μάλα μὲν κλύον ἠδ' ἐπίθοντο, 

καί οἱ πεφραδέτην, ὅσα περ πέπρωτο γενέσθαι \num{475}

ἀμφὶ Κρόνῳ βασιλῆι καὶ υἱέι καρτεροθύμῳ· 

πέμψαν δ' ἐς Λύκτον, Κρήτης ἐς πίονα δῆμον,

ὁππότ' ἄρ' ὁπλότατον παίδων ἤμελλε τεκέσθαι, 

Ζῆνα μέγαν· τὸν μέν οἱ ἐδέξατο Γαῖα πελώρη 

Κρήτῃ ἐν εὐρείῃ τρεφέμεν ἀτιταλλέμεναί τε. \num{480} 

ἔνθά μιν ἷκτο φέρουσα θοὴν διὰ νύκτα μέλαιναν, 

πρώτην ἐς Λύκτον· κρύψεν δέ ἑ χερσὶ λαβοῦσα 

ἄντρῳ ἐν ἠλιβάτῳ, ζαθέης ὑπὸ κεύθεσι γαίης,

Αἰγαίῳ ἐν ὄρει πεπυκασμένῳ ὑλήεντι.

τῷ δὲ σπαργανίσασα μέγαν λίθον ἐγγυάλιξεν \num{485}

Οὐρανίδῃ μέγ' ἄνακτι, θεῶν προτέρων βασιλῆι.

τὸν τόθ' ἑλὼν χείρεσσιν ἑὴν ἐσκάτθετο νηδύν, 

σχέτλιος, οὐδ' ἐνόησε μετὰ φρεσίν, ὥς οἱ ὀπίσσω 

ἀντὶ λίθου ἑὸς υἱὸς ἀνίκητος καὶ ἀκηδὴς

λείπεθ', ὅ μιν τάχ' ἔμελλε βίῃ καὶ χερσὶ δαμάσσας \num{490}

τιμῆς ἐξελάαν, ὁ δ' ἐν ἀθανάτοισιν ἀνάξειν. 

καρπαλίμως δ' ἄρ' ἔπειτα μένος καὶ φαίδιμα γυῖα

ηὔξετο τοῖο ἄνακτος· ἐπιπλομένου δ' ἐνιαυτοῦ, 

Γαίης ἐννεσίῃσι πολυφραδέεσσι δολωθείς, 

ὃν γόνον ἂψ ἀνέηκε μέγας Κρόνος ἀγκυλομήτης, \num{495} 

νικηθεὶς τέχνῃσι βίηφί τε παιδὸς ἑοῖο.

πρῶτον δ' ἐξήμησε λίθον, πύματον καταπίνων·

τὸν μὲν Ζεὺς στήριξε κατὰ χθονὸς εὐρυοδείης

Πυθοῖ ἐν ἠγαθέῃ, γυάλοις ὕπο Παρνησσοῖο, 

σῆμ' ἔμεν ἐξοπίσω, θαῦμα θνητοῖσι βροτοῖσι. \num{500} 

λῦσε δὲ πατροκασιγνήτους ὀλοῶν ὑπὸ δεσμῶν, 

Οὐρανίδας, οὓς δῆσε πατὴρ ἀεσιφροσύνῃσιν· 

οἵ οἱ ἀπεμνήσαντο χάριν εὐεργεσιάων,

δῶκαν δὲ βροντὴν ἠδ' αἰθαλόεντα κεραυνὸν

καὶ στεροπήν· τὸ πρὶν δὲ πελώρη Γαῖα κεκεύθει· \num{505} 

τοῖς πίσυνος θνητοῖσι καὶ ἀθανάτοισιν ἀνάσσει.

κούρην δ' Ἰαπετὸς καλλίσφυρον Ὠκεανίνην

ἠγάγετο Κλυμένην καὶ ὁμὸν λέχος εἰσανέβαινεν.

ἡ δέ οἱ Ἄτλαντα κρατερόφρονα γείνατο παῖδα,

τίκτε δ' ὑπερκύδαντα Μενοίτιον ἠδὲ Προμηθέα, \num{510}

ποικίλον αἰολόμητιν, ἁμαρτίνοόν τ' Ἐπιμηθέα· 

ὃς κακὸν ἐξ ἀρχῆς γένετ' ἀνδράσιν ἀλφηστῇσι· 

πρῶτος γάρ ῥα Διὸς πλαστὴν ὑπέδεκτο γυναῖκα

παρθένον. ὑβριστὴν δὲ Μενοίτιον εὐρύοπα Ζεὺς

εἰς ἔρεβος κατέπεμψε βαλὼν ψολόεντι κεραυνῷ \num{515} 

εἵνεκ' ἀτασθαλίης τε καὶ ἠνορέης ὑπερόπλου.

Ἄτλας δ' οὐρανὸν εὐρὺν ἔχει κρατερῆς ὑπ' ἀνάγκης, 

πείρασιν ἐν γαίης πρόπαρ' Ἑσπερίδων λιγυφώνων 

ἑστηώς, κεφαλῇ τε καὶ ἀκαμάτῃσι χέρεσσι· 

ταύτην γάρ οἱ μοῖραν ἐδάσσατο μητίετα Ζεύς. \num{520}

δῆσε δ' ἀλυκτοπέδῃσι Προμηθέα ποικιλόβουλον,

δεσμοῖς ἀργαλέοισι, μέσον διὰ κίον' ἐλάσσας· 

καί οἱ ἐπ' αἰετὸν ὦρσε τανύπτερον· αὐτὰρ ὅ γ' ἧπαρ 

ἤσθιεν ἀθάνατον, τὸ δ' ἀέξετο ἶσον ἁπάντῃ 

νυκτός, ὅσον πρόπαν ἦμαρ ἔδοι τανυσίπτερος ὄρνις. \num{525} 

τὸν μὲν ἄρ' Ἀλκμήνης καλλισφύρου ἄλκιμος υἱὸς

Ἡρακλέης ἔκτεινε, κακὴν δ' ἀπὸ νοῦσον ἄλαλκεν

Ἰαπετιονίδῃ καὶ ἐλύσατο δυσφροσυνάων, 

οὐκ ἀέκητι Ζηνὸς Ὀλυμπίου ὕψι μέδοντος,

ὄφρ' Ἡρακλῆος Θηβαγενέος κλέος εἴη \num{530}

πλεῖον ἔτ' ἢ τὸ πάροιθεν ἐπὶ χθόνα πουλυβότειραν.

ταῦτ' ἄρα ἁζόμενος τίμα ἀριδείκετον υἱόν·

καί περ χωόμενος παύθη χόλου, ὃν πρὶν ἔχεσκεν,

οὕνεκ' ἐρίζετο βουλὰς ὑπερμενέι Κρονίωνι.

καὶ γὰρ ὅτ' ἐκρίνοντο θεοὶ θνητοί τ' ἄνθρωποι \num{535}

Μηκώνῃ, τότ' ἔπειτα μέγαν βοῦν πρόφρονι θυμῷ

δασσάμενος προύθηκε, Διὸς νόον ἐξαπαφίσκων. 

τῷ μὲν γὰρ σάρκάς τε καὶ ἔγκατα πίονα δημῷ 

ἐν ῥινῷ κατέθηκε, καλύψας γαστρὶ βοείῃ,

τοῖς δ' αὖτ' ὀστέα λευκὰ βοὸς δολίῃ ἐπὶ τέχνῃ \num{540} 

εὐθετίσας κατέθηκε, καλύψας ἀργέτι δημῷ. 

δὴ τότε μιν προσέειπε πατὴρ ἀνδρῶν τε θεῶν τε· 

``Ἰαπετιονίδη, πάντων ἀριδείκετ' ἀνάκτων, 

ὦ πέπον, ὡς ἑτεροζήλως διεδάσσαο μοίρας.'' 

ὣς φάτο κερτομέων Ζεὺς ἄφθιτα μήδεα εἰδώς· \num{545} 

τὸν δ' αὖτε προσέειπε Προμηθεὺς ἀγκυλομήτης,

ἦκ' ἐπιμειδήσας, δολίης δ' οὐ λήθετο τέχνης· 

``Ζεῦ κύδιστε μέγιστε θεῶν αἰειγενετάων, 

τῶν δ' ἕλευ ὁπποτέρην σε ἐνὶ φρεσὶ θυμὸς ἀνώγει.'' 

φῆ ῥα δολοφρονέων· Ζεὺς δ' ἄφθιτα μήδεα εἰδὼς \num{550} 

γνῶ ῥ' οὐδ' ἠγνοίησε δόλον· κακὰ δ' ὄσσετο θυμῷ 

θνητοῖς ἀνθρώποισι, τὰ καὶ τελέεσθαι ἔμελλε.

χερσὶ δ' ὅ γ' ἀμφοτέρῃσιν ἀνείλετο λευκὸν ἄλειφαρ, 

χώσατο δὲ φρένας ἀμφί, χόλος δέ μιν ἵκετο θυμόν,

ὡς ἴδεν ὀστέα λευκὰ βοὸς δολίῃ ἐπὶ τέχῃ. \num{555}

ἐκ τοῦ δ' ἀθανάτοισιν ἐπὶ χθονὶ φῦλ' ἀνθρώπων

καίουσ' ὀστέα λευκὰ θυηέντων ἐπὶ βωμῶν. 

τὸν δὲ μέγ' ὀχθήσας προσέφη νεφεληγερέτα Ζεύς· 

``Ἰαπετιονίδη, πάντων πέρι μήδεα εἰδώς, 

ὦ πέπον, οὐκ ἄρα πω δολίης ἐπελήθεο τέχνης.'' \num{560} 

ὣς φάτο χωόμενος Ζεὺς ἄφθιτα μήδεα εἰδώς. 

ἐκ τούτου δἤπειτα χόλου μεμνημένος αἰεὶ 

οὐκ ἐδίδου μελίῃσι πυρὸς μένος ἀκαμάτοιο

θνητοῖς ἀνθρώποις οἳ ἐπὶ χθονὶ ναιετάουσιν· 

ἀλλά μιν ἐξαπάτησεν ἐὺς πάις Ἰαπετοῖο \num{565}

κλέψας ἀκαμάτοιο πυρὸς τηλέσκοπον αὐγὴν

ἐν κοίλῳ νάρθηκι· δάκεν δ' ἄρα νειόθι θυμὸν

Ζῆν' ὑψιβρεμέτην, ἐχόλωσε δέ μιν φίλον ἦτορ,

ὡς ἴδ' ἐν ἀνθρώποισι πυρὸς τηλέσκοπον αὐγήν.

αὐτίκα δ' ἀντὶ πυρὸς τεῦξεν κακὸν ἀνθρώποισι· \num{570} 

γαίης γὰρ σύμπλασσε περικλυτὸς Ἀμφιγυήεις

παρθένῳ αἰδοίῃ ἴκελον Κρονίδεω διὰ βουλάς· 

ζῶσε δὲ καὶ κόσμησε θεὰ γλαυκῶπις Ἀθήνη

ἀργυφέῃ ἐσθῆτι· κατὰ κρῆθεν δὲ καλύπτρην 

δαιδαλέην χείρεσσι κατέσχεθε, θαῦμα ἰδέσθαι· \num{575} 

ἀμφὶ δέ οἱ στεφάνους νεοθηλέας, ἄνθεα ποίης,

ἱμερτοὺς περίθηκε καρήατι Παλλὰς Ἀθήνη· 

ἀμφὶ δέ οἱ στεφάνην χρυσέην κεφαλῆφιν ἔθηκε,

τὴν αὐτὸς ποίησε περικλυτὸς Ἀμφιγυήεις

ἀσκήσας παλάμῃσι, χαριζόμενος Διὶ πατρί. \num{580}

τῇ δ' ἔνι δαίδαλα πολλὰ τετεύχατο, θαῦμα ἰδέσθαι,

κνώδαλ' ὅσ' ἤπειρος δεινὰ τρέφει ἠδὲ θάλασσα·

τῶν ὅ γε πόλλ' ἐνέθηκε, χάρις δ' ἐπὶ πᾶσιν ἄητο,

θαυμάσια, ζωοῖσιν ἐοικότα φωνήεσσιν.

αὐτὰρ ἐπεὶ δὴ τεῦξε καλὸν κακὸν ἀντ' ἀγαθοῖο, \num{585} 

ἐξάγαγ' ἔνθά περ ἄλλοι ἔσαν θεοὶ ἠδ' ἄνθρωποι,

κόσμῳ ἀγαλλομένην γλαυκώπιδος Ὀβριμοπάτρης· 

θαῦμα δ' ἔχ' ἀθανάτους τε θεοὺς θνητούς τ' ἀνθρώπους,

ὡς εἶδον δόλον αἰπύν, ἀμήχανον ἀνθρώποισιν.

ἐκ τῆς γὰρ γένος ἐστὶ γυναικῶν θηλυτεράων, \num{590}

τῆς γὰρ ὀλοίιόν ἐστι γένος καὶ φῦλα γυναικῶν,

πῆμα μέγα θνητοῖσι, σὺν ἀνδράσι ναιετάουσαι,

οὐλομένης Πενίης οὐ σύμφοροι, ἀλλὰ Κόροιο.

ὡς δ' ὁπότ' ἐν σμήνεσσι κατηρεφέεσσι μέλισσαι

κηφῆνας βόσκωσι, κακῶν ξυνήονας ἔργων· \num{595}

αἱ μέν τε πρόπαν ἦμαρ ἐς ἠέλιον καταδύντα

ἠμάτιαι σπεύδουσι τιθεῖσί τε κηρία λευκά,

οἱ δ' ἔντοσθε μένοντες ἐπηρεφέας κατὰ σίμβλους 

ἀλλότριον κάματον σφετέρην ἐς γαστέρ' ἀμῶνται· 

ὣς δ' αὔτως ἄνδρεσσι κακὸν θνητοῖσι γυναῖκας \num{600}

Ζεὺς ὑψιβρεμέτης θῆκε, ξυνήονας ἔργων

ἀργαλέων. ἕτερον δὲ πόρεν κακὸν ἀντ' ἀγαθοῖο,

ὅς κε γάμον φεύγων καὶ μέρμερα ἔργα γυναικῶν

μὴ γῆμαι ἐθέλῃ, ὀλοὸν δ' ἐπὶ γῆρας ἵκηται

χήτει γηροκόμοιο· ὁ δ' οὐ βιότου γ' ἐπιδευὴς \num{605} 

ζώει, ἀποφθιμένου δὲ διὰ ζωὴν δατέονται

χηρωσταί. ᾧ δ' αὖτε γάμου μετὰ μοῖρα γένηται, 

κεδνὴν δ' ἔσχεν ἄκοιτιν, ἀρηρυῖαν πραπίδεσσι, 

τῷ δέ τ' ἀπ' αἰῶνος κακὸν ἐσθλῷ ἀντιφερίζει

ἐμμενές· ὃς δέ κε τέτμῃ ἀταρτηροῖο γενέθλης, \num{610} 

ζώει ἐνὶ στήθεσσιν ἔχων ἀλίαστον ἀνίην

θυμῷ καὶ κραδίῃ, καὶ ἀνήκεστον κακόν ἐστιν.

ὣς οὐκ ἔστι Διὸς κλέψαι νόον οὐδὲ παρελθεῖν.

οὐδὲ γὰρ Ἰαπετιονίδης ἀκάκητα Προμηθεὺς

τοῖό γ' ὑπεξήλυξε βαρὺν χόλον, ἀλλ' ὑπ' ἀνάγκης \num{615}

καὶ πολύιδριν ἐόντα μέγας κατὰ δεσμὸς ἐρύκει.

Ὀβριάρεῳ δ' ὡς πρῶτα πατὴρ ὠδύσσατο θυμῷ 

Κόττῳ τ' ἠδὲ Γύγῃ, δῆσε κρατερῷ ἐνὶ δεσμῷ, 

ἠνορέην ὑπέροπλον ἀγώμενος ἠδὲ καὶ εἶδος

καὶ μέγεθος· κατένασσε δ' ὑπὸ χθονὸς εὐρυοδείης. \num{620} 

ἔνθ' οἵ γ' ἄλγε' ἔχοντες ὑπὸ χθονὶ ναιετάοντες

εἵατ' ἐπ' ἐσχατιῇ μεγάλης ἐν πείρασι γαίης 

δηθὰ μάλ' ἀχνύμενοι, κραδίῃ μέγα πένθος ἔχοντες.

ἀλλά σφεας Κρονίδης τε καὶ ἀθάνατοι θεοὶ ἄλλοι 

οὓς τέκεν ἠύκομος Ῥείη Κρόνου ἐν φιλότητι \num{625} 

Γαίης φραδμοσύνῃσιν ἀνήγαγον ἐς φάος αὖτις· 

αὐτὴ γάρ σφιν ἅπαντα διηνεκέως κατέλεξε, 

σὺν κείνοις νίκην τε καὶ ἀγλαὸν εὖχος ἀρέσθαι.

δηρὸν γὰρ μάρναντο πόνον θυμαλγέ' ἔχοντες

ἀντίον ἀλλήλοισι διὰ κρατερὰς ὑσμίνας \num{631}   %problema com os números dos versos. verificar com o documento do werner

Τιτῆνές τε θεοὶ καὶ ὅσοι Κρόνου ἐξεγένοντο, \num{630}

οἱ μὲν ἀφ' ὑψηλῆς Ὄθρυος Τιτῆνες ἀγαυοί, \num{632} 

οἱ δ' ἄρ' ἀπ' Οὐλύμποιο θεοὶ δωτῆρες ἐάων 

οὓς τέκεν ἠύκομος Ῥείη Κρόνῳ εὐνηθεῖσα.

οἵ ῥα τότ' ἀλλήλοισι πόνον θυμαλγέ' ἔχοντες \num{635} 

συνεχέως ἐμάχοντο δέκα πλείους ἐνιαυτούς· 

οὐδέ τις ἦν ἔριδος χαλεπῆς λύσις οὐδὲ τελευτὴ

οὐδετέροις, ἶσον δὲ τέλος τέτατο πτολέμοιο.

ἀλλ' ὅτε δὴ κείνοισι παρέσχεθεν ἄρμενα πάντα,

νέκταρ τ' ἀμβροσίην τε, τά περ θεοὶ αὐτοὶ ἔδουσι, \num{640}

πάντων \textless{}τ'\textgreater{} ἐν στήθεσσιν ἀέξετο θυμὸς ἀγήνωρ,

ὡς νέκταρ τ' ἐπάσαντο καὶ ἀμβροσίην ἐρατεινήν,

δὴ τότε τοῖς μετέειπε πατὴρ ἀνδρῶν τε θεῶν τε·

``κέκλυτέ μευ Γαίης τε καὶ Οὐρανοῦ ἀγλαὰ τέκνα, 

ὄφρ' εἴπω τά με θυμὸς ἐνὶ στήθεσσι κελεύει. \num{645} 

ἤδη γὰρ μάλα δηρὸν ἐναντίοι ἀλλήλοισι

νίκης καὶ κάρτευς πέρι μαρνάμεθ' ἤματα πάντα, 

Τιτῆνές τε θεοὶ καὶ ὅσοι Κρόνου ἐκγενόμεσθα.

ὑμεῖς δὲ μεγάλην τε βίην καὶ χεῖρας ἀάπτους

φαίνετε Τιτήνεσσιν ἐναντίον ἐν δαῒ λυγρῇ, \num{650}

μνησάμενοι φιλότητος ἐνηέος, ὅσσα παθόντες

ἐς φάος ἂψ ἀφίκεσθε δυσηλεγέος ὑπὸ δεσμοῦ

ἡμετέρας διὰ βουλὰς ὑπὸ ζόφου ἠερόεντος.''

ὣς φάτο· τὸν δ' αἶψ' αὖτις ἀμείβετο Κόττος ἀμύμων· 

``δαιμόνι', οὐκ ἀδάητα πιφαύσκεαι, ἀλλὰ καὶ αὐτοὶ \num{655} 

ἴδμεν ὅ τοι περὶ μὲν πραπίδες, περὶ δ' ἐστὶ νόημα,

ἀλκτὴρ δ' ἀθανάτοισιν ἀρῆς γένεο κρυεροῖο, 

σῇσι δ' ἐπιφροσύνῃσιν ὑπὸ ζόφου ἠερόεντος

ἄψορρον ἐξαῦτις ἀμειλίκτων ὑπὸ δεσμῶν

ἠλύθομεν, Κρόνου υἱὲ ἄναξ, ἀνάελπτα παθόντες. \num{660}

τῷ καὶ νῦν ἀτενεῖ τε νόῳ καὶ πρόφρονι θυμῷ

ῥυσόμεθα κράτος ὑμὸν ἐν αἰνῇ δηιοτῆτι, 

μαρνάμενοι Τιτῆσιν ἀνὰ κρατερὰς ὑσμίνας.'' 

ὣς φάτ'· ἐπῄνησαν δὲ θεοὶ δωτῆρες ἐάων 

μῦθον ἀκούσαντες· πολέμου δ' ἐλιλαίετο θυμὸς \num{665} 

μᾶλλον ἔτ' ἢ τὸ πάροιθε· μάχην δ' ἀμέγαρτον ἔγειραν 

πάντες, θήλειαί τε καὶ ἄρσενες, ἤματι κείνῳ,

Τιτῆνές τε θεοὶ καὶ ὅσοι Κρόνου ἐξεγένοντο,

οὕς τε Ζεὺς ἐρέβεσφιν ὑπὸ χθονὸς ἧκε φόωσδε,

δεινοί τε κρατεροί τε, βίην ὑπέροπλον ἔχοντες. \num{670}

τῶν ἑκατὸν μὲν χεῖρες ἀπ' ὤμων ἀίσσοντο

πᾶσιν ὁμῶς, κεφαλαὶ δὲ ἑκάστῳ πεντήκοντα

ἐξ ὤμων ἐπέφυκον ἐπὶ στιβαροῖσι μέλεσσιν.

οἳ τότε Τιτήνεσσι κατέσταθεν ἐν δαῒ λυγρῇ

πέτρας ἠλιβάτους στιβαρῇς ἐν χερσὶν ἔχοντες· \num{675} 

Τιτῆνες δ' ἑτέρωθεν ἐκαρτύναντο φάλαγγας

προφρονέως· χειρῶν τε βίης θ' ἅμα ἔργον ἔφαινον 

ἀμφότεροι, δεινὸν δὲ περίαχε πόντος ἀπείρων,

γῆ δὲ μέγ' ἐσμαράγησεν, ἐπέστενε δ' οὐρανὸς εὐρὺς

σειόμενος, πεδόθεν δὲ τινάσσετο μακρὸς Ὄλυμπος \num{680}

ῥιπῇ ὕπ' ἀθανάτων, ἔνοσις δ' ἵκανε βαρεῖα

τάρταρον ἠερόεντα ποδῶν αἰπεῖά τ' ἰωὴ

ἀσπέτου ἰωχμοῖο βολάων τε κρατεράων. 

ὣς ἄρ' ἐπ' ἀλλήλοις ἵεσαν βέλεα στονόεντα· 

φωνὴ δ' ἀμφοτέρων ἵκετ' οὐρανὸν ἀστερόεντα \num{685}

κεκλομένων· οἱ δὲ ξύνισαν μεγάλῳ ἀλαλητῷ. 

οὐδ' ἄρ' ἔτι Ζεὺς ἴσχεν ἑὸν μένος, ἀλλά νυ τοῦ γε

εἶθαρ μὲν μένεος πλῆντο φρένες, ἐκ δέ τε πᾶσαν

φαῖνε βίην· ἄμυδις δ' ἄρ' ἀπ' οὐρανοῦ ἠδ' ἀπ' Ὀλύμπου 

ἀστράπτων ἔστειχε συνωχαδόν, οἱ δὲ κεραυνοὶ \num{690} 

ἴκταρ ἅμα βροντῇ τε καὶ ἀστεροπῇ ποτέοντο

χειρὸς ἄπο στιβαρῆς, ἱερὴν φλόγα εἰλυφόωντες, 

ταρφέες· ἀμφὶ δὲ γαῖα φερέσβιος ἐσμαράγιζε 

καιομένη, λάκε δ' ἀμφὶ περὶ μεγάλ' ἄσπετος ὕλη·

ἔζεε δὲ χθὼν πᾶσα καὶ Ὠκεανοῖο ῥέεθρα \num{695}

πόντός τ' ἀτρύγετος· τοὺς δ' ἄμφεπε θερμὸς ἀυτμὴ 

Τιτῆνας χθονίους, φλὸξ δ' αἰθέρα δῖαν ἵκανεν

ἄσπετος, ὄσσε δ' ἄμερδε καὶ ἰφθίμων περ ἐόντων

αὐγὴ μαρμαίρουσα κεραυνοῦ τε στεροπῆς τε.

καῦμα δὲ θεσπέσιον κάτεχεν χάος· εἴσατο δ' ἄντα \num{700} 

ὀφθαλμοῖσιν ἰδεῖν ἠδ' οὔασιν ὄσσαν ἀκοῦσαι

αὔτως, ὡς ὅτε γαῖα καὶ οὐρανὸς εὐρὺς ὕπερθε 

πίλνατο· τοῖος γάρ κε μέγας ὑπὸ δοῦπος ὀρώρει, 

τῆς μὲν ἐρειπομένης, τοῦ δ' ὑψόθεν ἐξεριπόντος· 

τόσσος δοῦπος ἔγεντο θεῶν ἔριδι ξυνιόντων. \num{705}

σὺν δ' ἄνεμοι ἔνοσίν τε κονίην τ' ἐσφαράγιζον

βροντήν τε στεροπήν τε καὶ αἰθαλόεντα κεραυνόν,

κῆλα Διὸς μεγάλοιο, φέρον δ' ἰαχήν τ' ἐνοπήν τε

ἐς μέσον ἀμφοτέρων· ὄτοβος δ' ἄπλητος ὀρώρει 

σμερδαλέης ἔριδος, κάρτευς δ' ἀνεφαίνετο ἔργον. \num{710}

ἐκλίνθη δὲ μάχη· πρὶν δ' ἀλλήλοις ἐπέχοντες 

ἐμμενέως ἐμάχοντο διὰ κρατερὰς ὑσμίνας.

οἱ δ' ἄρ' ἐνὶ πρώτοισι μάχην δριμεῖαν ἔγειραν, 

Κόττος τε Βριάρεώς τε Γύγης τ' ἄατος πολέμοιο·

οἵ ῥα τριηκοσίας πέτρας στιβαρέων ἀπὸ χειρῶν \num{715} 

πέμπον ἐπασσυτέρας, κατὰ δ' ἐσκίασαν βελέεσσι

Τιτῆνας· καὶ τοὺς μὲν ὑπὸ χθονὸς εὐρυοδείης 

πέμψαν καὶ δεσμοῖσιν ἐν ἀργαλέοισιν ἔδησαν,

νικήσαντες χερσὶν ὑπερθύμους περ ἐόντας, 

τόσσον ἔνερθ' ὑπὸ γῆς ὅσον οὐρανός ἐστ' ἀπὸ γαίης· \num{720} 

τόσσον γάρ τ' ἀπὸ γῆς ἐς τάρταρον ἠερόεντα.

ἐννέα γὰρ νύκτας τε καὶ ἤματα χάλκεος ἄκμων

οὐρανόθεν κατιών, δεκάτῃ κ' ἐς γαῖαν ἵκοιτο· 

{[}ἶσον δ' αὖτ' ἀπὸ γῆς ἐς τάρταρον ἠερόεντα·{]}

ἐννέα δ' αὖ νύκτας τε καὶ ἤματα χάλκεος ἄκμων

ἐκ γαίης κατιών, δεκάτῃ κ' ἐς τάρταρον ἵκοι. \num{725} 

τὸν πέρι χάλκεον ἕρκος ἐλήλαται· ἀμφὶ δέ μιν νὺξ 

τριστοιχὶ κέχυται περὶ δειρήν· αὐτὰρ ὕπερθε

γῆς ῥίζαι πεφύασι καὶ ἀτρυγέτοιο θαλάσσης.

ἔνθα θεοὶ Τιτῆνες ὑπὸ ζόφῳ ἠερόεντι

κεκρύφαται βουλῇσι Διὸς νεφεληγερέταο, \num{730} 

χώρῳ ἐν εὐρώεντι, πελώρης ἔσχατα γαίης.

τοῖς οὐκ ἐξιτόν ἐστι, θύρας δ' ἐπέθηκε Ποσειδέων 

χαλκείας, τεῖχος δ' ἐπελήλαται ἀμφοτέρωθεν.

ἔνθα Γύγης Κόττος τε καὶ Ὀβριάρεως μεγάθυμος 

ναίουσιν, φύλακες πιστοὶ Διὸς αἰγιόχοιο. \num{735}

ἔνθα δὲ γῆς δνοφερῆς καὶ ταρτάρου ἠερόεντος 

πόντου τ' ἀτρυγέτοιο καὶ οὐρανοῦ ἀστερόεντος

ἑξείης πάντων πηγαὶ καὶ πείρατ' ἔασιν,

ἀργαλέ' εὐρώεντα, τά τε στυγέουσι θεοί περ·

χάσμα μέγ', οὐδέ κε πάντα τελεσφόρον εἰς ἐνιαυτὸν \num{740}

οὖδας ἵκοιτ', εἰ πρῶτα πυλέων ἔντοσθε γένοιτο,

ἀλλά κεν ἔνθα καὶ ἔνθα φέροι πρὸ θύελλα θυέλλης 

ἀργαλέη· δεινὸν δὲ καὶ ἀθανάτοισι θεοῖσι 

τοῦτο τέρας· καὶ Νυκτὸς ἐρεμνῆς οἰκία δεινὰ

ἕστηκεν νεφέλῃς κεκαλυμμένα κυανέῃσι. \num{745} 

τῶν πρόσθ' Ἰαπετοῖο πάις ἔχει οὐρανὸν εὐρὺν

ἑστηὼς κεφαλῇ τε καὶ ἀκαμάτῃσι χέρεσσιν

ἀστεμφέως, ὅθι Νύξ τε καὶ Ἡμέρη ἆσσον ἰοῦσαι

ἀλλήλας προσέειπον ἀμειβόμεναι μέγαν οὐδὸν 

χάλκεον· ἡ μὲν ἔσω καταβήσεται, ἡ δὲ θύραζε \num{750} 

ἔρχεται, οὐδέ ποτ' ἀμφοτέρας δόμος ἐντὸς ἐέργει,

ἀλλ' αἰεὶ ἑτέρη γε δόμων ἔκτοσθεν ἐοῦσα

γαῖαν ἐπιστρέφεται, ἡ δ' αὖ δόμου ἐντὸς ἐοῦσα 

μίμνει τὴν αὐτῆς ὥρην ὁδοῦ, ἔστ' ἂν ἵκηται· 

ἡ μὲν ἐπιχθονίοισι φάος πολυδερκὲς ἔχουσα, \num{755} 

ἡ δ' Ὕπνον μετὰ χερσί, κασίγνητον Θανάτοιο, 

Νὺξ ὀλοή, νεφέλῃ κεκαλυμμένη ἠεροειδεῖ.

ἔνθα δὲ Νυκτὸς παῖδες ἐρεμνῆς οἰκί' ἔχουσιν,

Ὕπνος καὶ Θάνατος, δεινοὶ θεοί· οὐδέ ποτ' αὐτοὺς 

Ἠέλιος φαέθων ἐπιδέρκεται ἀκτίνεσσιν \num{760}

οὐρανὸν εἰσανιὼν οὐδ' οὐρανόθεν καταβαίνων. 

τῶν ἕτερος μὲν γῆν τε καὶ εὐρέα νῶτα θαλάσσης 

ἥσυχος ἀνστρέφεται καὶ μείλιχος ἀνθρώποισι,

τοῦ δὲ σιδηρέη μὲν κραδίη, χάλκεον δέ οἱ ἦτορ

νηλεὲς ἐν στήθεσσιν· ἔχει δ' ὃν πρῶτα λάβῃσιν \num{765} 

ἀνθρώπων· ἐχθρὸς δὲ καὶ ἀθανάτοισι θεοῖσιν. 

ἔνθα θεοῦ χθονίου πρόσθεν δόμοι ἠχήεντες

ἰφθίμου τ' Ἀίδεω καὶ ἐπαινῆς Περσεφονείης

ἑστᾶσιν, δεινὸς δὲ κύων προπάροιθε φυλάσσει, 

νηλειής, τέχνην δὲ κακὴν ἔχει· ἐς μὲν ἰόντας \num{770} 

σαίνει ὁμῶς οὐρῇ τε καὶ οὔασιν ἀμφοτέροισιν,

ἐξελθεῖν δ' οὐκ αὖτις ἐᾷ πάλιν, ἀλλὰ δοκεύων

ἐσθίει, ὅν κε λάβῃσι πυλέων ἔκτοσθεν ἰόντα.

ἰφθίμου τ' Ἀίδεω καὶ ἐπαινῆς Περσεφονείης.

ἔνθα δὲ ναιετάει στυγερὴ θεὸς ἀθανάτοισι, \num{775}

δεινὴ Στύξ, θυγάτηρ ἀψορρόου Ὠκεανοῖο

πρεσβυτάτη· νόσφιν δὲ θεῶν κλυτὰ δώματα ναίει 

μακρῇσιν πέτρῃσι κατηρεφέ'· ἀμφὶ δὲ πάντῃ 

κίοσιν ἀργυρέοισι πρὸς οὐρανὸν ἐστήρικται.

παῦρα δὲ Θαύμαντος θυγάτηρ πόδας ὠκέα Ἶρις \num{780}

ἀγγελίῃ πωλεῖται ἐπ' εὐρέα νῶτα θαλάσσης. 

ὁππότ' ἔρις καὶ νεῖκος ἐν ἀθανάτοισιν ὄρηται, 

καί ῥ' ὅστις ψεύδηται Ὀλύμπια δώματ' ἐχόντων,

Ζεὺς δέ τε Ἶριν ἔπεμψε θεῶν μέγαν ὅρκον ἐνεῖκαι

τηλόθεν ἐν χρυσέῃ προχόῳ πολυώνυμον ὕδωρ, \num{785} 

ψυχρόν, ὅ τ' ἐκ πέτρης καταλείβεται ἠλιβάτοιο 

ὑψηλῆς· πολλὸν δὲ ὑπὸ χθονὸς εὐρυοδείης 

ἐξ ἱεροῦ ποταμοῖο ῥέει διὰ νύκτα μέλαιναν· 

Ὠκεανοῖο κέρας, δεκάτη δ' ἐπὶ μοῖρα δέδασται· 

ἐννέα μὲν περὶ γῆν τε καὶ εὐρέα νῶτα θαλάσσης \num{790}

δίνῃς ἀργυρέῃς εἱλιγμένος εἰς ἅλα πίπτει,

ἡ δὲ μί' ἐκ πέτρης προρέει, μέγα πῆμα θεοῖσιν. 

ὅς κεν τὴν ἐπίορκον ἀπολλείψας ἐπομόσσῃ

ἀθανάτων οἳ ἔχουσι κάρη νιφόεντος Ὀλύμπου,

κεῖται νήυτμος τετελεσμένον εἰς ἐνιαυτόν· \num{795} 

οὐδέ ποτ' ἀμβροσίης καὶ νέκταρος ἔρχεται ἆσσον

βρώσιος, ἀλλά τε κεῖται ἀνάπνευστος καὶ ἄναυδος

στρωτοῖς ἐν λεχέεσσι, κακὸν δ' ἐπὶ κῶμα καλύπτει.

αὐτὰρ ἐπὴν νοῦσον τελέσει μέγαν εἰς ἐνιαυτόν,

ἄλλος δ' ἐξ ἄλλου δέχεται χαλεπώτερος ἆθλος· \num{800} 

εἰνάετες δὲ θεῶν ἀπαμείρεται αἰὲν ἐόντων,

οὐδέ ποτ' ἐς βουλὴν ἐπιμίσγεται οὐδ' ἐπὶ δαῖτας

ἐννέα πάντ' ἔτεα· δεκάτῳ δ' ἐπιμίσγεται αὖτις 

εἴρας ἐς ἀθανάτων οἳ Ὀλύμπια δώματ' ἔχουσι. 

τοῖον ἄρ' ὅρκον ἔθεντο θεοὶ Στυγὸς ἄφθιτον ὕδωρ, \num{805} 

ὠγύγιον· τὸ δ' ἵησι καταστυφέλου διὰ χώρου. 

ἔνθα δὲ γῆς δνοφερῆς καὶ ταρτάρου ἠερόεντος 

πόντου τ' ἀτρυγέτοιο καὶ οὐρανοῦ ἀστερόεντος

ἑξείης πάντων πηγαὶ καὶ πείρατ' ἔασιν, 

ἀργαλέ' εὐρώεντα, τά τε στυγέουσι θεοί περ. \num{810}

ἔνθα δὲ μαρμάρεαί τε πύλαι καὶ χάλκεος οὐδός, 

ἀστεμφὲς ῥίζῃσι διηνεκέεσσιν ἀρηρώς, 

αὐτοφυής· πρόσθεν δὲ θεῶν ἔκτοσθεν ἁπάντων 

Τιτῆνες ναίουσι, πέρην χάεος ζοφεροῖο. 

αὐτὰρ ἐρισμαράγοιο Διὸς κλειτοὶ ἐπίκουροι \num{815}

δώματα ναιετάουσιν ἐπ' Ὠκεανοῖο θεμέθλοις,

Κόττος τ' ἠδὲ Γύγης· Βριάρεών γε μὲν ἠὺν ἐόντα 

γαμβρὸν ἑὸν ποίησε βαρύκτυπος Ἐννοσίγαιος,

δῶκε δὲ Κυμοπόλειαν ὀπυίειν, θυγατέρα ἥν.

αὐτὰρ ἐπεὶ Τιτῆνας ἀπ' οὐρανοῦ ἐξέλασε Ζεύς, \num{820} 

ὁπλότατον τέκε παῖδα Τυφωέα Γαῖα πελώρη

Ταρτάρου ἐν φιλότητι διὰ χρυσῆν Ἀφροδίτην· 

οὗ χεῖρες †μὲν ἔασιν ἐπ' ἰσχύι ἔργματ' ἔχουσαι,†

καὶ πόδες ἀκάματοι κρατεροῦ θεοῦ· ἐκ δέ οἱ ὤμων 

ἦν ἑκατὸν κεφαλαὶ ὄφιος δεινοῖο δράκοντος, \num{825} 

γλώσσῃσι δνοφερῇσι λελιχμότες· ἐκ δέ οἱ ὄσσων 

θεσπεσίῃς κεφαλῇσιν ὑπ' ὀφρύσι πῦρ ἀμάρυσσεν· 

πασέων δ' ἐκ κεφαλέων πῦρ καίετο δερκομένοιο· 

φωναὶ δ' ἐν πάσῃσιν ἔσαν δεινῇς κεφαλῇσι,

παντοίην ὄπ' ἰεῖσαι ἀθέσφατον· ἄλλοτε μὲν γὰρ \num{830} 

φθέγγονθ' ὥς τε θεοῖσι συνιέμεν, ἄλλοτε δ' αὖτε

ταύρου ἐριβρύχεω μένος ἀσχέτου ὄσσαν ἀγαύρου, 

ἄλλοτε δ' αὖτε λέοντος ἀναιδέα θυμὸν ἔχοντος,

ἄλλοτε δ' αὖ σκυλάκεσσιν ἐοικότα, θαύματ' ἀκοῦσαι,

ἄλλοτε δ' αὖ ῥοίζεσχ', ὑπὸ δ' ἤχεεν οὔρεα μακρά. \num{835}

καί νύ κεν ἔπλετο ἔργον ἀμήχανον ἤματι κείνῳ,

καί κεν ὅ γε θνητοῖσι καὶ ἀθανάτοισιν ἄναξεν,

εἰ μὴ ἄρ' ὀξὺ νόησε πατὴρ ἀνδρῶν τε θεῶν τε· 

σκληρὸν δ' ἐβρόντησε καὶ ὄβριμον, ἀμφὶ δὲ γαῖα

σμερδαλέον κονάβησε καὶ οὐρανὸς εὐρὺς ὕπερθε \num{840}

πόντός τ' Ὠκεανοῦ τε ῥοαὶ καὶ Τάρταρα γαίης. 

ποσσὶ δ' ὕπ' ἀθανάτοισι μέγας πελεμίζετ' Ὄλυμπος

ὀρνυμένοιο ἄνακτος· ἐπεστονάχιζε δὲ γαῖα.

καῦμα δ' ὑπ' ἀμφοτέρων κάτεχεν ἰοειδέα πόντον

βροντῆς τε στεροπῆς τε πυρός τ' ἀπὸ τοῖο πελώρου \num{845} 

πρηστήρων ἀνέμων τε κεραυνοῦ τε φλεγέθοντος· 

ἔζεε δὲ χθὼν πᾶσα καὶ οὐρανὸς ἠδὲ θάλασσα· 

θυῖε δ' ἄρ' ἀμφ' ἀκτὰς περί τ' ἀμφί τε κύματα μακρὰ

ῥιπῇ ὕπ' ἀθανάτων, ἔνοσις δ' ἄσβεστος ὀρώρει· 

τρέε δ' Ἀίδης ἐνέροισι καταφθιμένοισιν ἀνάσσων \num{850}

Τιτῆνές θ' ὑποταρτάριοι Κρόνον ἀμφὶς ἐόντες 

ἀσβέστου κελάδοιο καὶ αἰνῆς δηιοτῆτος. 

Ζεὺς δ' ἐπεὶ οὖν κόρθυνεν ἑὸν μένος, εἵλετο δ' ὅπλα,

βροντήν τε στεροπήν τε καὶ αἰθαλόεντα κεραυνόν,

πλῆξεν ἀπ' Οὐλύμποιο ἐπάλμενος· ἀμφὶ δὲ πάσας \num{855}

ἔπρεσε θεσπεσίας κεφαλὰς δεινοῖο πελώρου.

αὐτὰρ ἐπεὶ δή μιν δάμασε πληγῇσιν ἱμάσσας,

ἤριπε γυιωθείς, στονάχιζε δὲ γαῖα πελώρη· 

φλὸξ δὲ κεραυνωθέντος ἀπέσσυτο τοῖο ἄνακτος

οὔρεος ἐν βήσσῃσιν ἀιδνῆς παιπαλοέσσης \num{860}

πληγέντος, πολλὴ δὲ πελώρη καίετο γαῖα 

αὐτμῇ θεσπεσίῃ, καὶ ἐτήκετο κασσίτερος ὣς

τέχνῃ ὑπ' αἰζηῶν ἐν ἐυτρήτοις χοάνοισι

θαλφθείς, ἠὲ σίδηρος, ὅ περ κρατερώτατός ἐστιν, 

οὔρεος ἐν βήσσῃσι δαμαζόμενος πυρὶ κηλέῳ \num{865}

τήκεται ἐν χθονὶ δίῃ ὑφ' Ἡφαίστου παλάμῃσιν· 

ὣς ἄρα τήκετο γαῖα σέλαι πυρὸς αἰθομένοιο.

ῥῖψε δέ μιν θυμῷ ἀκαχὼν ἐς τάρταρον εὐρύν. 

ἐκ δὲ Τυφωέος ἔστ' ἀνέμων μένος ὑγρὸν ἀέντων,

νόσφι Νότου Βορέω τε καὶ ἀργεστέω Ζεφύροιο· \num{870}

οἵ γε μὲν ἐκ θεόφιν γενεήν, θνητοῖς μέγ' ὄνειαρ.

αἱ δ' ἄλλαι μὰψ αὖραι ἐπιπνείουσι θάλασσαν·

αἳ δή τοι πίπτουσαι ἐς ἠεροειδέα πόντον,

πῆμα μέγα θνητοῖσι, κακῇ θυίουσιν ἀέλλῃ· 

ἄλλοτε δ' ἄλλαι ἄεισι διασκιδνᾶσί τε νῆας \num{875}

ναύτας τε φθείρουσι· κακοῦ δ' οὐ γίνεται ἀλκὴ 

ἀνδράσιν, οἳ κείνῃσι συνάντωνται κατὰ πόντον. 

αἱ δ' αὖ καὶ κατὰ γαῖαν ἀπείριτον ἀνθεμόεσσαν

ἔργ' ἐρατὰ φθείρουσι χαμαιγενέων ἀνθρώπων, 

πιμπλεῖσαι κόνιός τε καὶ ἀργαλέου κολοσυρτοῦ. \num{880}

αὐτὰρ ἐπεί ῥα πόνον μάκαρες θεοὶ ἐξετέλεσσαν, 

Τιτήνεσσι δὲ τιμάων κρίναντο βίηφι,

δή ῥα τότ' ὤτρυνον βασιλευέμεν ἠδὲ ἀνάσσειν

Γαίης φραδμοσύνῃσιν Ὀλύμπιον εὐρύοπα Ζῆν

ἀθανάτων· ὁ δὲ τοῖσιν ἐὺ διεδάσσατο τιμάς. \num{885}

Ζεὺς δὲ θεῶν βασιλεὺς πρώτην ἄλοχον θέτο Μῆτιν, 

πλεῖστα θεῶν εἰδυῖαν ἰδὲ θνητῶν ἀνθρώπων. 

ἀλλ' ὅτε δὴ ἄρ' ἔμελλε θεὰν γλαυκῶπιν Ἀθήνην

τέξεσθαι, τότ' ἔπειτα δόλῳ φρένας ἐξαπατήσας

αἱμυλίοισι λόγοισιν ἑὴν ἐσκάτθετο νηδύν, \num{890} 

Γαίης φραδμοσύνῃσι καὶ Οὐρανοῦ ἀστερόεντος· 

τὼς γάρ οἱ φρασάτην, ἵνα μὴ βασιληίδα τιμὴν

ἄλλος ἔχοι Διὸς ἀντὶ θεῶν αἰειγενετάων.

ἐκ γὰρ τῆς εἵμαρτο περίφρονα τέκνα γενέσθαι· 

πρώτην μὲν κούρην γλαυκώπιδα Τριτογένειαν, \num{895}

ἶσον ἔχουσαν πατρὶ μένος καὶ ἐπίφρονα βουλήν,

αὐτὰρ ἔπειτ' ἄρα παῖδα θεῶν βασιλῆα καὶ ἀνδρῶν

ἤμελλεν τέξεσθαι, ὑπέρβιον ἦτορ ἔχοντα·

ἀλλ' ἄρα μιν Ζεὺς πρόσθεν ἑὴν ἐσκάτθετο νηδύν,

ὥς οἱ συμφράσσαιτο θεὰ ἀγαθόν τε κακόν τε. \num{900}

δεύτερον ἠγάγετο λιπαρὴν Θέμιν, ἣ τέκεν Ὥρας,

Εὐνομίην τε Δίκην τε καὶ Εἰρήνην τεθαλυῖαν,

αἵ τ' ἔργ' ὠρεύουσι καταθνητοῖσι βροτοῖσι,

Μοίρας θ', ᾗς πλείστην τιμὴν πόρε μητίετα Ζεύς, 

Κλωθώ τε Λάχεσίν τε καὶ Ἄτροπον, αἵ τε διδοῦσι \num{905} 

θνητοῖς ἀνθρώποισιν ἔχειν ἀγαθόν τε κακόν τε.

τρεῖς δέ οἱ Εὐρυνόμη Χάριτας τέκε καλλιπαρήους,

Ὠκεανοῦ κούρη πολυήρατον εἶδος ἔχουσα,

Ἀγλαΐην τε καὶ Εὐφροσύνην Θαλίην τ' ἐρατεινήν·

τῶν καὶ ἀπὸ βλεφάρων ἔρος εἴβετο δερκομενάων \num{910}

λυσιμελής· καλὸν δέ θ' ὑπ' ὀφρύσι δερκιόωνται. 

αὐτὰρ ὁ Δήμητρος πολυφόρβης ἐς λέχος ἦλθεν· 

ἣ τέκε Περσεφόνην λευκώλενον, ἣν Ἀιδωνεὺς

ἥρπασεν ἧς παρὰ μητρός, ἔδωκε δὲ μητίετα Ζεύς. 

Μνημοσύνης δ' ἐξαῦτις ἐράσσατο καλλικόμοιο, \num{915} 

ἐξ ἧς οἱ Μοῦσαι χρυσάμπυκες ἐξεγένοντο

ἐννέα, τῇσιν ἅδον θαλίαι καὶ τέρψις ἀοιδῆς.

Λητὼ δ' Ἀπόλλωνα καὶ Ἄρτεμιν ἰοχέαιραν 

ἱμερόεντα γόνον περὶ πάντων Οὐρανιώνων

γείνατ' ἄρ' αἰγιόχοιο Διὸς φιλότητι μιγεῖσα. \num{920}

λοισθοτάτην δ' Ἥρην θαλερὴν ποιήσατ' ἄκοιτιν· 

ἡ δ' Ἥβην καὶ Ἄρηα καὶ Εἰλείθυιαν ἔτικτε 

μιχθεῖσ' ἐν φιλότητι θεῶν βασιλῆι καὶ ἀνδρῶν. 

αὐτὸς δ' ἐκ κεφαλῆς γλαυκώπιδα γείνατ' Ἀθήνην, 

δεινὴν ἐγρεκύδοιμον ἀγέστρατον ἀτρυτώνην, \num{925} 

πότνιαν, ᾗ κέλαδοί τε ἅδον πόλεμοί τε μάχαι τε· 

Ἥρη δ' Ἥφαιστον κλυτὸν οὐ φιλότητι μιγεῖσα

γείνατο, καὶ ζαμένησε καὶ ἤρισεν ᾧ παρακοίτῃ,

ἐκ πάντων τέχνῃσι κεκασμένον Οὐρανιώνων.

ἐκ δ' Ἀμφιτρίτης καὶ ἐρικτύπου Ἐννοσιγαίου \num{930}

Τρίτων εὐρυβίης γένετο μέγας, ὅς τε θαλάσσης 

πυθμέν' ἔχων παρὰ μητρὶ φίλῃ καὶ πατρὶ ἄνακτι

ναίει χρύσεα δῶ, δεινὸς θεός. αὐτὰρ Ἄρηι

ῥινοτόρῳ Κυθέρεια Φόβον καὶ Δεῖμον ἔτικτε, 

δεινούς, οἵ τ' ἀνδρῶν πυκινὰς κλονέουσι φάλαγγας \num{935}

ἐν πολέμῳ κρυόεντι σὺν Ἄρηι πτολιπόρθῳ,

Ἁρμονίην θ', ἣν Κάδμος ὑπέρθυμος θέτ' ἄκοιτιν. 

Ζηνὶ δ' ἄρ' Ἀτλαντὶς Μαίη τέκε κύδιμον Ἑρμῆν,

κήρυκ' ἀθανάτων, ἱερὸν λέχος εἰσαναβᾶσα.

Καδμηὶς δ' ἄρα οἱ Σεμέλη τέκε φαίδιμον υἱὸν \num{940} 

μιχθεῖσ' ἐν φιλότητι, Διώνυσον πολυγηθέα, 

ἀθάνατον θνητή· νῦν δ' ἀμφότεροι θεοί εἰσιν. 

Ἀλκμήνη δ' ἄρ' ἔτικτε βίην Ἡρακληείην

μιχθεῖσ' ἐν φιλότητι Διὸς νεφεληγερέταο. 

Ἀγλαΐην δ' Ἥφαιστος ἀγακλυτὸς ἀμφιγυήεις \num{945} 

ὁπλοτάτην Χαρίτων θαλερὴν ποιήσατ' ἄκοιτιν.

χρυσοκόμης δὲ Διώνυσος ξανθὴν Ἀριάδνην,

κούρην Μίνωος, θαλερὴν ποιήσατ' ἄκοιτιν· 

τὴν δέ οἱ ἀθάνατον καὶ ἀγήρων θῆκε Κρονίων.

Ἥβην δ' Ἀλκμήνης καλλισφύρου ἄλκιμος υἱός, \num{950} 

ἲς Ἡρακλῆος, τελέσας στονόεντας ἀέθλους,

παῖδα Διὸς μεγάλοιο καὶ Ἥρης χρυσοπεδίλου,

αἰδοίην θέτ' ἄκοιτιν ἐν Οὐλύμπῳ νιφόεντι· 

ὄλβιος, ὃς μέγα ἔργον ἐν ἀθανάτοισιν ἀνύσσας

ναίει ἀπήμαντος καὶ ἀγήραος ἤματα πάντα. \num{955}

Ἠελίῳ δ' ἀκάμαντι τέκε κλυτὸς Ὠκεανίνη 

Περσηὶς Κίρκην τε καὶ Αἰήτην βασιλῆα.

Αἰήτης δ' υἱὸς φαεσιμβρότου Ἠελίοιο

κούρην Ὠκεανοῖο τελήεντος ποταμοῖο

γῆμε θεῶν βουλῇσιν, Ἰδυῖαν καλλιπάρηον· \num{960}

ἣ δή οἱ Μήδειαν ἐύσφυρον ἐν φιλότητι

γείναθ' ὑποδμηθεῖσα διὰ χρυσῆν Ἀφροδίτην. 

ὑμεῖς μὲν νῦν χαίρετ', Ὀλύμπια δώματ' ἔχοντες, 

νῆσοί τ' ἤπειροί τε καὶ ἁλμυρὸς ἔνδοθι πόντος· 

νῦν δὲ θεάων φῦλον ἀείσατε, ἡδυέπειαι \num{965}

Μοῦσαι Ὀλυμπιάδες, κοῦραι Διὸς αἰγιόχοιο,

ὅσσαι δὴ θνητοῖσι παρ' ἀνδράσιν εὐνηθεῖσαι

ἀθάναται γείναντο θεοῖς ἐπιείκελα τέκνα.

Δημήτηρ μὲν Πλοῦτον ἐγείνατο δῖα θεάων,

Ἰασίῳ ἥρωι μιγεῖσ' ἐρατῇ φιλότητι \num{970} 

νειῷ ἔνι τριπόλῳ, Κρήτης ἐν πίονι δήμῳ,

ἐσθλόν, ὃς εἶσ' ἐπὶ γῆν τε καὶ εὐρέα νῶτα θαλάσσης

πᾶσαν· τῷ δὲ τυχόντι καὶ οὗ κ' ἐς χεῖρας ἵκηται, 

τὸν δὴ ἀφνειὸν ἔθηκε, πολὺν δέ οἱ ὤπασεν ὄλβον.

Κάδμῳ δ' Ἁρμονίη, θυγάτηρ χρυσῆς Ἀφροδίτης, \num{975}

Ἰνὼ καὶ Σεμέλην καὶ Ἀγαυὴν καλλιπάρηον 

Αὐτονόην θ', ἣν γῆμεν Ἀρισταῖος βαθυχαίτης,

γείνατο καὶ Πολύδωρον ἐυστεφάνῳ ἐνὶ Θήβῃ.

κούρη δ' Ὠκεανοῦ Χρυσάορι καρτεροθύμῳ

μιχθεῖσ' ἐν φιλότητι πολυχρύσου Ἀφροδίτης \num{980}

Καλλιρόη τέκε παῖδα βροτῶν κάρτιστον ἁπάντων,

Γηρυονέα, τὸν κτεῖνε βίη Ἡρακληείη

βοῶν ἕνεκ' εἰλιπόδων ἀμφιρρύτῳ εἰν Ἐρυθείῃ.

Τιθωνῷ δ' Ἠὼς τέκε Μέμνονα χαλκοκορυστήν,

Αἰθιόπων βασιλῆα, καὶ Ἠμαθίωνα ἄνακτα. \num{985}

αὐτάρ τοι Κεφάλῳ φιτύσατο φαίδιμον υἱόν, 

ἴφθιμον Φαέθοντα, θεοῖς ἐπιείκελον ἄνδρα· 

τόν ῥα νέον τέρεν ἄνθος ἔχοντ' ἐρικυδέος ἥβης

παῖδ' ἀταλὰ φρονέοντα φιλομμειδὴς Ἀφροδίτη

ὦρτ' ἀνερειψαμένη, καί μιν ζαθέοις ἐνὶ νηοῖς \num{990} 

νηοπόλον μύχιον ποιήσατο, δαίμονα δῖον. 

κούρην δ' Αἰήταο διοτρεφέος βασιλῆος

Αἰσονίδης βουλῇσι θεῶν αἰειγενετάων

ἦγε παρ' Αἰήτεω, τελέσας στονόεντας ἀέθλους,

τοὺς πολλοὺς ἐπέτελλε μέγας βασιλεὺς ὑπερήνωρ, \num{995}

ὑβριστὴς Πελίης καὶ ἀτάσθαλος ὀβριμοεργός· 

τοὺς τελέσας ἐς Ἰωλκὸν ἀφίκετο πολλὰ μογήσας

ὠκείης ἐπὶ νηὸς ἄγων ἑλικώπιδα κούρην

Αἰσονίδης, καί μιν θαλερὴν ποιήσατ' ἄκοιτιν.

καί ῥ' ἥ γε δμηθεῖσ' ὑπ' Ἰήσονι ποιμένι λαῶν \num{1000} 

Μήδειον τέκε παῖδα, τὸν οὔρεσιν ἔτρεφε Χείρων

Φιλλυρίδης· μεγάλου δὲ Διὸς νόος ἐξετελεῖτο. 

αὐτὰρ Νηρῆος κοῦραι ἁλίοιο γέροντος,

ἤτοι μὲν Φῶκον Ψαμάθη τέκε δῖα θεάων

Αἰακοῦ ἐν φιλότητι διὰ χρυσῆν Ἀφροδίτην· \num{1005} 

Πηλεῖ δὲ δμηθεῖσα θεὰ Θέτις ἀργυρόπεζα

γείνατ' Ἀχιλλῆα ῥηξήνορα θυμολέοντα.

Αἰνείαν δ' ἄρ' ἔτικτεν ἐυστέφανος Κυθέρεια,

Ἀγχίσῃ ἥρωι μιγεῖσ' ἐρατῇ φιλότητι 

Ἴδης ἐν κορυφῇσι πολυπτύχου ἠνεμοέσσης. \num{1010}

Κίρκη δ' Ἠελίου θυγάτηρ Ὑπεριονίδαο

γείνατ' Ὀδυσσῆος ταλασίφρονος ἐν φιλότητι

Ἄγριον ἠδὲ Λατῖνον ἀμύμονά τε κρατερόν τε· 

{[}Τηλέγονον δὲ ἔτικτε διὰ χρυσῆν Ἀφροδίτην·{]}

οἳ δή τοι μάλα τῆλε μυχῷ νήσων ἱεράων \num{1015}

πᾶσιν Τυρσηνοῖσιν ἀγακλειτοῖσιν ἄνασσον.

Ναυσίθοον δ' Ὀδυσῆι Καλυψὼ δῖα θεάων

γείνατο Ναυσίνοόν τε μιγεῖσ' ἐρατῇ φιλότητι. 

αὗται μὲν θνητοῖσι παρ' ἀνδράσιν εὐνηθεῖσαι

ἀθάναται γείναντο θεοῖς ἐπιείκελα τέκνα. \num{1020}

{[}νῦν δὲ γυναικῶν φῦλον ἀείσατε, ἡδυέπειαι

Μοῦσαι Ὀλυμπιάδες, κοῦραι Διὸς αἰγιόχοιο.{]}
